\documentclass[output=paper,nobabel,draftmode  ,colorlinks, citecolor=brown]{langscibook}
\ChapterDOI{10.5281/zenodo.7142662}

\title{Three sources of head effects} 

\author{Yury Lander\orcid{0000-0003-1798-7174}\affiliation{HSE University, Moscow}}


\abstract{This paper elaborates on the idea that properties which are usually ascribed to heads have one of three sources: wide scope in semantic composition, information load (relevance), and origin from an appositive structure. Starting with constructions combining property words with words denoting objects, we proceed to possessive constructions, adpositional constructions and even clause-level phenomena, and argue that in all of them, the assignment of the relevant head properties to different elements may be motivated by the diversity of the sources. Given this picture, we tentatively conclude that in most cases we need not think of heads, but rather of head properties.}


\begin{document}

\maketitle

\largerpage[-1]
\section{Introduction}\label{sec-intro-lan}
This paper develops the claim that headedness, or more precisely, head effects owe their existence to several different factors. Taking constructions with adjectival words as an illustration, I will argue for three sources of head effects and then show that the same sources are relevant for other constructions as well. 
While the concept of ``head'' is basic in many linguistic theories, calling something ``head'' is
often rooted in nothing more than linguistic tradition. The authors of grammatical descriptions and
theoretical treatments make precise what they mean by ``being the head'' very rarely. Here I rely on
the following properties which are frequently ascribed to the head of a construct\footnote{I use the
  term construct rather than, for example, constituent, since the properties listed here are in
  principle applicable to discontinuous dependencies.}, cf.\ discussion of head properties in
\citet{Zwicky85a}; \citet{CFMcG93a-ed}; \citet{Croft1996};  \citet[242--254]{Croft2002} \emph{inter
  alia}:\footnote{Many other tests proposed in the literature are not discussed here. First, I do
  not use semantic tests, since I think of headedness as a grammatical rather than a semantic
  phenomenon. Second, I avoid tests that require theory-specific analyses. An example of such a test
  requires that the head is the category determinant, i.e.\ ``[i]t determines the syntactic category of the construct as a whole'' \citep[297]{Zwicky1993}. Presumably, this test is highly dependent on our view on syntactic categories, which, however, is not stable enough.}

\begin{itemize}
    \item the head is required in the construct,\footnotemark
    \item the head can determine the external syntax of the construct: the syntactic distribution of the construct (including the forms of the elements that combine with the construct) is often predictable from properties of the head,
    \item the head can determine the internal syntax of the construct: it makes it possible to predict what elements (simplex or complex) may appear within the construct and assigns syntactic functions to these elements; such function assignment may manifest itself in rules governing the word order and form of any participant of a construction,
    \item the head can be chosen as the locus of morphosyntactic marking,
    \item the head can appear as a distributional equivalent of the construct (i.e.\ it can appear alone in the same positions as the construct).
\footnotetext{Importantly, here I only mean overt elements and abstract away from the issue of null elements in syntax. Note, further, that in most syntactic theories not only heads are assumed to be obligatory but also their arguments/complements. This may lead to confusion: according to this criterion, the same elements may be depicted as heads and as their arguments. It is not obvious to me that this does not reflect the actual situation, though. For example, whenever one speaks of the grammatical category of definiteness, one assumes a parameter whose value must be specified, and this looks more like a specification of an argument. At the same time, definite articles are often assumed to head the nominal phrases on the basis of this and other criteria. Thus, indeed, the same elements sometimes can be treated as heads and as arguments depending on the perspective.}
\end{itemize}

\noindent
Though commonly accepted, these properties deserve a few comments. 

First, I admit that head properties are applicable to both words and phrases\footnote{Phrasal heads fit well into the definitions of ‘heads’ provided by some theories, e.g., by Categorial Grammar (\cite{Dowty2003}). Furthermore, it is normal to think about phrasal heads when discussing such patterns as relative clause constructions (\cite{KC77a}). Though it is possible that in the relevant discussions of relative clauses, the term ``head'' is used differently from the discussions of many other grammatical patterns (since it is based primarily on semantics), the ``heads'' of these constructions often display the properties listed above.} (but in theories that allow only lexical heads, the points provided below should be reformulated using the notions of ``projection'', ``percolation'', etc.). Furthermore, head properties can be discussed with respect to roots and affixes, but here I abstract away from this issue. Second, the original notion of headedness in non-coordinating constructions presupposes asymmetry but many of the head properties do not. For example, obligatoriness often holds for several parts of the construction (e.g., in \emph{the dog} both the noun and the determiner are obligatory). Third, head properties do not unambiguously point to the head, since they are sometimes distributed among different elements. Fourth, sometimes a property which is typical for alleged heads allows an alternative explanation. For example, the locus of morphosyntactic marking is often determined with respect to the edge of a phrase, cf.\ \citet{Klavans1985}; \citet[210]{Anderson92a-u},
or such marking occurs on all words of a constituent that are available to such marking, see \citet{LanderNichols2020}  for a preliminary typology. 

With all this in mind, I prefer to speak not of the heads but of the head effects and head properties. This is not to deny the very idea that something may be treated as the head of a constituent. Head properties probably tend to converge, but this is still worthy of cross-linguistic and cross-constructional investigation. At the same time, I admit that head effects are facts of grammar and as such result from grammaticalization of certain principles leading to asymmetry between elements.\footnote{The term grammaticalization is used here broadly, as ``the shifting from relatively freely constructed utterances in discourse, whose idiosyncratic form is motivated only by the speaker's goals for the immediate speech event (…) to relatively fixed constructions in grammar, seen as arbitrary (though ultimately not necessarily unmotivated) constraints on the speaker's output'' \citep[346]{Dubois1984}.} 
In Section~\ref{sec-adjprob}, I discuss the problem posed by the fact that adjective-like words sometimes have properties of heads of nominal phrases. In Section~\ref{sec-headeffects}, it is argued that this phenomenon receives different explanations in different constructions. Section~\ref{sec-extending} shows that similar explanations are applicable to possessive, adpositional and even clause-level patterns. The final section~\ref{sec-conclu-lan} summarizes the paper.

\section{The Adjectival Problem}\label{sec-adjprob}

Below I assume that from the semantic perspective we can think of more ``adjectival'' words (\textsc{adj}s) and more ``nouny'' words (\textsc{n}s), irrespective of part-of-speech distinctions. This is in accordance with current typological practices. For example, \citet{wals-87} in his discussion of the order of ``modifying adjective'' and ``noun'' states that for his purposes

\begin{quote}
    the term \emph{adjective} should be interpreted in a semantic sense, as a word denoting a
    descriptive property, with meanings such as `big', `good', or `red'. [\ldots] In some languages, like
    \ili{English}, adjectives form a distinct word class. In other languages, however, adjectives do not
    form a distinct word class and are verbs or nouns [\ldots{}]. \citep{wals-87}
\end{quote}
A similar semantic understanding of ``adjectives'' is found in many other typological works; cf.\ \citet*[670]{Haspelmath2010}, \citet*[6]{Riessler2016} among others. 
Essentially, it is intended for comparing languages with very different systems and providing generalizations which are not bound by specific grammatical characteristics, irrespectively of whether \textsc{n}s and \textsc{adj}s have the same grammatical distribution and are contrasted with other content words, \textsc{adj}s and clearly verbal expressions constitute one part of speech grammatically contrasted with nouns, or any other situation. Notably, however, I do not discuss all ``property words'' here: while being interested in \textsc{adj}s that apparently serve as heads of \textsc{np}s, I remove from consideration all kinds of \textsc{adj}s which behave in parallel to relative clauses.\footnote{The fact that in many languages \textsc{adj}s pattern together with verbs is well-known, cf. \citet{Beck2002}  and the literature cited there. However, there is evidence that sometimes \textsc{adj}s can be described as reduced relative clauses even in languages where adjectives are contrasted with verbs. For example, in \ili{Tanti Dargwa} (East Caucasian), adjectives are clearly distinct from verbs in many morphosyntactic properties. Yet when appearing as attributes, they manifest a subtype of relative clause and can have overt subjects which may but need not coincide with the modified noun \citep[198--199]{Sumbatova2014}. \citet{Cinque2010} argued that even in some Standard Average European languages, adjectival expressions can be divided into reduced relative clauses and base-generated expressions (which he considered to represent functional heads).}

Traditional European linguistics seemingly assumes that in combinations like (\ref{ex-smallanimal}) \textsc{adj} is a modifier of \textsc{n}.

\eal\label{ex-smallanimal}
\ex `small' $+$ `animal' \\ 
\ex `old' $+$ `person'\\
\ex `private' $+$ `person' \\ 
\ex `old-fashioned' $+$ `book'\\
\ex `principal' $+$ `investigator'
\zl
This assumption is reflected in the discussions of the concept of head. For example, among the
criteria of headedness listed by \citet{Zwicky85a} in his now classic paper, we find a test for
\emph{semantic headedness} which is described in the following way: ``in a combination X $+$ Y, X is
the `semantic head' if, speaking very crudely, X $+$ Y describes a kind of the thing described by X''
\citep[4]{Zwicky85a}. According to this test, the \textsc{n} in the combinations such as
(\ref{ex-smallanimal}) should be the head, at least if we follow \citet*[359]{Wierzbicka1986} in
accepting that unlike an adjective, ``a noun designates `a kind of (person, thing, or whatever)',
rather than merely a single property'' (as an adjective does).\footnote{A reviewer pointed out that
  nouns like \emph{thing} and \emph{person}, which may be thought of as semantically almost vacuous, present a
  problem for this approach.} 

The ``Adjectival Problem'' I discuss below is related to the fact that
in many languages the reality does match this picture, so that \textsc{adj}s combined with
\textsc{n}s display head effects. For example, diminutive (`small') and augmentative (`big')
%\itdopt{S: Sind \emph{small} und \emph{big} Wörter? Dann kursiv.}
expressions are sometimes based on the constructions with words with the meaning of `child’ and
`mother’, cf.\ \citet{Matisoff1992}; \citet{Jurafsky1996}; \citet*[65--67]{HeineKuteva2002} among
others. This kind of construction often develops from possessives: `child of X' turns into `small
X', `mother of X' turns into `big X', etc. In adnominal possessives the possessee normally has head
properties, so \textsc{adj}s with the meaning `small' and `big' are expected to behave as syntactic
heads then. For example, in (\ref{ex-smallbasket}), where the noun for `daughter' refers to the
property of being small, it takes a ``head-marking'' suffix, which normally marks the feminine
gender possessee in possessive constructions (\ref{ex-boysname}) and assigns the following noun the
possessor function. This pattern goes against the assumption that in a combination of an
\textsc{adj} and a \textsc{n}, the former should be syntactically a modifier of the latter. 

\ea
\langinfo{Miya}{Afro-Asiatic, Chadic}{glosses are mine – YuL}
\ea\label{ex-smallbasket} 
\gll wùn-a	baday \\ daughter-\textsc{poss.f}	basket \\
\glt `small basket' \citep[54, 258]{Schuh1998}

\ex\label{ex-boysname}
\gll ngə̀n-a	          və́rkə \\ 
     name-\textsc{poss.f} boy \\ 
\glt `the boy's name' \citep[249]{Schuh1998}
\z
\z
%\itdopt{St: warum kursiv?}
% \itdopt{Sonderzeichen einfügen: ngə̀n-a	və́rkə̀}

%\largerpage
\noindent
This phenomenon is not restricted to occasional combinations. 
\citet{Ross1998} and \citet{Malchukov2000} describe numerous languages which display the phenomenon
dubbed ``possessive-like attribute constructions'' by the first author and ``dependency reversal''
by the second author. In such constructions, an assumed semantic modifier appears as an apparent
syntactic head of the phrase. The relevant pattern is illustrated in (\ref{ex-alberto}), whose
comparison with the adnominal possessive construction (\ref{ex-shortman}) suggests that the
\textsc{adj} here appears as the possessee-like head (taking a marker which normally indexes the
possessor on the possessee) and the \textsc{n} behaves as the possessor-like dependent (preceding its
presumable head in accordance with the general ``head-final'' order in the language;
cf.\ \citet*[69--70]{Green1999}):\footnote{The glosses are changed according to \citegen{Koontz-GarbodenFrancez2010} treatment.}

\ea
\langinfo{Ulwa}{Misumalpan}{\citealt[200]{Koontz-GarbodenFrancez2010}}\\
\ea\label{ex-alberto}
\gll Alberto pan-ka \\ Alberto stick-\textsc{pr.3sg} \\
\glt `Alberto's stick'
\ex\label{ex-shortman}
\gll al	 adah-ka as \\ 
     man short-\textsc{pr.3sg} \textsc{indf} \\
\glt `a short man'
\z\z

\noindent
\citet{Ross1998}, in his detailed study of the construction in Oceanic languages, showed that the
range of apparent possessee-like elements in Oceanic possessive-like attribute constructions is
closed and typically includes the concepts belonging to the semantic domains of \textsc{dimension},
\textsc{value} and \textsc{age}. This list is remarkable because, as was noted by Ross himself, it
consists of almost all the categories which belong to the core of the semantical adjectival category
as according to \citet{Dixon1979}, the exception being the domain of \textsc{color}. 

\largerpage[-1]
In other languages, however, the range of adjectival concepts participating in this kind of construction is open. An example is presented by West Caucasian languages, here illustrated with \ili{West Circassian}. In this language, \textsc{n}s and \textsc{adj}s constitute a complex stem, cf.\ \citet{Lander2017}, where the \textsc{n} precedes the \textsc{adj}  and is incorporated into it (\ref{ex-river}). There are several arguments for this direction of incorporation. First, such a description makes the nominal complex consistent in branching, since in other similar patterns the preposed \textsc{n} is incorporated into the following element (\ref{ex-iron}). This goes in line with the overall left-branching of the \ili{West Circassian} stems and morphosyntax in general, cf.\ \citet{KorotkovaLander2010}.  
Second, and more importantly, the distribution of a nominal complex sometimes depends on an \textsc{adj}. In particular, it is the \textsc{adj} that determines the (pragmatic) possibility of adding a comparative marker to the whole nominal complex, as in (\ref{ex-worldbright}).\footnote{The description according to which the \textsc{adj} heads the nominal complex in \ili{West Circassian} was originally proposed by Svetlana Toldova together with the author.}

\newpage
\ea
\langinfo{West Circassian}{West Caucasian}{corpus data: \href{http://adyghe.web-corpora.net}{adyghe.web-corpora.net}}\\
\ea
\label{ex-river}	
\gll adəɡjejə-m	    jə-qʷəŝhe-xe-r,    \emph{jə-psəχʷe-čer-xe-r}   		daxe-x\\
     Adyghea-\textsc{obl}   \textsc{poss}-mountain-\textsc{pl}-\textsc{abs} \textsc{poss}-river-tumultuous-\PL-\ABS{}	beautiful-\PL\\
\glt `The mountains of Adyghea, its tumultuous rivers are 	beautiful.'
\ex\label{ex-iron}
\gll	\emph{ʁʷəč’ə-maste-m}	r-e-lažʲe\\
		iron-needle-\OBL{}	\textsc{instr}-\textsc{dyn}-work\\
\glt `S/he is working with an iron needle.'
\ex\label{ex-worldbright}
\gll	mə	λ’ə-m	\emph{nah}	\emph{c’əf-halel}	mə	dwənaje-nefəne-m      		tje-b-ʁʷete-n-ep\\
	this	man-\textsc{obl} more	person-generous	this	world-bright-\textsc{obl}		\LOC-2\SG.\ERG-find-\textsc{mod}-\textsc{neg}\\
\glt `You will not find a person who is more generous than this man in this bright world.'

\z
\z
%\itdopt{S an YL: Is INSTR instrumental? Leipzig Glossing Rules is INS}
%If possible, I would retain INSTR. The problem is that for West \ili{Circassian} we use INS for the instrumental case and INSTR for the instrumental preverb. Fortunately, the instrumental case does not occur in the examples in this paper, hence it is in principle possible to replace INSTR with INS here, but it will make this paper less consistent with other papers on West \ili{Circassian} published by our group. 

\noindent
Thus, \textsc{adj}s can behave as apparent heads – a phenomenon which is probably not that rare. But in some languages \textsc{adj}s have head properties outside of the ``dependency reversal'' phenomenon patterns as well. As mentioned earlier, \citet{Dixon1979} argued that the core of the semantic adjectival category includes the words denoting \textsc{dimension}, \textsc{value}, \textsc{age} and \textsc{color}. This conclusion partly relied on languages with a grammatically distinct closed class of adjectives covering exactly the semantic domains listed above. Some of these languages (e.g.\ Hausa) distinguish the adjectival class on the basis of dependency reversal, but others do not. If we look at the morphosyntactic properties mentioned by Dixon for such classes, we will find that they include the expression of certain categories of NPs such as number or gender. Now, while the expression of such categories as number, case and gender on attributes is usually treated as agreement which marks their dependent status, the logic can be reversed as easily as dependency relations can be. Marking of a category of a whole phrase makes its host a morphosyntactic locus. Being a morphosyntactic locus can be a head property. Therefore, the members of small adjectival classes sometimes display head effects, even though this does not make them unambiguous heads of the nominal phrases.

\largerpage[-1]
This logic can be further extended to many systems with an open adjectival class whose members display the so-called NP-internal agreement \citep[cf.][21--23 for \ili{Russian}]{Corbett93a}. Curiously, \citet{Moravcsik1995} noticed that adjectives are more likely to agree with their nominal heads than possessors (at least possessors displaying the Suffixaufnahme, i.e.\ double case marking). \citet{Lander2010} argued that if a presumable modifier agrees in NP-categories, it normally can be used without the noun head, hence representing the whole NP and demonstrating a head property.\footnote{Languages differ in whether there is a need to postulate a null modified noun in such structures. In \ili{Russian}, for example, the adjective used without a noun nevertheless takes inflection based on the formal (not semantic) gender of the omitted noun, which evidences for a null noun controlling NP-internal agreement. However, this does not deprive the adjective of its formal head properties.}
Thus, \textsc{adj}s may have head properties even where they are usually thought to be modifiers.\footnote{In the generative tradition, adjectives are often thought as heading some functional projections of the noun (and hence presumably being able to take some head properties) at least since \citet{Abney87a}. See \citet{Cinque2010} for a  discussion.} 

\section{Ways of capturing head effects}\label{sec-headeffects}

In this section, I discuss explanations that can be offered for head effects. As we will see, there are several factors at play here.

\subsection{Scope-based compositional effects}\label{subsec-scopebased}

When adjectives are considered modifiers of nouns, trivially, they are assumed to be added to
nouns. If so, they should be semantically added higher than nouns (``after nouns'') and should have
scope over a noun. Some languages might rely on this in constructing their morphosyntactic
structures. Here, a compositionally higher element (i.e.\ an element having semantic scope over other
relevant elements) displays head effects.

Not all \textsc{adj}s need to have scope over a noun, though. As known from formal semantic studies
of adjectives (see \citet{Kamp1975} and \citet{Siegel1976} for original discussion and
\citet{McNally2016} for a recent overview), many \textsc{adj}s can be interpreted as predicates
restricting sets of individuals. Their combinations with \textsc{n}s are interpreted as
intersections of two sets: e.g., \emph{black flags} refers to the intersection of a set of black
individuals and a set of flags. Hence, combinations of such \textsc{adj}s (called \emph{intersective}
\textsc{adj}s) and \textsc{n}s need not involve semantic asymmetry, although the asymmetric option of
composition – when an adjectival predicate narrows down the set of possible referents provided by
the \textsc{n} – is still retained. 

On a par with intersective \textsc{adj}s, we find \emph{non-intersective} \textsc{adj}s, whose
interpretation requires the knowledge of the \textsc{n} being modified and as such has scope over
that \textsc{n}'s denotation. Some of them (e.g.\ \emph{main}, \emph{skillful}) still determine a subset of
the set restricted by the \textsc{n} with which they combine. Interestingly, though, such \emph{subsective
non-intersective} meanings are often conveyed by nouns and hence pretend to have head properties
within an NP: for example, the meaning `main' is regularly expressed as `the head of' and the
meaning `skillful' is often expressed as `the master of'. 

Finally, we find \emph{non-subsective} (or \emph{privative}) \textsc{adj}s with the meaning `former', whose combinations with
\textsc{n}s do not even establish a subset of the denotation of the latter (see \cite{Kamp1975}, \cite{KampPartee1995}, \cite{Partee2010} for discussion).\footnote{A reviewer pointed out that similar but different problems arise with intensional expressions such as \emph{alleged}. Interestingly, however, it is not even very clear that such expressions should be treated as \textsc{adj}s – in fact, in many languages they are served by relative clause constructions (including participial ones).} Occasionally such concepts are expressed by basically subsective or even intersective \textsc{adj}s with the meaning `old' (e.g., in \ili{Turkish}), and in some languages concepts like `former' are conveyed by grammatical means such as a specific derivational morpheme (like \ili{English} \emph{ex}-) or nominal tense (\cite{NordlingerSadler2004}). Still, whenever the privative concepts are expressed by dedicated words, non-subsective \textsc{adj}s may show head properties. For example, the concept `former' sometimes is expressed by the noun for `trace of' appearing as the possessee in a possessive construction (see \citealt{Lander2009} for discussion):

\ea 
\langinfo{Sundanese}{Austronesian}{\citealt[36]{Hardjadibrata1985}}\\
\gll urut 	pamajikana-na \\ 	
     trace	wife-\textsc{pr.3sg} \\
\glt `his ex-wife'
\z
Unfortunately, I am not aware of studies investigating the differences of expression of different types of \textsc{adj}s in this perspective cross-linguistically.

\subsection{Relevance}\label{sec-relevance}

\citet*[55]{Malchukov2000} suggested that the dependency reversal may have a functional motivation, namely the ``discourse-pragmatic salience of the attributive constituent'' and provided facts from various languages that point in this direction. For example, in \ili{Latin} the dependency reversal construction like that in (\ref{ex-latin}) was typically used either when the semantic modifier was non-restrictive, or when it was contrastive or emphatic. 


\ea\label{ex-latin} 
\langinfo{Latin}{Indo-European, Italic}{\citealt[949]{Pinkster2015}}\\
\gll  arbor-um	quae	hum-i		arid-o atque harenos-o	gign-untur \\ 
      tree-\textsc{gen.pl} which soil-\textsc{gen.sg}	dry-\textsc{sg}	and	sandy-\textsc{abl.sg} grow-\textsc{3pl.pass.prs} \\
\glt `trees, which grow in a dry and sandy soil'
\z

\noindent
Following this line, I propose that an element of a constituent sometimes shows head effects due to
its extraordinary information load, called \emph{relevance} below.\footnote{\textcite[257--259]{Croft1996, Croft2002} relates the status of head with the ``primary information-bearing unit'' (PIBU), which certainly reflects this factor. Note, however, that for \textsc{adj+n} combinations, it is not that easy to determine what the PIBU is.}
%Thus, a particularly relevant element may have head properties. 
(It is true, however, that defining the relevance and measuring the information load precisely is a
problem.) 

There can be different reasons to assign relevance to elements. In combinations of \textsc{adj}s and
\textsc{n}s, the latter are presumably relevant by default as bearers of the lexical content which
is normally needed for the identification of the referent. That is why quite often, when a
(non-predicative) NP consists of a sole \textsc{adj}, some ``assumed noun'' is recovered from the
context. This is probably a \emph{raison d’être} of the notion of semantic headedness in Zwicky’s
approach. 

Yet an element can receive sufficient relevance due to other factors as well. For example, the increased relevance accompanies non-restrictiveness because there should be a specific motivation for the appearance of an element which does not help to identify a referent. Restrictiveness, however, cannot be given in absolute terms either. Some expressions determine classes of objects more or less easily. For example, the word \emph{crocodile} determines the class of crocodiles, \emph{green} determines the class of green things, and \emph{insane} determines the class of what is thought to be insane in a given society. Surely, insaneness may be questioned (even in a court), a word can be used indirectly, there are color shades which are classified as green by some people and blue by others, and speakers do not always distinguish between crocodiles and alligators. Nonetheless, when one uses words like these, it is normally assumed that the speaker and the addressee determine what is meant relatively identically. Now, for most basic \textsc{adj}s in Dixon's sense (except for \textsc{color}), the situation is different because their use relies heavily on the speaker's evaluation. Since the speaker's evaluation need not be shared by the addressee, these \textsc{adj}s are the worst candidates to function as restrictive modifiers. This is not to say that they cannot be: the addressee often has to take the speaker's perspective. Yet, such \textsc{adj}s should not be that convenient when other means of restricting the reference are possible. This makes their use marked, increases their relevance and makes it more possible for them to display head effects as shown in (\ref{sec-adjprob}).\footnote{\citet{Thompson1989} studied the function of ``Property Concept Words'' in natural discourse. According to her, \textsc{adj}s do not typically restrict the meaning of a \textsc{n}, which – if present – is often either anaphoric or ``empty'' (i.e.\ describing only a very general category). Rather they are usually used either predicatively (in the absolute majority of cases) or as a means of referent introducing. While the predication function goes in line with the speaker’s evaluation, the referent-introducing function is not, at least at first glance. It is not obvious, however, that the latter function is not fulfilled by \textsc{n}s, even where they are semantically empty.}


\subsection{Appositive structures}\label{sec-appositive}

So far we assumed that \textsc{adj}s should syntactically interact with \textsc{n}s. But in some
languages \textsc{adj}s themselves constitute phrases which syntactically are not necessarily
subordinate to \textsc{n}s, cf.  \citet*[13--14]{Riessler2016}. Probably the most well-known
illustration of this is provided by ``non-configurational'' Australian (primarily, Pama-Nyungan)
languages where apparent combinations of \textsc{n}s and \textsc{adj}s actually consist of
autonomous nominal expressions which describe the same individual, cf.\ \citet{Blake1983};
\citet{Heath1984} among others. For example, \citet*[145]{Blake1983} argued for the oft-cited
sentence (\mex{1}) that ``there are in fact no noun phrases but [\ldots] where an argument is represented by
more than one word we have nominals in parallel or in apposition'':\footnote{But see \citet{LouagieVerstraete2016} and \citet{Blake2001} for arguments that nominal expressions in Australian languages are often more integrated than it is often assumed.} 


\ea
\langinfo{Kalkatungu}{Pama-Nyungan}{\citealt[145]{Blake1983}}\\
\gll \emph{cipa-yi }      \emph{t̪uku-yu}	\emph{yaun-tu}	yaɲi	       icayi\\ 
     this-\textsc{erg}	dog-\textsc{erg}	big-\textsc{erg}	white.man	bite \\
\glt `This big dog bit/bites the white man.'
\z

\noindent
The idea that some combinations of \textsc{n}s and \textsc{adj}s either manifest apposition of two
(or more) nominals or have developed from appositive structures was developed for languages of other
areas too; cf., e.g.\ \citet*[651-654]{Testelec1998} on \ili{Georgian}, as well as numerous recent studies
on the rise of configurationality in Indo-European languages \parencites{Luraghi2010, Ledgeway2012,
  Spevak2015, Reinoehl2016}, see also \citet*[19--22]{Rijkhoff2002} and
\citet{LouagieReinoehl2022} for typologically informed discussion of such patterns. \citet*[46]{Reinoehl2016} summarized the relevant diachronic scenario in the following way:

\begin{quote}
    Several authors have described how syntactically independent and coranking elements with a shared reference, for example local particles and local case forms, or demonstratives and nominals (typically in core cases), frequently co-occurred in a sentence. They would often stand adjacent to each other in accordance with Behaghel's principle that what belongs together semantically also stands together [$\dots$]. At some point, elements would co-occur in such a symmetrical group so frequently that the string is reanalysed as a single syntactic unit, that is as a phrase. \citep[46]{Reinoehl2016}
\end{quote}

\noindent
This  suggests that even in configurational structures originating from appositives, \textsc{adj}s  which formerly constituted independent nominals and naturally had head properties there\footnote{Historically, such adjectives may originate from combinations of modifiers and nominalizing pronouns, but this does not affect the reasoning presented here.} retain these properties for historical reasons. This concerns the morphosyntactic locus criterion (i.e.\ \textsc{adj}s may retain marking characterizing the whole NP) and the related capability of appearance without a companion \textsc{n} (\cite{Lander2010}).

\section{Extending the perspective}\label{sec-extending}
While I only illustrated the sources of head effects by the examples of combinations of
\textsc{adj}s with \textsc{n}s, the same factors play a role in other constructions as well. Below I
briefly consider the three sources of head effects in the context of adnominal possessive
constructions, adpositional phrases and clause level constructions. 

\subsection{Possessive constructions}\label{subsec-possessive}

\largerpage
Scope-based effects in possessive constructions relate to the fact that possessive relations involving the most prototypical (primarily, definite) possessors are used to establish the reference of the possessee (\cite{Keenan1974}), so the latter tends to be definite (\cite{Haspelmath1999}).
Since the possessive relation operates with the denotation of the whole NP, a combination of the
possessor with the marker of this relation must be compositionally higher and may display head
effects. The fact that in phrases like \emph{John's enemy} the phrase \emph{John's} is as obligatory
as the possessee may have resulted from grammaticalization of this. Another possible manifestation
of the same phenomenon is observed in indirect possessive constructions in mostly right-branching
Oceanic languages. In these constructions the reference to the possessor is accompanied by a
classifier specifying the kind of possessive relation, which arguably shows some head properties
(\cite{PalmerBrown2007}). The following examples demonstrate that the possessive classifier which
characterizes the relation as that of consumption and contains the possessor indexing appears to be
a distributional equivalent of the whole construction (in the examples in this section brackets
enclose possessive NPs):\footnote{\citet{PalmerBrown2007} suggested that the possessive classifier
  is a kind of noun. \citet{Lichtenberk2009} contended this view and argued that such classifiers
  should not be considered heads. Even then, however, we may still think of them as displaying head
  properties.} 

\ea
\langinfo{Kokota}{Austronesian}{\citealt[205]{PalmerBrown2007}}\\
\ea 
\gll n-e	     \r{ŋ​}ɑ​=di              mɑ​nei [\emph{ɣ​e-gu}	                     \emph{kaku=ro}] \\ 
     \textsc{rl}-\textsc{3.s} eat=\textsc{3pl.obj} s/he  \spacebr{}\textsc{cnsm-1sg.pr} banana=\textsc{dem} \\
\glt `He ate my bananas.'

\ex 
\gll n-e	      \r{ŋ​}ɑ​=di              mɑ​nei [\emph{ɣ​e-gu=ro}] \\ 
     \textsc{rl}-\textsc{3.s} eat=\textsc{3pl.obj} s/he  \spacebr{}\textsc{cnsm-1sg.pr=dem} \\
\glt `He ate my food.'
\z
\z

\largerpage
\noindent
Relevance comes into play in possessive constructions when highly relevant possessors determine the features of NPs. Prominent NP-internal possessors control agreement in some languages, for example, in Amazonia \parencites[cf.][]{Dixon2000, Ritchie2017} and Northern Australia (\cite{MeakinsNordlinger2017}, although the details of these constructions vary; see also the recent volume \citep{BaranyNikolaeva2019}, which includes a detailed discussion of such patterns \citep{Nikolevaetal2019}).\footnote{Note that there are other ways to describe these constructions. Thus, for (\ref{ex-russian}) one might suggest either that it represents semantic agreement or that the noun for `majority' can control either singular agreement (not illustrated here) or plural agreement. Further, in many examples, the agreement with internal possessors is apparent only: either one can postulate a covert clause-level argument which controls agreement but is coreferent to the internal possessor (as suggested for (\ref{ex-juan}) by Ritchie) or one can assume that some features of the possessor are transferred to the possessee (cf.\ \cite{Lander2011} for \ili{Tanti Dargwa}).}   
A related phenomenon is found in patterns where arguments of quantifiers are formally represented as their possessors, but still affect the behavior of the whole NP. For example, in (\ref{ex-juan}), the internal possessor (which agrees with the possessee) seemingly controls the object gender/number agreement on the verb, while in (\ref{ex-russian}), the genitive possessor arguably controls the subject number agreement on the verb:


\ea\label{ex-juan} 
\langinfo{Chimane}{unclassified; Bolivia}{\citealt[663]{Ritchie2017}}\\
\gll Juan	täj-je-bi-\emph{te} [un mu' \emph{Sergio-s}] \\ {Juan(\textsc{m})}	touch-\textsc{clf-poss.appl-3sg.m.o}  \spacebr{}{hand(\textsc{f})} the.\textsc{m}  Sergio\textsc{(m)-f} \\
\glt `Juan touched Sergio's hand.'
\z


\ea\label{ex-russian} 
\langinfo{Russian}{Indo-European, Slavic}{corpus data: \href{https://ruscorpora.ru}{ruscorpora.ru}}\\
\gll {}[bol’šinstv-o \emph{passažir-ov}] vyxodj-\emph{at}\\ 
     \spacebr{}majority-\textsc{nom.sg}	passenger-\textsc{gen.pl}	exit-\textsc{npst.3pl}	\\
\glt `Most passengers exit.'
\z

\noindent Finally, possessive constructions may develop from appositive structures, where the possessor
expression evidently has properties of the head of a nominal itself. The Oceanic indirect possessive
construction presumably developed from the apposition of a possessive classifier and a possessee, so
the head properties of the possessive classifiers may be due to this and not only their semantic
function. 

In many languages, appositive structures arguably serve as a source of the phenomenon of Suffixaufnahme, where the possessor displays head properties by taking the ``external'' case (becoming the locus of case marking of the whole NP) in addition to genitive \citep{Plank1995}; cf.\ (\ref{ex-awngi}), where genitive markers arguably originated from pronouns bound by possessa, i.e.\ the construction could be interpreted as ``in the ones of the one of the woman, in the ones of the nice (one), in the ones of the house, in the doorways'' (\cite{Aristar1995}). Moreover, there are languages like \ili{Bilin}, where the possessor can remain the only host for the external case marking and does not share this head property with the possessee (\ref{ex-bilin}). 


\ea\label{ex-awngi}
\langinfo{Awngi}{Afro-Asiatic, Cushitic}{\citealt[435]{Aristar1995}, after \cite[37]{Hetzron1976}}\\
\gll ​ɣ​unɑ​-w-s-kʷ-dɑ​ ceŋ​kut-əkʷ-dɑ​ ŋ​ən-əkʷ-dɑ​ ɑ​bjel-kɑ​-dɑ​\\ 
     woman-\textsc{gen.m-dat-gen.pl-loc}   nice-\textsc{gen.pl-loc} house-\textsc{gen.pl-loc} doorway-\textsc{pl-loc}\\
\glt `in the doorways of the woman's nice house'
\z

\ea\label{ex-bilin} 
\langinfo{Bilin}{Afro-Asiatic, Cushitic}{\citealt[435]{Aristar1995}, after \citealt[37]{Hetzron1976}}
\gll ti'idɑ​d ɑ​d\"ɑ​ri-​ɣ​ʷ-əd\\ 
     order   lord-\textsc{gen.m-dat}\\
\glt `by the order of the lord'
\z

\largerpage[2]
\noindent
According to \citet{Aristar1995}, the pattern (\ref{ex-bilin}) continues an appositive structure like `(by) the order, by the lord's one'. For us, this construction is interesting because it shows that the appositive origin of head properties does not imply their symmetric distribution on several elements of a construction.

\subsection{Adpositional constructions}\label{subsec-adpositional}

In adpositional constructions, scope-based head effects are trivial and widely assumed. Adpositions typically provide the semantic and syntactic information relating nominals to their context and as such they have scope over these nominals. This explains why adpositions can show such head effects as being obligatory, governing the form of nominals in dependent-marking constructions and marking their function in head-marking constructions.

Less discussed is the fact that the ``adpositional object'' displays head properties in adpositional
structures, presumably because of its informational relevance. At least in non-head-marking patterns
it is usually at least as obligatory as the adposition itself. Sometimes we even find that an
adposition can be omitted, so that its ``object'' turns out to be a distributional equivalent of the
whole phrase: a well-known example is the optionality of \emph{to} in British English\il{English!British} \emph{I gave it (to)
him}. Another non-canonical situation is presented when an adposition specifies the relation
provided by the case, so the apparent object of a adposition serves as a locus of marking of some
external relation. 

\largerpage[2]
\ea\label{ex-deutsch} 
\langinfo{German}{Indo-European, Germanic}{\citealt[208]{Donaldson2007}}\\
\ea
\gll Ich habe die	Zeitung \emph{auf} \emph{den} Tisch gelegt.\\ 
     I.\textsc{nom}	 have.\textsc{1sg} the.\textsc{acc}	newspaper  on the.\textsc{acc} table put.\textsc{ptcp}\\
\glt `I put the newspaper on the table.'

\ex
\gll Die Zeitung lieg-t \emph{auf} \emph{dem} Tisch.\\ the.\textsc{nom}  newspaper        lie-\textsc{3sg} on the.\textsc{dat} table\\
\glt `The newspaper is lying on the table.'
\z\z

\ea\label{ex-newspaper}  
\langinfo{Russian}{Indo-European, Slavic}{personal knowledge}\\ 
\ea
\gll ja	položi-l   gazet-u	 \emph{na} \emph{stol}\\ 
     I.\textsc{nom}	put-\textsc{pst} newspaper-\textsc{acc.sg} on table(\textsc{acc.sg}) \\
\glt `I put the newspaper on the table.'

\ex
\gll gazet-a lež-it  \emph{na} \emph{stol-e}\\ newspaper-\textsc{nom.sg} lie-\textsc{npst.3sg} on table-\textsc{loc.sg}\\
\glt `The newspaper is lying on the table.'
\z\z

\noindent
Curiously, it is usually assumed that the direction (`to', `from') and essive (`in') meanings have scope
over the search domain (`on', `front', `under', `behind', etc.); cf.\ \emph{from} [\emph{under} [\emph{the bridge}]], see
e.g.\ \citet{CR2010a}. If we follow this assumption, in (\ref{ex-deutsch})--(\ref{ex-newspaper}) case
marking should be higher than the prepositional marking in the semantic structure. This looks
confusing under the traditional account which assigns the head status to adpositions and assumes
that the case appears deeper in the syntactic structure than the adposition. The assumption that
both an adposition and its associate NP are allowed to have head properties, presumably, opens the
door to more sophisticated modes of the semantic composition of such constructions.

Head effects originating from appositive structures are observed when an adposition develops from an
adverb while its associate NP bears a case with the same function as an adverb, as in
(\ref{ex-bagwalal}). Presumably, in such examples both the adposition and the case-marked NP are
distributional equivalents of the phrase, allowing omission of the other element.  

\ea\label{ex-bagwalal} 
\langinfo{Bagwalal}{East Caucasian}{\citealt[169]{Sosenskaja2001}}\\
\gll \emph{hinc'-ib-a-la}                  \emph{č’ihi}  r-isa-n partal	r-uk’a	qanč-ibi  \\ 
     stone-\textsc{pl-obl.pl-super} on \textsc{n.pl}-find-\textsc{ptcp.n.pl}	things \textsc{n.pl}-be	cross-\textsc{pl} \\
\glt `crosses were found on stones' (lit.: on stones, things that were found are crosses)
\z

\subsection{Clause level}\label{subsec-clause}

The issue of headedness in the clause is very complex, partly because clauses themselves may be very
different in what candidates for having head properties they contain. Yet several observations can
be made. For example, scope-based head effects can be found for auxiliaries and similar functional
elements (cf.\ \cite{Zwicky85a}), which presumably have scope over the predicate.\footnote{In fact,
  there may be other candidates to the highest elements in the semantic structure. For example, some
  adverbials (e.g., modal adverbials) have scope over the whole clause, but the expressions
  involving such adverbs regularly allow complex paraphrases with the relevant meaning expressed in
  a matrix clause (e.g., \emph{It is possible that...}) and the very event described in a
  subordinate clause. Another candidate is the topic (or the subject, when it has grammaticalized
  from the topic), and here the situation is probably similar to the special properties of the
  possessor described above. Here I refrain from the discussion of these issues.} The predicate,
which is usually described as the head in the absence of auxiliaries, is normally the most relevant
element of the clause, which further seems to have wide semantic scope over much of the clause. Most
informationally loaded elements different from the predicate occur as well and they can have head
properties, as seen in languages where the focused element (which presumably has the highest
relevance value) takes clause-level morphosyntactic marking. For example, in \ili{Udi}, the focused
element takes a marker of agreement with the subject even when it is the subject itself;
cf.\ (\ref{ex-lezgi}) with the focused subject and (\ref{ex-guests}) with the predicate
focus:\footnote{The \ili{Udi} agreement markers are often described as clitics (\cite{Harris2002}), but
  the main reason for this is the fact that they can be hosted by different constituents. This
  vision comes from a very strong association of heads with particular lexical categories, which is
  unnecessary if we think of head effects rather than of heads.} 

\newpage
\ea
\langinfo{Udi}{East Caucasian}{corpus data}\\
\ea\label{ex-lezgi}
\gll sa	läzgi-n	kːoj-a  \emph{qːonaʁ-χo-tːun} eʁ-o.  \\ 
     one Lezgi-\textsc{gen}  house-\textsc{dat} guest-\textsc{pl-3pl}	come-\textsc{pot} \\
\glt `Some \textsc{guests} are coming to a Lezgi.'

\ex\label{ex-guests}
\gll qːonaʁ-χo	\emph{har-i-tːun}\\ guest-\textsc{pl} come-\textsc{aor-3pl} \\
\glt `The guests came!'
\z\z

\noindent
Traces of appositive-like structures at the clause level can be observed in serial constructions
lacking formal restrictions on their components, called symmetrical serial verb constructions
\citep[3]{Aikhenvald2003}.\footnote{Asymmetrical serial verb constructions, which put restrictions
  on one of the components, are associated with head effects of different origin.}  A construction
of this type is illustrated in (\ref{ex-abui}): 

\ea\label{ex-abui} 
\langinfo{Abui}{Trans-New Guinea, West}{\citealt[354]{Kratochvil2007}}\\
\gll ko pi \emph{yaa} \emph{mit} \emph{nate-a} \emph{tanga} \emph{ananra} naha  \\ 
     soon    we.\textsc{incl}	  go	sit	stand.up-\textsc{dur}  speak.\textsc{cnt} tell.\textsc{cnt}	\textsc{neg} \\
\glt `we will not negotiate'
\z
Here, the whole conventionalized sequence of verbs
refers to negotiation, so the negation has scope over all of these verbs. 


\citet*[22]{Aikhenvald2006} states that ``[s]ymmetrical serial constructions are not `headed' in the way asymmetrical ones are: all their components have equal status in that none of them determines the semantic or syntactic properties of the construction as a whole'', but this claim involves a presupposition that head properties deny equality. If we abandon this presupposition, we can instead suggest that in symmetrical serial constructions several predicates may have head properties and this is due to the fact that these constructions originate from appositive-like constructions.

\section{Conclusion}\label{sec-conclu-lan}

To sum up, I propose that head properties arise (at least) due to one of the three factors: (i) the
higher position of an element in a compositional structure, (ii) the informational prominence, and
(iii) the development of a construction from an appositive(-like) structure. These factors are
logically independent and may lead to the assignment of head properties to different elements of a
construction. As a result, it is more accurate to speak not of the heads but rather of head effects,
which may – but need not – concentrate around a single component of a construction. 

It is worth noting that the list of factors contributing to head effects should not be thought of as
including both synchronic and diachronic properties. In fact, all these factors can be interpreted
as diachronic. Grammaticalization of a construction may lead to the development of a hierarchical
structure out from a flat non-configurational sequence of words and groups of words. Such
development relies on the informational prominence and/or on the most typical compositional
combinations, but this development may apply to syntactic units that are already grammaticalized and
display morphosyntactic asymmetries. Hence, in diachrony, we suspect to find a kind of competition
or interaction between various factors affecting the shape of a construction that we observe. Such
processes, the ways that languages meet such conflicts and escape from them, seem to be a fruitful
subject for further studies.

Finally, this paper did not attempt to answer the question of why the concrete head effects appear where they appear. Hawkins (\citeyear{Hawkins1993}, \citeyear[343--358]{Hawkins1995}) tried to explain the head properties by the role that head-like elements play in efficient processing of utterances. If his work is on the right track, it makes sense to look at the limits of cross-linguistic and cross-constructional variation of structures with respect to this role. 

\section*{Abbreviations}

\begin{tabularx}{.99\textwidth}{@{}lX}
\textsc{aor}	& aorist\\
\textsc{cnsm}	& `consumed' (possessive classifier)\\
\textsc{cnt}	& continuative\\
\textsc{dyn}	& dynamic\\
\textsc{instr}	& instrumental preverb\\
\textsc{mod}	& modal (tense)\\
\textsc{npst}	& non-past\\
\textsc{pot}	& potential\\
\textsc{pr}	& possessor\\
\textsc{rl}	& realis\\
\textsc{super}	& `on the surface'\\
\end{tabularx}


\section*{\acknowledgmentsUS}

This paper originates from my talks at the conferences ``Moscow Syntax and Semantics 2009'' (Moscow, 2009) and ``Köpfigkeit und/oder grammatische Anarchie?'' (Berlin, 2017). I am grateful to the audience of these conferences for discussion. I also thank Johanna Nichols, Paul Phelan, the editors and anonymous reviewers of the volume for their comments on earlier drafts of the paper. Support from the Basic Research Program of HSE University is gratefully acknowledged. All errors are mine.

{\sloppy
\printbibliography[heading=subbibliography,notkeyword=this]
}

\end{document}


% en
%      <!-- Local IspellDict: en_US-w_accents -->
