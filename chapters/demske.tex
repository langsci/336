\documentclass[output=paper,colorlinks,citecolor=brown]{langscibook} 

\IfFileExists{../localcommands.tex}{%hack to check whether this is being compiled as part of a collection or standalone
%\bibliography{localbibliography} 
\usepackage{orcidlink}

% add all extra packages you need to load to this file


% Haitao Liu
%\usepackage{xeCJK}
%\setCJKmainfont{SimSun}
%\setCJKmainfont[Scale=MatchUppercase,
%                Path=fonts/
%]{SourceHanSerifSC-Regular}

% instead use option:  ,chinesefont % for references in raffelsiefen.tex
% loading the package changes some spacings


\usepackage{multicol}
\usepackage{tikz}\usetikzlibrary{decorations.pathreplacing}
\usepackage{url}
\urlstyle{same}

%\usepackage{listings}
%\lstset{basicstyle=\ttfamily,tabsize=2,breaklines=true}

\usepackage{langsci-basic}
\usepackage{langsci-optional}
\usepackage[danger]{langsci-lgr}

% toggle danger in texlive 2021
%\newcommand{\M}{\textsc{m}\xspace}

% toggle danger in texlive 2021 or uncomment this
% \newcommand{\N}{\textsc{n}\xspace}
% \newcommand{\F}{\textsc{f}\xspace}


\usepackage{./styles/biblatex-series-number-checks}




\usepackage{langsci-gb4e}




% Demske

\usepackage{tipa}
\usepackage{styles/avm+}
%\usepackage{styles/merkmalstruktur}
\avmfont{\sc}
\usepackage{langsci-forest-setup}
\usepackage{xspace}
%\usepackage{styles/abbrev} 

\usepackage{soul}
\usepackage{color}
\newcommand{\rem}[1]{\textcolor{red}{\st{#1}}}
\newcommand{\add}[1]{\textcolor{blue}{\ul{#1}}}


% Salzmann

\usepackage[nocenter]{qtree}


% Müller

% add this to the default preamble 
\forestset{default preamble={
    for tree={anchor=north},
}}


\usepackage{german}

%\usepackage{german}
\selectlanguage{USenglish}

% Mit Babel geht irgendwie die hyphenation nicht richtig
%\usepackage[ngerman,english]{babel}
%\useshorthands{"} 
%\addto\extrasenglish{\languageshorthands{ngerman}}

\usepackage{styles/makros.2020,
styles/abbrev,
styles/merkmalstruktur,
styles/article-ex,styles/eng-date}


\usepackage{todonotes}
\newcommand{\todostefan}[1]{\todo[color=green!40]{\footnotesize #1}\xspace}
\newcommand{\inlinetodo}[1]{\todo[color=green!40,inline]{\footnotesize #1}\xspace}

\newcommand{\inlinetodoopt}[1]{\todo[color=green!40,inline]{\footnotesize #1}\xspace}
\newcommand{\inlinetodoobl}[1]{\todo[color=red!40,inline]{\footnotesize #1}\xspace}

\newcommand{\itdobl}[1]{\inlinetodoobl{#1}}
\newcommand{\itdopt}[1]{\inlinetodoopt{#1}}

\newcommand{\addpages}{\todostefan{add pages}}

%\newcommand{\iaddpages}{\yel[add pages]{pages}\xspace}



% subfigure
\usepackage{subcaption}



% Nolda
%\usepackage[main=british,nil,german,french]{babel}
\newcommand{\foreignlanguagedummy}[2]{#2}
\usepackage{tagpair}
\usepackage{hang}
\usepackage[noconfig]{ntheorem}
\usepackage{pstricks,pst-node,pst-tree}
\usepackage{newunicodechar}



%% hyphenation points for line breaks
%% Normally, automatic hyphenation in LaTeX is very good
%% If a word is mis-hyphenated, add it to this file
%%
%% add information to TeX file before \begin{document} with:
%% %% hyphenation points for line breaks
%% Normally, automatic hyphenation in LaTeX is very good
%% If a word is mis-hyphenated, add it to this file
%%
%% add information to TeX file before \begin{document} with:
%% %% hyphenation points for line breaks
%% Normally, automatic hyphenation in LaTeX is very good
%% If a word is mis-hyphenated, add it to this file
%%
%% add information to TeX file before \begin{document} with:
%% \include{localhyphenation}
\hyphenation{
Arsch
anaph-o-ra
Bü-cking
con-stit-u-ents
Dor-drecht
For-schungs-ge-mein-schaft
Ge-schich-te
ha-ben
pho-nol-o-gy
pro-so-dic
pro-so-di-cally
Sal-pe-ter
sei-nen
Wil-liams
}
\hyphenation{
Arsch
anaph-o-ra
Bü-cking
con-stit-u-ents
Dor-drecht
For-schungs-ge-mein-schaft
Ge-schich-te
ha-ben
pho-nol-o-gy
pro-so-dic
pro-so-di-cally
Sal-pe-ter
sei-nen
Wil-liams
}
\hyphenation{
Arsch
anaph-o-ra
Bü-cking
con-stit-u-ents
Dor-drecht
For-schungs-ge-mein-schaft
Ge-schich-te
ha-ben
pho-nol-o-gy
pro-so-dic
pro-so-di-cally
Sal-pe-ter
sei-nen
Wil-liams
}
\newcommand*{\orcid}[1]{}

% do not show the chapter number. It is redundant, since most references to figures are within the
% same chapter.
\renewcommand{\thefigure}{\arabic{figure}}

\newcommand{\rlapsub}[1]{\rlap{\sub{#1}}}

% \SetupAffiliations{output in groups = false, 
%                    separator between two = {\bigskip\\},
%                    separator between multiple = {\bigskip\\},
%                    separator between final two = {\bigskip\\}
%                    }


%%%%%%%%Alte Umlaute
\newcommand{\oldae}{$\stackrel{\textrm{\tiny e}}{\textrm{a}}$}
\newcommand{\oldoe}{$\stackrel{\textrm{\tiny e}}{\textrm{o}}$}
\newcommand{\oldue}{$\stackrel{\textrm{\tiny e}}{\textrm{u}}$}

\newcommand{\refl}{\REFL}
\newcommand{\pst}{\PST}


% Müller
\let\vref\ref


\let\citew\citet

\newcommand{\page}{}

% biblatex stuff
% get rid of initials for Carl J. Pollard and Carl Pollard in the main text:
\ExecuteBibliographyOptions{uniquename=false}




\newcommand{\nom}{\textsc{nom}}
\newcommand{\gen}{\textsc{gen}}
\newcommand{\dat}{\textsc{dat}}
\newcommand{\acc}{\textsc{acc}}


%\newcommand{\spacebr}{\hspaceThis{[}}



\newcommand{\acknowledgmentsEN}{Acknowledgements}
\newcommand{\acknowledgmentsUS}{Acknowledgments}


% no bf!!!111!
\let\textbfemph\emph

\newcommand{\textbfremoved}[1]{#1}
%\newcommand{\emphremoved}[1]{#1}


\newcommand{\noemph}[1]{#1}
\newcommand{\underlineemph}[1]{\emph{#1}}




% for editing, remove later
\usepackage{xcolor}
\newcommand{\added}[1]{{\red #1}}
\newcommand{\addedthis}{\todostefan{added this}}

\newcommand{\changed}[1]{\textcolor{orange}{#1}}




% Nolda

\theorembodyfont{\normalfont}
\let\restriction\relax
\renewtheoremstyle{break}{\item{\itshape ##1\ ##2}\newline\nopagebreak}{\item{\itshape ##1\ ##2\ (##3)}\newline\nopagebreak}
\theoremstyle{break}
\newtheorem{definition}{Definition}
\newtheorem{pattern}{Pattern}
\newtheorem{restriction}{Restriction}
\newunicodechar{‑}{\hbox{-}}
\newunicodechar{…}{\dots}
\newunicodechar{⁡}{\relax}
\newunicodechar{⁣}{\relax}
\newunicodechar{⁀}{\raisebox{+1ex}{\ensuremath\frown}}
\newunicodechar{⁐}{\raisebox{+1ex}{\ensuremath\frown}\setbox1=\hbox{\ensuremath\smile}\hspace{-\wd1}\raisebox{-1ex}{\ensuremath\smile}}
\newunicodechar{⪪}{\ensuremath{<\mathrel{\llap{\ensuremath{-}}}}}
\setkomafont{descriptionlabel}{\normalfont}
\ExecuteBibliographyOptions{labeldate=comp,labelnumber=true,defernumbers=true}
\defbibenvironment{sources}{\list{\printfield{labelprefix}\,\printfield{labelnumber}}{\settowidth{\labelwidth}{S\,0}\setlength{\labelsep}{\biblabelsep}\setlength{\leftmargin}{\labelwidth}\addtolength{\leftmargin}{\labelsep}\setlength{\itemsep}{\bibitemsep}\setlength{\parsep}{\bibparsep}}\renewcommand{\makelabel}[1]{##1\hfil}}{\endlist}{\item}
\newcommand{\citesource}[1]{\citefield{#1}{labelprefix}\,\citefield{#1}{labelnumber}}


% Was soll das machen?
\newcommand{\textstyleFootnoteSymbol}{}



% Was ist das???? St. Mü. 30.10.2021
%Kann weg. Damit waren die bücker transkripte aligniert. Habe das jetzt mit tabularx und hphantom gemacht

%\newlength{\calength} %tmp length to store the space 1. until [; 2. until ].
%
%%first argument speaker ID, second argument text. Optional argument left margin indicator (arrow or similar)
%\newcommand{\cabox}[3][]{\parbox{0mm}{\hspace*{-1cm}#1}%
%\parbox{1.5cm}{#2}%
%\parbox{9.6cm}{#3}\\%
%}
%
%%translation. First parbox is empty, second parbox takes the translation text
%\newcommand{\trsbox}[1]{\parbox{1.5cm}{~}%
%\parbox{9.6cm}{\itshape #1}\\%
%}
%
%%store the width of a string.
%\newcommand{\settablength}[1]{\settowidth{\calength}{#1}\global\calength=\calength}
%
%%print string and store its width. Useful if the first item of the aligned set is also the longest
%\newcommand{\inittab}[1]{#1\settablength{#1}}
%
%%insert horizontal white space equivalent to the stored width
%\newcommand{\skiptab}{\parbox{\calength}{~}}
%
%%print the argument and fill up with horizontal white space until the stored width is reached.
%\newcommand{\filledtab}[1]{\parbox{\calength}{#1}}



% for standalone compilations Felix: This is in the class already
%\let\thetitle\@title
%\let\theauthor\@author 
\makeatletter
\newcommand{\togglepaper}[1][0]{ 
\bibliography{../bib-abbr,../stmue,../localbibliography,
collection.bib}
  %% hyphenation points for line breaks
%% Normally, automatic hyphenation in LaTeX is very good
%% If a word is mis-hyphenated, add it to this file
%%
%% add information to TeX file before \begin{document} with:
%% %% hyphenation points for line breaks
%% Normally, automatic hyphenation in LaTeX is very good
%% If a word is mis-hyphenated, add it to this file
%%
%% add information to TeX file before \begin{document} with:
%% \include{localhyphenation}
\hyphenation{
Arsch
anaph-o-ra
Bü-cking
con-stit-u-ents
Dor-drecht
For-schungs-ge-mein-schaft
Ge-schich-te
ha-ben
pho-nol-o-gy
pro-so-dic
pro-so-di-cally
Sal-pe-ter
sei-nen
Wil-liams
}
\hyphenation{
Arsch
anaph-o-ra
Bü-cking
con-stit-u-ents
Dor-drecht
For-schungs-ge-mein-schaft
Ge-schich-te
ha-ben
pho-nol-o-gy
pro-so-dic
pro-so-di-cally
Sal-pe-ter
sei-nen
Wil-liams
}
  % \memoizeset{
  %   memo filename prefix={hpsg-handbook.memo.dir/},
  %   % readonly
  % }
  \papernote{\scriptsize\normalfont
    \@author.
    \titleTemp. 
    To appear in: 
    Ulrike Freywald \& Horst Simon (eds.) Headedness and/or grammatical anarchy?
    Berlin: Language Science Press. [preliminary page numbering]
  }
  \pagenumbering{roman}
  \setcounter{chapter}{#1}
  \addtocounter{chapter}{-1}
}
\makeatother



% This does a linebreak for \gll for long sentences leaving space for the language at the right
% margin. The factor .0989 is needed since otherwise starred examples cause a linebreak.
% St.Mü. 17.06.2021 08.02.2021
\newcommand{\longexampleandlanguage}[2]{%
%\begin{tabularx}{.99\linewidth}[t]{@{}X@{}p{\widthof{(#2)}}@{}}%
%\begin{minipage}[t]{.99\linewidth}%
\begin{tabularx}{\linewidth}[t]{@{}X@{}p{\widthof{(#2)}}@{}}%
\begin{minipage}[t]{\linewidth}%
#1%
\end{minipage} & (\ili{#2})%
\end{tabularx}}

% ORCIDs in langsci-affiliations 
\usepackage{orcidlink}
\definecolor{orcidlogocol}{cmyk}{0,0,0,1}
\ProvideDocumentCommand{\LinkToORCIDinAffiliations}{ +m }
  {%
    \orcidlink{#1}
  }
 
}{}


%%%%%%%%%%%%%%%%%%%%%%%%%%%%%%%%%%%%%%%%%
\title{Silent heads in Early New High German} 
\author{Ulrike Demske\orcid{0000-0002-8427-648X}\affiliation{Universität Potsdam}}
\ChapterDOI{10.5281/zenodo.7142708}

\abstract{The rising standard language in Early New High German (1350--1650) provides particularly interesting cases for the question of missing heads on all levels of language structure. A well-known example are subordinate clauses lacking a finite auxiliary verb, traditionally called Afinite Constructions. Based on new data, drawn from two treebanks of Early New High German, the present paper will briefly sketch the distribution of ACs, before establishing that they are in fact a type of ellipsis and do not cluster with other non-finite clauses in German. The remainder of the paper addresses the question what kind of information is missing in ACs and how this information is retrieved. Obviously, auxiliary drop in \textsc{enhg}\il{German!Early New High} represents a type of ellipsis rarely attested in present-day German.}

%%%%%%%%%%%%%%%%%%%%%%%%%%%%%%%%%%%%%%%%%%%%
\begin{document}
\maketitle


%%%%%%%%%%%%%%%%%%%%%%%%%%%%%%%%%%%%%
\section{Introduction}
The language of the Early New High \ili{German} period (= \textsc{enhg}\il{German!Early New High}) is set apart from the language in other periods of \ili{German} by its overwhelming array of contexts where graphemes, bound morphemes and words may be left out. Two well-known examples are provided below: in (\ref{print}), the inflectional suffix \textit{-e}, indicating first person singular, is dropped. And the unbound morpheme \textit{landt} is omitted in the first conjunct of the N N-coordination in (\ref{bound_morph}).
\eal 
\ex \label{print}
\gll Jch \textbfemph{clopffet} an man ließ mich ein  \\ I knocked.\textsc{1sg.pst} at one let me in  \\
\glt `I knocked at the door and was let in.'
\hfill (1490: KalF s1052)\footnote{For each source, the publication date and a siglum of the respective text in the treebank is provided along with the number of the clause.}
\ex \label{bound_morph}
\gll Weilen der Anstandt beschlossen/ kömpt teglich viel Volcks aus \textbfemph{Holl:} vnd Seelandt allhero \\ since the armistice declared comes daily much people from Holl: and Zeeland here  \\ \hfill (1609: Aviso 151)
\glt `Since the armistice is declared, a lot of people from Holland and Zeeland arrive here each day.'
\zl
A prominent example for dropping a segment of syntactic structure is provided by so-called Afinite Constructions (= ACs) widely attested in the \ili{German} language of the 16th and 17th century. The finite auxiliary is absent in the relative clause in (\ref{rc_intro}) and the adjunct clause in (\ref{ac_intro}). The probable position of the omitted auxiliary is indicated by an underscore in clause-final position. This assumption of the gap in clause-final position proves reasonable in view of the fact that ACs are restricted to syntactically dependent clauses exhibiting either a subordinator or a relative pronoun at the left periphery. The type of ellipsis illustrated below is hence different from coordinate ellipsis which will  be addressed in the present paper only in passing.
\eal \label{afinite_patterns}
\ex \label{rc_intro}
\gll Diesen Monat seynd auß Holland vnd Seeland/ vber 80. Schiff/ so nach Spania gewählt \_\_/ abgeseglet/  \\ this month are from Holland and Zeeland over 80 ships which to Spain wanted {} sailed \\ \hfill (1597: AC s118)
\glt `This month 80 ships have left Holland and Zeeland heading for Spain.'
\ex \label{ac_intro}
\gll Ob schon der Friede mit Engeland vor richtig gehalten  \_\_/ so wird dennoch dieses Reiches Krieges-Floote wenig vermindert \\ even if the peace with England for right considered {}  so is however this country's war-fleet hardly diminished  \\ 
\glt `Even if the peace with England is considered to be good, the war fleet of this country is hardly diminished.' \hfill (1667: M s4619)
\zl
This type of auxiliary drop is still attested in present-day \ili{German}, as witnessed by an example found in \textit{Buddenbrooks} from the beginning of the 20th century, an example taken from \citet{schroeder85}. Its use in present-day \ili{German}, however, is very restricted.
\ea \label{buddenbrooks}
\gll Begreift man das stille Entzücken, mit dem die Kunde, als das erste, leise, ahnende Wort gefallen \_\_, von der Breiten in die Mengstraße getragen worden \_\_ \\ comprehends one the quiet delight, with which the news, when the first, soft, guessing word dropped {}, from the Breite to the Mengstraße carried been    \\ \hfill (1901: Buddenbrooks 408)
\glt `If one comprehends the quiet delight characterizing the news that has been carried from Breite to Mengstraße when the first, soft, guessing word has been dropped \dots'
\z
Regarding the data in (\ref{afinite_patterns}) and (\ref{buddenbrooks}), the question arises whether the respective subordinate clauses in fact include a gap, i.e. are headless clauses, or rather pattern with infinitival and participial clauses. Assuming that we are dealing with ellipsis here, how do we retrieve the missing information? With respect to present-day \ili{German}, two ways are distinguished to fill an obvious gap: either by referring to a suitable antecedent or to the current situation and world knowledge \citep{reich2011}.

The outline of the paper is as follows: before tackling the aforementioned questions in Section~\ref{analysis}, the basic facts about ACs in \textsc{enhg}\il{German!Early New High} are presented in Section~\ref{data}. The remainder of the paper addresses the question why this particular type of auxiliary ellipsis is productive for only 200 years in the history of \ili{German}. The data used in the present paper mainly come from two constituent-based treebanks for \textsc{enhg}\il{German!Early New High} comprising a total of 770.000 tokens with 2.789 attestations of ACs from the whole language period. 

%%%%%%%%%%%%%%%%%%%%%%%%%%%%%%%%%%%%%%%%%%%%%%%%%%%%%%%%%%%%%%%%%%%%%%%%%%%%%%%%%%%%%%%
%%%%%%%%%%%%%%%%%%%%%%%%%%%%%%%%%%%%%%%%%%%%%%%%%%%%%%%%%%%%%%%%%%%%%%%%%%%%%%%%%%%%%%%
\section{Afinite Constructions in Early New High German \label{data}}

\subsection{Background \label{background}}
ACs are a phenomenon characteristic of \textsc{enhg}\il{German!Early New High} syntax. While they are only rarely attested for the 14th century, and their occurrences remain sparse even throughout the 15th century, the 16th and the 17th centuries represent the heyday of the construction. In the 18th century, attestations of ACs are scattered  again \citep{admoni80,haerd81,breitbarth2005,ERSW93}.\footnote{\citet[171, 172]{haerd81} provides the time course of ACs in three-place verb complexes of the type ``\textit{sein} `be'+ \textit{worden} `been' + past participle'' or ``\textit{haben} `have' + past participle + modal verb''.} As regards register as a parameter driving the distribution of ACs, \citet{admoni67}, \citet{biener25} and \citet{stammler54} observe the first instances in formal registers, namely chancery texts of the 14th century, indicating that ACs are characteristic of rather formalized varieties of language. Recent corpus studies however suggest that there are in fact considerable differences not only with respect to register but also with respect to individual authors \citep{breitbarth2005,thomas2018}. Future corpus studies covering even more variation across registers will contribute to this still open issue.  

\largerpage
The frequent use of ACs in \textsc{enhg}\il{German!Early New High} gave rise to a significant number of accounts for this type of auxiliary ellipsis, starting with the assumption that ACs are due to language contact with \ili{Latin}, in particular with its participial constructions \citep{biener25}. Their early attestations in chancery language were taken to support such a view. But \citet[491]{behaghel28} and \citet{breitbarth2005} point out that the patterns in \ili{German} and \ili{Latin} are too different in crucial respects to suggest the influence of language contact as a plausible explanation. As regards language contact, \citet{blum2018} also considers the possible impact of Low \ili{German} and \ili{Swedish} on \textsc{enhg}\il{German!Early New High}, because both languages likewise include some version of afinite constructions. He argues that in the first case, we are probably dealing with an independent development, while the impact is the other way round in the latter: the omission of the auxiliary verb \textit{hava} ‘have' in \ili{Swedish} is due to language contact with (Low) \ili{German} because of its chronology. A more promising account for the rise of ACs is based on the syncretism of finite and non-finite past tense forms, emerging for instance in cases where \textit{ge-} is prefixed to both forms, a proposal put forward by \citet{biener25}. As illustrated by example (\ref{geform2}), the prefix \textit{ge-} may be attached to a finite verb in past tense in order to focus the result state of a complex event. Note that the strongly inflected verb \textit{sprechen} `speak' has distinct stem forms for past tense and present perfect in \ili{German}. Weakly inflected verbs such as \textit{machen} `make' on the other hand give rise to an ambiguity between the non-finite participle and the finite preterite form, when the person and number marker \textit{-e} is absent (\ref{geform}), cf.\ \citet[242]{ERSW93} for more information on \textit{e-}apocope in \textsc{enhg}\il{German!Early New High}. The dental suffix as well as the prefix are interpreted as past tense marker. 
\eal
\ex \label{geform2}
\gll So bald sy die wort ge-sprach, f\oldue{}r sy durch die lüft \\ as soon she the words \textsc{pfv}-spoke, went she through the air \\ 
\glt `As soon as she had spoken these words, she flew away' \\ \hfill (1509: Fortunatus 150.6)
\ex \label{geform}
\gll wann sie got auz dem selben ertreich ge-machet als uns \\ because them God from the same soil \textsc{pfv}-made like us \\ 
\glt `Because God has created them from the same soil like us.' \\   \hfill  \citep[386]{ERSW93}
\zl
As stated by \citet[386]{ERSW93}, the use of \textit{ge-} with finite verbs is most frequent in the 14th century when only a few examples of ACs are attested. Therefore there are some doubts in the literature that the chronology does fit the syncretism scenario \citep{ERSW93,breitbarth2005}. We will come back to the syncretism of finite and non-finite verbs in Section~\ref{enhg_context}. 

The particular historical development of ACs is certainly at odds with another proposal interpreting the rise of ACs in subordinate clauses as an analogy to main clauses with the right sentence bracket in both clause types restricted to non-finite verbs. According to \citet{bock75}, the V-second pattern might have affected V-final clauses in such a way that finite verbs were omitted from the right sentence bracket. \citet{breitbarth2005}, however, can show in her study that the right sentence bracket is attested earlier in dependent clauses than its counterpart in V-second clauses. 

Further accounts include possible influences from auxiliary ellipsis in other contexts \citep{behaghel28,bock75,schroeder85} as well as the use of ACs as a marker for syntactic dependency \citep{admoni67,breitbarth2005,demske90}. We will come back to a possible trigger for the rise of ACs in Section~\ref{enhg_context}, when the particular history is embedded in the \textsc{enhg}\il{German!Early New High} context. 

%%%%%%%%%%%%%%%%%%%%%%%%%%%%%%%%%%%%%%%%%%%%%%%%%%%%%%%%%%%%%%%
\subsection{Distribution of Afinite Constructions \label{distribution}}
This section will present the properties of ACs in more detail, based on data from two treebanks of \textsc{enhg}\il{German!Early New High}. Table (\ref{table1}) shows absolute frequencies for the period from 1400 until 1700 as well as relative frequencies referring to the number of all dependent clauses in texts of the respective century.\footnote{The number of dependent clauses in each century comprises clauses with a subordinator or relative pronoun. Infinitival clauses are not considered. cf.\ discussion in Section~\ref{arg_ellipsis}.} Regarding the size of the underlying corpus for each subperiod, the treebank data confirm earlier observations that the frequency of ACs increases rapidly from the 15th to the 16th century.\footnote{Many thanks go to Iskra Fodor who drew the numbers from both treebanks.}  
\begin{table} 
  \centering
  \begin{tabular}{lrrr}
    \lsptoprule
   \textsc{enhg}\il{German!Early New High} & Tokens & \multicolumn{2}{c}{Afinite Constructions} \\
   & & absolute & relative \\
    \midrule
    1401--1500 & 201,870 & 65/5,917 & 1,1\% \\
    1501--1600 & 241,238 & 1,648/10,824 & 15,2\% \\
    1601--1700 & 229,184 & 1,076/15,953 & 6,7\% \\
     \midrule
    Total & 672,292 & 2,789/32,694 & 8,5\% \\
      \lspbottomrule
    \end{tabular}
   \caption{Afinite constructions from the 15th to the 17th century \label{table1}}
\end{table}

\largerpage
The gap in ACs concerns the finite auxiliary in syntactically dependent clauses. All auxiliary verbs can be omitted, \textit{haben} `have' and \textit{sein} `be' outnumber the auxiliary verb \textit{werden} `get' and the modal verbs by far, as already noted by \textcite{breitbarth2005} and \textcite{blum2018}. The following examples provide different contexts, each of them suitable for one of the auxiliary verbs, disregarding possible changes affecting the alternation between \textit{haben} and \textit{sein} in combination with the past participle \citep[387]{ERSW93,sapp2011a}.\footnote{Consider for instance the use of \textit{haben} `have' with \textit{gelingen} `succeed' by Friedrich Schiller, a verb that is restricted to the perfect auxiliary \textit{sein} `be' in present-day \ili{German}: 
\ea
Man unterrichte sich demnach im Verfolg dieser Geschichte, wie weit ihr's gelungen hat -- Ich denke, ich habe die Natur getroffen.  \\ `One may inform oneself by following the story how far it has succeeded -- I think I pictured the nature quite well.' \hfill (Die Räuber; \citet[275]{behaghel23})
\z

\noindent
Further discussion and data for the changes concerning the alternation of perfect auxiliaries are provided by \citet[273ff.]{behaghel23}.
} The missing auxiliaries are given in the gloss of each example.  

\eal
\ex\label{ex-Mistral}
\gll vnd bekamen gleich am Morgen vor tags widerumb den Maistral, welchen wir \dots {} mit frewden angenommen \_\_/ \\ and got right {in the} morning before day again the Mistral which we {} with pleasure accepted (have)   \\ 
\glt `Right in the morning before daylight, we got the Mistral which we welcomed with pleasure.' \hfill  (1582: RW s64)
\ex
\gll {Zu dem} ist ein Gallion so auß Spania / nach Genoua abgefahren \_\_/ durch vngewitter/ in den Hauen Diff. in Prouinz eingelauffen / \\ furthermore is a galleon which from Spain {} to Genoa gone (is) {due to} thunderstorm in the harbor Diff. in Provence landed  \\   \hfill (1597: AC s190)
\glt `Furthermore a galleon which left Spain heading towards Genoa has landed in the harbor of Diff. in Provence.'
\ex
\gll Ob schon der Friede mit Engeland vor richtig gehalten  \_\_/ so wird dennoch dieses Reiches Krieges-Floote wenig vermindert \\ even if the peace with England for right considered (gets) so is however this country's war-fleet hardly diminished  \\ 
\glt `Even if the peace with England is considered to be good, the war fleet of this country has not been diminished.' \hfill (1667: M s4619)
\zl
Auxiliary ellipsis means that the dependent clause contains at least one non-finite form of a main verb.\footnote{There is disagreement in the literature regarding the inclusion of copula constructions. In contrast to the present paper, \citet{ERSW93} for instance do not consider examples as (\ref{copula}) as instantiations of ACs. Since this question has no impact for the present discussion, I will not delve any further into the matter.
\ea \label{copula}
\gll Alexander, Khuenig zu Polln vnd Großfuerstn in Littn der mer rhue vnd fridens begierig \_\_/ hat das alles lassen hin / geen \\ Alexander, king of Poland and Grand.Duke of Lithuania who more qietness and peace longing (was) has that everything let by {} go  \\   \hfill (1557: H s249)
\glt `Alexander, king of Poland and Grand Duke of Lithuania, who was longing for more quietness and peace, has everything let go by.'
\z
} Besides the past participle, the infinitive is attested either with or without the infinitival marker \textit{zu} `to', though there is a clear preference for the past participle in ACs, cf.\ Table (\ref{table2}).\footnote{As \textcite{breitbarth2005} points out, the auxiliary verbs \textit{haben} `have' and \textit{sein} `be' are dropped by far more frequently than the auxiliary \textit{werden} `get'. Auxiliary ellipsis is therefore a quite common phenomenon with the present and past perfect and much less so with passive constructions, cf.\ \textcite[78]{breitbarth2005} for the numbers in her corpus.} The table records all instances of auxiliary ellipsis attested in two-place verbal complexes. Auxiliary ellipsis in three-place verbal complexes is not taken into account, since we only find a few instances in our data.\footnote{Regarding the small number of instances, the role of ordering in the verbal complex, in particular the supposed position of the gap, cannot be addressed in this context, as suggested by one of the reviewers.} 
{\small
\begin{table}
  \centering
  \begin{tabular}{lrrr}
    \lsptoprule
   \textsc{enhg}\il{German!Early New High} & \sc past participle & \sc $^{zu}$infinitive & \sc bare infinitive  \\
    \midrule
    1401--1500 & 45 (90\%) & 1 (2\%) & 4 (8\%) \\
    1501--1600 & 1,606 (89\%) & 108 (6\%) & 81 (5\%)  \\
    1601--1700 & 964 (92\%) & 47 (4\%) & 41 (4\%)  \\
    \midrule
    & 2,615 & 156 & 126 \\
    \lspbottomrule
    \end{tabular}
   \caption{Non-finite forms in Afinite Constructions
   \label{table2}}
\end{table}
}

The examples in (\ref{nonfiniteV}) provide instances for auxiliary ellipsis in two-place and three-place verbal complexes with the first example comprising \textit{zu} `to' + infinitive  and the second example two participles in a passive construction. Our findings from the \textsc{enhg}\il{German!Early New High} treebanks hence match findings such as \citet{breitbarth2005} and \citet{thomas2018} regarding the form of the non-finite verb in ACs in qualitative and quantitative respects.
\eal \label{nonfiniteV}
\ex 
\gll Vnd ist sich zu verwunderen/  daß sie in solchen Gewehren also ge\oldue{}bet/  daß sie ohne f\oldae{}hlen dieselbigen in jhre feind werffen/ vnd sonderlich mit den Messeren  welche den breiten Sch\r{u}macher-messern \textbfemph{zu} \textbfemph{vergleichen} \_\_, dem feind seinen kopff mit werffen voneinander spalten. \\ and is \refl {} to astonish that they in such weapons also practiced that they without missing {the.very. same} towards their enemies throw and {in.particular} with the knives which the wide shoemaker-knives to compare (are) the enemy his head with throwing apart split \\  \hfill (1624: Brun s94)
\glt `It is astonishing that they are so skillful in using such weapons that they can throw those towards their enemies without missing them, in particular the knives -- comparable to a shoemaker's knife -- in order to split the heads of their enemies apart.'
\ex 
\gll hingegen sollen die Gelder so unsern Schiffen zu Ostende \textbfemph{abgefordert} \textbfemph{worden} \_\_/ wieder erleget/ und die Abforderer zur Straffe gezogen werden. \\ however shall the means which our ships at Ostende demanded been (have) again reimbursed and the demanders to account called are   \\ \hfill (1667: M s186)
\glt `The means, however, which have been demanded from our ships at Ostende, are supposed to be reimbursed and the wrongdoers are called to account for this.'
\zl

The particular distribution of ACs regarding the form of the non-finite verb will be taken up again in Section~\ref{enhg_context}.

Dependent clauses without a finite auxiliary occur in all positions where dependent clauses can appear in \textsc{enhg}\il{German!Early New High}: apart from the right periphery which is their most frequent position (\ref{postfield}), they occur at the left edge of the clause (\ref{prefield}) and sometimes even in the middlefield (\ref{middlefield}). At the left edge, the auxiliary is omitted especially when it is identical to the fronted auxiliary of the main clause, cf.\ (\ref{prefield_adja}). As the example in (\ref{postfield}) illustrates, auxiliaries may be omitted more than once in a complex sentence. Both ACs are not related by a coordination relation. 

\eal
\ex  \label{postfield}
\gll Vmb den 20. diß/ ist bey Mödling in Oesterreich ein erschröcklich Wetter abgangen/ welches so viel stein geworffen \_\_/ daß sie in den Gräben eines Knie tieff gelegen \_\_/  \\ around the 20. of.this is at Mödling in Austria a terrible thunderstorm broken which so many rocks thrown (has) that they in the trenches a knee's deep lain (have)   \\ \hfill (1597: AC s423)
\glt `Around the 20th of May, a terrible storm has broken over Mödling in Austria such that many rocks fell down filling the trenches about knee-deep.'
\ex \label{middlefield}
\gll du hast mich Oliuiers/ den ich vnder allen menschen am liebsten gehabt \_\_/ beraubt \\ you have me Olivier's whom I among all people the best liked (have) bereaved \\ \hfill (1532: FB s488) 
\glt `You have bereaved me of Olivier whom I have liked the best among all people.'
\zl
\eal \label{prefield}
\ex 
\gll Alß wir vns nun v\~{m} Mittagszeiten z\r{u}ruck nach dem Portu wider gewendet \_\_/ ersahen wir zur lincken von ferne ain Schiff \\ when we us now around noon back to the harbor again turned (have) discovered we {at the} left from afar a ship \\ \hfill (1582: RW s99)
\glt `When we turned back to the harbor at noon, we discovered afar a ship to our left.'
\ex \label{prefield_adja}
\gll Hernach da die Kirchen die Gefahr gesehen \_\_/ hat sie diß abstellen koennen. \\ thereafter when the church the danger recognized (has) has it this stop can  \\ \hfill (1650: FP s318)
\glt `When the church recognized the danger eventually, it could have stopped it.'
\zl
ACs are particularly frequent in relative and adjunct clauses as witnessed by the preceding data, whilst ACs in object and subject clauses occur much less frequently. Two examples for the latter types of subordinate clauses are given below. Table \ref{table3} provides an overview of the instances occuring in the treebanks of \textsc{enhg}\il{German!Early New High} depending on clause type, merging subject and object clauses into one type of subordinate clause (= complement).
\eal
\ex
\gll was sie verricht \_\_ möchte in volgenden Monaten erzehlet werden. \\ what they fullfilled (have) wants in following months told been \\  \hfill (1597: AC s120)
\glt `In a few months time, it might be told what they have fullfilled.'
\ex
\gll Jn disem Land haben die Holaender vor der zeit gewunnen, was sie begert \_\_. \\ in this country have the Dutch before the time obtained what they wanted (have)  \\ \hfill (1624: Brun s555)
\glt `In this country, the Dutch have obtained some time ago whatever they wanted.'
\zl

{\small
\begin{table}
  \centering
  \begin{tabular}{lrrrr}
    \lsptoprule
   \sc enhg & \sc relative & \sc adjunct & \sc complement \\
    \midrule
    1401--1500 & 20/1,856 (1\%) & 33/2,784 (2\%) & 12/1,277 (1\%)   \\
    1501--1600 & 827/3,250 (25\%) & 701/3,184 (22\%) & 113/4,390 (3\%) \\
    1601--1700 & 488/2,919 (17\%) & 466/2,874 (16\%) & 110/10,160 (1\%) \\
    \midrule
   \sc total & 1,335/8,025 (17\%) & 1,200/8,842 (14\%) & 235/15,827 (1\%)  \\
    \lspbottomrule
    \end{tabular}
   \caption{Afinite constructions depending on function of subclause   \label{table3}}
\end{table}
}

\largerpage
Accounting for a similar distribution across functions of subclauses, \citet[139ff.]{breitbarth2005} suggests to consider ACs as markers for pragmatic dependency. According to her, relative and adjunct clauses are pragmatically more dependent from their matrix clause than subject and object clauses and therefore tend to trigger auxiliary ellipsis. \citet{demske90} on the other hand argues in line with previous work by \citet{admoni67} that ACs are markers of syntactic dependency. In her view, their preference for relative and adjunct clauses results from the ambiguity of adverbial and relative connectors: lexical items such as \textit{so} `so' may introduce verb-final relative clauses (\ref{so_final}) as well as verb-second clauses (\ref{so_second}), indicating that they do not signal syntactic dependency of a clause. Even the placement of the finite verb is not a reliable marker for syntactic dependency in \textsc{enhg}\il{German!Early New High}, as argued by \textcite{demske2018}. Instead, the omission of the finite auxiliary may be taken as an unambiguous marker (\ref{so_ac}).   
\eal
\ex \label{so_final}
\gll vnd keret ich im Fundique der Frantzosen ein/ wie alle Teutschen \textbfemph{so} dahin kommen zuthon pflegen/ \\ and stopped I in.the shelter {of.the} French off as all Germans who there come to.do use   \\ \hfill (1582: RW 68.3)
\glt `And I stopped off in the French's shelter as all Germans do who come here.'
\ex \label{so_second}
\gll \textbfemph{So} seind sie auch mit Rettich/ Knoblauch/ Zwybel zimlich wol versehen.\\ so are they also with radish garlic onion very well supplied   \\ \hfill (1582: RW 73.13)
\glt `They are well supplied with radish, garlic and onions.'
\ex \label{so_ac}
\gll  Dise seind aber in wenig Jaren von vngest\oldue{}minen deß M\oldoe{}hrs so gar verw\oldue{}stet/ vnd mit dem sand/ \textbfemph{so} das Wasser darüber außgeworffen \_\_/ dermassen bedeckt worden/ das {heütigs tags} an denen orten sonderlichs nichts/ dann ain sandechter boden (wie im w\oldue{}sten Arabien) z\r{u}finden \_\_. \\ those are however in few years by storms of.the ocean so very devastated and with the sand which the water over.it casted (has) so covered been that today at these places else nothing but a sandy soil (as in.the arid Arabia) to.find (is)  \\  \hfill (1582: RW s167)
\glt `They have been devastated by stormy weather and so covered by sand which has been cast over by the water such that today one will find nothing but sand (as in the arid Arabia).'
\zl

\noindent
Whatever the ultimate motivation of using ACs in \textsc{enhg}\il{German!Early New High}, be it a pragmatic or a syntactic one, the question arises how to analyze this type of headless construction. This will be the topic of the following section. The probable motivation for using ACs in \textsc{enhg}\il{German!Early New High} will be addressed in the final section of the paper.

%%%%%%%%%%%%%%%%%%%%%%%%%%%%%%%%%%%%%%%%%%%%%%%%%%%%%%%%%%%%%%%%%%%%%%
%%%%%%%%%%%%%%%%%%%%%%%%%%%%%%%%%%%%%%%%%%%%%%%%%%%%%%%%%%%%%%%%%%%%%%%
\section{Afinite Constructions as auxiliary ellipsis \label{analysis}}

\subsection{Afinite Constructions are a type of ellipsis \label{arg_ellipsis}}

\largerpage[2]
According to \citet{schroeder85}, ACs are not instantiations of ellipsis. He claims that any type of ellipsis requires the identity of a gap with a suitable antecedent, cf.\ also \citet{biener25} for a similar argument. The following example shows that this condition need not hold for ACs in \textsc{enhg}\il{German!Early New High}, going on the assumption that the auxiliary \textit{sein} `be' is omitted in the relative clause comprising a non-finite form of the motion verb \textit{kommen} `come', while the only other auxiliary in the present context is the auxiliary \textit{haben} `have' in the root clause. 
\ea \label{AC}
\gll Sein Schiff hat vom Feuer/ so in die Brandwein Fässer gekommen \_\_/ grossen Schaden erlitten/ \\ his boat has from fire which in the spirits barrels come (is) major damage suffered \\ \hfill (1667: M s831)
\glt `His boat has suffered major damage from the barrels filled with spirits which have been on the boat.'
\z
\citet{schroeder85} therefore suggests treating ACs as a member of the class of non-finite clauses
along with participial and infinitival clauses as illustrated in (\ref{satzwertig}). The first
clause involves a present participle heading a relative clause, the second one a $^{zu}$-infinitive
heading an adjunct clause.

\eal \label{satzwertig}
\ex
\gll Die Dünkircher Capers haben Königl. Ordre/ alle Ostender Schiffe [mit Wahren von Contrebande nach Engeland fahrend]/
wegzunehmen.  \\ 
     the Dunkirk privateers have royal order all Ostende ships with goods of contraband to England going to.capture \\  \hfill (1667: M s841)
\glt `The Dunkirk privateers have royal order to capture all ships heading for England with contraband.'
\ex
\gll Jch were zwar gern außgestigen/ [dise Jnsulen besser zu besichtigen] \\ I were indeed with.pleasure got.off these islands better to visit    \\ \hfill (1624: Brun s22)
\glt `I would have loved to get off in order to better visit these islands.'
\zl

\noindent
Even if the AC in (\ref{AC}) and the non-finite clauses in (\ref{satzwertig}) have in common that
they do not include a finite verb, they differ in three crucial respects: ACs are restricted to
dependent clauses introduced by either a relative pronoun or a subordinating conjunction, and ACs
include subjects in contrast to non-finite clauses as in (\ref{satzwertig}).\footnote{\ili{English} in
  contrast allows for subjects in non-finite clauses, as one reviewer adds, cf.\ \textit{for} \dots\
  \textit{to} infinitival clauses.} In addition, ACs differ from participial and infinitival
clauses, because the omitted auxiliary obviously governs the status of the non-finite verb: the
so-called `infinitivus-pro-participio' effect (IPP-effect) provides strong support for the presence
of an auxiliary at least at one point in the derivation. Pertinent examples include a modal verb
that does not appear as a past participle as expected but as an infinitive as shown with the modal
verb \textit{müssen} `must' governed by \textit{haben} `have' in (\ref{ersatzinfinitiv_a}). The AC
in (\ref{ersatzinfinitiv_b}) involves two non-finite verbs, one being the modal verb \textit{müssen}
`must'. An interpretation of the modal verb as finite is excluded because of the singular subject,
i.e. \textit{man} `one'. Obviously, the IPP-effect in (\ref{ersatzinfinitiv_b}) is triggered by the
omitted auxiliary verb. A further example for an AC exhibiting an IPP-effect is given in
(\ref{ersatzinfinitiv_c}) and involves the modal verb \textit{sollen} `be supposed to'.

\eal 
\ex \label{ersatzinfinitiv_a}
\gll Vñ seind von 20. gewaltigen Türckischen Meer-r\oldae{}uberen besprungen wordẽ/ haben aber weichen \textbfemph{m\oldue{}ssen/} die T\oldue{}rcken weren sonst meister worden.\\ and are by 20 powerful Turkish pirates overrun been have however retreat must.\textsc{inf} the Turkish were otherwise victor been \\ \hfill (1624: Brun s671)
\glt `They have been overrun by Turkish pirates and had to retreat. Otherwise the Turkish would have beaten them.'
\ex \label{ersatzinfinitiv_b}
\gll Sie aber hat beschuetzt der enge Weg/ darauff man nacheinander gehen \textbfemph{m\oldue{}ssen} \_\_/ daß der Feind jhnen nicht zu komen kondte. \\ them however has protected the small path where one after.another go must.\textsc{inf} (have) that the enemy them not towards come could  \\ \hfill (1624: Brun s645)
\glt `The small path where one had to go one after another however protected them, so that the enemy could not get to them.'
\ex \label{ersatzinfinitiv_c}
\gll die haben erstlich auf den Malkutsch Aga, der die Janitscharn herauff gen Raab führen
\textbfemph{sollen} \_\_/ getroffen \\ they have at.first upon the Malkutch Aga who the janissaries up towards Rab lead should.\textsc{inf} (have) come \\ 
\glt `At first, they have come upon the Malkutch Aga, who was supposed to lead the janissaries up towards Rab.' \hfill (1597: AC s743)
\zl

\noindent
ACs therefore do not pattern with non-finite clauses, but are rather a type of ellipsis. The following section will look into the analysis of ACs as elliptical variants of full clauses.
 
%%%%%%%%%%%%%%%%%%%%%%%%%%%%%%%%%%%%%%%%%%%%%%%%%%%%%%%%%%%%%%%%%%%
\subsection{Dropping the finite auxiliary \label{finiteness}}

ACs are subordinate clauses where a finite auxiliary is dropped. The missing information includes agreement features such as person and number as well as finiteness features, including tense, mood and assertion features according to \textcite{klein2006} and \textcite{repp2009}. On \textcite{repp2009}'s approach, the latter bundle of features is needed to anchor a finite clause in the possible world where the proposition is evaluated. In contrast to syntactically independent clauses, the anchoring of a subordinate clause is supposed to happen indirectly by way of its subordinator which anchors the dependent clause in its matrix clause.\footnote{Non-finite independent clauses are anchored through their modal interpretation as pointed out by \citet{reis2003} with respect to examples like the following:
\eal
\ex 
\glt Ich -- (und) die Fenster putzen? Niemals! \\ `I -- (and) cleaning the windows? Never!'
\ex 
\glt Wohin fahren? \\ `Where to go?'
\ex 
\glt Den Eierkuchen wenden. \\ `Flip over the pancake.'
\zl
} 

In present-day \ili{German}, the finite verb can be omitted in Gapping and Sluicing constructions. In Gapping constructions, the elliptical clause is part of a coordination structure -- including embedded clauses -- and the finite verb is absent in the second conjunct. The following examples illustrate the observation that parallelism and identity promote the elision of language material. They all display alternatives with the focus being set on the local PP. Gapping also allows for the omission of more than the finite verb, namely the verbal complex (\ref{gapping_b}), and is hence a type of non-constituent ellipsis.\footnote{cf.\ \citet{reich2011} for a comprehensive typology of ellipsis (in \ili{German}).} 
\eal \label{gapping_a}
\ex  
\gll dass Fred in München wohnt und Selma in Berlin \_\_. \\ that Fred in Munich lives and Selma in Berlin (lives)  \\ 
\glt `that Fred lives in Munich and Selma in Berlin.'
\ex \label{gapping_b} 
\gll dass Fred in München gewohnt hat und Selma in Berlin \_\_. \\ that Fred in Munich lived has and Selma in Berlin (lived has)  \\
\glt `that Fred has lived in Munich and Selma in Berlin.'
\zl
Both examples exhibit instances of ellipsis in conjoined subordinate clauses. Note that Gapping in subordinate clauses requires that the complementizer in the second conjunct is dropped as well. Following \citet{repp2009}, I assume that anchoring of the proposition is due to the complementizer in the first conjunct, respectively. 

Sluicing is another type of ellipsis in present-day \ili{German} involving a finite verb. In contrast to Gapping, Sluicing is restricted to subordinate clauses with the anchoring implemented by the \textit{wh}-phrase in a functional projection such as CP \citep{repp2009}. The omitted material always includes the whole subordinate clause except the \textit{wh}-element at the left periphery as shown by the following examples from \textcite{repp2009}. It does not matter whether the matrix clause is syntactically independent or dependent, cf.\ (\ref{sluicing_a}) vs. (\ref{sluicing_b}).
\eal
\ex \label{sluicing_a}
\gll Max hat etwas gegessen, aber ich weiß nicht, was \_\_\_. \\ Max has something eaten, but I know not what  \\
\glt `Max ate something, but I don't know what.'
\ex \label{sluicing_b}
\gll Max liebt diese Gruppe, obwohl ich nicht weiß, warum \_\_\_.\\ Max adores this band, although I not know why  \\
\glt `Max loves this band, although I don't know why.'
\zl
Obviously none of these types of ellipsis patterns with ACs: Gapping does allow the omission of a finite auxiliary but is restricted to coordination structures, while Sluicing is restricted to subordinate clauses but does not allow the omission of just the finite (auxiliary) verb. Going on the assumption that ACs are a type of ellipsis, they instantiate a type no longer productive in \ili{German}.

%%%%%%%%%%%%%%%%%%%%%%%%%%%%%%%%%%%%%%%%%%%%%%%%%%%%%%%%%%%%%%%%%%%%%%%%%%%%%%%%%%%%%%%
\subsection{Resolving the missing information}

Depending on the way missing information is resolved, two types of ellipsis are distinguished: antecedent-based ellipsis and situation-based ellipsis \citep{reich2011}. The latter type relies on the current situation as well as world knowledge to retrieve the omitted information, cf.\ (\ref{situation}) for an example from present-day \ili{German}. Antecedent-based ellipsis reconstructs the missing information from a suitable antecedent as illustrated by the Gapping construction in (\ref{antecedent}) where the finite verb is ommited in the second conjunct.
\eal
\ex \label{situation}
\gll Mit Senf, bitte.  \\ with mustard please   \\
\glt `With mustard, please.'
\ex  \label{antecedent}
\gll Selma arbeitet an ihrem neuen Roman und Wilma an ihrem Blog. \\ Selma works at her new novel and Wilma at her blog    \\
\glt `Selma works at her new novel and Wilma at her blog.'
\zl
How do ACs fit into this typology? The omitted finite auxiliary is certainly not resolved by the non-linguistic context and ACs cannot be considered instantiations of the situation-based type of ellipsis. But do they belong to the antecedent-based type of ellipsis? This type requires identity between the omitted information and a linguistic antecedent as in (\ref{antecedent}): the finite verb of the first conjunct \textit{arbeitet} `works' is taken to fill the gap in the second conjunct. Even if ACs do not occur in coordination but subordination structures, identity between gap and a possible antecedent might play a role as suggested by examples like (\ref{identity}). The finite auxiliary is dropped in a temporal clause preceding its matrix clause. Assuming that it is the auxiliary \textit{haben} `have' which is missing here, identity holds not only for the choice of the auxiliary verb but also for the grammatical features person and number.\footnote{In contrast to present-day \ili{German}, the verb \textit{stoßen} `hit' appears with the auxiliary \textit{haben} `have' instead of \textit{sein} `be', testifying to the change as regards the use of auxiliary verbs in the history of \ili{German}  \textcite[387]{ERSW93} and \textcite{sapp2011a}.}
\ea \label{identity}
\gll Als der 33 Jar geregiert \_\_/ hat er mit seinem fueß an seines gestorbnen Roß khopf gestossen/ ist durch ain vergiffts thier gepissen worden/ vnd dauon gestorben. \\ when he 33 years governed (had) has he with his foot at his dead horse's head hit is by a poisonous animal bitten been and thereof died  \\  \hfill (1557: H s93)
\glt `When he has ruled 33 years, he hit the head of his dead horse with his foot and was bitten by a poisonous animal whereof he died.'
\z
There are, however, many attested examples for ACs where either identity between a possible antecedent and the dropped auxiliary is not given or where a suitable antecedent is entirely absent. The first case concerns examples like (\ref{non-identity_a}) where the linguistic context displays another auxiliary verb or (\ref{non-identity_b}) where auxiliaries do not agree with respect to the grammatical feature person (second-person vs. first-person). 
\eal
\ex \label{non-identity_a}
\gll Sein Schiff hat vom Feuer/ so in die Brandwein Fässer gekommen \_\_/ grossen Schaden erlitten/ \\ his boat has from fire which in the spirits barrels come (is) major damage suffered \\ \hfill (1667: M s831)
\glt `His boat has suffered major damage from the barrels filled with spirits which have been on the boat.'
\ex \label{non-identity_b}
\gll du hast mich Oliuiers/ den ich vnder allen menschen am liebsten gehabt \_\_/ beraubt \\ you have me Olivier's whom I among all people the best liked (have) bereaved \\ \hfill (1532: FB s488) 
\glt `You have bereaved me of Olivier whom I have liked the best among all people.'
\zl
Identity is not given in a trivial sense in examples without a possible antecedent altogether as in (\ref{non-identity_c}): the auxiliary \textit{haben} `have' is dropped in a relative clause, embedded in a clause with a finite form of the main verb \textit{begleiten} `accompany' but no finite auxiliary. Examples like the one below therefore suggest that ACs are not a type of antecedent-based ellipsis either, giving rise to the question how we reconstruct information in this type of ellipsis.
\ea \label{non-identity_c}
\gll sie belaiten auch {die jhenige}/ vber welche der Cadi oder Oberrichter den sententz gesprochen
\_\_/ zur Hauptstatt hinauß/ damit sie sehen/ das der execution gn\r{u}gsame volziehung geschehe: \\
they accompany also those about whom the Cadi or judge the sentence pronounced (has) of.the capital out so.that they see that of.the execution sufficient performing happened    \\ \hfill (1582: RW s284)
\glt `They also accompanied those out of the capital whose sentence has been pronounced by the Cadi or judge, in order to see whether the execution would be enforced in a sufficient manner.'
\z
Let us turn back to the question what kind of information we drop when we drop the finite auxiliary. We have seen that finiteness comprises tense and mood features besides the agreement features person and number. As stated by \citet{repp2009}, tense and mood features expressed by the finite verb in syntactically independent clauses are used to anchor the proposition in a possible world where the proposition is evaluated. As a consequence, Gapping in independent clauses is restricted to coordination structures exhibiting identity between gap and antecedent. As regards subordinate clauses, anchoring happens through the same functional projection in the C-domain as in independent clauses, hosting here either a subordinator or a pronominal element instead of the finite verb as in syntactically independent clauses. \citet{repp2009} observes that Gapping in subordinate clauses is only available when the complementizer is dropped in the second conjunct under identity, cf.\ the example below taken from \citet{repp2009}. Gapping thus aims at elements anchoring the proposition in a possible world. 
\ea \label{gapping_d}
\gll Ich denke, dass Max draußen spielt und (*dass) Pia drinnen. \\ I think that Max outside plays and (that) Pia inside \\
\glt `I think that Max plays outside and Pia inside.'
\z
In contrast to Gapping, ACs do not occur in coordination but subordination structures with the complementizer or relative particle present in all instances. This is expected in view of the fact that the subordinator is required to anchor the proposition by way of connecting it to the tense and mood features of its matrix clause. Anchoring is done by the subordinating conjunction \textit{dass} `that' in (\ref{subordinator_a}) and by the relative pronoun \textit{davon} `thereof' in (\ref{subordinator_b}), while a finite verb is not necessary to anchor subordinate clauses. This is cross-linguistically reflected by the fact that many subordinate clauses do not include a finite verb \citep{repp2009}. Note that anchoring by a subordinator does not hinge on the fact that the subordinator itself carries temporal meaning, tense is rather contributed by the matrix predicate.\footnote{It is therefore not necessary, as claimed by \citet[105]{breitbarth2005}, to establish a relationship between the rise of ACs and the unfolding of a system of temporal subordinators.}

\eal
\ex \label{subordinator_a}
\gll Dem seye nun wie jm w\oldoe{}ll/ so m\r{u}ß ich gleichwol bekennen/ \textbfemph{das} sie noch {heütigs tages} in dem Acker z\r{u}finden \_\_. \\ this.one be now as it may so must I nonetheless confess that they still nowadays in the field to.find (are) \\ 
\glt `Be it as it may, I have to confess nevertheless that they are to be found in the field still nowadays.' \hfill (1582: RW s385)
\ex \label{subordinator_b}
\gll Die Holländer haben von unsrer Virginischen Floote nur 17. Kauff-Schiffe und die Fregat Elisabeth bekommen/ \textbfemph{davon} sie 13. mitgenommen und 4. mit der Fregat verbrennt \_\_. \\ The Dutchmen have of our Virginia fleet only 17 trading.vessels and the frigate Elisabeth got whereof they 13 captured and 4. with the frigate burnt (have)  \\ 
\glt `The Dutchmen have got only 17 trading vessels and the frigate Elisabeth from our Virgina fleet whereof they captured 13 and burnt 4 together with the frigate Elisabeth.' \hfill (1667: M s4959)
\zl
Likewise, the subordinator anchors the embedded propositions in coordinated dependent clauses, appearing in the first conjunct. The finite auxiliary may be dropped in only one (\ref{coord_a}) or both conjuncts (\ref{coord_b}).
\eal
\ex \label{coord_a}
\gll \textbfemph{Alß} sich nun das wetter für vns widerumb geschicket \_\_/ wir auch unser Schiff mit holtz vnnd frischem wasser gnuogsam hetten versehen/ liessen wir die Segel fliegen/ vnnd fuohren daruon:  \\  when \refl {} now the weather for us again befitted (has) we also our boat with wood and fresh water sufficiently had equipped let we the sails fly and went away   \\ 
\glt `When the weather became suitable for us and we had our boat sufficiently equipped with wood and fresh water, we let the sails fly and went away.'  \hfill (1582: RW s500)
\ex \label{coord_b}
\gll \textbfemph{Dieweil} wir aber bey 7. Monat allda gelegen \_\_ mit vnserem Schiff sampt einem Jacht-schiff/ vnd sie vnser gemueth genugsam erfahren vnd erkandt \_\_/ haben sie vns alles guts erzeigt/ \\ while we however about 7 months there anchored (have) with our boat along.with a hunting-boat and they our mind sufficiently understood and known (have) have they us all good shown \\  \hfill (1624: Brun s211)
\glt `While we anchored there with our boat along with a hunting boat for about seven months and they had understood our mind well enough, they showed us all interesting things.'
\zl

\noindent
Unlike infinitival and participial clauses, the finite auxiliary is present in the syntactic structure of ACs, cf.\ arguments provided in Section~\ref{arg_ellipsis}. The agreement features carried by the auxiliary in T assign nominative case to the subject in question and the auxiliary \textit{haben} `have' governs the third status of the non-finite verb, cf.\ example (\ref{port}). The strikethrough in Figure \ref{tree} indicates that the auxiliary is given a silent spellout in the phonological component of the grammar and testifies to the deletion approach adopted here.\footnote{Three approaches to ellipsis are distinguished: (i) the movement approach, (ii) the anaphora approach, and (iii) the deletion approach. cf.\ \citet{reich2011} for a detailled discussion.}
\ea \label{port}
\gll Da fieng es den Portugalesern am wasser zu manglen/ jnmassen sie grossen durst gelitten \_\_. \\ then started it the Portugese in water to lack such.that they big thirst suffered (have) \\ \hfill (1624: Brun s416)
\glt `Then there was a shortage of water, such that the Portugese severely suffered from thirst.'
\z

\begin{figure} 
\begin{forest} 
[{CP}
[{C} [\textit{jnmassen}\\ `such.that']] 
[{TP}
[{SpecTP} [\textit{sie} \\ `they']]
[{T\rlap{$'$}}
[{VP} 
[{DP} [\textit{grossen durst} \\ `big thirst']]
[{V} [\textit{gelitten} \\ `suffered']]]
[{T}[\rem{\textit{haben}} \\ `have']]]
]]
\end{forest}
\caption{Afinite Construction \textit{jnmassen sie grossen durst gelitten} \label{tree}}
\end{figure}

\noindent
In line with \citet{klein2006} and \citet{repp2009}, Figure \ref{tree} subscribes to the assumption that finiteness is rather a property of clauses than of VPs. Irish data -- (\ref{irish}) is taken from \citet{repp2009} -- show that finiteness markers can attach to complementizers, providing cross-linguistic evidence for this assumption.\footnote{\textcite{weiss1998} discusses a \ili{German} dialect exhibiting a similar behavior as regards inflected complementizers.} An analysis of ACs along the lines of Figure \ref{tree} further suggests that ACs are only superficially headless clauses, whereas they actually contain a silent head.  
\ea \label{irish}
\gll D'éag sé sula-\textbfemph{r} thánig an sagart \\ die.\pst {} he before-\pst {} come.\pst {} the priest \\  \hfill \citep{repp2009}
\glt `He died, before the priest arrived.' 
\z

\largerpage
\noindent
Like other types of ellipsis, ACs drop redundant information for economy reasons. In contrast to other types of ellipsis, however, redundancy in ACs does not arise because of identical material in the linguistic context. The finite auxiliary may rather be omitted because it carries grammatical features which are already contributed by the subordinator and the tense features of the matrix predicate. The example (\ref{tense}) with ACs in a sentence-initial temporal clause and a relative clause following the matrix clause illustrates how tense information is provided by the context: the event denoted by the temporal clause precedes the event expressed in the matrix clause. Since the latter uses narrative past tense, it is the past tense of the auxiliary \textit{haben} `have' which can be reconstructed. The missing finite verb in the relative clause has to be the past tense of \textit{sein} `be', because the proposition denotes an unalterable property of the small town \citep[27]{rothstein2008}.\footnote{Regarding individual state predicates, \textcite[27]{rothstein2008} presents a clear grammaticality contrast between past tense and present perfect.}  

\ea \label{tense}
\gll Als wir die gantze weite herumb gnuogsam besichtiget \_\_/ giengen wir hinab zur stett des fleckens Bethphagae, welcher jhenseyt an der hoehe des oelberges gelegen \_\_/ vnd den Priestern zu Jerusalem hat zu gehoret:  \\ when we the whole plateau around sufficiently visited (had) went we down to.the place of.the small.town B. which beyond at the incline of.the mount.of.olives situated (was) and the priests of Jerusalem has to belonged    \\ \hfill(1582: RW s348)
\glt `When we had sufficiently visited the whole plateau, we went down towards the small town of B. which was beyond, at the  incline of the mount of olives and belonged to the priests of Jerusalem.'
\z

\largerpage
\noindent
The finite verb in a subordinate clause is rather not omitted when it carries information in addition to the finiteness features: according to \citet[442]{ERSW93}, the finite auxiliary tends to be present when it is marked for subjunctive, i.e. when there is grammatical information that cannot be provided otherwise. In (\ref{subjunctive}), the perfect auxilary \textit{haben} `have' is marked for subjunctive in a context favorable for auxiliary drop, i.e. an IPP-effect context \citep{haerd81}. Another case is provided by subordinate clauses with the finite verb being either a modal verb or a main verb accounting for the observation that modal verbs are rarely omitted from subordinate clauses \citep[486--492]{behaghel28}. Here again, the finite verb conveys information that goes beyond finiteness information.\footnote{This does not hold for a contrast observed by \citet{warner95} between full verbs and auxiliary verbs in \ili{English}: while only full verbs may be dropped in the following context in present-day \ili{English}, the auxiliary \textit{be} lost this option at the beginning of the 19th century; cf.\ the examples taken from \citet[537]{warner95}:
\eal
\ex If Paul comes in, then Mary will too. (sc. come in)
\ex `I wish our options were the same. But in time they will.' (sc. be the same)\\ \hfill (1816 Jane Austen, Emma) 
\zl
According to \citet{warner95}, the auxiliary \textit{be} loses this option due to a change affecting the possible decomposability of finite verbs into lexical information and tense information.  
}
This also holds for \textit{sein} `be' and \textit{haben} `have' when they trigger a modal meaning, building complex verbs with a $^{zu}$infinitive.

\ea \label{subjunctive}
\gll  vnd darmit jnen nit leichtlich hilff zukommen möchte/ ist der Rheinstromb/ mit den Außlägern dermassen verlegt worden/ das kein Schiff leichtlich \textbfemph{hette} vberkommen mögen. \\ and so.that them not easily assistance provide may is the Rhine with the ships so dammed.up been that no ship easily had pass may   \\  \hfill (1597: AC s804)
\glt `And in order to prevent any assistance for them, the Rhine has been dammed up with ships such that no ship may easily have passed.'
\z

%%%%%%%%%%%%%%%%%%%%%%%%%%%%%%%%%%%%%%%%%%%%%%%%%%%%%%%%%%%%%%%%%%%%%%%%%%%%%%%%%%%%%%%%%
%%%%%%%%%%%%%%%%%%%%%%%%%%%%%%%%%%%%%%%%%%%%%%%%%%%%%%%%%%%%%%%%%%%%%%%%%%%%%%%%%%%%%%%%%
\section{The larger picture: Early New High German \label{enhg_context}}

ACs are a type of ellipsis with a striking historical development as noted above: while they are frequently attested in the language of \textsc{enhg}\il{German!Early New High}, they scarcely occur in present-day \ili{German}. Taking into account the preceding discussion, I will come back to the question why ACs are a characteristic feature of \textsc{enhg}\il{German!Early New High} grammar. 

Dropping the finite auxiliary verb in a syntactically dependent clause is licensed by the subordinator linking the subordinate clause to its matrix clause and thus supplying its anchor to a possible world. But what kind of changes in the grammar of \ili{German} trigger the rise and subsequent spreading of ACs just in the 15th century? The asymmetry of main and subclauses regarding verb position is reasonably well established by the end of Old High \ili{German}, cf.\ \citet{axel2007} among others. The same holds for various types of coordinate ellipsis. The prerequisites are hence in place for some time already, before the first instances of ACs are attested. What is indeed new in \textsc{enhg}\il{German!Early New High}, is the coincidence of two changes affecting the coding of past tense. On the one hand, we observe the increasing use of the present perfect at the expense of past tense forms which gained momentum throughout the 16th century according to \citet{dentler1997}.\footnote{Cf.\ \citet{fischer2018} for a comprehensive discussion of the increasing use of the present perfect at the expense of the simple past in the history of \ili{German}. \citet{amft2018} informs on the development in a particular register.} An example from \citet{amft2018} is given in (\ref{loss}) where a perfect tense form is used in an otherwise all simple past context. 

\ea \label{loss}
\gll Jnn dißem allem weyß gott woll/ hab ich auß seynen gnaden mit lachendem hertzen vnd muthun gethan als merckt ichs gar nicht \\ in this all knows God well have I from his mercy with laughing heart and courage done as.if noticed I=it at.all not    \\ \\  \hfill (Korn; \citet[262]{amft2018})
\glt `God knows very well that I acted by his mercy and courage with a laughing heart as if I noticed nothing at all.'
\z

\noindent
On the other hand, we note the decline of finite forms of the simple past including the prefix \textit{ge-}, which are available until the 16th century \citep[386]{ERSW93}. As a consequence, verb forms including the prefix \textit{ge-} are rather interpreted as non-finite rather than finite forms in syntactic contexts admitting both forms, i.e. the right sentence bracket in subordinate clauses.\footnote{So far there are not enough data across dialects available to validate the suggested correlation between the loss of simple past forms and the increase of ACs in the history of \ili{German}.} An interpretation as a non-finite form is further supported by a prominent feature of \textsc{enhg}\il{German!Early New High} syntax, the omnipresence of coordinate ellipsis in this variety of \ili{German}, including instances of parallel and non-parallel coordination patterns in subordinate clauses \citep{behaghel28,schroeder85}.\footnote{Both types of ellipsis are not restricted to subordinate clauses, but are also attested in syntactically independent clauses \citep{biener25}. To account for non-parallel coordinate ellipsis, we therefore need an analysis allowing for non-identity of gap and antecedent. This problem does not arise for ACs, because licensing the gap works differently in coordination and subordination patterns. Non-parallel patterns of coordinate ellipsis in subordinate clauses may be instances of ACs with the gap licensed by a subordinator instead of an antecedent, cf.\ also \citet[63]{breitbarth2005} on this issue. But then we still have to account for non-parallel coordinate ellipsis in main clauses. And I have nothing to say in the present paper about coordinate ellipsis in \textsc{enhg}\il{German!Early New High}.} Note that auxiliary drop in parallel coordination patterns of subordinate clauses may be licensed either by a suitable antecedent or by a subordinator. Coordinate ellipsis is therefore promoting the spread of ACs, but not causing it, as claimed by \citet{behaghel28} and \citet{schroeder85}. 

Other features of \textsc{enhg}\il{German!Early New High} grammar support the spreading of ACs as well: as argued by \citet{demske2018}, verb position may be pragmatically motivated as in (\ref{vfinal}) where the final position of the finite verb indicates background information rather than syntactic dependency. The clause introduced by \textit{der selbig brack} `this tracker dog' elaborates on a referent introduced earlier into the discourse, and the clause introduced by the pronominal connector \textit{dardurch} `in this way' in the second example provides a scribe's comment on the reporting itself. 
\eal \label{vfinal}
\ex Do man geessen het, ward geschickt nach Wilhalems wappen unnd corperthur, desgeleichen nach seinem bracken, der im von der abentheüre hauptman gegeben was. \textbfemph{Der selbig brack} auff das mal nitt gefunden \textbfemph{ward}, dann er sich inn dem wald verlauffen het. \\ 
`After dinner, they sent for Wilhelm's coat of arms and armor, likewise for his tracker dog, that was given to him by the capitain. This tracker dog was not found in time, because he got lost in the forest.'  \hfill (1481: WvÖ 265.21)
\ex Do sprach auch maniger in der stat: \glqq Wer mag der mit dem feür sein, er furt die bosten corperthur, die ye kein man gesach, unnd anders vil, das er an im het.\grqq {} \textbfemph{Dardurch} er gelopt \textbfemph{ward}, das nit alles z\r{u} schreiben ist. \hfill  (1481: WvÖ 243.13) \\
`Some men in town also said: ``Who might be the one with the fire, he has the best armor ever seen by man, and other things more.'' {} Thus he was praised in a way that refrains from any description.’
\zl
At the same time verb position is a syntactic device in \textsc{enhg}\il{German!Early New High}, used to discriminate between
main clauses with verb-second and syntactically dependent clauses with the finite verb in final
position just as in present-day \ili{German}. As shown by \textcite{demske2018}, ambiguities arise with
pronominal elements like \textit{darauf} `afterwards' which are attested with the finite verb either
in second or in final position, cf.\ (\ref{darauf}a) vs. (\ref{darauf}b). In view of examples like
(\ref{vfinal}), there is no way to indicate whether verb-final clauses like (\ref{darauf}b) are
syntactically dependent or independent.\footnote{As \citet{loetscher2000} points out, clauses
  introduced by pronominal connectives like \textit{darauf} `afterwards' in \textsc{enhg}\il{German!Early New High} are always
  pragmatically dependent -- irrespective of their syntactic status.} Since only subordinators
license the omission of the finite auxiliary in dependent clauses, ACs provide a means to clearly
mark syntactic dependency in \textsc{enhg}\il{German!Early New High} (\ref{darauf}c). Recall that ACs are particularly
frequent in adverbial and relative clauses (cf.\ Table \ref{table3}), often introduced by an
ambiguous element such as \textit{wann} `when' or \textit{so} `so'. The following examples display
all three variants attested with the pronominal \textit{darauf} `afterwards', i.e., verb-second,
verb-final and AC.

\largerpage
\eal \label{darauf}
\ex Den folgenden Morgen haben die Türcken im Schloß angefangen zu parlamentieren vnd mit jhren Gütern abzuziehen begert/ weil man dann verstanden/ das in 400. gefangner Christen in der Vöstung sein sollen/ ist jhnen der Abzug mit jhren Wöhren/ vmb die gefangne Christen desto ehe bey dem Leben zuerhalten/ bewilliget worden. \textbfemph{Darauff} \underline{hat} man drey fürnemmer Türcken herauß/ vnd drey von den vnsern zu Geisel hinein gegeben.  \hfill (1597: AC s658) \\
`The next morning the Turkish people started to negotiate in the castle and desired to withdraw with their belongings, because there was an understanding that about 400 arrested Christian people are supposed to be in the fortress. The withdrawal with their stockpile of weapons has been granted to them in order to keep the arrested Christians alive. Afterwards there was an exchange of three noble Turkish men and three of our people.'
\ex König Agrant vonn Zisia reit von einem künig cz\r{u} dem anderen unnd bate sy all unnd jecklichen in sunderheit, iren fleiß z\r{u} th\r{u}nde wider den künig vonn Maroch, ob sach w\oldae{}re, das sy im abgesigen m\oldoe{}chtent. \textbfemph{Darauff} sy im all cz\r{u} \underline{sagtent}, das sy all das b\oldoe{}ste th\r{u}n w\oldoe{}ltent.\\  \hfill (1481: WvÖ 249.31)\\
`King Agrant of Zisia was riding from one king to the next and asked them all and each in person to fight with all power against the king of Morocco. Since they had to defeat him. They all promised him then to do their best.'
\ex Zu Parma hat es ein erschrecklich Wetter gehabt/ so ein Tag vnd Nacht gewehrt/  welches viel Bäum vnd Schornstein nidergerissen/ \textbfemph{darauff} es auch 6. Tag vnd Nacht stets geregnet \_\_ / \hfill (1609: R09 69.31) \\
`There was an awful storm in Parma, which lasted day and night. Many trees and chimneys were destroyed, then it did rain for 6 days and nights incessantly.'
\zl

\largerpage
\noindent
Probably, the ambiguity of examples like (\ref{darauf}b) is another feature of \textsc{enhg}\il{German!Early New High} grammar strongly promoting the rapid expansion of ACs in the 16th century. The marking of syntactic dependency does not, however, trigger the rise of ACs as claimed by \citet{admoni67, breitbarth2005, demske90} and \citet{senyuk2014}.\footnote{Apart from the ambiguity of verb position and pronominal connectors, \citet{admoni67} adduces the increasing complexity of sentences throughout the period of \textsc{enhg}\il{German!Early New High} to justify the necessity of an additional marker for syntactic dependency which is according to him most needed in chancery texts. The high frequency of ACs in this register may therefore be due to its complexity instead of \ili{Latin} influence. \citet{breitbarth2005} refers to the decreasing use of the subjunctive as a marker. None of these proposals can however explain why this type of ellipsis is confined to finite auxiliaries.} Even if this assumption can explain the fact that this type of ellipsis is restricted to syntactically dependent clauses, it fails to account for its restriction to finite auxiliaries. The same goes for Breitbarth's claim that ACs are markers for pragmatic dependency, cf.\ Section~\ref{distribution}.

Obviously, there are good reasons for ACs to become a prominent feature of \textsc{enhg}\il{German!Early New High} syntax. But why are ACs lost again throughout the 18th century? The general understanding is that the decrease is due to the growing conventionalization of syntactic means as part of the emergence of a written standard language in \ili{German}. Verb position for instance is no longer a pragmatic device to distinguish between foreground and background information, but marks exclusively syntactic (in-)dependency, and coordinate ellipsis is much more constrained in present-day \ili{German} than in \textsc{enhg}\il{German!Early New High}. Even if the omission of a finite auxiliary in a dependent clause could be licensed by its subordinator, this option is less and less used in \ili{German}. Late witnesses are examples from the 20th century, cf.\ (\ref{buddenbrooks}) above and further examples in \citet[18]{blum2018}. 

%%%%%%%%%%%%%%%%%%%%%%%%%%%%%%%%%%%%%%%%%
%%%%%%%%%%%%%%%%%%%%%%%%%%%%%%%%%%%%%%%%%%%
\section{Summary} 

ACs are a special type of ellipsis, rarely attested in present-day \ili{German}. They resemble the antecedent-based type of ellipsis, because they drop grammatical information provided by their matrix clauses. The ellipsis is licensed by the respective subordinator which links the proposition of the dependent clause to the proposition of the matrix clause. ACs are, however, only superficially headless clauses. The finite auxiliary is present in the syntactic structure of ACs as indicated by the assignment of nominative case and verb status, but is given a silent spellout in the phonological component of the grammar. ACs result from ambiguous verb forms in past tense, their overwhelming numbers in sources of the 16th and 17th century is due to remarkably favorable conditions in the \textsc{enhg}\il{German!Early New High} grammar, among others a strong bias towards ellipsis on all levels of language structure. ACs are lost again throughout the 18th century. 

What do we learn from the history of ACs regarding the question of headedness? Obviously, syntactic changes concerning auxiliary ellipsis do not affect the hierarchical structure of the clause. The verbal head is at all times present in the finite subordinate clause, even if it may remain silent in the phonological component of the grammar during the period of \textsc{enhg}\il{German!Early New High}. In my view, the rise of ACs is best understood against the background of a rising standard language with a great deal of variation not only in the lexicon but also in the grammar, fed by a large number of dialects. The increasing conventionalization of grammatical means in the history of \ili{German} from the 17th century onwards constrains the occurrence of ellipsis considerably: the grammatical system requires the spellout of silent heads except for instances comprising complete identity. 

%%%%%%%%%%%%%%%%%%%%%%%%%%%%%%%%%%%%
\section*{Acknowledgements}

I would like to thank the organizers Ulrike Freywald and Horst Simon as well as the audience at the Workshop `Köpfigkeit und/oder grammatische Anarchie?' in May 2017 for interesting comments and questions on the topic of auxiliary ellipsis in \ili{German}. In September 2018, I had the opportunity to present the present paper at the University of Tokyo. For an intense and fruitful discussion I owe many thanks to Manshu Ide, Jiro Inaba, Yoshiki Mori and Shin Tanaka in particular. Most appreciated comments to an earlier written version of the paper came from two anonymous reviewers.

\section*{Sources}


\begin{description}
\item[Baumbank.UP] Demske, Ulrike. 2019. Referenzkorpus Frühneuhochdeutsch: Baumbank.UP. Universität Potsdam: Institut für Germanistik.
\item[Fortunatus] Fortunatus. Nach der Editio Princeps von 1509. Hg. von H. G. Roloff. Stuttgart: Reclam 1981.
\item[Mercurius] Demske, Ulrike. 2007. Mercurius-Baumbank. Universität des Saarlandes: Fachrichtung Germanistik.
\item[R09] Die Relation des Jahres 1609. Hg. von W. Schöne, Facsimile. Leipzig: Harrassowitz 1940.
\item[WvÖ] Wilhelm von Österreich. Druck Augsburg 1481. Hg. von F. Podleiszek, in: Deutsche Literatur in Entwicklungsreihen, Reihe Volks- und Schwankbücher Bd. 2. Darmstadt: 1964, 191--284.
\end{description}

%\largerpage

%%%%%%%%%%%%%%%%%%%%%%%%%%%%%%%%%%%%%%%%%%%%%%%
{\sloppy
\printbibliography[heading=subbibliography,notkeyword=this]
}
\end{document}
