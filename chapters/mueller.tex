% Antonio: In Abbildung haben wir plötzlich S. Was ist der Kopf?

% Todo: Netter/VanEynde

% 1) Reviews lesen
% 2) Aufsatz selbst ansehen
% 3) Van Eynde lesen



\documentclass[output=paper
  ,nobabel
  ,draftmode
  ,uniformtopskip % manual adjustment of pagebreaks
  ,colorlinks, citecolor=brown
]{langscibook}


\IfFileExists{../localcommands.tex}{%hack to check whether this is being compiled as part of a collection or standalone
   \usepackage{orcidlink}

% add all extra packages you need to load to this file


% Haitao Liu
%\usepackage{xeCJK}
%\setCJKmainfont{SimSun}
%\setCJKmainfont[Scale=MatchUppercase,
%                Path=fonts/
%]{SourceHanSerifSC-Regular}

% instead use option:  ,chinesefont % for references in raffelsiefen.tex
% loading the package changes some spacings


\usepackage{multicol}
\usepackage{tikz}\usetikzlibrary{decorations.pathreplacing}
\usepackage{url}
\urlstyle{same}

%\usepackage{listings}
%\lstset{basicstyle=\ttfamily,tabsize=2,breaklines=true}

\usepackage{langsci-basic}
\usepackage{langsci-optional}
\usepackage[danger]{langsci-lgr}

% toggle danger in texlive 2021
%\newcommand{\M}{\textsc{m}\xspace}

% toggle danger in texlive 2021 or uncomment this
% \newcommand{\N}{\textsc{n}\xspace}
% \newcommand{\F}{\textsc{f}\xspace}


\usepackage{./styles/biblatex-series-number-checks}




\usepackage{langsci-gb4e}




% Demske

\usepackage{tipa}
\usepackage{styles/avm+}
%\usepackage{styles/merkmalstruktur}
\avmfont{\sc}
\usepackage{langsci-forest-setup}
\usepackage{xspace}
%\usepackage{styles/abbrev} 

\usepackage{soul}
\usepackage{color}
\newcommand{\rem}[1]{\textcolor{red}{\st{#1}}}
\newcommand{\add}[1]{\textcolor{blue}{\ul{#1}}}


% Salzmann

\usepackage[nocenter]{qtree}


% Müller

% add this to the default preamble 
\forestset{default preamble={
    for tree={anchor=north},
}}


\usepackage{german}

%\usepackage{german}
\selectlanguage{USenglish}

% Mit Babel geht irgendwie die hyphenation nicht richtig
%\usepackage[ngerman,english]{babel}
%\useshorthands{"} 
%\addto\extrasenglish{\languageshorthands{ngerman}}

\usepackage{styles/makros.2020,
styles/abbrev,
styles/merkmalstruktur,
styles/article-ex,styles/eng-date}


\usepackage{todonotes}
\newcommand{\todostefan}[1]{\todo[color=green!40]{\footnotesize #1}\xspace}
\newcommand{\inlinetodo}[1]{\todo[color=green!40,inline]{\footnotesize #1}\xspace}

\newcommand{\inlinetodoopt}[1]{\todo[color=green!40,inline]{\footnotesize #1}\xspace}
\newcommand{\inlinetodoobl}[1]{\todo[color=red!40,inline]{\footnotesize #1}\xspace}

\newcommand{\itdobl}[1]{\inlinetodoobl{#1}}
\newcommand{\itdopt}[1]{\inlinetodoopt{#1}}

\newcommand{\addpages}{\todostefan{add pages}}

%\newcommand{\iaddpages}{\yel[add pages]{pages}\xspace}



% subfigure
\usepackage{subcaption}



% Nolda
%\usepackage[main=british,nil,german,french]{babel}
\newcommand{\foreignlanguagedummy}[2]{#2}
\usepackage{tagpair}
\usepackage{hang}
\usepackage[noconfig]{ntheorem}
\usepackage{pstricks,pst-node,pst-tree}
\usepackage{newunicodechar}



   \newcommand*{\orcid}[1]{}

% do not show the chapter number. It is redundant, since most references to figures are within the
% same chapter.
\renewcommand{\thefigure}{\arabic{figure}}

\newcommand{\rlapsub}[1]{\rlap{\sub{#1}}}

% \SetupAffiliations{output in groups = false, 
%                    separator between two = {\bigskip\\},
%                    separator between multiple = {\bigskip\\},
%                    separator between final two = {\bigskip\\}
%                    }


%%%%%%%%Alte Umlaute
\newcommand{\oldae}{$\stackrel{\textrm{\tiny e}}{\textrm{a}}$}
\newcommand{\oldoe}{$\stackrel{\textrm{\tiny e}}{\textrm{o}}$}
\newcommand{\oldue}{$\stackrel{\textrm{\tiny e}}{\textrm{u}}$}

\newcommand{\refl}{\REFL}
\newcommand{\pst}{\PST}


% Müller
\let\vref\ref


\let\citew\citet

\newcommand{\page}{}

% biblatex stuff
% get rid of initials for Carl J. Pollard and Carl Pollard in the main text:
\ExecuteBibliographyOptions{uniquename=false}




\newcommand{\nom}{\textsc{nom}}
\newcommand{\gen}{\textsc{gen}}
\newcommand{\dat}{\textsc{dat}}
\newcommand{\acc}{\textsc{acc}}


%\newcommand{\spacebr}{\hspaceThis{[}}



\newcommand{\acknowledgmentsEN}{Acknowledgements}
\newcommand{\acknowledgmentsUS}{Acknowledgments}


% no bf!!!111!
\let\textbfemph\emph

\newcommand{\textbfremoved}[1]{#1}
%\newcommand{\emphremoved}[1]{#1}


\newcommand{\noemph}[1]{#1}
\newcommand{\underlineemph}[1]{\emph{#1}}




% for editing, remove later
\usepackage{xcolor}
\newcommand{\added}[1]{{\red #1}}
\newcommand{\addedthis}{\todostefan{added this}}

\newcommand{\changed}[1]{\textcolor{orange}{#1}}




% Nolda

\theorembodyfont{\normalfont}
\let\restriction\relax
\renewtheoremstyle{break}{\item{\itshape ##1\ ##2}\newline\nopagebreak}{\item{\itshape ##1\ ##2\ (##3)}\newline\nopagebreak}
\theoremstyle{break}
\newtheorem{definition}{Definition}
\newtheorem{pattern}{Pattern}
\newtheorem{restriction}{Restriction}
\newunicodechar{‑}{\hbox{-}}
\newunicodechar{…}{\dots}
\newunicodechar{⁡}{\relax}
\newunicodechar{⁣}{\relax}
\newunicodechar{⁀}{\raisebox{+1ex}{\ensuremath\frown}}
\newunicodechar{⁐}{\raisebox{+1ex}{\ensuremath\frown}\setbox1=\hbox{\ensuremath\smile}\hspace{-\wd1}\raisebox{-1ex}{\ensuremath\smile}}
\newunicodechar{⪪}{\ensuremath{<\mathrel{\llap{\ensuremath{-}}}}}
\setkomafont{descriptionlabel}{\normalfont}
\ExecuteBibliographyOptions{labeldate=comp,labelnumber=true,defernumbers=true}
\defbibenvironment{sources}{\list{\printfield{labelprefix}\,\printfield{labelnumber}}{\settowidth{\labelwidth}{S\,0}\setlength{\labelsep}{\biblabelsep}\setlength{\leftmargin}{\labelwidth}\addtolength{\leftmargin}{\labelsep}\setlength{\itemsep}{\bibitemsep}\setlength{\parsep}{\bibparsep}}\renewcommand{\makelabel}[1]{##1\hfil}}{\endlist}{\item}
\newcommand{\citesource}[1]{\citefield{#1}{labelprefix}\,\citefield{#1}{labelnumber}}


% Was soll das machen?
\newcommand{\textstyleFootnoteSymbol}{}



% Was ist das???? St. Mü. 30.10.2021
%Kann weg. Damit waren die bücker transkripte aligniert. Habe das jetzt mit tabularx und hphantom gemacht

%\newlength{\calength} %tmp length to store the space 1. until [; 2. until ].
%
%%first argument speaker ID, second argument text. Optional argument left margin indicator (arrow or similar)
%\newcommand{\cabox}[3][]{\parbox{0mm}{\hspace*{-1cm}#1}%
%\parbox{1.5cm}{#2}%
%\parbox{9.6cm}{#3}\\%
%}
%
%%translation. First parbox is empty, second parbox takes the translation text
%\newcommand{\trsbox}[1]{\parbox{1.5cm}{~}%
%\parbox{9.6cm}{\itshape #1}\\%
%}
%
%%store the width of a string.
%\newcommand{\settablength}[1]{\settowidth{\calength}{#1}\global\calength=\calength}
%
%%print string and store its width. Useful if the first item of the aligned set is also the longest
%\newcommand{\inittab}[1]{#1\settablength{#1}}
%
%%insert horizontal white space equivalent to the stored width
%\newcommand{\skiptab}{\parbox{\calength}{~}}
%
%%print the argument and fill up with horizontal white space until the stored width is reached.
%\newcommand{\filledtab}[1]{\parbox{\calength}{#1}}



% for standalone compilations Felix: This is in the class already
%\let\thetitle\@title
%\let\theauthor\@author 
\makeatletter
\newcommand{\togglepaper}[1][0]{ 
\bibliography{../bib-abbr,../stmue,../localbibliography,
collection.bib}
  %% hyphenation points for line breaks
%% Normally, automatic hyphenation in LaTeX is very good
%% If a word is mis-hyphenated, add it to this file
%%
%% add information to TeX file before \begin{document} with:
%% %% hyphenation points for line breaks
%% Normally, automatic hyphenation in LaTeX is very good
%% If a word is mis-hyphenated, add it to this file
%%
%% add information to TeX file before \begin{document} with:
%% \include{localhyphenation}
\hyphenation{
Arsch
anaph-o-ra
Bü-cking
con-stit-u-ents
Dor-drecht
For-schungs-ge-mein-schaft
Ge-schich-te
ha-ben
pho-nol-o-gy
pro-so-dic
pro-so-di-cally
Sal-pe-ter
sei-nen
Wil-liams
}
\hyphenation{
Arsch
anaph-o-ra
Bü-cking
con-stit-u-ents
Dor-drecht
For-schungs-ge-mein-schaft
Ge-schich-te
ha-ben
pho-nol-o-gy
pro-so-dic
pro-so-di-cally
Sal-pe-ter
sei-nen
Wil-liams
}
  % \memoizeset{
  %   memo filename prefix={hpsg-handbook.memo.dir/},
  %   % readonly
  % }
  \papernote{\scriptsize\normalfont
    \@author.
    \titleTemp. 
    To appear in: 
    Ulrike Freywald \& Horst Simon (eds.) Headedness and/or grammatical anarchy?
    Berlin: Language Science Press. [preliminary page numbering]
  }
  \pagenumbering{roman}
  \setcounter{chapter}{#1}
  \addtocounter{chapter}{-1}
}
\makeatother



% This does a linebreak for \gll for long sentences leaving space for the language at the right
% margin. The factor .0989 is needed since otherwise starred examples cause a linebreak.
% St.Mü. 17.06.2021 08.02.2021
\newcommand{\longexampleandlanguage}[2]{%
%\begin{tabularx}{.99\linewidth}[t]{@{}X@{}p{\widthof{(#2)}}@{}}%
%\begin{minipage}[t]{.99\linewidth}%
\begin{tabularx}{\linewidth}[t]{@{}X@{}p{\widthof{(#2)}}@{}}%
\begin{minipage}[t]{\linewidth}%
#1%
\end{minipage} & (\ili{#2})%
\end{tabularx}}

% ORCIDs in langsci-affiliations 
\usepackage{orcidlink}
\definecolor{orcidlogocol}{cmyk}{0,0,0,1}
\ProvideDocumentCommand{\LinkToORCIDinAffiliations}{ +m }
  {%
    \orcidlink{#1}
  }

   %% hyphenation points for line breaks
%% Normally, automatic hyphenation in LaTeX is very good
%% If a word is mis-hyphenated, add it to this file
%%
%% add information to TeX file before \begin{document} with:
%% %% hyphenation points for line breaks
%% Normally, automatic hyphenation in LaTeX is very good
%% If a word is mis-hyphenated, add it to this file
%%
%% add information to TeX file before \begin{document} with:
%% %% hyphenation points for line breaks
%% Normally, automatic hyphenation in LaTeX is very good
%% If a word is mis-hyphenated, add it to this file
%%
%% add information to TeX file before \begin{document} with:
%% \include{localhyphenation}
\hyphenation{
Arsch
anaph-o-ra
Bü-cking
con-stit-u-ents
Dor-drecht
For-schungs-ge-mein-schaft
Ge-schich-te
ha-ben
pho-nol-o-gy
pro-so-dic
pro-so-di-cally
Sal-pe-ter
sei-nen
Wil-liams
}
\hyphenation{
Arsch
anaph-o-ra
Bü-cking
con-stit-u-ents
Dor-drecht
For-schungs-ge-mein-schaft
Ge-schich-te
ha-ben
pho-nol-o-gy
pro-so-dic
pro-so-di-cally
Sal-pe-ter
sei-nen
Wil-liams
}
\hyphenation{
Arsch
anaph-o-ra
Bü-cking
con-stit-u-ents
Dor-drecht
For-schungs-ge-mein-schaft
Ge-schich-te
ha-ben
pho-nol-o-gy
pro-so-dic
pro-so-di-cally
Sal-pe-ter
sei-nen
Wil-liams
}
   \togglepaper[4]
}{}

\ChapterDOI{10.5281/zenodo.7142691}
\selectlanguage{USenglish}

% fix, remove once forest is updated
\makeatletter
\apptocmd\forest@pgfmathhelper@attribute@dimen{\global\pgfmathunitsdeclaredtrue}{\typeout{patching succeeded}}{patching failed}
\apptocmd\forest@pgfmathhelper@register@dimen{\global\pgfmathunitsdeclaredtrue}{\typeout{patching succeeded}}{patching failed}
\makeatother 


\title{Headless in Berlin: Headless (nominal) structures in Head-Driven Phrase Structure Grammar} 
%\runningtitle{Kopf"|lose Strukturen in HPSG}


\author{Stefan Müller\orcid{0000-0003-4413-5313}\affiliation{Humboldt-Universität zu Berlin}}


\abstract{This paper deals with the status of heads in Head-Driven Phrase Structure Grammar (HPSG). Firstly, background assumptions are presented: the lexical representation of valence at the head and projection of head features. Secondly, I discuss criteria for determining the head of a phrase. I use nominal structures as an example, since the DP/NP debate is still undecided across frameworks, and exploring the arguments from an HPSG perspective may be interesting for readers. \citegen{Zwicky85a} criteria are discussed, and I show that most of them do not decide the issue for German nominal structures, but assignment of semantic roles by relational nouns and selectional relations in idioms \citep{OG2012a-u,Bruening2020a} support NP structures. I discuss nominal structures with non-overt nouns and copulaless sentences in African American Vernacular English (AAVE) and argue for an empty nominal head. I show that empty elements can be eliminated from grammars but argue that they are nevertheless useful in nominal structures and copula constructions in AAVE, since they capture generalizations. However, there are other structures like \citegen{Jackendoff2008a} N-P-N construction that should be analyzed as unheaded. The paper closes with general considerations about the use of empty elements in grammars, arguing that they should be detectable in the input by systematic variation with overt material. This excludes the assumption of empty elements like AgrO or Topic in grammars of languages like German, since there is no overt material associated with these heads.}

\newtoggle{rest}\togglefalse{rest}


\begin{document}

\maketitle
\rohead{\thechapter\hspace{0.5em}Headless structures in HPSG} % Display short title


% \papernote{Submitted to Horst Simon \& Ulrike Freywald (Eds). \emph{Headedness and/""or grammatical
%     anarchy?} (Empirically Oriented Theoretical Morphology and Syntax). Berlin: Language Science Press.}


\section{Introduction}


Ulrike Freywald and Horst Simon asked proponents of various linguistic theories to take part in
their workshop \emph{Headedness and/""or grammatical anarchy?} and explain the notion of head used
in the respective theories. They asked the following questions:
\begin{itemize}
\item Are structural asymmetries a precondition for structure building?
\item Or do ``real'' non-headed structures exist?
\item If so, how are non-headed structures built?

%a. Entstehen sie aus asymmetrischen Strukturen (Verlust von Kopfmerkmalen)?
%b. Entstehen sie durch ein eigenständiges, separates Verfahren der Strukturbildung?

\item How does headedness/a headed structure work, if there is no head?

%• Wie verändert sich Köpfigkeit in der diachronen Entwicklung von Sprache?
\item Do we need the concept of ``head'' in grammatical theory?
\end{itemize}

The current paper addresses these questions. I start with an introduction of Head-Driven Phrase
Structure Grammar (HPSG, \citealt{ps,ps2,MuellerLehrbuch3}) in Section~\ref{sec-koepfe-in-hpsg}; there, I
explain how lexical heads determine the internal structure and external distribution of phrases.
Following the introduction of the basic machinery in Section~\ref{sec-koepfe-in-hpsg}, I step back a bit
and discuss more general, theory-neutral criteria for an element being a head in
Section~\ref{sec-np-dp}. \citegen{Zwicky85a} criteria for being a head are applied in the notorious
DP/NP debate. \ili{German} data shows that most of the criteria deliver inconclusive results, but some seem
to argue for N as the head. After comparing the complexity of NP and DP
structures and discussing the assignment of semantic roles in nominalizations, selection, and idioms, I argue for assuming N
as the head in nominal structures.

Section~\ref{sec-unsichtbar} deals with the question of how to deal with structures in which there
is no visible head. Again, I discuss nominal structures and show how nounless nominal structures can
be described by assuming an empty nominal head. Furthermore, I explain the analysis of predicative
structures in African American Vernacular English\il{English!African American Vernacular} and why the assumption of an empty head was
suggested in \citew[Section~15.3.5]{SWB2003a}.

Frameworks like Construction Grammar reject empty elements dogmatically \parencites[\page
219]{Goldberg2003b}[\page 10]{Goldberg2006a}[\page 3]{HT2013a-u}[\page 112]{Fillmore2013a-u}[\page
134]{Michaelis2013a-u} since they are said to be unacquirable. I show in
Section~\ref{sec-grammatikkonversion} that grammars with empty elements may be transformed into
grammars without empty elements, and I argue that the NP grammar with empty nominal heads is in fact
easier to learn than the grammar without empty elements, since it captures the facts about omissible
elements directly. 

Apart from nominal structures in which we have a head but it is invisible, there are other
structures in which it is impossible to identify one central element that determines the structure
of the whole unit and where the stipulation of an empty head cannot be motivated by anything theory-external. 
Section~\ref{sec-kopflos} shows how such phrases can be analyzed and why they are
unproblematic for HPSG even though the theory has ``head-driven'' in its name, which seems to
suggest that all structures have to have a head. 

Section~\ref{sec-lang-acqui} discusses language acquisition and provides criteria for when the
assumption of empty elements is appropriate. Section~\ref{sec-summary} provides a summary of the paper.


\section{Heads and HPSG}
\label{sec-koepfe-in-hpsg}
        
The notion of head is crucial for Head-Driven Phrase Structure Grammar: most phrases in grammars
have a central element that is responsible for the internal structure of the phrase and for its
distribution. For example, prepositions determine the case of the NP they combine with:
\eal
\ex[]{
\gll  zu diesem Termin\\
      to this.\dat{} appointment\\
}
\ex[*]{
\gll  zu diesen Termin\\
      to this.\acc{} appointment\\
}
\zl
The form of the preposition in prepositional objects is important, since it is responsible for the
external distribution of the whole phrase: while an \emph{auf} PP can function as the object of
\emph{warten} `to wait', an \emph{an} PP cannot:
\eal
\ex[]{
\gll Ich warte auf den Mann.\\
     I   wait  on  the man\\
\glt `I am waiting for the man.'
}
\ex[*]{
\gll Ich warte an den Mann.\\
     I wait    at the man\\
}
\zl
It is clear for prepositions, verbs, and adjectives that they have valence and that their
form and/or inflectional properties are responsible for the distribution of the whole phrase. The problematic
cases (determiner and noun) and respective criteria for head status are discussed in Section~\ref{sec-nominalstrukturen}.

I turn now to the foundational assumptions of HPSG (treatment of valence and percolation of
head information) in order to be able to explain headless structures with reference to these more
common structures.

In HPSG, valence information is expressed by means of lists. For example, valence lists of two-place
verbs contain two elements (in \ili{German}).\footnote{
  It is commonly assumed that finite verbs in OV languages select all their arguments in one valence
  list \parencites[\page 295--296]{Pollard90a-Eng}[Section~3.1.1]{Kiss95a}[]{Mueller2002b}, while there are two valence lists for SVO languages like \ili{English}: one list for preverbal
  arguments (specifiers) and one for post-verbal arguments (complements;
  \citealp*[Section~4.3]{SWB2003a}; \citealp[Section~4.3]{MuellerGermanic}).
}
One of these arguments gets the nominative and the other one the accusative case. 

I assume binary branching structures for \ili{German}, as most authors working on \ili{German} in HPSG do (see for instance
\citealt{HN94a,Kiss95a,Meurers99a,Mueller99a,Kathol2000a,Holler-Feldhaus2001a}). Which argument is
combined with the head is not constrained, so the head can combine with the nominative or with the accusative
first. The argument that is not combined with the head is passed up in the
tree. Figure~\ref{abb-niemand-ihn-kennt} shows this for the example in (\mex{1}):
\ea
\gll {}[dass] niemand ihn kennt\\
     \spacebr{}that nobody him knows\\
\glt `that nobody knows him'
\z
\begin{figure}
\centerfit{
\begin{forest}
sm edges
[{V \eliste}
  [{NP[\type{nom}]} [niemand;nobody] ]
  [{V \sliste{ NP[\type{nom}] } }
    [{NP[\type{acc}]} [ihn;him] ]
    [{V \sliste{ NP[\type{nom}], NP[\type{acc}] }} [kennt;knows]] ] ]
\end{forest}}
\caption{Analysis of \emph{niemand ihn kennt} `nobody knows him'}\label{abb-niemand-ihn-kennt}
\end{figure}

The verb \emph{kennt} `knows' requires one NP in the nominative and one in the accusative. After combining 
\emph{kennt} `knows' with the accusative object \emph{ihn} `him', one gets a linguistic object that requires
an NP in the nominative. If this linguistic object is combined with the nominative, an element with an
empty valence list results. Since the head of this linguistic object is a verb, the whole linguistic
object is a sentence.

Many theoretical papers discuss tree structures without providing the rules that actually license the
trees. HPSG uses abstract dominance schemata to license linguistic objects. A
representation of such a schema is shown in Figure~\ref{abb-spr-h}.
\begin{figure}
\hfill\begin{forest}
[{H[\spr \eliste ]}
  [\ibox{1}]
  [{H[\spr \sliste{ \ibox{1} }]}]]
\end{forest}
\hfill\mbox{}
\caption{Visualization of the  Specifier-Head Schema}\label{abb-spr-h}
\end{figure}
%% \hfill
%% \scalebox{.7}{%
%% \begin{forest}
%% [{H[\spr \ibox{1}, \comps \ibox{2}]}
%%   [\textsc{mod} \ibox{3}]
%%   [{\ibox{3} H[\spr \ibox{1}, \comps  \ibox{2}]}]]
%% \end{forest}}
This treelet shows how heads can be combined with an element of their specifier list (\spr stands
for specifier). Usually the \sprl of a head contains exactly one element (the subject of SVO
languages and the determiner in NP structures\footnote{%
  Some authors assume an additional valence feature for subjects, namely \subj \parencites{Borsley87a}[Chapter~9]{ps2}. I assume a head feature
  \subj for control and raising. Subjects in SVO languages like \ili{English} and the Scandinavian
  languages are treated as specifiers \citep{MuellerGermanic}. There are analyses of the NP assuming
  that the determiner is a marker of the head rather than a dependent selected via valence features
  \citep{VanEynde2006b,Allegranza2007a-u,Sag2012a}. See also \citew{VanEynde2021a} for an
  overview of alternative approaches to nominal structures within HPSG. 
  The marker-based approaches provide a simple analysis of determinerless nominal structures
  \citep[\page 167, 174--175]{VanEynde2006b}, but the syntactic simplicity of syntactic structures
  comes at a price: the resulting structures are missing the quantifier usually contributed by the
  determiner. The only solution to the problem I am aware of was suggested by
  \citet{Allegranza98a-u}. \citet{Allegranza98a-u} suggests an analysis in which nouns that may appear
  without determiner (plurals and mass nouns) introduce a quantifier lexically. If these nouns are
  used without a determiner or a quantifier, the lexically introduced quantifier is used. In all
  other cases the lexically introduced quantifier is removed (p.\,103). The solution involves
  disjunctions and subtraction operations over sets and is rather complex. Furthermore, Allegranza's account fails
  on examples like \emph{alledged water}, since in his setup the quantifier scopes over \emph{water}
  directly. While this probably can be fixed, any imaginable solution is probably not simpler than
  what was suggested so far and hence I prefer the approach described in this paper.

In addition, marker-based approaches do not capture the parallelism between verbs and nouns in nominalizations like (i):
\eal
\ex Caesar destroyed the city.
\ex Caesar's destruction of the city
\zl
In the approach assumed here, both the subject \emph{Caesar} and the determiner phrase \emph{Caesar's} will be selected by the
verb and by the noun derived from the verb, respectively. The heads will assign semantic roles to
the selected element \citep{MyPM2021a}. In the marker-based analysis
\emph{Caesar's} would select \emph{destruction of the city} despite the fact that it fills a
semantic role of \emph{destruction} (Frank Van Eynde, p.c. 2019). I prefer the more uniform analysis of
the examples in (i). Nominalizations will be discussed further in Section~\ref{sec-relational-nouns-and-semantic-roles}.
%\inlinetodo{check}
%A further shortcoming of the marker approaches is   
}, \ibox{1} in Figure~\ref{abb-spr-h}), which means that the \sprl of the mother node is the empty
list.\footnote{%
  See \citet{MOe2013b} for an analysis of object shift in \ili{Danish} in which objects appearing to the
  left of the verb (like subjects) are treated as specifiers. In this analysis the \sprl may contain
  more than one element. See also \citew[Sections~5.3, 6.3]{Ng97a} for an analysis of nominal
structures in \ili{English} and \ili{Chinese} with multiple specifiers and \citew{WL2007a-u} for such an
analysis of \ili{Chinese} nominal structures.
}

Heads are marked by H in the figures. This is supposed to indicate that all head information, that
is, information that is relevant for the distribution of the phrase, is present at both the head
daughter and the mother. For sentences, this would be the information that the part of speech of the
head is \type{verb} and whether the verb is finite, a participle, or an infinitive with or without
\emph{to}. Of course other information 
about required arguments, extracted arguments and adjuncts, and relative
pronouns within a phrase, among other things, are also relevant for the external distribution of a phrase. This information is part
of the complex categories that are assumed in HPSG. The head information is the information
that is directly shared between lexical heads and intermediate and maximal projections of the lexical head.

The trees we saw so far are convenient for visualization, but HPSG uses typed feature value structures to
model all aspects of linguistic objects; even the internal configuration of complex syntactic
objects is represented by feature value pairs. (\mex{1}) shows how the Specifier-Head Schema can be
described with feature value pairs.\footnote{ 
  Even though the schema is more formal than the little treelets, it is still a simplification in that not
  all feature-value paths are fully specified as they would have to be according to the
  theory. (\mex{1}) leaves out the paths leading to \spr and a path in the list of daughters. For
  details see \citew[\page 30]{GSag2000a-u} and \citew{MuellerCurrentApproaches}.%
}
\ea
\label{specifier-head-phrase}
Specifier-Head Schema:\\*
\type{specifier-head-phrase} \impl\\*
\ms{
spr      & \eliste\\
head-dtr & \ibox{1} \ms{ spr \sliste{ \ibox{2} } }\\
dtrs     & \sliste{ \ibox{1}, \ibox{2} }\\
}
\z
The symbol \impl{} stands for a logical implication: if a feature structure is of type
\type{specifier-head-phrase}, the restrictions on the right side of the implication have to
hold. Types and implicational constraints are discussed further below.
The daughters in a tree are represented in a list, which is the value of the \textsc{daughters}
feature \citep[\page 30]{GSag2000a-u}. In the case at hand, we have two daughters: \ibox{1} and
\ibox{2}. The daughter \ibox{1} is the head daughter. In addition to being in the \dtrs list, it is
identified with the value of the feature \textsc{head-dtr}. The \sprl contains a description of the
other daughter \iboxb{2}. Figure~\ref{fig-spec-head-phrase} shows the tree representation for
structures licensed by the schema in (\mex{0}). \emph{Mannes} `man' selects a determiner. 
\begin{figure}
\begin{forest}
sm edges
[N{[\spr \eliste ]}
  [\ibox{1} Det [des;the.\gen]]
  [N{[\spr \sliste{ \ibox{1} } ]} [Mannes;man.\gen]]]
\end{forest}
\caption{Specifier"=Head structure}\label{fig-spec-head-phrase}
\end{figure}


Verbs project information about part of speech and inflection. The part of speech and case
information determines the distribution of nominal projections. HPSG groups information that belongs
together into one attribute value matrix (AVM). Part of speech information and case information
form the value of \head in (\mex{1}).\footnote{%
  Of course other properties like number, gender, and declension class are relevant for the distribution as
  well. Some authors bundle case, person, number, gender, and declension class as agreement features inside of \head
  \citep[\page 262]{Kathol99b}, and others refer to the number and gender information contained in the semantic
  index contributed by nouns \parencites[Section~2.5.1]{ps2}[Section~13.2]{MuellerLehrbuch1}. I omit
  declension class here, since it is not relevant for the current discussion.
} \type{noun} is the type of the feature description. Feature
structures of type \type{noun} always have a \textsc{case} feature. This feature may have the values \type{nom}, \type{gen},
\type{dat}, or \type{acc} in \ili{German}. For \emph{Frau} `woman', we may leave the value underspecified, since
\emph{Frau} is compatible with any of the four cases in \ili{German}, but for \emph{Mannes} `man', which is
in the genitive, it has to be \type{gen}.
\ea
\label{le-mannes}
\ms{
phon & \phonliste{ Mannes }\\
head & \ms[noun]{
       case & gen\\
       }\\
spr  & \sliste{ \normalfont Det }\\
comps & \sliste{ }\\
}
\z
Since HPSG allows values of features to be internally complex, features that have to be projected
from lexical items can be grouped together and the projection of head
features can be set up in a general way: all information that is present under \head is shared
between head daughter and mother, that is, the information in the description of the head daughter
is identical to the respective information at the mother. Figure~\ref{abb-spec-head} shows the
analysis of the nominal phrase \emph{des Mannes} `the.\gen man.\gen'.
The information concerning part of speech and case of the noun is shared with the respective
information for the whole phrase.
\begin{figure}
\begin{forest}
sm edges
[\ms{ head & \ibox{1} }
  [Det [des;the.\gen]]
  [ \ms{ head & \ibox{1} \ms[noun]{ case & gen }}  [Mannes;man.\gen]]]
\end{forest}
\caption{Specifier"=Head structure}\label{abb-spec-head}
\end{figure}

Apart from Specifier"=Head structures, there are also Head-Complement structures. These play a role
in the combination of relational nouns with their complements. Figure~\ref{abb-head-comp} gives an example.

\begin{figure}
\begin{forest}
sm edges, for tree={base=top}
[\ms{\textsc{head} \ibox{1}\\
     \comps \sliste{}}
  [\ms{\textsc{head} \ibox{1} \type{noun}\\
       \comps        \sliste{ \ibox{2} }}  [Eroberung;conquest]]
  [{\ibox{2} NP[\type{gen}]} [der Stadt;of.the city, roof]]]
\end{forest}
\caption{Head-Complement structure}\label{abb-head-comp}
\end{figure}

\largerpage[2]
Head-Complement phrases are parallel to Specifier-Head phrases. The only difference is that
complements are selected via another valence feature (\comps rather than \spr). The schema for
head-complement combinations that is parallel to Figure~\ref{abb-spr-h} is shown in Figure~\ref{abb-kopf-comp}.
\begin{figure}
\hfill
\begin{forest}
[{H[\comps \ibox{1}]}
  [{H[\comps  \sliste{ \ibox{2} } $\oplus$ \ibox{1}  ]}]
  [\ibox{2}]]
\end{forest}
\hfill\mbox{}
\caption{Visualization of the Head-Complement Schema}\label{abb-kopf-comp}
\end{figure}
As with Specifier-Head phrases, the valence list is split into two parts: one list with exactly one element \sliste{
  \ibox{2} } and another list with the rest \iboxb{1}. \ibox{2} is identified with the other
daughter, and \ibox{1}, the list containing the rest, is identified with the \compsv of the mother
node. The combination of a head with its complements as it is given in Figure~\ref{abb-kopf-comp}
combines the head with the first element in the valence list. This is exactly what we find in SOV
languages. For SOV languages and languages with scrambling see
\cites[Sections~9.1.1, 9.4]{MuellerGT-Eng4}[Sections~3, 4]{MuellerOrder}. The order of combination differs from the one in Specifier-Head
structures, in which the head is combined with the last element in the \sprl first. Again see
\citew{MOe2013b} for details.

\largerpage[2]
(\mex{1}) shows the schema that licenses Head-Complement phrases:
\ea
\label{head-complement-phrase}
Head-Complement Schema:\\
\type{head-complement-phrase} \impl\\*
\ms{
comps & \ibox{1}\\
head-dtr & \ibox{2} \ms{ comps \ibox{1} $\oplus$ \sliste{ \ibox{3} } }\\
dtrs     & \sliste{ \ibox{2}, \ibox{3} }\\
}
\z
Apart from the schemata introduced so far, HPSG has schemata for head"=adjunct combinations and for
nonlocal dependencies (for less general schemata see Section~\ref{sec-kopflos}). The two schemata
above are sufficient to be able to explain the assumptions about heads and headedness made in
HPSG.\footnote{%
  A reviewer asked how agreement between determiners, adjectives, and nouns can be accounted for in
  HPSG. These items agree in case, number, gender, and match in declension class. Agreement is usually
  analyzed using structure sharing of features of items that select others/are selected by
  others. Since nouns select their determiners, agreement between nouns and determiners can be
  assured. Similarly, adjectives select nouns, and hence adjective-noun agreement can be taken
  care of. Since nouns agree with their determiners, the agreement between all three elements is
  accounted for. See \citew[Section~2.5.1]{ps2} and \citew[Section~13.2]{MuellerLehrbuch1} for
  worked-out proposals for agreement in \ili{German} noun phrases and \citew{WZ2003a} on agreement in HPSG in general.
}

%\begin{sloppypar}
All feature structures in HPSG have to be of a certain type. These types are organized in hierarchies.
All feature structures modeling linguistic signs are of type \type{sign}. Linguistic signs are
divided into phrases and words. For these objects, we have the types \type{phrase} and
\type{word}. Phrases can be categorized into phrases that have a head (\type{headed-phrase}) and
phrases without a head (\type{non-headed-phrase}). \type{specifier-head-phrase} and
\type{head-complement-phrase} are subtypes of the type \type{head\-ed-phrase}.
%\end{sloppypar}

We want to say the following about structures: if there is a head in the structure, then the head
features of the head daughter have to be identical to the head features of the mother. HPSG allows
for an elegant expression of this fact using an implicational constraint:\footnote{%
  An alternative formulation of the Head Feature Principle, the so-called Generalized Head Feature
  Principle, is suggested by \citet[\page 33]{GSag2000a-u}. They suggest that all syntactic and semantic
  information of the head daughter is shared with the information of the mother by default. As
  (\ref{specifier-head-phrase}) and (\ref{head-complement-phrase}) show,
  the valence information at mother nodes differs from the valence information at the head daughter
  in Specifier-Head phrases and in Head-Complement phrases. The same is true for semantic
  information: usually the semantic information at the mother node differs from the information at
  the head daughter, since the mother node has a collection of the semantic contributions of all
  daughters. This is captured by Ginzburg \& Sag because sharing all information is a default that
  is overwritten in subtypes of \type{headed-phrase}. Defaults are often used in linguistics to
  describe unmarked cases, but what the Generalized Head Feature Principle sets as a default never actually holds. In fact,
  there is not a single structure in any HPSG theory I am aware of in which all syntactic and
  semantic features of head daughter and mother are identical. So the Generalized Head Feature
  Principle is not a generalization. It is never true, and hence I do not use it. 
}
\ea
\type{headed-phrase} \impl \ms{ head & \ibox{1}\\
                                head-dtr & \ms{ head & \ibox{1} }
                                }
\z
(\mex{0}) specifies a constraint that holds for all feature structures of type \type{headed-phrase},
including those that are subtypes of \type{headed-phrase}. The constraint identifies the head
features of the head daughter with the head features of the mother.

The fact that we have an implication in (\mex{0}) cannot be emphasized enough. This means that the
conclusion has to hold only if the antecedent is true. If the antecedent is false, nothing is said
about the presence of head daughters or the values of head features. This means that one can assume
headless structures in HPSG, and there are plenty of examples of headless constructions in the
literature \parencites{Mueller99b}[Chapter~10]{Mueller99a}. Hence it would be wrong to claim that HPSG assumes that all
structures must be headed. I will return to headless constructions in Section~\ref{sec-kopflos}.

\largerpage
Before turning to such truly headless constructions, in the following section I want to discuss
nominal structures, which are interesting for two reasons. For one, researchers still disagree as to
which element in a nominal structure is the head. And for another, both determiner and noun may be
omitted in \ili{German}, which means that nominal structures could be problematic for linguistic theories
in general.  


\section{Nominal structures}
\label{sec-np-dp}
\label{sec-nominalstrukturen}

Since the 1970s there have been proposals to treat the determiner as the head of nominal structures.
Such proposals became popular within the framework of GB \citep{Chomsky81a} but are entertained in
other frameworks as well: there are proposals in Categorial Grammar, LFG, HPSG, and Dependency
Grammar that assume the determiner to be the head. See \textcites[\page 6]{Ajdukiewicz35a-u}{VH77a-u,
%Lyons77a-u, 392
Brame82a}[\page 90--92]{Hudson84a-u}{Hellan86a,Abney87a,Netter94,Netter98a-Eng,vanLangendonck94a,Salzmann2020a,chapters/salzmann} for DP
proposals and \textcites[\page 49]{ps2,Demske2001a}[Section~6.6.1]{MuellerLehrbuch1}{Bruening2009a,Bruening2020a}
for NP proposals in various frameworks. \citet{Hudson2004a} working in Word Grammar, a version of
Dependency Grammar, suggests mutual dependency between determiner and noun.\footnote{%
  Due to space limitations, it is not possible to go into the details of a comparison, but such mutual
  dependencies are also assumed in HPSG: the noun selects the determiner via the valence feature
  \spr and the determiner selects the noun via the feature \textsc{specified} \citep[\page 50]{ps2}.
}

\subsection{Tests for head status}

I talked about prepositions, verbs, and adjectives
at the beginning of the previous section. It is clear that these categories are
heads of their respective phrasal units. This begs the question whether there are criteria for headedness
that could help deciding the question for nominal structures. \citet{Zwicky85a} looked at tests for
headhood in the 80s more carefully. The tests will be repeated below and it will be examined
whether they are useful in the DP/NP debate.

\subsubsection{The subcategorizand}

\largerpage
\citet[Section~2.1.2]{Zwicky85a} states that the subcategorizand is likely to be the head. The
subcategorizand is the lexical element, in contrast to the phrasal one(s), and it may appear in certain configurations. For instance,
the verb \emph{give} can appear with two NP arguments (e.g.\ in \emph{give her a book}) or with an NP and a PP as arguments (e.g.\ in \emph{give a book to her}). 
On the other hand, \emph{donate} is restricted to NP and PP. 
In both cases, the lexical element (the verb) is the head of the respective phrase.
For nominal structures, Zwicky argues that Det must be the subcategorizand, since the Det is the sole lexical
element in Det-\nbar combinations, and hence the determiner is the only plausible candidate for a
lexical head.
%% Zwicky begrüßt dieses Ergebnis, weil
%% seiner Meinung nach Determinatoren lexikalisch subkategorisiert dafür sind, ob sie mit Singular-
%% oder Pluralnomina bzw. Stoff- oder Massennomina vorkommen:
%% \eal
%% \ex each penguin/*penguins/*sand  % jeder Pinguin
%% \ex many *penguin/penguins/*sand  % 
%% \ex much *penguin/*penguins/sand
%% \zl
Unfortunately, he missed the fact that the determiner may be complex both in \ili{English} and other
languages as well:\footnote{%
  See \citew[\page 51--54]{ps2} and \citew[\page 193]{GSag2000a-u} for analyses of \emph{'s} as determiner and
  of complex prenominal phrases as determiner phrases.
}
\newpage

\eal
\ex the Queen of England's son
\ex 
\gll unter des Körpersportlers Haut\footnotemark\\
     below the body.sportsman's skin\\
\glt `below the body builder's skin'
\footnotetext{
taz [\ili{German} newspaper], 1995-01-04, p.\,15, quoted from \citew[\page 59]{Mueller99a}}
\zl
Since both the determiner and the \nbar can be phrasal, this test does not really help here.

\subsubsection{The morphosyntactic locus}

A further test discussed by \citet[Section~2.1.3]{Zwicky85a} is the test for the morphosyntactic
locus. The inflectional features are located at the noun in \ili{English}:
\eal
\ex the child
\ex the children
\zl
However, this test does not help for \ili{German}, since determiners are inflected as well:
\eal
\ex 
\gll das / dieses Kind\\
     the.\SG.\N{} {} this.\SG.\N{} child(\N)\\
\ex 
\gll die / diese Kinder\\
     the.\PL{} {} these.\PL{} children.\PL\\
\zl

\subsubsection{Determinant of concord}

\citet[Section~2.2.2]{Zwicky85a} looks at the element that determines concord within a
phrase. Sometimes it is claimed that the determiner is responsible for the inflection class of the
adjective. The determiners in (\mex{1}) have a fixed inflection class and the other elements in the
nominal structure have to be appropriate for the respective class with it:
\eal
\ex 
\gll ein kluger Mann\\
     a   smart  man\\
\ex
\gll der kluge  Mann\\
     the smart  man\\
\zl
% % Zu schwach und zu kompliziert.
% und das Papier ist eh zu lang.
% The situation is not that clear since there are nouns like \emph{Gesandter} `envoy',
% \emph{Verwandter} `relative', and \emph{Bekannter} `acquaintance' that also inflect according to the declension class:
% \eal
% \ex 
% \gll ein kluger Gesandter\\
%      a   smart  envoy\\
% \ex 
% \gll der kluge Gesandte\\
%      the smart envoy\\
% \zl
% So at least in these examples it could be argued that the noun is the important element.

However, it is equally possible to argue the other way round, and \citet[\page 9]{Zwicky85a} does exactly
this: gender is an inherent property of most nouns and the determiner has to match the gender of the noun:
\eal
\ex 
\gll der Mann\\
     the.\M{} man(\M)\\
\ex 
\gll die Frau\\
     the.\F{} woman(\F)\\
\zl
This suggests that the noun is the head in nominal structures. Therefore we can conclude that this test
fails as well for \ili{German}: sometimes the determiner, sometimes the noun determines concord.



\subsubsection{Semantic functor}

A further criterion suggested by \citet[Section~2.1.1]{Zwicky85a} is the one of the semantic
selector. Unfortunately, this criterion does not really decide the issue either. It is true -- as
Zwicky notes on page~4 -- that for instance the universal quantifier selects the semantic contribution
of the nominal part and incorporates it into the complete formula. The nominal part \emph{Frauen}
`women' corresponds to the Q in (\mex{1}b):
\eal
\ex 
\gll alle Frauen\\
     all  women\\
\ex $\lambda Q (\lambda P(\forall x(Q(x) \rightarrow P(x))))$
\zl
But on the other hand, we have relational nouns like \emph{conquest} whose arguments may be realized in the position of
the determiner. The meaning representation of (\mex{1}a) has to contain (\mex{1}b) somewhere:
\eal
\ex 
\gll Peters Eroberung der Stadt\\
     Peter's conquest of.the town\\
\ex conquest(Peter, town)
\zl 
This means that this criterion is not reliable either. The determiner embeds the semantic
contribution of the remaining nominal group, and the remaining nominal group may embed parts
contributed by the determiner.

\subsubsection{The distributional equivalent}

\citet[\page 12]{Zwicky85a} states that the noun is the distributional equivalent of the whole
phrase, including the determiner. Proper names like \emph{Kim} and plural nouns like \emph{penguins}
can be used instead of \emph{the penguins}. 

% If the distributional equivalent criterion is taken as basis, \citegen[\page 219]{Postal69a-u} examples in (\mex{1}) suggest that the prenominal element is the head:
% \eal
% \ex we sailors
% \ex you troops
% \ex us three men
% \zl
% I would argue that these constructions should be treated independently of the DP/NP debate (see Section~\ref{sec-pronoun-noun-combinations}).

As a reviewer pointed out, the criterion is a rather odd one since it could not be
applied to all heads that obligatorily require arguments. Examples are prepositions in \ili{German} and
verbs like \emph{devour} in \ili{English}. Since a single preposition cannot be used anywhere without its
NP argument, the preposition is not distributionally equivalent to the PP and hence would not
qualify as the head. Clearly an unwanted result.

%%
%% Das macht die Analyse mit leerem Nomen schwieriger ....


\subsubsection{Obligatoriness}
\label{sec-obligatheit}

Both the determiner (\mex{1}a) and the noun (\mex{1}b) may be omitted in \ili{German}. It is even possible to omit both of them,
as (\mex{1}c) shows:\footnote{
  The adjectives in (\mex{1}) could be nominalizations, but I am talking about elliptical
  constructions here. Nominalizations would be written with capital letters.
}
\eal
\ex 
\gll Er hilft Frauen.\\
     he helps women\\
\glt `He helps women.'
\ex 
\gll Er hilft den klugen.\\
     he helps the smart\\
\glt `He helps the smart ones.'
\ex 
\gll Er hilft klugen.\\
     he helps smart\\
\glt `He helps smart ones.'
\zl
As the translation of the examples shows, the pronoun \emph{one} is used in the parallel \ili{English}
structures. However, \ili{English} also permits nominal structures without a visible noun \parencites[\page
  13]{Zwicky85a,AS2015a}. Zwicky notes that structures with omitted noun are always elliptical. This means
that nouns are obligatorily present, and if they are missing, their omission is due to ellipsis.
%\todostefan{Abgrenzung Ellipse}
So, if this criterion is accepted, it decides in favor of N as the head.

A reviewer pointed out that there are certain cases in which the nominal part is optional, but when
it does not occur, this is not due to ellipsis. Examples are \emph{this}, \emph{that}, \emph{we}, and
\emph{you}:
\eal
\ex this man
\ex we sailors
\zl
I think that a double categorization of these elements as determiners and full NPs
is justified. So \emph{we} would be a full DP/NP if it is used without other material and something
different in constructions like \emph{we sailors}. See Section~\ref{sec-pronoun-noun-combinations} for more on pronoun-noun combinations.
% \eal
% \ex[]{
% \gll Ich helfen den Männern.\\
%      I help the men\\
% }
% \ex[*]{
% \gll Ich helfe den.\\
%      I   help the\\
% }
% \ex[]{
% \gll Ich helfe denen.\\
%      I   help those\\
% }
% \zl
In any case, examples like the ones just mentioned show that the criterion cannot be applied without
further qualifications.



\subsubsection{Language acquisition, uniformity, and Poverty of the Stimulus}

Abney's dissertation \citep{Abney87a} made the DP analysis very popular within the generative world. Abney argued for
a treatment of \ili{English} nominal structures that is parallel to the structures assumed for the
sentential domain. Many authors assume an IP/VP analysis for \ili{English}, not just those working in
Mainstream Generative Grammar (MGG), but also in LFG \citep*[\page 102]{BATW2015a}. In such analyses, there is a functional I
projection in addition to verbal projections. An advantage of the DP analysis is that one has parts
of the theory that are similar, and hence one can claim to have found deeper laws. Apart from this,
language acquisition is used as an argument in the DP/NP discussion: Chomsky still believes that
language cannot be learned from input alone \citep*{BPYC2011a}.\footnote{%
The authors discuss auxiliary inversion. See \citew[Section~13.8.2.4]{MuellerGT-Eng4} and \citew[Section~1]{Sag2020a} for a
critical discussion of these claims.%
} 
Since -- according to Chomsky -- language is acquired despite this Poverty of the Stimulus, there must be innate
language-specific knowledge which helps us to acquire language from the input that is available.
The claim was that children can acquire language because all phrases have the same internal
structure and knowledge about this structure is innate and therefore helps to acquire language
\citep[\page 106]{Haegeman94a-u}.\footnote{%
  \citet[\page 106]{Haegeman94a-u} states that ``the principles of X$'$ theory will be part of UG,
  they are innate. The ordering constraints found in natural languages vary cross-linguistically and
  thus have to be learned by the child through exposure. Very little data will suffice to allow the
  child to fix the ordering constraints of the language he is learning. A child learning \ili{English}
  will only need to be exposed to a couple of transitive sentences to realize that in \ili{English} verbs
  precede their complements.''
} In particular, nominal structures have a DP structure which is parallel to the IP/VP structures.

\citet[739]{Fodor2001b} points out that the situation is not as simple if movement to places in
otherwise invisible structure is possible. For example, it is not obvious in some cases whether
verbs are in V, I, or C, whether a language is V2 or not, or whether we have an SVO or an SOV
language, as the following three examples from \ili{English}, \ili{Danish}, and \ili{German} illustrate
\citep[Section~6.2.2, Figure~6.11]{MuellerGermanic}:
% \eal
% \ex {}[\sub{S} Conny [\sub{VP} reads [\sub{NP} a book]]].
% \ex {}[\sub{S} Conny$_i$ [\sub{S/NP} læser$_k$ [\sub{S/NP} \_$_i$ [\sub{VP}  \_$_k$ [\sub{NP} en bog]]]].
% \ex {}[\sub{S} Conny$_i$ [\sub{S/NP} liest$_k$ [\sub{S/NP} \_$_i$ [\sub{\vbar} [\sub{NP} ein Buch] \_$_k$]]]].
% \zl
\eal
\ex Conny reads a book. \hfill $-$V2, SVO
\ex Conny læser en bog. \hfill $+$V2, SVO
\ex Conny liest ein Buch.\hfill $+$V2, SOV
\zl
So, having simple transitive sentences in the input is not enough to decide. Sentences with
auxiliaries would help the linguist to decide between SOV and SVO languages, but this wouldn't help
to distinguish between $-$V2 and $+$V2; for this the linguist would need examples with fronted
objects, rather than subjects as in (\mex{0}). As Fodor
%\addpages the whole paper?
points out, there are many questions concerning
how language acquisition is supposed to work in a Principles \& Parameters setting. It is unclear
how a child can determine which way to set the parameters. Fodor suggests a model assuming innate treelets
that can be used in analyses of utterances and shows that this avoids problems of alternative
approaches. While this seems to be the most plausible approach with the Principles \& Parameters
framework, there are still serious issues (discussed by Fodor herself), and of course the overall
question is how information about treelets distinguishing between V2 and non-V2 languages are
supposed to make it into our genome \citep*{HCF2002a}. Assuming data-driven approaches without a rich
UG \citep{FPAG2007a} seems to be preferable.  
 
But let us assume for the sake of the argument that uniformity of basic \xbar structure
would help in language acquisition. Even with this assumption, there remains a problem with this argument, namely that many
researchers (from different frameworks) believe that the assumption of an IP structure is not plausible for \ili{German}
\citep{BK90a,Haider93a,Berman2003a}. If \ili{German} does not have an IP, it is not reasonable to assume
that the DP is parallel to the sentential domain and that constraints on both domains are part of
our innate linguistic knowledge.\footnote{%
  \citet{Haider92b} assumes a functional head in the \ili{German} clause that is not IP. In his DP approach
  he assumes a parallel between his FP (Functional Projection) and the DP. While there is a parallel in the
  functional/lexical structure of FP and DP, the makeup of the respective phrases in \ili{German} is quite
  different. The head of the FP is the place for the finite verb or a complementizer and SpecCP is
  the target for fronted constituents in V2 clauses.%
} Hence, uniformity of nominal and sentential domain is not an
argument for the DP analysis.

I want to close this section with a somewhat ironic remark. Although I do not believe in the
``parallelism in structure helps language acquisition'' argument, I want to point out that the NP
analysis suggested here is parallel to the analysis of the sentential domain assumed in HPSG:
auxiliaries are treated as verbs, not as Is or Ts \citep{Sag2020a}. The subject of verbs in SVO languages are treated
as specifiers, and so are determiners in nominal structures \citep[Sections~4.3--4.4]{MuellerGermanic},
so we have arrived at parallelism all the same.


\subsubsection{Pronoun-noun combinations, selection, agreement, and idioms}

\citegen{Zwicky85a} criteria for head status discussed so far are theory-neutral, as far as this is
possible. The DP analysis is considered the standard in Mainstream Generative Grammar and authors
usually refer to \citew{Abney87a} for a thorough argumentation for the DP analysis. However,
\citet{chapters/salzmann}, this volume, points out that all previous arguments for the DP analysis depend on theory-internal
assumptions. If these assumptions are not made, the arguments collapse. Since MGG changed
considerably since the 80s, none of the original arguments holds any longer. Salzmann suggests a new
argument based on agreement data from \ili{Bosnian}/""\ili{Croatian}/""\ili{Serbian} (BCS). In what follows, I want to discuss
three phenomena that seem to argue for DP analyses (pronoun-noun combinations, selection, and agreement) and one
controversial phenomenon that is very interesting in the DP/NP debate: idioms.

\subsubsubsection{Pronoun-noun combinations}
\label{sec-pronoun-noun-combinations}


Let us start with pronoun-noun combinations: 
% There seems to be a clear case suggesting that the initial element is the head: pronoun noun
% combinations.
\eal
\ex[]{ 
\gll Ich Idiot  habe mich gefreut.\\
     I   idiot  have myself been.glad\\
\glt `I idiot was glad.'
}
\ex[]{ 
\gll Du Idiot hast dich gefreut.\\
     you idiot   have yourself been.glad\\
\glt `You idiot were glad.'
}
\ex[]{ 
\gll Wir Idioten haben uns gefreut.\\
     we idiots   have ourselves been.glad\\
\glt `We idiots were glad.'
}
\ex[*]{
\label{ex-er-idiot}
\gll Er Idiot hat sich gefreut.\\
     he idiot has himself been.glad\\
\glt Intended: `He was glad and he is an idiot.'
}
\zl
The examples in (\mex{0}) show that the pronoun agrees with the verb in person and number, while the
noun together with a determiner is always third person.
 (\ref{ex-er-idiot}) shows that third person pronouns are not possible in
this construction, so there is something idiosyncratic about it. \citet[\page 139--140]{Simon2003a-u} and
references cited there see data like this as evidence for the DP analysis, but I think this construction should
not be treated as an instance of the normal NP or DP construction. 
% weil dir Idiot dabei keiner helfen wird
Note also that some languages have this construction and combine the pronoun with a full nominal
projection \citep[Section~5.3]{Hoehn2016a-u}. In these languages, one would not say that a D head
selects an NP, but the pronoun would have to select a full DP. This is actually the solution suggested by \citet[\page
568]{Hoehn2016a-u}: he assumes a PersP with the pronoun as head selecting a DP.\footnote{%
  Note that this begs the question why governing heads selecting for DPs can take PersPs as well.
} So, it seems
reasonable to treat the pronoun as a head, but the whole construction should not be decisive in the DP/NP discussion.

% % Das ist Quatsch, denn das geht genauso ohne das Nomen. Durch das Nomen wird nur die Referenz klar
% % und dann muss das Adverbial passen. 
% % Apart from this more general issue, there is evidence that the gender of the pronoun noun
% % combination is determined by the noun:
% % \eal
% % \ex Wir Idioten sind dann einer nach dem anderen ausgestiegen.
% % \ex Wir Idiotinnen sind dann eine nach der anderen ausgestiegen.
% % \zl
% % The inflectional paradigm of adjectives does not distinguish 

\subsubsubsection{Selection}

\citet[Section~\ref{salzmann:sec-categorial-selection}]{chapters/salzmann}, this volume, mentions the fact that incorporation seems to
require selection of nominal structures without determiners (NPs), while otherwise, verbs select
nominal structures with determiners (DPs). For this to have any force as an argument for DP, one
needs the assumption that only maximal projections can be selected. However, this assumption is not
made in HPSG. For example, partial verb phrase fronting is explained by assuming that non-maximal
verbal projections may be combined with governing heads (Müller \citeyear{Mueller96a};
\citeyear[Section~2.2.2]{Mueller2002b}; \citealp{Meurers99a}). And once non-maximal projections can
be combined with heads, we can have heads combining with bare nouns, \nbars, and NPs, and hence
there is no argument for DP. See for example \citet[\page 632]{MuellerPersian} for the suggestion
that light verbs in Persian may combine with lexical nouns.

Salzmann also states that it is impossible to ``select the absence of structure'' \parencites[28, 32]{Salzmann2020a}[Section~\ref{salzmann:sec-agreement}]{chapters/salzmann}, but this does not apply to HPSG. Since \nbar is defined as a nominal projection without a
specifier, one can select for something with absent structure. Furthermore, material that is
combined contributes to the properties of a complex category. The respective contributions can be
selected for. This is independent of the question at which projection level the respective
combination takes place. See for example the use of the marking feature in \citew[\page 45--46]{ps2} or \citew{VanEynde2006b}. 



\subsubsubsection{Agreement}

\citet[Section~4.3]{Salzmann2020a} discusses agreement patterns from
\ili{Bosnian}/""\ili{Croatian}/""\ili{Serbian} (BCS) and argues that they show that there has to be a DP layer over an
NP layer to get the facts right. While he is very careful to show that \citegen{Abney87a} arguments
for the DP analysis are theory-internal, the same holds for Salzmann's new argument: agreement is
established via the Probe/Goal mechanism of Minimalism \citep{Chomsky2001a-u}. This crucially relies
on c-command and the proximity of agreement source and agreement target. In comparison to this,
agreement is dealt with differently in frameworks like HPSG: the main expressive tool is structure
sharing. It is not assumed that there is an agreement source and an agreement target
\citep[Section~2.2]{ps2}, but instead, both exponents of morpho-syntactic features are treated alike
and the information on both sides is simply identified
\parencites[Chapter~2]{ps2}{Kathol99b}{WZ2003a}{VanEynde2021a}. Salzmann points out that there are
two types of agreement in BCS: adjectives and determiners may agree in grammatical and semantic
gender. If a certain head agrees in semantic gender, the next higher head cannot go back to
grammatical gender. Salzmann concludes from this that there must be a DP layer, since in terms of
c-command the determiner would be the highest head and its features would be decisive for agreement
with elements outside of the nominal structure (p.\,38). But this conclusion is dependent on many
theory-internal assumptions. \citet[Section~4]{Bruening2020a}, working in the same framework and
assuming an NP approach, developed an alternative theory of the agreement facts.

% \citet{Zlatic2014a} assumes an NP analysis of
% nominal structures in \ili{Serbian}. Adjectives attach to \nbar and determiners are categorized as
% adjectives as well and hence also attach to \nbar. The nominal projection is closed off by a unary
% projection projecting \nbar to NP. It is unclear how the analysis of agreement can be integrated
% into the general approach to agreement suggested by \citew{WZ2003a},\footnote{%
%   \citet{Malouf2000b-u} suggested an analysis of case-stacking languages in HPSG assuming a list of case
%   features. Something along this line could also work for the BCS agreement puzzle.%
% } but there is this final unary
% projection and whatever a determiner in a DP projection could contribute can be contributed in this
% projection. 
So again, the argument that Salzmann suggests instead of Abney's arguments is also a
theory-internal one.\footnote{
  Salzmann's puzzle is not solved for HPSG yet. \citet{VanEynde2020b} discusses agreement in BCS but
  does not solve the problem of inaccessibility of one of the two agreement options. In his account
  both agreement options are always available. So further research and modification of the general theory of
  agreement is needed but it is not necessary to assume an agreement theory based on c-command.
}

\subsubsubsection{Idioms}

A very interesting argument comes from \citet[Section~5]{Bruening2020a}. Bruening argues that idioms make reference to
dependency chains. This was also suggested within Dependency Grammar
\parencites{OGrady98a-u}[Section~4.2]{OG2012a-u}. \citet[Section~4.2]{OG2012a-u}
argue for the importance of dependency relations in linguistic descriptions and explicitly claim
that all idioms are based on dependency chains. They assume that determiners are dependents of nouns
and explicitly state that idioms with fixed verbs, free nouns, and fixed determiners do not exist (p.\,180). 

If the Catena claim is correct, this is 100\% compatible with the NP analysis suggested
here. \citet{Sag2007a} and \citet*{KSF2015a} developed a local theory of idioms that is based on
selection. This is compatible with the claims made by \citet{Bruening2020a} and
\citet[Section~4.2]{OG2012a-u}.

\citet[\page 31]{Salzmann2020a} argues that examples like the ones in (\mex{1}) are counter-examples to the
dependency chain claim:
\ea
She plays the piano/trombone/flute.
\z
In the specific collocation at hand one has to use the definite determiner and the actual instrument
is open. Salzmann sees this as a data point that could be used to argue against the dependency chain
claim. I would argue, however, that the fact that an instrument has to be inserted shows that
\emph{piano}/\emph{trombone}/\emph{flute} are part of the idiom. The material that is fixed in idioms varies to a great
degree. Sometimes case is fixed, sometimes it is not. Sometimes idioms can be passivized or used in
relative clauses, sometimes they cannot \citep{NSW94a}. In the case at hand, the semantic properties of the noun
slot are specified: \emph{She plays the volleyball} is not possible. Hence the nouns are part of
the collocation and the determiner depends on a collocation element as predicted by the theory.


\subsubsection{Summary}

Summing up the discussion of \citegen{Zwicky85a} tests for headedness and their application to the DP/NP issue, it
can be said that these tests deliver inconclusive results. Further arguments for either NP or DP are
either theory-internal or pro-NP (the idiom data).


\subsection{The DP analysis}

Having discussed criteria for head status in the previous subsection, I now turn to the DP analysis
and show why an NP analysis should be preferred.


\subsubsection{Personal pronouns}

Figure~\ref{abb-personalpronomina-dp} shows the analysis of personal pronouns in the DP analysis.
% figure moved below because of type setting
\begin{figure}[b]
\centerline{\begin{forest}
sm edges
[DP
    [D$'$
      [\dnull [er]]]]
\end{forest}}
\caption{Personal pronouns in the DP analysis}\label{abb-personalpronomina-dp}
\end{figure}
Personal pronouns are complete and stand for a full nominal structure. Hence they are \dnull rather
than \nnull. \dnull is projected to the maximal level, that is, to DP. This begs the question how
languages without determiners are analyzed (\eg \ili{Slavic} languages; \citealp{Zlatic2014a}). Since there are no determiners, maximal projections
within nominal structures have to be NPs. Since personal pronouns are placeholders for the whole
structure, they should be NPs as well. Hence we had languages in which pronouns are DP and others in
which pronouns are NPs, which would be somewhat unsatisfying.
%
%\inlinetodo{Read \citet{Boskovic2009b-u}.}


\if 0

Wenn man die ganze X-bar-Theorie annimmt, dann ist DetP, Det', Det genauso hohl wie die unnötigen
NP-Projektionen. Wenn man Det=DetP zulässt, dann kann man auch NP -> Mann annehmen. Adjektive
modifizieren dann NPen.

\subsubsection{Überflüssige Projektionsstufen}
\begin{figure}
\hfill
\begin{forest}
sm edges
[DP
    [D$'$
      [\dnull [der]]
        [NP
          [N$'$
            [\nnull [Mann]]]]]]
\end{forest}
\hfill
\begin{forest}
sm edges
[NP
  [DetP [Det$'$ [Det [der]]]]
  [N$'$
    [\nnull [Mann]]]]
\end{forest}
\hfill\mbox{}
\caption{Überflüssige NP"=Projektionsstufen in der DP"=Analyse}\label{abb-analyse-eigennamen-dp-np}
\end{figure}

\fi

\if 0
\subsubsection{Eigennamen}

Eine andere Frage ist die nach der Analyse der
Eigennamen. Abbildung~\ref{abb-analyse-eigennamen-dp-np} zeigt die Analyse von Eigennamen in der DP-
bzw.\ in der NP"=Welt.
\begin{figure}
\hfill
\begin{forest}
sm edges
[DP
    [D$'$
      [\dnull [\trace]]
        [NP
          [N$'$
            [\nnull [Peter]]]]]]
\end{forest}
\hfill
\begin{forest}
sm edges
[NP
  [N$'$
    [\nnull [Peter]]]]
\end{forest}
\hfill\mbox{}
\caption{Analyse von Eigennamen in der DP- bzw.\ NP"=Welt}\label{abb-analyse-eigennamen-dp-np}
\end{figure}
Verzichtet man wie in HPSG oder in jüngeren Varianten des Minimalismus auf unäre Verzweigungen, dann
sind Eigennamen einfach NPen. Mit ausgebauter \xbar"=Syntax stellt sich aber die Frage, wozu man in
der DP"=Analyse die Projektionsstufen bei N braucht. Entweder N$'$ oder NP ist überflüssig, da ja an einen Eigennamen
kein Artikel angeschlossen wird.\todostefan{Das gilt doch allgemein für \nbar}
\fi

\subsubsection{Possessive pronouns}

The next question concerns possessive pronouns. Possessive pronouns and possessor phrases in general
will play a major role in my argument for an NP analysis in HPSG, which is the reason why the
proposals to treat them in a DP approach are discussed here. There are proposals to analyze possessives like
determiners, that is, as \dnull (left figure in Figure~\ref{abb-dp-possessiva}). Since possessives may be complex and since possessive pronouns
alternate with such possessive phrases, both should occupy the same position. This is the reason for
G. \citet[\page 18]{GMueller2007a} to analyze possessive pronouns as specifiers. The middle figure
in Figure~\ref{abb-dp-possessiva} shows a GB rendering of this analysis.
\begin{figure}
\hfill
\scalebox{.8}{
\begin{forest}
sm edges
[DP
  [D$'$
    [\dnull [seine;his]]
    [NP
      [N$'$
        [\nnull [Stadt;town]]]]]]
\end{forest}}
\hfill
\scalebox{.8}{
\begin{forest}
sm edges
[DP
  [DP
    [D$'$
      [\dnull [seine;his]]]]
  [D$'$
    [\dnull [\trace]]
        [NP
          [N$'$
            [\nnull [Stadt;town]]]]]]
\end{forest}}
\hfill
\scalebox{.8}{
\begin{forest}
sm edges
[DP
  [DP
    [D$'$
      [\dnull [sein-;his]]]]
  [D$'$
    [\dnull [-e]]
        [NP
          [N$'$
            [\nnull [Stadt;town]]]]]]
\end{forest}}
\hfill\mbox{}
\caption{Possible analyses for possessive pronouns: left as \dnull, middle as specifier of an empty
  \dnull, right as specifier with inflection in \dnull following \citet[\page 53]{Olsen91b}}\label{abb-dp-possessiva}
\end{figure}
Finally, \citet[\page 52]{Olsen91b} observes that a DP like \emph{seine Stadt} `his town' is third person but
\emph{seine} `his' is first person, and that this is evidence against the possessive pronoun being the
head. Therefore, she assumes that \suffix{e} is the D head and \stem{mein} is a DP functioning
as the specifier. While this seems to be convincing at first, one could assume that \emph{seine} has
a first person referential index but syntactic features for third person for DP-internal
agreement. The rightmost analysis in Figure~\ref{abb-dp-possessiva}, therefore, is not the only possibility. One could assume the one in
the middle as well. The analysis to the right would not be an option within HPSG anyway, since
usually, fully inflected words are inserted into syntax rather than bound morphemes like \suffix{e}.
% Karl s Auto  s = D-Kopf bei Olsen91a:50

The assignment of thematic roles by relational nouns also plays a role in the analysis of
possessives. These are discussed in the following section.

\subsubsection{Relational nouns and assignment of semantic roles}
\label{sec-relational-nouns-and-semantic-roles}

If possessives were analyzed as D heads as in Figure~\ref{abb-dp-possessiva} (left) and in
Figure~\ref{abb-possessivpronomen-dnull}, relational nouns would have to assign a semantic role to a
head position that is higher up in the tree \citep[\page 51]{Olsen91b}. This is prohibited since, according to \citet[\page
  47]{Chomsky81a}, semantic roles may be assigned to argument positions (A positions) only. Chomsky
explicitly does not count head positions among these.
\begin{figure}
\centerline{
\scalebox{.9}{
\begin{forest}
sm edges
[DP
  [D$'$
    [\dnull [seine;his]]
    [NP
      [N$'$
        [\nnull [Eroberung;conquest]]
        [DP [der Stadt;of.the town,roof]]]]]]
\end{forest}}}
\caption{Possessive pronouns as \dnull and assignment of a semantic role by a relational noun}\label{abb-possessivpronomen-dnull}
\end{figure}
One could claim that the possessive pronoun is a \dnull and that the agent gets its semantic role
within the NP and is then moved out of the NP into the head position, as shown in the left figure
in Figure~\ref{abb-rollenzuweisung-rel-nom}. 
\begin{figure}
\hfill
\scalebox{.8}{
\begin{forest}
sm edges
[DP
  [D$'$
    [\dnull [seine$_i$;his]]
        [NP
          [DP [\trace$_i$]]
          [N$'$
            [\nnull [Eroberung;conquest]]
            [DP [der Stadt;of.the town,roof]]]]]]
\end{forest}}
\hfill
\scalebox{.8}{
\begin{forest}
sm edges
[DP
  [DP$_i$
    [D$'$
      [\dnull [seine;his]]]]
  [D$'$
    [\dnull [\trace]]
        [NP
          [DP [\trace$_i$]]
          [N$'$
            [\nnull [Eroberung;conquest]]
            [DP [der Stadt;of.the town,roof]]]]]]
\end{forest}}
\hfill\mbox{}
\caption{Assignment of semantic role to SpecNP and successive movement from a phrase position into a
  head position or a specifier position, respectively}\label{abb-rollenzuweisung-rel-nom}
\end{figure}
In such a setting, the movement out of the NP would have
to target a head position, which is also prohibited (see \eg \citet[Sections~5.1, 6.1, 7.1]{Radford2004a-u} on various types
of movement).
This means that possessive pronouns have to be placed into SpecDP. It follows that the semantic role
assigned by the relational noun is either assigned nonlocally, that is, from within the NP to a
specifier position of the DP, or that the assignment is local within the NP and the receiving
element is moved into the specifier position of the DP (see Figure~\ref{abb-rollenzuweisung-rel-nom}
(right)).\footnote{%
  HPSG does not refer to movement within NPs, but, as a reviewer pointed out, similar effects can be
  obtained by assuming that D embeds an \nbar and that the specifier of the embedded \nbar is shared
  with the specifier of the determiner (see \citew{HN94a} for argument attraction in general).
}

Given what was said in this and the previous section, it follows that possessive pronouns have to be
in SpecDP in the DP system. Figure~\ref{abb-poss-dp-np} shows the structures of the DP and NP
analysis in a fully fledged \xbar system.

\begin{figure}
\hfill
\begin{forest}
sm edges
[DP
  [DP
    [D$'$
      [\dnull [seine;his]]]]
  [D$'$
    [\dnull [\trace]]
        [NP
          [N$'$
            [\nnull [Stadt;town]]]]]]
\end{forest}
\hfill
\begin{forest}
sm edges
[NP
  [DP
    [D$'$
      [\dnull [seine;his]]]]
          [N$'$
            [\nnull [Stadt;town]]]]]]
\end{forest}
\hfill\mbox{}
\caption{Comparison of the construction with possessive pronouns in a fully fledged \xbar system in
  the DP and in the NP analysis}\label{abb-poss-dp-np}
\end{figure}
It is obvious that the NP analysis is much simpler.

\subsubsection{The DP analysis in Minimalism and the NP analysis in HPSG}

The HPSG analysis with an NP structure is really minimal: the lexical noun is combined directly with
the determiner/""possessive pronoun. Since the noun does not require anything but the determiner, it has the
category \nbar\unskip\hspace{1pt}.%
  \footnote{Categories are feature bundles. \nbar is an abbreviation for a nominal object selecting a
  determiner. (\ref{le-mannes}) is an example.}
The possessive pronoun is complete as well and need not be combined with other
elements. Hence it may be combined with the noun, directly resulting in a fully saturated nominal
projection: an NP. The left figure in Figure~\ref{abb-np-hpsg-dp-minimalismus} shows the HPSG analysis.
\begin{figure}[b]
\hfill
\begin{forest}
sm edges
[NP
  [Det [seine;his]]
  [N$'$
    [Stadt;town]]]
\end{forest}
\hfill
\begin{forest}
sm edges
[DP  
  [DP [seine;his]]
  [D$'$
  [D [\trace]]
  [NP
    [Stadt;town]]]]
\end{forest}
\hfill\mbox{}
\caption{NP and DP analysis in HPSG and Minimalism}\label{abb-np-hpsg-dp-minimalismus}
\end{figure}
The figure on the right-hand side of Figure~\ref{abb-np-hpsg-dp-minimalismus} shows the respective analysis in Minimalism.
Although the problem with unnecessary unary branching nodes does not exist in the framework of Bare Phrase
Structure \citep{Chomsky95b-u},\footnote{%
In Bare Phrase Structure Grammar, unary projections of determiners or nouns do not exist. Linguistic
objects are combined with Merge and the category of the result is determined by Labeling. The Label
is basically part of speech information, bar levels are not used. A noun and its dependent form a
phrase and if the noun does not require any further arguments, the result of the combination will be
a complete nominal object, which corresponds to the classical NP. Assuming a
DP analysis of \emph{the house}, the noun like \emph{house} is categorized as NP right away. So in Bare Phrase Structure,
lexical items can be both minimal and maximal at the same time \parencites[\page
64]{Chomsky95a-u}. This is parallel to what Categorial Grammar assumed since
\citew{Ajdukiewicz35a-u} and what is assumed in the analysis of nominal structures in HPSG as
well. See also \citet{Muysken82a} for an early suggestions to collapse bar-levels in the GB framework.
See \citew[Section~4.6.2]{MuellerGT-Eng4} for further discussion of Bare Phrase Structure Grammar.
} the problem with role assignment remains. In contrast, in the NP analysis, possessives
are in the specifier position of the noun and can receive their semantic role there. If one assumes
a DP analysis, one would have to assign the semantic role to the governing head (in HPSG) or to an
even higher element -- the specifier of the governing head. Within Minimalism, one could -- or rather,
had to -- assume non-local role assignment across phrase boundaries or movement of the possessive
pronoun out of the NP into the dominating DP \citep[\page 18]{Salzmann2020a}.

In conclusion, one can say that there is almost no theory-external evidence for a DP or NP analysis. The
criteria for headedness are inconclusive in the DP/NP area. Only few tests clearly decide
the issue, and these are in favour of an NP analysis. Theory-internal considerations show, however, that
the NP analysis must be preferred in non-transformational approaches. 


%% Auch relationale Nomina können semantische Rollen direkt an ihren Spezifikator zuweisen.
%% \nocite{Muysken82a}

%% \subsection{Zuweisung semantischer Rollen}

The next section deals with invisible heads, and nominal structures will play an important role in
this section as well.


\section{Invisible heads}
\label{sec-unsichtbar}

\largerpage[-1]
\citet[\page 37]{Wunderlich87d} writes the following on empty elements in syntax:
\begin{quote}
\label{page-wunderlich-on-empty-elements}
Eine sinnfällige Sprachtheorie sollte die Prinzipien der Sprache so nahe wie möglich nachzeichnen
und nicht Repräsentationen für Positionen vorsehen, die aus funktionalen Gründen gar nicht
erscheinen. Dem erwähnten sprachinhärenten Prinzip möchte ich daher das methodologische Prinzip
"`Vermeide leere Kategorien"' zur Seite stellen.
\citep[\page 37]{Wunderlich87d}\footnote{%
Language theory should model the principles of language as closely as possible. It should not assume
representations for positions that do not appear for functional reasons. In addition to the
principle inherent to language mentioned already [Avoid Pronoun], I would like to add the
methodological principle Avoid Empty Categories. [my translation, St.M.]
}
\end{quote}
While early HPSG used empty elements in nonlocal dependencies (traces, \citealp[\page 164]{ps2}) and empty heads for the
analysis of relative clauses \citep[\page 216]{ps2}, later publications tried to avoid empty elements
\citep*{SF94a,Sag97a,BMS2001a}. This section discusses two examples of empty heads: Subsection~\ref{sec-empty-nominal-heads} deals
with nominal structures again and suggests an empty nominal head, and
Subsection~\ref{sec-empty-copula} deals with copula constructions in African American
  Vernacular English\il{English!African American Vernacular} (AAVE), for which an empty verbal head was suggested. 

Another empty verbal head is assumed in the analysis of \ili{German} by almost all HPSG theoreticians
working on \ili{German}. I followed an approach without such an empty verbal head from 1993 until 2003
(see \citealp[Section~1.9]{Mueller2002b}), but I am now convinced that the assumption of the empty
verbal head is necessary to account for apparent multiple frontings in \ili{German}
\citep{Mueller2003b,Mueller2005d}. The discussion of the arguments for an analysis with an empty
verbal head cannot be included here due to space limitations, but the reader is referred to a
book-length discussion of the data, the analysis, and its alternatives in \citew{MuellerGS}.

See also \citew{Borsley99c-u,Borsley2009a-u,Borsley2013a-u} for further explicit suggestions of analyses
with empty heads.
%\itdopt{Dieser Absatz steht hier etwas unvermittelt und unverbunden.}

\subsection{Nominal heads}
\label{sec-empty-nominal-heads}

\largerpage
In the previous section, I argued for an NP analysis, that is, for an analysis in which the noun is the
head. This begs the question how to analyze phrases that distributionally behave like NPs but
do not contain a noun. The phrases in (\mex{1}f--k) may appear in places in which the NPs in
(\mex{1}a--e) may appear:
\eal
\ex 
\gll die kluge Frau\\
     the smart woman\\
\ex 
\gll die Frau aus Hamburg\\
     the woman from Hamburg\\
\ex 
\gll die kluge Frau aus Hamburg\\
     the smart woman from Hamburg\\
\ex 
\gll die kluge Frau, die wir kennen\\
     the smart woman who we know\\
\ex 
\gll die kluge Frau aus Hamburg, die wir kennen\\
     the smart woman from Hamburg who we know\\
\ex 
\gll die kluge\\
     the smart\\
\glt `the smart one'
\ex 
\label{ex-die-aus-hamburg} 
\gll die aus Hamburg\\
     the from Hamburg\\
\glt `the one from Hamburg'
\ex\label{ex-die-kluge-aus-hamburg} 
\gll die kluge aus Hamburg\\
     the smart from Hamburg\\
\glt `the smart one from Hamburg'
\ex 
\gll die, die wir kennen\\
     the who we know\\
\glt `the one who we know'
\ex 
\gll die kluge, die wir kennen\\
     the smart  who we know\\
\glt `the smart one who we know'
\ex 
\gll die kluge aus Hamburg, die wir kennen\\
     the smart from Hamburg who we know\\
\glt `the smart one from Hamburg who we know'
\zl
For instance, all phrases in (\mex{0}) may function as the subject of the verb \emph{lacht}
`laughs'. Therefore it is appropriate to categorize all these phrases with the same label, rather
than to assume that those in (\mex{0}a--e) are NPs and those in (\mex{0}f--k) DPs, say. If we want
to analyze (\mex{0}f--k) as NPs, we either have to assume an empty nominal head or we have to
formulate rules for NPs that say that an NP may consist of a determiner and one or several
adjectives, or of a determiner and PPs, or relative clauses, or some variation of these elements. The
set of rules would grow and the generalizations would not be captured (see
Section~\ref{sec-psg}). Instead of this, one can simply assume an empty nominal head. The advantage
of this is that all phrases in (\mex{0}) can be analyzed with the same set of rules and that they
have the same structure. Figure~\vref{fig-die-kluge-aus-hamburg} shows the analysis of
(\ref{ex-die-kluge-aus-hamburg}). A simple trick to get rid of the empty element in
Figure~\ref{fig-die-kluge-aus-hamburg} is to assume a unary branching rule that projects the
adjective to \nbar \citep{Wunderlich87d}. Note though that this unary branching rule does not
account for (\ref{ex-die-aus-hamburg}). We will come back to this in Section~\ref{sec-grammar-conversion}.

\begin{figure}
\centerline{%
\begin{forest}
sm edges
[NP
  [Det [die;the]]
  [N$'$
    [N$'$
      [Adj [kluge;smart]]
      [N$'$ [\trace]]]
  [PP [aus Hamburg;from Hamburg,roof]]]]
\end{forest}}
\caption{Analysis of \emph{die kluge aus Hamburg} `the smart one from Hamburg' with empty nominal head}\label{fig-die-kluge-aus-hamburg}
\end{figure}

\largerpage
As was noted in Section~\ref{sec-obligatheit}, determiners can be omitted as well. This is possible
for all nouns in the plural:
\eal
\ex\label{ex-Frauen}
\gll Frauen\\
     women\\
\ex 
\gll Frauen, die wir kennen\\
     women   who we know\\
\ex 
\gll kluge Frauen\\
     smart women\\
\ex 
\gll kluge Frauen, die wir kennen\\
     smart women   who we know\\
\zl
Mass nouns may be used without a determiner in the singular as well:
\eal
\ex 
\gll Getreide\\
     grain\\
\ex 
\gll Getreide, das gerade gemahlen wurde\\
     grain    that just  ground  was\\
\glt `grain that was just ground'
\ex 
\gll frisches Getreide\\
     fresh grain\\
\ex 
\gll frisches Getreide, das gerade gemahlen wurde\\
     fresh    grain    that just  ground was\\
\glt `fresh grain that was just ground'
\zl
%
As I did for structures without a noun, one may assume an empty determiner \citep[\page 90]{ps2}. The analysis of
(\ref{ex-Frauen}) is shown in Figure~\vref{fig-NP-ohne-Det}.

\begin{figure}
\centering
\begin{forest}
sm edges
[NP
  [Det [\trace] ]
  [\nbar [Frauen;women] ] ] 
\end{forest}
\caption{Analysis of nominal structures without a determiner}\label{fig-NP-ohne-Det}
\end{figure}

Interestingly, both determiner and noun may be omitted in a phrase, resulting in phrases consisting
of one or several adjectives and possibly PPs and relative clauses:
\eal
\ex 
\gll Ich helfe klugen.\\
     I help smart\\
\glt `I help smart ones.'
\ex 
\gll Dort drüben steht frisches, das gerade gemahlen wurde.\\
     there over  stands fresh    that just ground was\\
\glt `Over there is fresh [grain] that was just ground.'
\zl
The structures for (\mex{0}a) and a similar NP including a modifying PP are shown in Figure~\ref{fig-no-det-no-noun}.
\begin{figure}
\hfill\begin{forest}
sm edges
[NP
  [Det [\trace] ]
  [\nbar
    [A [klugen;smart] ]
    [\nbar [\trace] ] ] ]
\end{forest}
\hfill
\begin{forest}
sm edges
[NP
  [Det [\trace] ]
  [\nbar
    [\nbar
      [A [klugen;smart] ]
      [\nbar [\trace] ] ]
    [PP [aus Hamburg;from Hamburg, roof]] ]]
\end{forest}
\hfill\mbox{}
\caption{Analysis of nominal structures lacking both determiner and noun: \emph{klugen} `smart ones' and \emph{klugen aus Hamburg} `smart ones from Hamburg'}\label{fig-no-det-no-noun}
\end{figure}

%\largerpage[-1]
\newpage
Instead of an empty determiner, one can assume a unary branching rule projecting an \nbar to NP
\citep[\page 88]{MuellerLehrbuch1}.\footnote{
  The computational implementation \citep{Babel} of the grammar described in \citew{Mueller99a} did not contain
  empty elements. I used a unary branching rule for structures without a determiner. A lexical rule,
  as suggested by \citet[\page 80]{Michaelis2006a}, is not an option. See \citew[Section~6.6.2]{MuellerLehrbuch1} on this point.
} This and other alternatives to empty elements will be discussed in
Section~\ref{sec-grammatikkonversion}. But before turning to alternatives to empty heads, I want to
discuss empty verbal heads in the next subsection.

%% {Leere Elemente}

%% \begin{itemize}
%% \item Ich verwende genau drei:
%% \begin{itemize}
%% \item leerer Determinator (kein Kopf)
%% \item leeres Nomen (Kopf)
%% \item leeres Verb für Verbbewegung (Kopf)
%% \end{itemize}
%% \end{itemize}


\subsection{Verbal heads}
\label{sec-empty-copula}

\citet{Bender2001a} discusses data from African American Vernacular English\il{English!African American Vernacular} (AAVE), in which the
copula can be omitted, resulting in sentences like (\mex{1}), taken from \citew*[\page 457]{SWB2003a}:
\eal
\label{ex-AAVE-copula}
\ex Chris at home.
\ex We angry with you.
\ex You a genius.
\ex They askin for help.
\zl
\citet*[Section~15.3.4]{SWB2003a} discuss a phrasal schema that combines a predicate selecting for an NP
in a certain form with this NP directly. 
\ea
S \to NP Pred
\z
While this provides an account for examples like (\ref{ex-AAVE-copula}), it fails on examples like (\mex{1}), also
taken from \citew[\page 463]{SWB2003a}:
\eal
\ex How old they say his baby $\phi$?
\ex Tha's the man they say $\phi$ in love.
\zl
The interesting fact about these examples is that the predicate is extracted in (\mex{0}a) and the
subject is extracted in (\mex{0}b). The rule in (\mex{-1}) cannot apply in the analysis of (\mex{0}),
since \citeauthor{SWB2003a} do not assume traces for extraction, and hence one would need a special
rule for the case in (\mex{0}a) for combining a subject with an extracted predicate and a special
rule for the case in (\mex{0}b) for combining a predicate with an extracted subject.\footnote{
  Note that this is basically the same problem as the one I pointed out in the discussion of phrasal
  approaches in Construction Grammar \citep[\page 854]{Mueller2006d}.
}
Rather than assuming three unrelated rules, \citeauthor{SWB2003a} follow \citet{Bender2001a} in
assuming an empty head for the copula. This is an interesting twist in the discussion of empty
elements, since the need to assume this empty copula was caused by eliminating empty elements in the
analysis of extraction phenomena by \citet*{BMS2001a}. 

\section{Grammar conversion}
\label{sec-grammatikkonversion}\label{sec-grammar-conversion}


In the previous section, I suggested using empty elements in the analysis of nominal
structures. Current grammatical theories have different views regarding empty elements. There are
hundreds of empty elements of various categories in Minimalism (\citealt[\page 76]{Webelhuth95a}; Newmeyer \citeyear[\page 194]{Newmeyer2004b};
  \citeyear[\page 82]{Newmeyer2005a}; \citealt[Section~4.6.1.1]{MuellerGT-Eng4}), in most
  Construction Grammar variants there is not a single one, and in other frameworks it depends on the
  author whether empty elements are assumed and, if so, which ones.\footnote{%
    \citew[Section~19.1]{MuellerGT-Eng4} gives an overview of approaches with and without empty
    elements in Categorial Grammar, GPSG, LFG, TAG, Dependency Grammar, and HPSG.
} Apart from the two empty elements mentioned
already, I am using only two further empty elements in my grammars: one for nonlocal movement and
one for head movement \citep{MuellerLehrbuch3,MuellerGS}.

I show in this section which formal means used in various frameworks correspond to each other and
how grammars using empty elements can be transformed into grammars without them. This may help to
objectify the discussion, which is a bit emotional sometimes.

\subsection{Phrase structure grammars}
\label{sec-psg}

It was shown as early as the 1960s that grammars with empty elements can be transformed into
grammars without them by inserting the empty elements into the grammar rules. This results in new
grammar rules in which the respective symbols do not appear any longer \citep{BHPS61a,Mueller2004e}.

Let's take the grammar in (\mex{1}a) as an example. We can eliminate the empty element for \nbar by
adding new rules for all rules where \nbar appears on the right-hand side of the rule. The result of
such a transformation is shown in (\mex{1}b):
\ea
\begin{tabular}[t]{@{}ll@{~}l@{~}l@{\hspace{2cm}}ll@{~}l@{~}l@{~}l}
a. & NP    & $\to$ & Det \nbar                   & {b.} & {NP}          & {$\to$} & {Det \nbar}\\
   &   & &                                       &    & {\red{NP}}    & \red{$\to$} & {\red{Det}}\\
   & \nbar & $\to$ &Adj \nbar                    &    & {\nbar}       & {$\to$} & {Adj \nbar}\\
   &    & &                                      &    & {\red{\nbar}} & \red{$\to$} & {\red{Adj}}\\
   & \nbar & $\to$ &\nbar PP                     &    & {\nbar}       & {$\to$} & {\nbar PP}\\
   &    & &                                      &    & {\red{\nbar}} & \red{$\to$} & {\red{PP}}\\
   & \red{\nbar} & \red{$\to$} &\raisebox{0.2ex}{\red{\_}}\rule{0cm}{0.7em}                   & \\\\
%Det   & \trace\\
   & Det & $\to$ &die                            &    & {Det} & {$\to$} &{die}\\
   & Adj & $\to$ &klugen                         &    & {Adj} & {$\to$} &{klugen}\\
   & \nbar & $\to$ &Frauen                       &    & {\nbar} & {$\to$} &{Frauen}\\
\end{tabular}
\z
When we insert empty elements into rules, it may happen that all elements on the right-hand side are
deleted, which has the effect of creating new empty elements. Hence the step of inserting empty elements into
rules has to be applied until it converges, no new empty elements are produced, and all empty
elements are eliminated from the grammar.

As is demonstrated by the simple example in (\mex{0}), the elimination of empty elements may result
in an increase of the number of rules.\footnote{
  \citet[\page 38]{Wunderlich87d}, discussing a proposal with an empty head by \citet{Olsen87b-u},
  suggests a rule projecting nouns from adjectives, but does not mention cases like
  (\ref{ex-die-aus-hamburg}), in which no adjective is present.
} One rule from grammar (\mex{0}a) (the one for the lexical
item of the empty element) was removed and we got three new rules in (\mex{0}b) instead. The generalization that
nouns may be omitted in \ili{German} is not captured directly in the new grammar any longer. Instead we
have a largish number of descriptions of constituents that can form an NP or an \nbar, respectively.

It is often argued that there cannot be empty elements since these would be invisible and hence not
learnable.
%\inlinetodo{add references} 
The acquisition problem seems to argue for empty elements in the nominal structures at
hand, though, since what has to be acquired is the fact that the noun can be omitted in elliptical constructions.

The empty determiner is not part of the example in (\mex{0}), but it is clear that its elimination by
the techniques described above results in a unary branching rule that projects an \nbar to an NP. See \citew[\page
31]{Zlatic2014a} for the assumption of such a unary projection.

\subsection{Lexical rules}

\largerpage
As was mentioned above, the number of empty heads that are assumed in Minimalist work is
significant. Some contribute semantic information and are important for valence alternations like the
one in (\mex{1}):
\eal
\ex 
\gll Er bäckt einen Kuchen.\\
     he.\nom{} bakes a.\acc{} cake\\
\ex 
\gll Er bäckt {ihr} einen Kuchen.\\
     he.\nom{} bakes her.\dat{} a.\acc{} cake\\
\zl
(\mex{0}a) shows the transitive verb \emph{backen} `to bake' with a nominative and an accusative
argument. (\mex{0}b) has a dative argument in addition.

Almost all linguistic theories handle such alternations without empty elements. LFG, HPSG, and SBCG
analyze such valence alternations via lexical rules instead \citep*{Toivonen2013a,MuellerLFGphrasal,SBK2012a}. Figure~\vref{fig-lr-vs-leerer-kopf}
is a comparison of the two analyses.
\begin{figure}
\begin{forest}
[, phantom, s sep = 1cm
[
 {V[\sliste{ NP$_x$, NP$_z$, NP$_y$ }]}
 [ {V[\sliste{ NP$_x$, NP$_y$ }]} ]]
[
 {BenefactiveP[\sliste{ NP$_x$, NP$_z$, NP$_y$ }]}
 [ {V[\sliste{ NP$_x$, NP$_y$ }]} ]
 [Benefactive [$\varnothing$]]]]
\end{forest}
\caption{Comparison of the analysis of valence alternations by lexical rule and empty head}\label{fig-lr-vs-leerer-kopf}
\end{figure}
The left-hand side shows a lexical rule-based analysis relating a word with two elements in the
valence list to a word with three elements in the valence list. The right-hand side shows an
analysis with an empty head: The benefactive head selects a verb with two arguments, and the result
of the combination is a projection that takes three arguments.\footnote{
  Analyses suggested in the literature usually combine a VP with one further head, that is, the
  verbal head is combined with its arguments first and the result of this combination is then
  combined with the benefactive head \citep[\page 75]{BB2011b-u}.
}

%% \scalebox{1}{%
%% \begin{forest}
%% [, phantom, s sep = 1cm
%% [
%%  {V[\sliste{ NP[\type{nom}], NP[\type{dat}], NP[\type{acc}] }]}
%%  [ {V[\sliste{ NP[\type{nom}], NP[\type{acc}] }]} ]]
%% [
%%  {BenefactiveP[\sliste{ NP[\type{nom}], NP[\type{dat}], NP[\type{acc}] }]}, visible on=<2->
%%  [ {V[\sliste{ NP[\type{nom}], NP[\type{acc}] }]} ]
%%  [Benefactive [$\varnothing$]]]]
%% \end{forest}}

Lexical rules in HPSG are basically unary branching rules \citep{BC99a,Meurers2001a} and hence it does not
come as a big surprise that they correspond to constructions with an empty head.

\subsection{Recategorization of phrases}
\label{sec-recategorization}

\largerpage
The previous section dealt with lexical rules. Lexical rules relate words or stems to other words or
stems. One way to model lexical rules is parallel to unary branching rules. But nothing prevents one
from relating phrases to one another. For instance, \citet{Partee86b-u} and, following her, \citet{MuellerPredication}
suggest recategorizing NPs like \emph{ein guter Lehrer} `a good teacher' as in
(\mex{1}a) via a unary projection into an NP that can be used predicatively as in (\mex{1}b). (The
lexical rule"=based analysis by \citet[\page 409]{GSag2000a-u} has scope problems, since lexical
rules can be applied to single lexical elements only, and other elements that can appear in NPs (adjectives for
example) cannot be part of the input of the lexical rule \citep{Kasper97a,MuellerCopula}.)
\eal
\ex 
\gll Ein guter Lehrer lobt.\\
     a good teacher praises\\
\glt `A good teacher praises.'
\ex 
\gll Er ist ein guter Lehrer.\\
     he is a good teacher\\
\glt `He is a good teacher.'
\zl
Figure~\ref{fig-typanhebung} shows the analysis with a unary branching syntactic rule.
The rule projects a NP[\textsc{prd}$-$] to NP[\textsc{prd}$+$], and of course the semantic type
of the NP is adapted as well. This is not shown in the figure, since I do not have the space to
introduce semantic representations in this paper.
%Die Zusammenhänge kann man natürlich genauso mit einem leeren Kopf erfassen. Die Strukturen sind
%dann aber komplexer, da man erstens einen zusätzlichen Lexikoneintrag für das leere Element braucht
%und
What is missing from the figure is that both valence and semantics of the dominating NP are
different from the one of the dominated NP. The predicative NP selects for a subject and introduces a
respective relation that relates the subject to the predicative noun.
\begin{figure}
\centerline{%
\begin{forest}
[
 {NP[\textsc{prd}+]}
 [ {NP[\textsc{prd}$-$]} ]]
\end{forest}
}
\caption{Semantic type raising via unary rule}\label{fig-typanhebung}
\end{figure}

%\largerpage[-1]
The same effect can be reached by assuming an empty nominal head selecting for a \textsc{prd}$-$
phrase and projecting a \textsc{prd}$+$ one.\footnote{%
Proposals in MGG sometimes use an empty Pred head
projecting a PredP \citep{Bowers93a-u}. This is not entirely equivalent to what is suggested here,
since all categories are projected as PredP, and this makes it impossible for governing heads to
select the syntactic category of the predicative element they combine with. As \citet[\page
105--106]{ps2} pointed out, verbs like \emph{grow}, \emph{get}, \emph{turn out}, \emph{become},
and \emph{end up} select different kinds of predicative elements. In the analysis in
Figure~\ref{fig-typanhebung}, an NP is projected into an NP. The syntactic category remains selectable.}

\largerpage
Another interesting case are free relative clauses. Free relative clauses have the form of relative
clauses but function like NP or PP arguments or adjuncts \citep{Bausewein90,Mueller99b}. For
example, in (\mex{1}) the relative clause \emph{wem er vertraut} `who he trusts' fills the slot of the dative object of \emph{helfen} `to help':
\ea
\gll \emph{Wem}  er vertraut, hilft er auch.\footnotemark\\
     who he trusts    helps he too\\
\footnotetext{%
        \citet[\page 234]{Engel77}
      }
\glt `He helps those he trusts.'
\z
%
\citet[Section~2]{GR81a} suggest an analysis with an empty head (XP) that is modified by a relative
clause and the properties of the relative phrase are related to the ones of the empty head. This
analysis as sketched in (\mex{1}a) is interesting since it is parallel to normal relative clause structures containing an
overt pronoun as in (\mex{1}b):
\eal
\ex {}[\sub{\npdat} \_\sub{\npdat} [Wem er vertraut]], hilft er auch.
\ex 
\gll {}[\sub{\npdat} Denen\sub{NP[\type{dat}]}, [denen er vertraut]], hilft er auch.\\
     {}              those              \spacebr{}who   he trusts    helps he too\\
\glt `He helps those he trusts.'
\zl
However, the problem is that relative clauses are adjuncts and modification by adjuncts is
optional. To maintain an analysis like the one by \citeauthor{GR81a}, one would have to assume that
modification of the empty head by a relative clause is obligatory, since otherwise one would have
complete XPs in the grammar that could function as arguments in other areas of the grammar
\citep[\page 97]{Mueller99b}. For example, one could derive sentences with ditransitive verbs and
all of the arguments could be saturated by the empty element:
\ea[*]{
\gll dass \_\sub{NP[nom]} \_\sub{NP[dat]} \_\sub{NP[acc]} gibt\\
     that {}              {}              {}              gives\\
\glt As in: `that she gives it to her'
}
\z
Usually adjunction is not obligatory however. While empty elements and unary projections are
equivalent in most cases, we have a clear difference here. If one analyzes free relatives using a
unary branching rule mapping a free relative clause to an XP, it is clear that this rule can only
apply if there is a free relative clause, while nothing ensures the presence of an adjunct in the
analysis using an empty head.

\subsection{Summary}

\largerpage
I have used nominal structures without overt nouns to further support the point that empty elements may
help capturing generalizations in some cases. They can be avoided in other parts of grammars, and
unary branching projections in the lexicon or in the syntax may be assumed instead. Sometimes unary
projections have an advantage over empty heads since they can ensure the obligatory presence of
constituents that would be treated as adjuncts to empty elements.




\section{Headless structures}
\label{sec-kopflos}

Section~\ref{sec-unsichtbar} dealt with structures in which we usually find certain elements, and it
was argued that it is reasonable to use empty elements in the places in which the heads would appear
when realized overtly. That is, it was assumed that there is a head even though it is invisible. The
argumentation for empty elements is based on the fact that the respective positions are usually
filled. In Section~\ref{sec-recategorization}, I argued that the assumption of an empty head in the
analysis of free relative clauses would permit ill-formed structures and argued for an analysis
without head. However, \citet{Jackendoff2008a} pointed out that there are sequences like those in
(\mex{1}), called N-P-N expressions, where it is not reasonable to assume that one of the involved
elements is the head or that there is some kind of bigger structure from which an element is missing:\footnote{
  See also \citew{Jacobs2008a} for further examples from \ili{German} and \citew[Section~21.10.1]{MuellerGT-Eng4}
  for discussion. Another example of a construction that is usually treated as headless in HPSG is
  coordination. \citet{Borsley2005a} shows that the Minimalist analysis of coordination as ConjP
  suggested by \citet[Chapter~6]{Kayne:94}, \citet[Chapter~3]{Johannessen98a-u}, and others fails in several
  respects. Categorial Grammar also assumes a functor-based approach with the conjunction as the
  head. However, the result of the combination of two X with a conjunction categorized as
  (X/X)$\backslash$X is an X, which gets the external distribution right as far as the main category
  is concerned, while this remains a puzzle in the Minimalist proposals (governing heads select a DP,
  not a ConjP). Nevertheless, there are
  cases in which the last conjunct determines the properties of the complete phrase, as Borsley has
  shown. So additional mechanisms seem to be needed to get the headed analysis right. I will not
  take a stand on this issue here but point the reader to \citew{AC2021a} for a general 
  discussion and an overview of approaches to coordination in HPSG.
}
\ea
student after student
\z
Such sequences can be used in NP positions within larger structures, but they do not have the
structure of NPs internally. For instance, there is no determiner and there is the restriction that
the second N has to be identical with the first one. The meaning of N-P-N expressions cannot be
determined compositionally from the meaning of the parts: N \emph{after} N roughly means
\emph{many Ns in succession} \citep[\page 26]{Jackendoff2008a}.

All theories assuming that all structures have a head/""functor (Minimalism, Dependency Grammar,
Categorial Grammar) have a problem. The previous section showed that one can charm away empty
heads if one does not like them. Similarly, one can conjure up empty heads if one needs
them. Figure~\vref{fig-leerer-Kopf-N-P-N} shows a hypothetical Dependency Grammar analysis.
\begin{figure}
\begin{forest}
dg edges
[N
  [\trace]
  [N [student]]
  [P [after]]
  [N [student]]]
\end{forest}
\caption{Analysis of the N-P-N Construction with empty head}\label{fig-leerer-Kopf-N-P-N}
\end{figure}
This analysis assumes an empty head that selects the two Ns and a P.

Since Minimalism allows for binary branching only, one would need two empty heads to model the N-P-N
construction with empty heads.\footnote{
  G.\ \citet{GMueller2011a} suggests an analysis in which the preposition selects a noun and bears a
  feature \textsc{redup}, which triggers a reduplication of the noun. It remains unclear why a
  structure with a preposition as the main element should project an NP and how sequences of the
  type N-P-N-P-N (\emph{student after student after student}) should be analyzed. See \citew[Section~4.4]{Jackendoff2008a} and
  \citew{Bargmann2015a} on sequences of the latter type. Note also that G.\ Müller stated that his
  analysis predicts that adjectival modifiers of the N in N-P-N constructions are not permitted in
  \ili{German}, a claim that is empirically false \citep[Section~4.1]{MuellerCxG}. 
}

Like for Constructional Grammar, Jackendoff's examples are entirely without any problem for HPSG: a
special schema combines N, P, and N (and possibly further Ps and Ns). Figure~\ref{fig-n-p-n} shows
the respective analysis.
\begin{figure}
\begin{forest}
sm edges
[NP
  [N [student]]
  [P [after]]
  [N [student]]]
\end{forest}
\caption{Construction-based analysis of N-P-N structures}\label{fig-n-p-n}
\end{figure}

I can hear the reproaches: ``But the assumption of a special schema is a stipulation!'' This is true,
but an empty head would be a stipulation as well. The N-P-N schema captures everything that can be
and must be said about the construction: three or more elements (see \citealt{Bargmann2015a}) are combined
idiosyncratically, resulting in an idiosyncratic meaning.

Having shown that empty elements can be removed from grammars (Section~\ref{sec-grammar-conversion})
and can be added if theories require heads (Figure~\ref{fig-leerer-Kopf-N-P-N}), I now turn to the
question of whether there are limits/style guides for the assumption of empty heads.

\section{Good and bad empty elements}
\label{sec-lang-acqui}

As mentioned on page~\pageref{page-wunderlich-on-empty-elements}, HPSG follows Wunderlich in
assuming that syntactic theories should avoid the stipulation of empty elements because of
methodological considerations. I have
demonstrated that sometimes the assumption of empty elements is warranted (empty nominal and verbal
heads) and sometimes it is not and should therefore be avoided (N-P-N construction). This section tries to give a more general
answer to the question when it is legitimate to assume an empty element and when it is not. In
general, it has to be explained how syntactic structures that are suggested can be acquired by
language learners. If one assumes a lean Universal Grammar as \citet*{HCF2002a} or none at all
(\citealt{Tomasello2003a,Goldberg2006a}), then there must be language-particular evidence for empty
elements. Analyses that are solely theory-internally motivated like, for instance, the analysis of PPs
by \citet[\page 452]{Radford97a-u} are not legitimate. Radford assumes that case can be checked in
specifier positions only. In addition, he assumes five empty elements and complicated movement
processes. His analysis is shown in Figure~\ref{fig-kaese}. 
\begin{figure}
\hfill
\begin{forest}
sm edges without translation
[pP
   [p
	[P [with]]
	[p [$\varnothing$]]]
   [AgrOP
	[D [\textbf{me}]]
	[$\overline{\mbox{AgrO}}$
		[AgrO
			[P [t$'$]]
			[AgrO [,phantom  ]]]
		[PP
			[P [t]]
			[D [\textbf{t}]]]]]]
\end{forest}
%\hfill
%% \begin{forest}
%% sm edges without translation
%% [PP
%%   [P [with]]
%%   [NP [me]]]
%% \end{forest}
\hfill\mbox{}
\caption{Theory-internally motivated analysis of a PP following Radford}\label{fig-kaese}
\end{figure}
The necessity for these empty elements follows from the theoretical apparatus that is assumed. Since
there is no independent evidence for the apparatus, it is not acquirable and hence has to be
innate. This contradicts Minimalist assumptions \citep*{HCF2002a}. A precondition to detect absence of
elements is that the positions of the prospective elements can be filled in
principle \citep[\page 40]{MuellerCoreGram}. It is not
legitimate to argue cross-linguistically for empty elements, since the cross-linguistic evidence is
available to us as linguists but not to those who acquire the language.

Summing up, one can say that empty determiners, nouns, and verbs can be acquired from linguistic
evidence from within the language that is acquired, but categories like
AgrO, Topic, Pred, etc.\ that are motivated with reference to other languages cannot. The respective
categories should be assumed for the languages in which they have visible forms (\eg AgrO for \ili{Basque}
and Topic for \ili{Japanese}) but not for languages without any morphological reflexes.
%, \citealp{Chomsky93a}

\section{Summary}
\label{sec-summary}


This paper discussed the role of heads in syntactic structures (within the framework of HPSG). Heads
project morpho-syntactic features (part of speech, case, verb form, and so on), and they have valence
specifications determining the structure of phrases. While it is clear for most types of phrases
which element is the head, theory-neutral criteria for determining heads often fail to decide the
question of whether N or D is the head in \ili{German} nominal structures. I used thematic role assignment
and selection in idioms to argue for an NP analysis. Apart from discussing the notorious DP/NP
issue, I discussed two cases in which empty heads were assumed (again nominal structures in \ili{German} and copula
constructions in AAVE\il{English!African American Vernacular}). These empty heads filled slots in which overt material can be realized. For
the N-P-N construction, an empty head could be stipulated as well, in order to save claims made in
other frameworks that all structures have to have a head. Since HPSG does not make this claim, I
argued for a headless construction instead. I have shown that grammars assuming empty elements can
be converted into ones without empty elements in a straightforward way. Nevertheless there are
conditions on the use of empty elements in grammatical theories: the elements should be recoverable
from input in the language under consideration. There has to be language-internal 
evidence for assuming an empty element, that is the position that is assumed should be filled in some
situations.


\section*{\acknowledgmentsUS}

I thank all participants of the workshop on Headedness and/or grammatical anarchy that took place in
2017 at the Freie Universität Berlin for interesting conversations and stimulating
discussion. Ulrike Freywald and Horst Simon are thanked for organizing this event.

I thank two anonymous reviewers, Jong-Bok Kim, and Antonio Machicao y Priemer for valuable comments.
Georg Walkden and András Bárány are thanked for pointers to relevant literature.
I thank Elizabeth Pankratz for proofreading and for comments on the paper.

{\sloppy
\printbibliography[heading=subbibliography,notkeyword=this]
}
\end{document}



% en
%      <!-- Local IspellDict: en_US-w_accents -->
