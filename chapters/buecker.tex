%% -*- coding:utf-8 -*-
\documentclass[output=paper
  ,nobabel
  ,draftmode
  ,babelshorthands
  ,colorlinks, citecolor=brown
]{langscibook}


\IfFileExists{../localcommands.tex}{%hack to check whether this is being compiled as part of a collection or standalone
   \usepackage{orcidlink}

% add all extra packages you need to load to this file


% Haitao Liu
%\usepackage{xeCJK}
%\setCJKmainfont{SimSun}
%\setCJKmainfont[Scale=MatchUppercase,
%                Path=fonts/
%]{SourceHanSerifSC-Regular}

% instead use option:  ,chinesefont % for references in raffelsiefen.tex
% loading the package changes some spacings


\usepackage{multicol}
\usepackage{tikz}\usetikzlibrary{decorations.pathreplacing}
\usepackage{url}
\urlstyle{same}

%\usepackage{listings}
%\lstset{basicstyle=\ttfamily,tabsize=2,breaklines=true}

\usepackage{langsci-basic}
\usepackage{langsci-optional}
\usepackage[danger]{langsci-lgr}

% toggle danger in texlive 2021
%\newcommand{\M}{\textsc{m}\xspace}

% toggle danger in texlive 2021 or uncomment this
% \newcommand{\N}{\textsc{n}\xspace}
% \newcommand{\F}{\textsc{f}\xspace}


\usepackage{./styles/biblatex-series-number-checks}




\usepackage{langsci-gb4e}




% Demske

\usepackage{tipa}
\usepackage{styles/avm+}
%\usepackage{styles/merkmalstruktur}
\avmfont{\sc}
\usepackage{langsci-forest-setup}
\usepackage{xspace}
%\usepackage{styles/abbrev} 

\usepackage{soul}
\usepackage{color}
\newcommand{\rem}[1]{\textcolor{red}{\st{#1}}}
\newcommand{\add}[1]{\textcolor{blue}{\ul{#1}}}


% Salzmann

\usepackage[nocenter]{qtree}


% Müller

% add this to the default preamble 
\forestset{default preamble={
    for tree={anchor=north},
}}


\usepackage{german}

%\usepackage{german}
\selectlanguage{USenglish}

% Mit Babel geht irgendwie die hyphenation nicht richtig
%\usepackage[ngerman,english]{babel}
%\useshorthands{"} 
%\addto\extrasenglish{\languageshorthands{ngerman}}

\usepackage{styles/makros.2020,
styles/abbrev,
styles/merkmalstruktur,
styles/article-ex,styles/eng-date}


\usepackage{todonotes}
\newcommand{\todostefan}[1]{\todo[color=green!40]{\footnotesize #1}\xspace}
\newcommand{\inlinetodo}[1]{\todo[color=green!40,inline]{\footnotesize #1}\xspace}

\newcommand{\inlinetodoopt}[1]{\todo[color=green!40,inline]{\footnotesize #1}\xspace}
\newcommand{\inlinetodoobl}[1]{\todo[color=red!40,inline]{\footnotesize #1}\xspace}

\newcommand{\itdobl}[1]{\inlinetodoobl{#1}}
\newcommand{\itdopt}[1]{\inlinetodoopt{#1}}

\newcommand{\addpages}{\todostefan{add pages}}

%\newcommand{\iaddpages}{\yel[add pages]{pages}\xspace}



% subfigure
\usepackage{subcaption}



% Nolda
%\usepackage[main=british,nil,german,french]{babel}
\newcommand{\foreignlanguagedummy}[2]{#2}
\usepackage{tagpair}
\usepackage{hang}
\usepackage[noconfig]{ntheorem}
\usepackage{pstricks,pst-node,pst-tree}
\usepackage{newunicodechar}



   \newcommand*{\orcid}[1]{}

% do not show the chapter number. It is redundant, since most references to figures are within the
% same chapter.
\renewcommand{\thefigure}{\arabic{figure}}

\newcommand{\rlapsub}[1]{\rlap{\sub{#1}}}

% \SetupAffiliations{output in groups = false, 
%                    separator between two = {\bigskip\\},
%                    separator between multiple = {\bigskip\\},
%                    separator between final two = {\bigskip\\}
%                    }


%%%%%%%%Alte Umlaute
\newcommand{\oldae}{$\stackrel{\textrm{\tiny e}}{\textrm{a}}$}
\newcommand{\oldoe}{$\stackrel{\textrm{\tiny e}}{\textrm{o}}$}
\newcommand{\oldue}{$\stackrel{\textrm{\tiny e}}{\textrm{u}}$}

\newcommand{\refl}{\REFL}
\newcommand{\pst}{\PST}


% Müller
\let\vref\ref


\let\citew\citet

\newcommand{\page}{}

% biblatex stuff
% get rid of initials for Carl J. Pollard and Carl Pollard in the main text:
\ExecuteBibliographyOptions{uniquename=false}




\newcommand{\nom}{\textsc{nom}}
\newcommand{\gen}{\textsc{gen}}
\newcommand{\dat}{\textsc{dat}}
\newcommand{\acc}{\textsc{acc}}


%\newcommand{\spacebr}{\hspaceThis{[}}



\newcommand{\acknowledgmentsEN}{Acknowledgements}
\newcommand{\acknowledgmentsUS}{Acknowledgments}


% no bf!!!111!
\let\textbfemph\emph

\newcommand{\textbfremoved}[1]{#1}
%\newcommand{\emphremoved}[1]{#1}


\newcommand{\noemph}[1]{#1}
\newcommand{\underlineemph}[1]{\emph{#1}}




% for editing, remove later
\usepackage{xcolor}
\newcommand{\added}[1]{{\red #1}}
\newcommand{\addedthis}{\todostefan{added this}}

\newcommand{\changed}[1]{\textcolor{orange}{#1}}




% Nolda

\theorembodyfont{\normalfont}
\let\restriction\relax
\renewtheoremstyle{break}{\item{\itshape ##1\ ##2}\newline\nopagebreak}{\item{\itshape ##1\ ##2\ (##3)}\newline\nopagebreak}
\theoremstyle{break}
\newtheorem{definition}{Definition}
\newtheorem{pattern}{Pattern}
\newtheorem{restriction}{Restriction}
\newunicodechar{‑}{\hbox{-}}
\newunicodechar{…}{\dots}
\newunicodechar{⁡}{\relax}
\newunicodechar{⁣}{\relax}
\newunicodechar{⁀}{\raisebox{+1ex}{\ensuremath\frown}}
\newunicodechar{⁐}{\raisebox{+1ex}{\ensuremath\frown}\setbox1=\hbox{\ensuremath\smile}\hspace{-\wd1}\raisebox{-1ex}{\ensuremath\smile}}
\newunicodechar{⪪}{\ensuremath{<\mathrel{\llap{\ensuremath{-}}}}}
\setkomafont{descriptionlabel}{\normalfont}
\ExecuteBibliographyOptions{labeldate=comp,labelnumber=true,defernumbers=true}
\defbibenvironment{sources}{\list{\printfield{labelprefix}\,\printfield{labelnumber}}{\settowidth{\labelwidth}{S\,0}\setlength{\labelsep}{\biblabelsep}\setlength{\leftmargin}{\labelwidth}\addtolength{\leftmargin}{\labelsep}\setlength{\itemsep}{\bibitemsep}\setlength{\parsep}{\bibparsep}}\renewcommand{\makelabel}[1]{##1\hfil}}{\endlist}{\item}
\newcommand{\citesource}[1]{\citefield{#1}{labelprefix}\,\citefield{#1}{labelnumber}}


% Was soll das machen?
\newcommand{\textstyleFootnoteSymbol}{}



% Was ist das???? St. Mü. 30.10.2021
%Kann weg. Damit waren die bücker transkripte aligniert. Habe das jetzt mit tabularx und hphantom gemacht

%\newlength{\calength} %tmp length to store the space 1. until [; 2. until ].
%
%%first argument speaker ID, second argument text. Optional argument left margin indicator (arrow or similar)
%\newcommand{\cabox}[3][]{\parbox{0mm}{\hspace*{-1cm}#1}%
%\parbox{1.5cm}{#2}%
%\parbox{9.6cm}{#3}\\%
%}
%
%%translation. First parbox is empty, second parbox takes the translation text
%\newcommand{\trsbox}[1]{\parbox{1.5cm}{~}%
%\parbox{9.6cm}{\itshape #1}\\%
%}
%
%%store the width of a string.
%\newcommand{\settablength}[1]{\settowidth{\calength}{#1}\global\calength=\calength}
%
%%print string and store its width. Useful if the first item of the aligned set is also the longest
%\newcommand{\inittab}[1]{#1\settablength{#1}}
%
%%insert horizontal white space equivalent to the stored width
%\newcommand{\skiptab}{\parbox{\calength}{~}}
%
%%print the argument and fill up with horizontal white space until the stored width is reached.
%\newcommand{\filledtab}[1]{\parbox{\calength}{#1}}



% for standalone compilations Felix: This is in the class already
%\let\thetitle\@title
%\let\theauthor\@author 
\makeatletter
\newcommand{\togglepaper}[1][0]{ 
\bibliography{../bib-abbr,../stmue,../localbibliography,
collection.bib}
  %% hyphenation points for line breaks
%% Normally, automatic hyphenation in LaTeX is very good
%% If a word is mis-hyphenated, add it to this file
%%
%% add information to TeX file before \begin{document} with:
%% %% hyphenation points for line breaks
%% Normally, automatic hyphenation in LaTeX is very good
%% If a word is mis-hyphenated, add it to this file
%%
%% add information to TeX file before \begin{document} with:
%% \include{localhyphenation}
\hyphenation{
Arsch
anaph-o-ra
Bü-cking
con-stit-u-ents
Dor-drecht
For-schungs-ge-mein-schaft
Ge-schich-te
ha-ben
pho-nol-o-gy
pro-so-dic
pro-so-di-cally
Sal-pe-ter
sei-nen
Wil-liams
}
\hyphenation{
Arsch
anaph-o-ra
Bü-cking
con-stit-u-ents
Dor-drecht
For-schungs-ge-mein-schaft
Ge-schich-te
ha-ben
pho-nol-o-gy
pro-so-dic
pro-so-di-cally
Sal-pe-ter
sei-nen
Wil-liams
}
  % \memoizeset{
  %   memo filename prefix={hpsg-handbook.memo.dir/},
  %   % readonly
  % }
  \papernote{\scriptsize\normalfont
    \@author.
    \titleTemp. 
    To appear in: 
    Ulrike Freywald \& Horst Simon (eds.) Headedness and/or grammatical anarchy?
    Berlin: Language Science Press. [preliminary page numbering]
  }
  \pagenumbering{roman}
  \setcounter{chapter}{#1}
  \addtocounter{chapter}{-1}
}
\makeatother



% This does a linebreak for \gll for long sentences leaving space for the language at the right
% margin. The factor .0989 is needed since otherwise starred examples cause a linebreak.
% St.Mü. 17.06.2021 08.02.2021
\newcommand{\longexampleandlanguage}[2]{%
%\begin{tabularx}{.99\linewidth}[t]{@{}X@{}p{\widthof{(#2)}}@{}}%
%\begin{minipage}[t]{.99\linewidth}%
\begin{tabularx}{\linewidth}[t]{@{}X@{}p{\widthof{(#2)}}@{}}%
\begin{minipage}[t]{\linewidth}%
#1%
\end{minipage} & (\ili{#2})%
\end{tabularx}}

% ORCIDs in langsci-affiliations 
\usepackage{orcidlink}
\definecolor{orcidlogocol}{cmyk}{0,0,0,1}
\ProvideDocumentCommand{\LinkToORCIDinAffiliations}{ +m }
  {%
    \orcidlink{#1}
  }

   %% -*- coding:utf-8 -*-
%%%%%%%%%%%%%%%%%%%%%%%%%%%%%%%%%%%%%%%%%%%%%%%%%%%%%%%%%%%%
%
% gb4e

% fixes problem with to much vertical space between \zl and \eal due to the \nopagebreak
% command.
\makeatletter
\def\@exe[#1]{\ifnum \@xnumdepth >0%
                 \if@xrec\@exrecwarn\fi%
                 \if@noftnote\@exrecwarn\fi%
                 \@xnumdepth0\@listdepth0\@xrectrue%
                 \save@counters%
              \fi%
                 \advance\@xnumdepth \@ne \@@xsi%
                 \if@noftnote%
                        \begin{list}{(\thexnumi)}%
                        {\usecounter{xnumi}\@subex{#1}{\@gblabelsep}{0em}%
                        \setcounter{xnumi}{\value{equation}}}
% this is commented out here since it causes additional space between \zl and \eal 06.06.2020
%                        \nopagebreak}%
                 \else%
                        \begin{list}{(\roman{xnumi})}%
                        {\usecounter{xnumi}\@subex{(iiv)}{\@gblabelsep}{\footexindent}%
                        \setcounter{xnumi}{\value{fnx}}}%
                 \fi}
\makeatother

   %% hyphenation points for line breaks
%% Normally, automatic hyphenation in LaTeX is very good
%% If a word is mis-hyphenated, add it to this file
%%
%% add information to TeX file before \begin{document} with:
%% %% hyphenation points for line breaks
%% Normally, automatic hyphenation in LaTeX is very good
%% If a word is mis-hyphenated, add it to this file
%%
%% add information to TeX file before \begin{document} with:
%% %% hyphenation points for line breaks
%% Normally, automatic hyphenation in LaTeX is very good
%% If a word is mis-hyphenated, add it to this file
%%
%% add information to TeX file before \begin{document} with:
%% \include{localhyphenation}
\hyphenation{
Arsch
anaph-o-ra
Bü-cking
con-stit-u-ents
Dor-drecht
For-schungs-ge-mein-schaft
Ge-schich-te
ha-ben
pho-nol-o-gy
pro-so-dic
pro-so-di-cally
Sal-pe-ter
sei-nen
Wil-liams
}
\hyphenation{
Arsch
anaph-o-ra
Bü-cking
con-stit-u-ents
Dor-drecht
For-schungs-ge-mein-schaft
Ge-schich-te
ha-ben
pho-nol-o-gy
pro-so-dic
pro-so-di-cally
Sal-pe-ter
sei-nen
Wil-liams
}
\hyphenation{
Arsch
anaph-o-ra
Bü-cking
con-stit-u-ents
Dor-drecht
For-schungs-ge-mein-schaft
Ge-schich-te
ha-ben
pho-nol-o-gy
pro-so-dic
pro-so-di-cally
Sal-pe-ter
sei-nen
Wil-liams
}
    \togglepaper[10]
}{}


\title[Burning down the phrase and heating up the head]{Burning down the phrase and heating up the head: The interjectionalization of German \emph{von wegen}}

\author{Jörg Bücker\orcid{0000-0003-4864-3374}\affiliation{Heinrich-Heine-Universität Düsseldorf}}
% \chapterDOI{} %will be filled in at production

% \epigram{}

\abstract{%
	While bare prepositional heads usually do not develop into interjections in German, the interjection \emph{von wégen} is an exception, it traces back to the complex preposition \emph{von wegen}. In this paper it will be shown that \emph{von wégen} arose substantially from dialogic language use as a verbal means to reestablish prior speech acts in order to react to them. While its semantic and pragmatic development followed the common diachronic path from a descriptive meaning to a text-/discourse-structuring and affect-/stance-related meaning, its structural development was less usual since it involved the structural reduction of an exocentric phrase to its head. This paper suggests that some aspects of this change might be addressed as head-status change, head-category change and head-feature change.
}


\begin{document}

\maketitle


\section{Introduction}\label{sec-intro-büc}

In \ili{German}, lexicalization of complete PPs is a common outcome of linguistic change, cf.\ entrenched PPs such as \emph{zum Beispiel} `for example', \emph{auf gut Glück} `haphazardly', \emph{vor Freude} `for joy'
and \emph{zwar} `indeed, admittedly'\footnote{\emph{Zwar} is a fused and decategorized descendant of \emph{ze wāre} `for real' \citep[cf.][1020]{Kluge2002}.}
%\itdopt{bib einfügen}
as some random examples. The lexicalization of PPs can also lead to de- and recategorization as an interjection, cf.\
deprepositional interjections such as \emph{um Himmels willen} `good heavens', \emph{in Dreiteufels Namen} `in three devils' name', \emph{bei Gott} `by Jove!', \emph{am Arsch} `my ass', \emph{zum Kuckuck} `dash it!' and \emph{fürwahr} `forsooth!'\footnote{\textit{Fürwahr} traces back to the Middle High \ili{German} PP \emph{vür wār/vür wāre} `for real' \citep[cf.][927]{GrimmGrimm1878}.}. 
Following \citeauthor{Fries1992}' (\citeyear{Fries1992}, \citeyear{Fries22}) approach of interjections as heads of interjection phrases (INT$^0$, INTP), this pathway of ``interjectionalization'' (``Interjektionalisierung'', \citealp{Nuebling2001}) can be reconstructed in terms of PP > INT$^0$ (see also \citealp{Ehlich1986,Reisigl1999,Nuebling2004} on interjections).
%\itdopt{bib einfügen}

At first sight, only full-fledged PPs with nominal complements (preferably from specific lexical domains, e.g., sacral and profane nouns) seem to be qualified to be reanalyzed as interjections in \ili{German}. The reanalysis of prepositional heads alone in terms of P$^0$ > INT$^0$ seems to be blocked instead. However, even though the path PP > INT$^0$ is certainly much more common in \ili{German}, there is at least one remarkable exception that reflects a development along the path P$^0$ > INT$^0$: The present-day \ili{German} interjection \emph{von wégen} `my foot' did not arise from a full PP but traces back to the Early Modern High \ili{German} prepositional head \emph{von wegen}. In this paper, I will look at the forms and functions of present-day \ili{German} \emph{von wégen} in spoken data in Section~\ref{sec-pres-vw} before reconstructing the diachronic steps of its development in Section~\ref{sec-past-vw} and arriving at some observations and considerations on why a development along the path P$^0$ > INT$^0$ is less frequent in \ili{German} in comparison to the alternative path PP > INT$^0$ in Section~\ref{sec-complex-vw}.

In particular, it will be argued that the reanalytic steps leading to the emergence of \emph{von wégen} arose essentially from peculiarities of dialogic language use such as the necessity to reestablish accessible speech acts in order to react to them. Accordingly, this study is substantially based on examples from dialogic contexts.\footnote{Methodologically, the study is qualitative.} The present-day \emph{von wégen} examples are taken from the ``Datenbank für Gesprochenes Deutsch (DGD)'' (`Database for spoken \ili{German}') and are transcribed according to GAT 2 conventions (cf.\ \citealp{Selting2009}).
%\itdopt{Selting et al. reference nicht gegeben}

The historical examples come from the ``Corpus der altdeutschen Originalur\-kunden bis zum Jahr 1300''
(`Corpus of Old German Original Charters up to the year 1300', 13th century, mainly
Central\il{German!Central} and Upper\il{German!Upper} German)\footnote{It contains the  oldest
  instances of the circumposition \emph{von -- wegen.}}  and from Early Modern High
German\il{German!Early New High} letters. The examples from letters in this study are primarily
taken from ``Actenstücke und Briefe zur Geschichte des Hauses Habsburg im Zeitalter Maximilian’s I''
(`Records and Letters Pertaining to the History of the House of Habsburg in the Age of
Maximilian I', 15th century, Upper German\il{German!Upper}) and from ``Aktensammlung zur Geschichte
der Basler Reformation in den Jahren 1519 bis Anfang 1534'' (`Collection of Records Pertaining to the
History of the Basel Reformation in the Years 1519 until Early 1534', 16th century, Upper
German\il{German!Upper}), but the study is also based on an analysis of examples from the corpus
``Frühneuzeitliche Fürstinnenkorrespondenzen im mitteldeutschen Raum'' (`Early Modern Correspondences
of Princesses in Central Germany', 16th -- 18th centuries, Middle German) and from letters by Hildebrand Veckinchusen (15th century, Low German). All examples mentioned in this paper will contain references to the corpus or edition they are taken from.

\section{The present-day German interjection \emph{von wegen}}\label{sec-pres-vw}

The present-day spoken \ili{German} interjection \emph{von wégen} is a common emphatic expression of
disagreement.\footnote{Note that present-day \ili{German} \emph{von wegen} can also be used to initiate
  direct speech in terms of a conversational ``reporting frame'' (cf.\ \citealp{Buecker2009,
    Buecker2013}, see also \citealp[307–310]{Androutsopoulos1998}). Since the reporting frame \emph{von wegen} is the outcome of a diachronic path of development in its own right (cf.\ \citealt{Buecker2022a}), this paper will restrict itself to the diachronic emergence of the disagreeing \emph{von wégen}.} The following example shows this use:\footnote{For the sake of simplicity, the line numbering of the transcripts in this section starts with 1.}


\vbox{
\ea\label{ex-interjec132} \emph{Von wégen} as an interjection in spoken \ili{German} (DGD, FOLK\_E\_00132)\\
%\cabox{JA:}{[einfach keine schönen MÖglichkeiten,]}
%\trsbox{simply no nice possibilities}
%
%\cabox{~}{\inittab{was zu MAchen.}]} %set tab length to "was zu MAchen."

%\cabox{AM, KA:}{\filledtab{[((giggle))   }]}%use tab length and fill with ((giggle))
%
%\smallskip
%
%\settablength{ja weil WIR die ganzen (kart hent)?} %explicit setting of tab length
%\cabox{PA:}{verHÖkere sie [\filledtab{an andere spieler (weiter).   }]}
%\trsbox{sell them to other players}
%
%\smallskip
%
%\settablength{verHÖkere sie }
%\cabox{AM:}{\skiptab{}[ja weil WIR die ganzen (kart hent)?]} %insert white space before the text starts
%\trsbox{\skiptab{}~yeah because we have all the cards}
%
%\smallskip
%
%\settablength{[<{}<smile voice> ja,>}
%\cabox{AM:}{\filledtab{[((giggles))        }]}
%
%\smallskip
%
%\cabox[→]{JA:}{\filledtab{[ja weil ihr HORtet.}]}
%\trsbox{yeah because you’re hoarding}
%
%\smallskip
%
%\cabox{AM:}{[<{}<smile voice> ja,>]}
%\trsbox{yes}
%
%\smallskip
%
%\settablength{((claps her/his hands one time))}
%\cabox{AM:}{[\filledtab{wo::w.}]}
%\trsbox{wow}
%
%\smallskip
%
%\cabox{?:}{[((claps her/his hands one time))]}
%\smallskip
%
%\cabox{KA:}{[\filledtab{((laughs))}]}
%
%\smallskip
%
%\cabox{JA:}{nicht SCHLECHT.}
%\trsbox{not bad}
%
%\smallskip
%
%\cabox[→]{KA:}{\textbfemph{von !WE!gen.=}}
%\trsbox{so much for / my foot}
%
%\settablength{=wir HORten?}
%
%\cabox[→]{~}{[=wir HORten?]}
%\trsbox{‘we’re hoarding’}
%
%\smallskip
%
%\cabox{AM:}{[\filledtab{triUMPH.}]}
%\trsbox{victory}
%
%\smallskip
%
%\cabox{AM:}{triUMPH. ((laughs))}
%\trsbox{victory}
\z
\begin{tabularx}{\textwidth}[t]{rlll@{}l@{}}
1 & & JA: & [einfach keine schönen MÖglichkeiten,\\
  & &     & `simply no nice possibilities'\\
2 & &     & was zu MAchen.]\\
  & &     & `to do something'\\
3 & & AM, KA: & [((giggle))\hphantom{mmmr}]\\
4 & & PA: & verHÖkere sie [an andere spieler (weiter).\hphantom{(kart}   &]\\
  & &     & `sell them to other players'\\
5 & & AM: & \hphantom{verHÖkere sie} [ja weil WIR die ganzen (kart hent)?&] \\
  & &     & \hphantom{verHÖkere sie} `yeah because we have all the cards' \\
6 & & AM: & [((giggles))\hphantom{HORtet]} ] \\
7 &→& JA: & [ja weil ihr HORtet.] \\
  & &     & `yeah because you’re hoarding' \\
8 & & AM: & <{}<smile voice> ja,> \\
  & &     & `yes' \\
 {[\ldots]} \\
88 & & AM: & [wo::w.\hphantom{her/his hands one time))} ] \\
89 & & ?:  & [((claps her/his hands one time))] \\
90 & & KA: & [((laughs))\hphantom{/his hands one time/]} ]\\
\end{tabularx}
}
\begin{tabularx}{\textwidth}[t]{rlll@{}l@{}}
91 & & JA: & nicht SCHLECHT.\\
   & &     & `not bad'\\
92 &→& KA: & \textbf{von !WE!gen.=} \\
   & &     & `so much for'/`my foot'\\
93 &→&     & [=wir HORten?] \\
   & &     & `we're hoarding'\\
94 & & AM: & [triUMPH.\hphantom{ten?} ]\\
   & &     & `victory'\\
95 & & AM: & triUMPH. ((laughs))\\
   & &     & `victory'\\
\end{tabularx}
\medskip

\noindent
This extract is taken from a conversation between four friends who are playing the board game ``Thurn und Taxis''. In the game, JA and PA on the one hand and KA and AM on the other are both trying to build up the most comprehensive network of post offices in a 17th century Germany setting. In the first segment of the extract (lines 1--8), JA complains that his possibilities to improve his situation in the game are restricted because AM and KA are ``hoarding'' (cf.\ line 7) a large number of city cards which are necessary to establish postal routes. About 50 seconds later, AM and KA make use of their city cards in order to finish a lengthy postal route (cf.\ the second segment of the extract, lines 88--95), and JA has to concede that this achievement was not bad at all (cf.\ line 91). In reaction to JA's concession, KA reestablishes JA's ``hoarding'' claim in line 7 in order to reject it by means of the emphatically stressed  \emph{von wégen} (cf.\ lines 92--93), whereas AM expresses her enthusiasm for the victory points they just gained (cf.\ 94--95).

In example (\ref{ex-interjec132}), \emph{von wégen} can be translated either as `so much for' or as `my foot (eye, ass, \ldots)', `my foot' probably being the better translation as  \emph{von wégen} clearly expresses disagreement. In example (\ref{ex-interjection}),  \emph{von wégen} is again used as an expression of disagreement, but this time it can only be translated as `my foot (eye, ass, \ldots)': 
\largerpage

%\begin{figure}
\vbox{
\ea\label{ex-interjection} \emph{Von wégen} as an interjection in spoken \ili{German} (DGD, FOLK\_E\_00255)\\
%\cabox{AG:}{jetzt geht der schöne winter (0.2) dem [ENde zu.]}
%\trsbox{now the beautiful winter is coming to an end}
%
%\smallskip
%\cabox{BS:} {\filledtab{[hm::.]}}%\\
%\trsbox{hm}
%    {\filledtab{(0.4)}}
%\smallskip
%
%\settablength{ja weil WIR die ganzen (kart hent)?} %explicit setting of tab length
%\cabox{AG:}{HEUT ist  [er pampig-   ]}
%\trsbox{today he's snotty}
%
%\smallskip
%
%\cabox{PD:}{[von !WE!gen]}
%\trsbox{my foot}
%
%\smallskip
%\cabox{PD:}{Es wird noch [KÄLter wieder]}
%\trsbox{it will get even colder again}
%\smallskip
%
%\cabox{BS:}     {\filledtab      {[also es ist GLATT im (.)]}\\
%            bei uns in der;}
%\trsbox{well it is icy in the; at our place in the}
%\itdobl{Stefan: Align BS: to the first line, not to the middle.}
\z
\hspace{1em}\begin{tabularx}{\textwidth}{rlll@{}l@{}}
1 &→& AG: & jetzt geht der schöne winter (0.2) dem [ENde zu. &]\\
  & &     & `now the beautiful winter is coming to an end'\\
2 & & BS: & \hphantom{jetzt geht der schöne winter (0.2) dem} [hm::.&]\\
3 & &     & (0.4)\\
4 & & AG: & HEUT ist [er pampig-\hphantom{\textbf{en.}}]\\
  & &     & `today he’s snotty'\\
5 &→& PD: & \hphantom{HEUT ist} \textbf{[von !WE!gen.]}\\
  & &     & \hphantom{HEUT ist} \spacebr`my foot'\\
\end{tabularx}
}
\begin{tabularx}{\textwidth}{rlll@{}l@{}}
6 &→& PD: & es wird noch [KÄLter wieder.]\\
  & &     & `it will get even colder again'\\
7 & & BS: & \multicolumn{2}{l}{\hphantom{es wird noch} [also es ist \hphantom{ieder}] GLATT im (.) bei uns in der;}\\
  & &     & `well it is icy in the – at our place in the'\\
\end{tabularx}
\medskip

\noindent
Example (\ref{ex-interjection}) is an extract from a ``coffee klatch'' conversation between four women. The extract starts with AG claiming that the winter weather is about to come to an end (line 1). Since her conversational partners do not take over the turn after that (cf.\ the pause in line 3), AG tries to carry on with another topic (cf.\ line 4). PD, however, interrupts AG by means of an emphatically stressed \emph{von wégen} and a counterclaim to her assumption of the winter weather ending (cf.\ lines 5--6). This keeps the focus on the weather topic for the subsequent turn (cf.\ line~7).

\largerpage
At first sight, the \emph{von wégen} instances in example (\ref{ex-interjec132}) and example (\ref{ex-interjection}) look very similar: Prosodically, \emph{von wégen} has a strong emphatic accent on the trochee \emph{wégen}; semantically, \emph{von wégen} expresses disagreement with a prior claim by one of the addressees; syntactically, \emph{von wégen} occupies an independent position right in front of a subsequent syntactic unit; functionally, \emph{von wégen} has substantial weight as an action in its own right, it expresses a full-fledged challenge of a prior claim.

However, the sequential context and the relationship between \emph{von wégen} and the subsequent utterance are actually quite different in the two examples:

\begin{itemize}
    \item In example (\ref{ex-interjec132}), \emph{von wégen} expresses disagreement concerning a prior claim that is accessible but not contextually active anymore. Against this background, the subsequent \emph{wir horten} `we're hoarding' serves as a quotative index that reestablishes a preceding speech act \emph{von wégen} is reacting to.
    
    \item In example (\ref{ex-interjection}), \emph{von wégen} expresses disagreement concerning a prior claim that is both accessible and contextually active. Against this background, the subsequent \emph{es wird noch kälter wieder} `it will get even colder again' is not a quotative index but a counterclaim reinforcing \emph{von wégen} and making the propositional content of the disagreement explicit. This is the reason why \emph{von wégen} cannot be translated as `so much for' in (\ref{ex-interjection}).
\end{itemize}

\noindent
The difference between the utterances immediately following \emph{von wégen} in examples (\ref{ex-interjec132}) and (\ref{ex-interjection}) is not only a matter of pragmatics but also a matter of syntactic positioning. Note that quotative indexes and counterclaims can easily combine after \emph{von wégen}, but in such cases, the quotative index (e.g., \emph{der Winter endet} `the winter weather ends') has to precede the counterclaim (e.g. \emph{es wird noch kälter wieder} `it will get even colder again'), cf.\ \emph{von wégen der Winter endet, es wird noch kälter wieder} `my foot the winter weather ends, it will get even colder again'. This is also reflected on the prosodic level: Just like in example (\ref{ex-interjec132}), quotative complex anaphors tend to follow \emph{von wégen} fast (``latching'') or are even a part of its intonation phrase. Counterclaims, in contrast, are usually separated from adjacent \emph{von wégen} intonation phrases more clearly (cf.\ example (\ref{ex-interjection}) without ``latching'').

In the next section, it will be shown from a diachronic point of view that the pragmatic relationship between \emph{von wégen} and subsequent quotative indexes is a reanalyzed remnant of a former syntactic relationship between \emph{von wegen} as a prepositional head P$^0$ and its complement, while the pragmatic relationship between \emph{von wégen} and subsequent counterclaims is a reanalyzed remnant of a former syntactic relationship between hanging topic instances of \emph{von wegen} PPs and their subsequent host clauses.

\section{The diachronic interjectionalization of \emph{von wégen}}\label{sec-past-vw}
\subsection{The rise of prepositional \emph{von wegen} in Early Modern High German}\label{ssec-frnhd-vw}

The complex preposition \emph{von wegen} is syntactically derived from the circumposition \emph{von – wegen} in terms of a linearization change beginning in the 14th century.%
%
\footnote{The earliest instances of \emph{von wegen} can be explained both by an influence of the well-known positional shift of attributive nouns from pre- to postposition (cf.\ \citealp[215–231]{Demske2001a} and \citealp[100–101]{NueblingEtAl2013}, among others) and by syntactic adaption to the prototypical head-initial pattern of \ili{German} prepositions. See \citet[106–120]{Paul1920}, \citet[404–124]{Bloomfield1933} and \citet[14ff]{Becker1990} on analogy formation in language change.
%	\itdobl{Stefan: Genaue Seitenangaben von--bis.}
}
The change from \emph{von – wegen} to \emph{von wegen} did not affect the semantic and pragmatic characteristics of \emph{von – wegen} but maintained them for the most part.\footnote{Structural and semantic changes that set the preposition \emph{von wegen} apart from the circumposition \emph{von – wegen} happened later.}
In the 13th century, \emph{von – wegen} PPs were predominantly used as one of the following three adjunct types:

\enlargethispage{-7pt}
\eanoraggedright\label{ex-causaladj} 
Causal adjunct (1284, Corpus der altdeutschen
Originalurkunden II, Doc. No. 673, lines 19--20; cf.\ \citealp[86]{WilhelmNewald1943}):\\[2pt]%\smallskip
ſo iſt von beiden ſiten gv\char"0364 tlich / vnd einmv\char"0364 tlich verzigen auf allen den ſchad\-en der \textbfemph{von deſ Chrieges wegen}/ biz auf diſen tac hivte iſt giſchehen\\
`so both sides amicably and consensually waive the compensation for the damage that has been done till this day \textbfemph{due to the war'}
\z


\eanoraggedright\label{ex-modaladj} 
Modal adjunct establishing a person or institution from whose side or  under whose authority the action being expressed in the host clause proposition is carried out (1286, Corpus der altdeutschen Originalurkunden II, Doc. No. 841, lines 21--23; cf.\ \citealp[192]{WilhelmNewald1943}):\\[2pt]
	dat men en geenrehande ſcade ſal doen noch mit roue -- noch mit brande -- noch in geenre maniren - \textbfemph{uan des hertogen Wegen} - noch \textbfemph{uan unſer wegen}\\
`that one shall not do any damage, neither with robbery nor with pillage nor in any other way,
\emph{on behalf/the part of the Duke or on our behalf/part}'
\z

\eanoraggedright\label{ex-domainadj} 
Domain adjunct establishing a domain of conceptual content with regard to which the validity of the host clause proposition is restricted (1297, Corpus der altdeutschen Originalurkunden IV, Doc. No. 2687, lines 34--36; cf.\ \citealp[78]{deBoorHaacke1963}):\\[2pt]
	vnde [wir, J.B.] verzihen vnſ / vúr vnſ / vnde vúr alle vnſere nahkomen / alleſ deſ rehtes / daz wir hatten an demme vorgenanten hove / \textbfemph{von der ſelben vorgenanten fúnzehen ſchefol koren geltes wegen}\\
`and [we, J.B.] abandon every of our and our descendants' rights that we had on the estate mentioned above \textbfemph{regarding those above-mentioned fifteen bushels' corn rent}'
\z

\noindent
The domain adjunct use (example (\ref{ex-domainadj})) is of particular importance for this study because it can be assumed to be the semantic, syntactic and pragmatic starting point of the diachronic development towards \emph{von wégen} as an interjection of disagreement (see Subsections~\ref{ssec-frnhd-scope-vw} and~\ref{ssec-mhd-vw}). 

\subsection{The increasing scope of \emph{von wegen} in Early Modern High German}\label{ssec-frnhd-scope-vw}

While the circumposition \emph{von – wegen} dates back to the first half of the 13th century, the
complex preposition \emph{von wegen} emerges in the mid-14th century and becomes productive in the
15th century. This study assumes the restrictive \emph{von wegen} to be the starting point of its
diachronic development towards the disagreeing interjection \emph{von wégen}. Let’s start with an
Early Modern High German\il{German!Early New High} instance of \emph{von wegen} as a head of a restrictive domain adjunct that looks quite similar to 13th century examples such as (\ref{ex-domainadj}) -- it provides a conceptual point of reference with regard to which the action or state being expressed in the proposition is restricted (example (\ref{ex-restrictive}) is taken from a letter by Heinrich of Puchheim to the court chancellery of the Holy Roman Emperor Maximilian I):

\eanoraggedright\label{ex-restrictive} Restrictive prepositional domain adjunct with propositional scope (1477, Geschichte des Hauses Habsburg II, Doc. No. 116; cf.\ \citealp[304]{Chmel1855}):\\
Als dann sein k. gnad \textbfemph{von wegen des umgeltz zu Liechtmwerd} ausrichtung und raittung begert, nun hab ich meinen brief und urkunnd damit ich sein k. gnad desselben hanndels unterrichtung mocht yetz nicht bey hanndn \\
	`As then his Imperial Highness is asking for a notification and accounting \textbfemph{regarding the dues of Lichtenwerd}, as a matter of fact, I don’t have my record and document at hand by which I intended to inform his royal grace about this business'
\z

\noindent
In example (\ref{ex-restrictive}), the proposition of the host clause is restricted to all contextually relevant matters that relate to ``the dues of Lichtenwerd'', and the subsequent clause is oriented towards this restriction (cf.\ \emph{desselben hanndels }‘this business’). Just like in example (\ref{ex-domainadj}), the grammatical shape of the complement of \emph{von wegen} indicates accessibility (cf.\ the definite article \emph{des} in \emph{des umgeltz zu Liechtmwerd}). This is because (\ref{ex-restrictive}) is part of a reply to a preceding letter by the court chancellery of Maximilian I that included the following passage:

\eanoraggedright\label{ex-context} Context of example (\ref{ex-restrictive}) (1477, Geschichte des Hauses Habsburg II, Doc. No. 15; cf.\ \citealp[302]{Chmel1855}):\\[2pt]
Item so sol er [= Heinrich of Puchheim, J.B.] seinen kaiserlichen gnaden ierleich von ungelt zu Liechttenwerde 8 pfunt pfenning geben die er lannge zeit nit geraicht hat, begert sein kaiserlich gnad daz er dauon raittung tu und was er mit raittung dauon schuldig wirdet daz er das seinen kaiserlichen gnaden ausrichtte und gebe.\\
	`Item he [= Heinrich of Puchheim, J.B.] is obliged to give his Imperial Highness 8 pound of pfennigs of the dues of Lichtenwerd each year which he has not been doing for a long time now, his Imperial Highness asks that he gives an account hereof, and about what he is owing according to the account, he shall inform his Imperial Highness and give it to him.'
\z

Example (\ref{ex-restrictive}) and its context (\ref{ex-context}) show us the following:

\begin{itemize}
\item The restrictive \emph{von wegen} can recycle linguistic material from the preceding verbal context in its complement position (cf.\ \emph{ungelt zu Liechttenwerde} in example (\ref{ex-context}) and the repetition \emph{umgeltz zu Liechtmwerd} in example (\ref{ex-restrictive})).

\item The recycling can be done to identify an accessible speech act in the context (cf.\ the request by the court chancellery of Maximilian I in example (\ref{ex-context})) in order to prepare a reaction to it (cf.\ the answer to the request by Heinrich of Puchheim in example (\ref{ex-restrictive})).
\end{itemize}

\noindent
With regard to the addressee of  example (\ref{ex-restrictive}) (i.e., the court chancellery of Maximilian I), the prior speech act in (\ref{ex-context}) can be assumed to be accessible, but not active for three reasons. First, an exchange of letters is a temporally ``stretched'' way to communicate, i.e. the letter with the request and the letter with the answer follow each other with significant delay. Second, a reply to a big court chancellery has to take into consideration that the chancellery is concerned with a multitude of different issues and thus needs hints to the specific background of the reply. Third, the two letters do not only consist of the request and the answer that are cited in (\ref{ex-restrictive})  and (\ref{ex-context}). They also deal with other topics and issues.

For these reasons, the request for the account has to be reactivated before it can be answered. This is achieved by the host clause of the \emph{von wegen} PP, but not by the \emph{von wegen} PP itself -- note that the \emph{was}--clause does not yet express a reaction but only prepares it, while the subsequent claim starting with \emph{nun hab ich meinen brief und urkunnd} […] is the reaction. Obviously, the \emph{von wegen} PP does not establish its syntactic host as a speech act reacting to a preceding speech act but just modifies it with propositional scope in terms of a domain adjunct.

\largerpage[-1]
While the \emph{von wegen} PP in  (\ref{ex-restrictive}) clearly has a propositional scope, the following example shows that specific contexts can allow for an interpretation of \emph{von wegen} either with propositional scope or with speech act scope (cf.\ example (\ref{ex-restprep}) from a synopsis of a letter from the Styrian administrative district to the court chancellery of Maximilian I):

\eanoraggedright\label{ex-restprep}  Restrictive prepositional domain adjunct with ambiguous scope (1479, Geschichte des Hauses Habsburg III, Doc. No. 144; cf.\ \citealp[331]{Chmel1858}):\\[2pt]
Item \textbfemph{von wegen der lehen} ist der lanndschaft antwurtt, daz sy getrawn seinen k.\,g.\ trew und gewerttig allzeit gewesen, und sich gehalten als die trewn unndertanen und auch lehenslewt.\\
	`Item \emph{regarding the fiefdom} is the answer of the district, that they have always been loyal and subservient to his Imperial Highness and acted like loyal subjects and vassals as well.'
\z

\noindent
The answer that is referred to in this synopsis was a reaction to a prior inquiry by Maximilian I (cf.\ (\ref{ex-contextrestprep})):

\eanoraggedright\label{ex-contextrestprep} Context of example (\ref{ex-restprep})
%\itdopt{context of example (\ref{ex-restprep})? It is also (\ref{ex-restrictive}) in the word file.}
(1479, Geschichte des Hauses Habsburg III, Doc. No. 144; cf.\ \citealp[330]{Chmel1858}):\\[2pt]
Und wolt sein k. gnad gern ain wissen von in [= the residents of the Styrian administrative district, J.B.] haben, was die dienst wern, die sy seinen k. gnaden von der lehen wegen zu tun schuldig sein.\\
	`And his Imperial Highness would like to know from them [= the residents of the Styrian administrative district, J.B.] what the services were that they owe his Imperial Highness regarding the fiefdom.'
\z

\noindent
Just like in (\ref{ex-restrictive}) and (\ref{ex-context}), the \emph{von wegen} PP in
(\ref{ex-restprep}) recycles linguistic material from a relevant part of the context (cf.\ example
(\ref{ex-contextrestprep})) in its complement position as a part of a relationship between an
accessible prior speech act (a request) and a corresponding reaction (an answer). However, there are
at least two important differences. First, the restrictive \emph{von wegen} PP in
(\ref{ex-restprep}) occupies the front field of its host clause and is preceded by the topic-changing \ili{Latin} particle \emph{item}. Second, the host clause of the \emph{von wegen} PP in  (\ref{ex-restprep}) is a reaction to the reactivated speech act. This is regularly the case with combinations of \emph{von wegen} PPs with host clause matrix predicates denoting a verbal or mental activity that is or at least could be a reaction to a prior speech act (e.g., verbs of speaking or thinking and related constructions).

Taken together, this leads to ambiguity because \emph{von wegen} can now be interpreted as providing a restrictive point of reference either for the host clause proposition (= [F [\emph{von wegen} NP [p]]]) or for the host clause as a full-fledged speech act (= [\emph{von wegen} NP [F [p]]]).\footnote{``F'' symbolizes the illocutionary force, ``p'' the propositional content of a given speech act \citep[cf.][31]{Searle1969}.} The latter interpretation is supported by the front field position of the \emph{von wegen} PP after \emph{item} that establishes it as a wide-ranging part of the background.

Example (\ref{ex-restprep}) shows that the use of \emph{von wegen} as a head of a restrictive domain adjunct results in scope ambiguities when (a) its complement contains linguistic material that appears to be recycled from a prior speech act and (b) its host clause shows features of a full-fledged reaction to this speech act. In this respect, examples such as (\ref{ex-restprep}) represent an important ``critical context'' \citep[cf.][]{Diewald2002} in which a new interpretation of \emph{von wegen} -- an interpretation as a head of a restrictive domain adjunct with an increased speech act scope – becomes available and plausible, albeit not obligatory yet.

A crucial ``isolating context'' \citep{Diewald2002} in which \emph{von wegen} can only be interpreted as a head of a domain adjunct with speech act scope is its use as a ``hanging topic'' in the sense of \citet{Altmann1981}. Example (\ref{ex-speechact}) from a letter to the mayor of Basel shows such a case:

\eanoraggedright\label{ex-speechact} Restrictive prepositional domain adjunct with speech act scope (1529, Geschichte der Basler Reformation IV, Doc. No. 36, lines 9--15; cf.\ \citealp[34]{Roth1941}):\\[2pt]
Aber \emph{von wegen des Mertzen und anderer prediger munch, die sich zu Gebwyler und an andern
  orten inn unser verwaltigung enthalten sollen}, des haben wir bitz auff das obgemelt ewer
schreyben dheyn wussen gehept; [wir, J.B.] wussen auch noch nit, wa oder an welchen enden sich die
gemelten prediger inn unser verwaltigung enthalten sollen. Das\footnote{The complementizer
  \emph{dass} seems to be used conditionally here, which was possible in Early Modern High
  German\il{German!Early New High} \citep[cf.][821]{GrimmGrimm1860}. If so, this would be an early instance of a so-called “relevance (speech act/utterance/pragmatic/biscuit) conditional” restricting not the propositional validity conditions of the matrix clause consequent but its relevance as a speech act in the given line of actions \citep[e.g.][]{Austin1956, Sweetser1990,Guenthner1999}.}
aber der Mertz zů Gebwyler sein solle, das ist nit inn unser Verwaltung, deshalben wir uns auch desselben nit beladen. \\
	`But \emph{regarding this Mertz and other preacher monks who are supposed to stay in Guebwiller and elsewhere in our district}, we had no knowledge of this up to your letter; [we, J.B.] also didn’t know where or in which parts the above-mentioned preachers are supposed to stay in our district. If this Mertz should be in Guebwiller, though, that is not within our district, therefore we will not deal with this issue.'
\z

\noindent
In example (\ref{ex-speechact}), the \emph{aber}-prefaced \emph{von wegen} PP reestablishes a request for administrative cooperation concerning the recovery of stolen goods in a prior letter (see Geschichte der Basler Reformation IV, Doc. No. 25, lines 40--27; cf.\ \citealt[26--27]{Roth1941})\footnote{Since the pragmatic operation of reactivating a speech act from a prior letter should have become obvious enough from the preceding examples, I will do without quoting the full context in the remainder of this paper.}  
%\itdopt{bib einfügen} 
in order to prepare its matrix clause as a reaction (= [\emph{von wegen} NP [F [p]]]). An interpretation of \emph{von wegen} with a narrow propositional scope over the subsequent clause (= [F [\emph{von wegen} NP [p]]]) is not possible, in contrast. Example (\ref{ex-speechact}) also shows that the use of restrictive \emph{von wegen} PPs as a hanging topic affects the referential characteristics of the complement: The NP \emph{des Mertzen und anderer prediger munch} [\ldots] `this Mertz and other preacher monks [\ldots]' clearly contributes to the function of \emph{von wegen} by means of its pragmatically established reference to a prior speech act in the context, not by means of its semantically established reference to a group of certain individuals. We should bear this in mind as it is important for the decategorization of the complement position of \emph{von wegen} (see Subsection~\ref{ssec-mhd-vw}).

\subsection{The isolation, decategorization and renewal of \emph{von wegen} in Modern High German}\label{ssec-mhd-vw}

The previous section has shown that in the 15th and 16th centuries, the restrictive \emph{von wegen}
underwent a scope extension and came to be used as a hanging topic in order to reestablish an
accessible but inactive speech act for its host as a reaction to it. This already bears obvious
resemblances to present-day \ili{German} examples with \emph{von wégen} reestablishing a non-adjacent
prior speech act in order to challenge it (cf.\ example (\ref{ex-interjec132}) on p.\,\pageref{ex-interjec132}). However, there are still crucial differences between the Early Modern High German\il{German!Early New High} hanging topic \emph{von wegen} on the one hand and present-day \ili{German} \emph{von wégen} as an emphatic interjection of disagreement on the other:

\begin{itemize}
\item The Early Modern High German\il{German!Early New High} preposition \emph{von wegen} is restricted to case-governed NPs (e.g., \emph{von wegen der lehen} `regarding the fiefdom' in example (\ref{ex-restprep})), while the present-day \ili{German} interjection \emph{von wégen} can be accompanied by linguistic material of any category (e.g., \emph{von wégen wir horten} `my foot we’re hoarding' in example (\ref{ex-interjec132})) and cannot assign a case to associated NPs anymore (NPs receive the nominative instead as a default case, cf.\ \emph{von wégen tolles Wetter} versus *\emph{von wégen tollem Wetter} or *\emph{von wégen tollen Wetters}).

\item The Early Modern High German\il{German!Early New High} preposition \emph{von wegen} must come with a complement and a host clause, while the present-day \ili{German} interjection \emph{von wégen} is inherently independent of both (cf.\ the examples in Section~\ref{sec-pres-vw}).

\item The Early Modern High German\il{German!Early New High} preposition \emph{von wegen} establishes its host clause as a reaction to a preceding speech act, whereas the disagreeing interjection \emph{von wégen} is a full-fledged reaction itself.
\end{itemize}

\largerpage
\noindent
Obviously, we are still only halfway between the 14th century domain adjunct \emph{von wegen} with propositional scope on the one hand and the present-day \ili{German} interjection \emph{von wégen} on the other. There are still some steps of change missing, in particular the structural loss of constraints on the prepositional complement position (up to the point where a complement position in a strict sense no longer exists) and the functional renewal as an emphatic expression of disagreement.

The loss of categorial constraints on the complement position was probably a result of its functional reduction to a quotative link to a preceding speech act when the \emph{von wegen} PP occupied the pre--front field as a hanging topic. Since this function did not require the referential semantics of a noun, other linguistic items than nouns became possible as heads of the complement (cf.\ the following 18th century example):

\eanoraggedright\label{ex-restsemi} Restrictive semi-prepositional domain adjunct with an AdvP as complement \citep[120–121]{Goethe1767}:\\[2pt]
Du hättest immer schweigen können, daß du drüben zu früh angekommen bist, es hilft uns nichts und ärgert uns nur; besonders den Horn, dem es unaufhörlich im Kopfe liegt daß du nicht noch hinunter gegangen bist. Apropos \textbfemph{von wegen unten}. Der Hr. Langer ist der Mutter und Tochter ums Tohr begegnet [\ldots] \\
	`You should have withheld that you arrived too early over there, it does not help us but just annoys us; especially Horn who constantly wonders why you did not come downstairs. Apropos \emph{regarding downstairs}. Mr. Langer met the mother and the daughter at the gate [\ldots]'
\z

\largerpage[-1]
\noindent
In example (\ref{ex-restsemi}), \emph{von wegen} takes the AdvP \emph{unten} `downstairs' as its complement. \emph{Unten} refers back to the verb in the preceding clause as its antecedent (cf.\ the past participle of \emph{hinuntergehen} `to go downstairs'), and \emph{apropos von wegen unten} can be analyzed as a ``modificative complex'' (`modifikativer Komplex', \citealp[1167–1172]{Zifonun1997})
%\itdobl{Stefan: Genaue Seitenangabenvon--bis.}
with \emph{apropos} as the head and \emph{von wegen unten} as a restrictive modifier (= [\emph{apropos} [\emph{von wegen unten}]]).

Example (\ref{ex-restsemi}) shows that in the 18th century at the latest, the complement position of \emph{von wegen} had lost its categorial restriction to NPs. This implies a substantial decategorization of \emph{von wegen}: Since it can take phrases of all categories as its complement, it is not a core member of the class of prepositions anymore. However, the loss of the categorial restriction to NPs does not lead to the interjection \emph{von wégen} directly. It instead took some time until the non-prepositional use of \emph{von wegen} became common enough to result in a complete categorial split between a case-assigning prepositional \emph{von wegen} on the one hand and a non-prepositional \emph{von wegen} on the other that could not assign a case anymore. In fact, it was not until the second half of the 19th century that \emph{von wegen} was mentioned in grammars and dictionaries as an independent non-prepositional expression of disagreement. For example, the ``hennebergisches Idiotikon'' describes \emph{von wegen} as a stand-alone reply accompanied by emphatic dissent, cf.\ \emph{ja, von wegen! dabei walten noch ganz andere Gründe ob, sind noch ganz andere Dinge zu bedenken, wenn’s wahr ist! das geht so nicht! das ist so nicht gemeint (mein Lieber!)} `well, my foot! there are also completely different reasons and issues involved, if that is true! that is not ok! that is not what I meant (my dear fellow!)' \citep[cf.][271–272]{Spiesz1881}. Similarly, \citet{Meyer1880} and \citet{Brendicke1897} mention a spirantized stand-alone \emph{von wejen!} as a common part of the urban vernacular of Berlin (``\ili{Berlinish}'').\footnote{Cf.\ \emph{Na von wejen – !} `well my foot' \citep[89]{Meyer1880} and \emph{von wejen! ach so! das ist nichts.} `my foot! I see! that is nothing.' \citep[190]{Brendicke1897}. See also \citet{Schlobinski1988}, \citet{SchoenfeldSchlobinski1998,Freywald2017} on ``\ili{Berlinish}''.}  The fact that \emph{von wegen} not only leaves the class of prepositions but simultaneously enters the class of interjections has at least two reasons: Interjections tend to occupy a syntactic position in the left periphery of utterances \citep[e.g.][31]{Nuebling2004}, and they can combine with casus rectus-NPs (e.g., \emph{Oh diese Philologen!} `Oh those philologists!', see \citealp[321–322]{Fries1992}).
%\itdopt{bib einfügen}

The reanalysis of \emph{von wegen} as an interjection cleared the formal way for the functional renewal of \emph{von wegen} as a full-fledged marker of disagreement. The relevant context of this renewal was the recurring use of \emph{von wegen} as a means to reestablish a prior speech act in order to challenge it by means of a disagreeing reaction. Such constellations already occurred in the Early Modern High German\il{German!Early New High} period (cf.\ example (\ref{ex-restscope})):

\eanoraggedright\label{ex-restscope} Restrictive prepositional domain adjunct with speech act scope (1532, Geschichte der Basler Reformation VI, Doc. No. 202, lines 22--25; cf.\ \citealp[161]{Roth1950}):\\
%\itdopt{bib einfügen}
%\begin{quote}
\textbfemph{Aber von wegen der XXXII} [\textbfemph{Kronen}, J.B.] \textbfemph{solden, so dem houptman sollen noch uszstan}, daran tragend wir dhein schuld, dann wir haben alle monat unnsere XI. [Kronen, J.B.] sold vollig abgericht, daran nut uffgeschlagen.\\
`\textbfemph{But regarding the XXXII} [\textbfemph{Kronen}, J.B.] \textbfemph{pay that are to be due to the bailiff}, we are not responsible for this as we delivered our XI. [Kronen, J.B.] pay completely every month, did not delay in that.'
%\end{quote}
\z

\noindent
Example (\ref{ex-restscope}) shows a prepositional hanging topic with \emph{von wegen} reestablishing a prior claim that had been raised in order to challenge its validity in the subsequent host clause. It arose in reaction to an attempt of the bailiff (\emph{Hauptmann}) of Zurich to gain remaining payments from Basel.

\largerpage
In contexts such as (\ref{ex-restscope}), it became possible to reanalyze the contextually emerging expression of disagreement as an inherent function of \emph{von wegen}. However, the inability of \emph{von wegen} as a preposition to express a full-fledged stance blocked this reanalysis, until the interjection \emph{von wegen} emerged that was independent enough to be resemanticized towards a full-fledged expression of disagreement.

After \emph{von wegen} was fully established as a disagreeing interjection without a complement position that had to be filled for grammatical reasons, it could be used without accompanying quotative material in direct reaction to a speech act that was still contextually active (cf.\ (\ref{ex-interjection}) on p.\,\pageref{ex-interjection}). This was important for the final step of the development – the lexicalization of the strong accent on the disagreeing interjection \emph{von wegen}. As we have seen in Section~\ref{sec-pres-vw}, the present-day disagreeing \emph{von wégen} only forms a prosodic unit with quotative material occupying its former complement position. If such material is missing, there is a more or less clear prosodic ``caesura'' \citep{Auer2010,Barth-Weingarten2016}
between \emph{von wégen} and the subsequent part of the utterance -- a caesura that is inherited from the prosodically independent hanging topic use of its (semi-)prepositional predecessor \emph{von wegen} in the left periphery of their syntactic hosts.\footnote{Cf.\ \citet{Selting1993} on the prosody of hanging topics in everyday spoken \ili{German}. Even though we do not have direct access to prosodic features of written historical data, the assumption that hanging topics had a substantial degree of prosodic independence not only in present-day Modern High \ili{German} but in Early Modern High German\il{German!Early New High} and in earlier stages of Modern High \ili{German} as well does not seem to be too daring.}
Against this background, the extra-strong accent on \emph{von wegen} probably emerged as follows:

\begin{itemize}
    \item When used as a direct reaction to a prior action, \emph{von wegen} forms an intonation phrase with a clear caesura after \emph{wegen} and the initial syllable of the trochee \emph{wegen} as the lexically and syntactically fixed position for a prominent emphatic accent of the phrase.
    
    \item As direct challenges of a prior action tend to be emphatic by nature, the accent on \emph{wegen} was consistently intensified and finally reanalyzed as an inherent prosodic feature of \emph{von wegen} (in terms of \emph{von wégen}).
\end{itemize}

\largerpage
\noindent
Accordingly, the prosodically prominent \emph{von wégen} can be assumed to be the most recent innovation within the development of \emph{von wegen}. Given that the earliest instances of the semi-prepositional \emph{von wegen} occurred in the second half of the 18th century and that the stand-alone disagreeing \emph{von wégen} is mentioned for the first time in dictionaries of the second half of the 19th century, the reanalysis of \emph{von wégen} probably took place at some point in the first half of the 19th century.

\section{Complexity of change as the reason why P$^0$ > INT$^0$ is less common than PP > INT$^0$ in German}\label{sec-complex-vw}

We have seen in Section~\ref{sec-past-vw} that the emphatic disagreeing interjection \emph{von wégen} is the result of four major reanalytic steps. The first step affected the Early Modern High German\il{German!Early New High} restrictive preposition \emph{von wegen}, extended its scope, and made it possible to use its maximal projection as a hanging topic in the pre-front field. The second step removed the restriction to case-governed NPs as complements (= semi-prepositional \emph{von wegen}), while the third step removed the complement position entirely (= non-prepositional \emph{von wegen}), yielding an interjection that adopted the function to express disagreement from a recurrent context of use. The final step brought out the stressed \emph{von wégen}.

This development clearly does not represent a case of grammaticalization in the traditional sense (``grammaticalization I'', \citealp{Wischer2000})
as none of the reanalytic steps increase the ``grammaticity'' of the item they affect (e.g., by means of attrition, paradigmaticization, obligatorification, condensation, coalescence and fixation; cf.\ \citealp[174]{Lehmann2015}).\footnote{See also \citet{Norde2012} on Lehmann’s parameters. \citet{Hopper1991,Hopper1996}
proposes five alternative parameters (or ``principles'') of grammaticalization (layering, divergence, specialization, persistence, de-categorialization), while \citet{Himmelmann2004}
criticizes the ``box metaphor'' lying behind many traditional structural approaches to
grammaticalization and lexicalization.}  Instead, the development can be related to two tendencies
of semantic\hyp pragmatic change discussed in \citet{TraugottKoenig1991} that are brought about by metonomy (contextual contiguity) rather than metaphor:\footnote{In former studies, Traugott distinguishes between ``propositional'', ``textual'' and ``expressive'' aspects of meaning \citep[cf.][]{Traugott1982}, while in more recent studies, Traugott prefers the cline ``non-/less subjective > subjective > intersubjective'' and holds that subjectification and intersubjectification as processes of linguistic change are inherently independent of grammaticalization (e.g., \citealp{Traugott2010}; see also \citealp[57–59]{Brinton1996}).
}
 
\begin{itemize}
    \item The first two reanalytic steps yielded an item that no longer contributes to the host clause proposition but to coherence in terms of the dialogical well-placedness of the host clause as a speech act. This corresponds to \citeauthor{TraugottKoenig1991}'s (\citeyear[208]{TraugottKoenig1991}) ``semantic-pragmatic tendency II'': ``Meanings based in the described external or internal situation > meanings based in the textual situation''.
    
    \item The last two steps yielded an item that expresses affect and stance concerning a contextually accessible and relevant speech act. This corresponds to \citeauthor{TraugottKoenig1991}'s (\citeyear[209]{TraugottKoenig1991}) ``semantic-pragmatic tendency III'': ``Meanings tend to become increasingly situated in the speaker’s subjective belief-state/attitude toward the situation''.
\end{itemize}

\noindent
Accordingly, we can consider the semantic-pragmatic development of \emph{von wégen} a case of change
along the frequently discussed diachronic path ``propositional/""descriptive meaning >
text-/discourse\hyp structuring meaning > affect-/""stance\hyp related meaning''. Of course, this does not mean that the diachronic emergence of \emph{von wégen} is a purely semantic and pragmatic change. As we have seen above, it involves substantial changes of the status, category and features of \emph{von wegen} as a head over time (cf.\ Subsections~\ref{ssec-frnhd-scope-vw} and~\ref{ssec-mhd-vw}). In general, heads and headedness can change in three different ways:%
\footnote{See \citet{Zwicky:1985:heads,Zwicky1993,Hudson1987,Croft1996} on the concept of headedness in linguistics. It is understood that the details of the three types of change that are distinguished here will differ substantially depending on the grammatical theory that is assumed.}

\begin{enumerate}
    \item The grammatical status of a linguistic item changes either from head to non-head (e.g. phrase) or vice versa (= head--status change). Recently, van Gelderen has argued in the theoretical framework of Minimalism that speakers tend to construct constituents as heads rather than as phrases due to processing economy (``Head Preference Principle (HPP)'', cf.\ \citealp[13–14]{VanGelderen2011}).
%    \itdobl{Stefan: Genaue Seitenangabenvon--bis.}%\itdopt{bib einfügen}
    According to van Gelderen, this tendency can also be observed in grammaticalization.%
\footnote{Van Gelderen discusses the grammaticalization of the pronominal specifier \emph{that} as the head of a complementizer phrase, for instance.}
    In the case of \emph{von wégen}, only head-status change from head to non-head played a marginal role because the reanalysis of \emph{Wegen} (= dative plural form of \emph{wec} `side') as a part of the discontinuous complex head of the circumposition \emph{von – wegen} – the predecessor of \emph{von wegen} as a continuous complex preposition (cf.\ Subsection~\ref{ssec-frnhd-vw}) – implies a loss of the head-status: As a noun in the complement position of \emph{von}, \emph{Wegen} was the lexical head of an NP, but as a part of the circumpositional head of \emph{von – wegen}, it became a head segment that was not a head in its own right anymore. Of course, this holds for \emph{von} as well when it was merged into the circumpositional head in terms of a head segment. However, since the emergence of the circumposition \emph{von – wegen} is not directly linked to the diachronic emergence of \emph{von wégen}, we can ignore this in the following.
    
    \item	The head-status of a linguistic item remains but its category changes (= head-category change). Head-category change has two aspects: In terms of its source (or linguistic input), it is decategorization as a certain head X$^0$, and in terms of its target (or linguistic output), it is recategorization as a certain head Y$^0$.\footnote{This sets head-category change apart from head-status change, which can potentially lead to categoryless ``junk'' (\citealp{Lass1990}, see also \citealp{Simon2010}). Head-category change, in contrast, cannot lead into a linguistic wasteland of categorylessness just by definition.}
    In this study, the de- and recategorization of the preposition \emph{von wegen} as the interjection \emph{von wégen} involves head-category change in terms of [\sub{PP} \textbf{P$^0$} NP] > [\sub{INTP} \textbf{INT$^0$} XP], if we follow \citeauthor{Fries1992}' (\citeyear{Fries1992}, \citeyear{Fries22}) concept of an interjection phrase (INTP). However, we have to bear in mind that [\sub{PP} P$^0$ NP] did not lead to [\sub{INTP} INT$^0$ XP] directly but via two intermediate steps – the semi-prepositional \emph{von wegen} and the interjection \emph{von wegen} without lexicalized stress (cf.\ Subsection~\ref{ssec-mhd-vw}). This shows that headedness can be a matter of degree (e.g., in terms of prototypical and peripheral types of P$^0$ and INT$^0$; see point 3 below) and that additional reanalytic steps can precede and follow head-category change (e.g., P$^0$\sub{prototypical} > \textbf{P$^0$\sub{peripheral} > INT$^0$\sub{peripheral}} > INT$^0$\sub{prototypical}; only the head-category change is marked bold). This leads us to the third principal type of change that can affect heads and headedness.
    
    \item The features of a linguistic item as a head change while its head-status and its category remain (= head-feature change). One case of head--feature change is the reduction of grammatical restrictions on the complement position that affected the preposition \emph{von wegen} (loss of a restriction to NPs, cf.\ P$^0$\sub{prototypical} > P$^0$\sub{peripheral}). When this reached the point where an identifiable complement position did not exist anymore, the category of the head changed from P$^0$ to INT$^0$ (see point 2 above). Another important head-feature change affected the scope of \emph{von wegen} as a domain adjunct: As we have seen, the early instances had scope over the host clause proposition only and it took a reanalytic step to develop the capability to take scope over the host clause as a full-fledged speech act (in terms of [F [\emph{von wegen} NP [p]]] > [\emph{von wegen} NP [F [p]]], cf.\ Subsection~\ref{ssec-frnhd-scope-vw}). Furthermore, the lexicalization of the emphatic accent on the interjection \emph{von wégen} can be regarded a case of head-feature change (cf.\ Subsection~\ref{ssec-mhd-vw}). At first sight, this feature does not seem to play a significant role for the headedness of \emph{von wégen}. However, if we follow \citet[18]{Nuebling2004} and consider strong stress to be a feature of prototypical interjections, the lexicalization of the emphatic accent reflects development towards the core of the category ``interjection'' (in terms of INT$^0$\sub{peripheral} > INT$^0$\sub{prototypical}), just like the reduction of grammatical restrictions on the complement position reflects development towards the periphery of the category “preposition”.
\end{enumerate}

\noindent
The head-feature changes mentioned here arose substantially from ambiguous contexts of use in which
conventional and nonconventional meaning aspects of a linguistic item could not be separated from
each other, and they resulted in semantic enrichment of this item (in terms of occasion becoming
convention): The quotative reactivation of a prior speech act, the increased scope, the expression
of disagreement and the presence of an emphatic accent all started as occasional side effects in
certain contexts of use and became conventional parts of the \emph{von wegen} items they are related
to over time. This can be classified as hypoanalysis in the sense of
\citet[126–130]{Croft2000}.%\itdopt{bib einfügen}
%\itdobl{Stefan: Genaue Seitenangaben von--bis.}

Considering now what we found out about the structural and semantic\hyp pragmatic changes that led to \emph{von wégen} and returning to our initial question of why the interjectionalization path PP > INT$^0$ with head-status change is much more common in \ili{German} in comparison to the path P$^0$ > INT$^0$ with head-category change, the main reason seems to be the striking complexity of change it requires:

\begin{itemize}
\item	Structurally, a prepositional head has to get rid of its complement position before it can become an interjection in terms of P$^0$ > INT$^0$. The case of \emph{von wégen} has shown, however, that it takes a massive reanalytic effort in very specific contexts (e.g., formal isolation as a hanging topic with a quotative complement) to remove the complement position of a fully grammaticalized prepositional head entirely (note that the semi-prepositional \emph{von wegen} could not be reanalyzed as an interjection yet; cf.\ Subsection~\ref{ssec-mhd-vw}). If such contexts are not given (and they are by no means taken for granted), the complement position will not drop and the prepositional head alone cannot be de- and recategorized as an interjection. A PP such as \emph{zum Teufel} `to the devil', in contrast, is not inherently tied to fixed accompanying syntactic material that needs to be removed before it can be reanalyzed in terms of PP > INT$^0$.

%\largerpage
\item Semantically and pragmatically, \ili{German} prepositional heads do not have an inherent affective
  and stance-related meaning that is directly qualified for use as an interjection, while PPs –
  especially ones with nominal complements from sacral and profane domains of the lexicon – can
  be emphatic expressions of affect and stance in their own right. Since prepositional heads have to
  acquire a completely new meaning in order to become an interjection, they are maximally dependent
  on dialogical contexts providing the possibility to adopt such a meaning. The case of \emph{von
    wégen} shows that this implies a long and complex way through the above-mentioned path
  ``propositional/descriptive meaning > text-/""discourse\hyp structuring meaning > affect-/stance-related meaning''. A PP such as \emph{zum Teufel} `to the devil', in contrast, is less dependent on such a context-driven import of meaning, and its semantic and pragmatic path towards a use as an interjection can be assumed to be shorter and less complex.
\end{itemize}

\noindent
Taken together, it is obviously the high complexity and context-dependency of changes that prevents a frequent development of new interjections from prepositional heads in \ili{German}. Full PPs, in comparison, can be reanalyzed much more easily as interjections as they require fewer changes and fewer contexts giving rise to these changes. However, if both the linguistic system and the contextual circumstances of language use provide an opportunity, the path P$^0$ > INT$^0$ is possible and can yield interjections such as \emph{von wégen} that are able to become a lasting part of the language.

\section{Summary}\label{sec-summ-büc}
The starting point of this paper was the observation that \ili{German} deprepositional interjections usually arise from full PPs (PP > INT$^0$) and not from prepositional heads alone (P$^0$ > INT$^0$), the present-day \ili{German} interjection \emph{von wégen} ‘my foot’ being an exception that traces back to the Early Modern High German\il{German!Early New High} prepositional head \emph{von wegen}. In order to find out how an isolated prepositional head can be reanalyzed as an interjection, we took a look at the forms and functions of \emph{von wégen} in present-day \ili{German} spoken data in Section~\ref{sec-pres-vw} and then reconstructed the diachronic steps of its development, beginning with the rise of the prepositional \emph{von wegen} in Section~\ref{sec-past-vw}. After that, we arrived at some observations and considerations on why a development along the path P$^0$ > INT$^0$ is far less frequent in \ili{German} in comparison to the alternative path PP > INT$^0$ in Section~\ref{sec-complex-vw}: Even though each step of the diachronic emergence of \emph{von wégen} for itself reflects common mechanisms and directions of semantic-pragmatic and structural change, the development as a whole was strikingly complex. Most of the reanalytic complexity has to do with the presence of a complement position and the absence of an affective and stance-related meaning that characterizes P$^0$: The removal of this position and the adoption of a new meaning required, so to say, hard reanalytic work in highly specific contexts. Full PPs, in contrast, can be reanalyzed as interjections much more easily as they are not inherently connected to fixed accompanying linguistic material that has to be removed and can come with complements from lexical domains that promote the use as an interjection (e.g., sacral and profane nouns).

\section*{\acknowledgmentsUS}
I would like to thank the editors of this volume and the two anonymous referees for many helpful comments and suggestions. Thanks to Daniel Ross for proofreading.

{\sloppy
\printbibliography[heading=subbibliography,notkeyword=this]
}
\end{document}