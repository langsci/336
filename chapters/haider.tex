%% -*- coding:utf-8 -*-
\documentclass[output=paper
  ,nobabel
  ,uniformtopskip % manual adjustment of pagebreaks
%  ,draftmode
%  ,colorlinks, citecolor=brown
]{langscibook}

\ChapterDOI{10.5281/zenodo.7142712}

\IfFileExists{../localcommands.tex}{%hack to check whether this is being compiled as part of a collection or standalone
   \usepackage{orcidlink}

% add all extra packages you need to load to this file


% Haitao Liu
%\usepackage{xeCJK}
%\setCJKmainfont{SimSun}
%\setCJKmainfont[Scale=MatchUppercase,
%                Path=fonts/
%]{SourceHanSerifSC-Regular}

% instead use option:  ,chinesefont % for references in raffelsiefen.tex
% loading the package changes some spacings


\usepackage{multicol}
\usepackage{tikz}\usetikzlibrary{decorations.pathreplacing}
\usepackage{url}
\urlstyle{same}

%\usepackage{listings}
%\lstset{basicstyle=\ttfamily,tabsize=2,breaklines=true}

\usepackage{langsci-basic}
\usepackage{langsci-optional}
\usepackage[danger]{langsci-lgr}

% toggle danger in texlive 2021
%\newcommand{\M}{\textsc{m}\xspace}

% toggle danger in texlive 2021 or uncomment this
% \newcommand{\N}{\textsc{n}\xspace}
% \newcommand{\F}{\textsc{f}\xspace}


\usepackage{./styles/biblatex-series-number-checks}




\usepackage{langsci-gb4e}




% Demske

\usepackage{tipa}
\usepackage{styles/avm+}
%\usepackage{styles/merkmalstruktur}
\avmfont{\sc}
\usepackage{langsci-forest-setup}
\usepackage{xspace}
%\usepackage{styles/abbrev} 

\usepackage{soul}
\usepackage{color}
\newcommand{\rem}[1]{\textcolor{red}{\st{#1}}}
\newcommand{\add}[1]{\textcolor{blue}{\ul{#1}}}


% Salzmann

\usepackage[nocenter]{qtree}


% Müller

% add this to the default preamble 
\forestset{default preamble={
    for tree={anchor=north},
}}


\usepackage{german}

%\usepackage{german}
\selectlanguage{USenglish}

% Mit Babel geht irgendwie die hyphenation nicht richtig
%\usepackage[ngerman,english]{babel}
%\useshorthands{"} 
%\addto\extrasenglish{\languageshorthands{ngerman}}

\usepackage{styles/makros.2020,
styles/abbrev,
styles/merkmalstruktur,
styles/article-ex,styles/eng-date}


\usepackage{todonotes}
\newcommand{\todostefan}[1]{\todo[color=green!40]{\footnotesize #1}\xspace}
\newcommand{\inlinetodo}[1]{\todo[color=green!40,inline]{\footnotesize #1}\xspace}

\newcommand{\inlinetodoopt}[1]{\todo[color=green!40,inline]{\footnotesize #1}\xspace}
\newcommand{\inlinetodoobl}[1]{\todo[color=red!40,inline]{\footnotesize #1}\xspace}

\newcommand{\itdobl}[1]{\inlinetodoobl{#1}}
\newcommand{\itdopt}[1]{\inlinetodoopt{#1}}

\newcommand{\addpages}{\todostefan{add pages}}

%\newcommand{\iaddpages}{\yel[add pages]{pages}\xspace}



% subfigure
\usepackage{subcaption}



% Nolda
%\usepackage[main=british,nil,german,french]{babel}
\newcommand{\foreignlanguagedummy}[2]{#2}
\usepackage{tagpair}
\usepackage{hang}
\usepackage[noconfig]{ntheorem}
\usepackage{pstricks,pst-node,pst-tree}
\usepackage{newunicodechar}



   \newcommand*{\orcid}[1]{}

% do not show the chapter number. It is redundant, since most references to figures are within the
% same chapter.
\renewcommand{\thefigure}{\arabic{figure}}

\newcommand{\rlapsub}[1]{\rlap{\sub{#1}}}

% \SetupAffiliations{output in groups = false, 
%                    separator between two = {\bigskip\\},
%                    separator between multiple = {\bigskip\\},
%                    separator between final two = {\bigskip\\}
%                    }


%%%%%%%%Alte Umlaute
\newcommand{\oldae}{$\stackrel{\textrm{\tiny e}}{\textrm{a}}$}
\newcommand{\oldoe}{$\stackrel{\textrm{\tiny e}}{\textrm{o}}$}
\newcommand{\oldue}{$\stackrel{\textrm{\tiny e}}{\textrm{u}}$}

\newcommand{\refl}{\REFL}
\newcommand{\pst}{\PST}


% Müller
\let\vref\ref


\let\citew\citet

\newcommand{\page}{}

% biblatex stuff
% get rid of initials for Carl J. Pollard and Carl Pollard in the main text:
\ExecuteBibliographyOptions{uniquename=false}




\newcommand{\nom}{\textsc{nom}}
\newcommand{\gen}{\textsc{gen}}
\newcommand{\dat}{\textsc{dat}}
\newcommand{\acc}{\textsc{acc}}


%\newcommand{\spacebr}{\hspaceThis{[}}



\newcommand{\acknowledgmentsEN}{Acknowledgements}
\newcommand{\acknowledgmentsUS}{Acknowledgments}


% no bf!!!111!
\let\textbfemph\emph

\newcommand{\textbfremoved}[1]{#1}
%\newcommand{\emphremoved}[1]{#1}


\newcommand{\noemph}[1]{#1}
\newcommand{\underlineemph}[1]{\emph{#1}}




% for editing, remove later
\usepackage{xcolor}
\newcommand{\added}[1]{{\red #1}}
\newcommand{\addedthis}{\todostefan{added this}}

\newcommand{\changed}[1]{\textcolor{orange}{#1}}




% Nolda

\theorembodyfont{\normalfont}
\let\restriction\relax
\renewtheoremstyle{break}{\item{\itshape ##1\ ##2}\newline\nopagebreak}{\item{\itshape ##1\ ##2\ (##3)}\newline\nopagebreak}
\theoremstyle{break}
\newtheorem{definition}{Definition}
\newtheorem{pattern}{Pattern}
\newtheorem{restriction}{Restriction}
\newunicodechar{‑}{\hbox{-}}
\newunicodechar{…}{\dots}
\newunicodechar{⁡}{\relax}
\newunicodechar{⁣}{\relax}
\newunicodechar{⁀}{\raisebox{+1ex}{\ensuremath\frown}}
\newunicodechar{⁐}{\raisebox{+1ex}{\ensuremath\frown}\setbox1=\hbox{\ensuremath\smile}\hspace{-\wd1}\raisebox{-1ex}{\ensuremath\smile}}
\newunicodechar{⪪}{\ensuremath{<\mathrel{\llap{\ensuremath{-}}}}}
\setkomafont{descriptionlabel}{\normalfont}
\ExecuteBibliographyOptions{labeldate=comp,labelnumber=true,defernumbers=true}
\defbibenvironment{sources}{\list{\printfield{labelprefix}\,\printfield{labelnumber}}{\settowidth{\labelwidth}{S\,0}\setlength{\labelsep}{\biblabelsep}\setlength{\leftmargin}{\labelwidth}\addtolength{\leftmargin}{\labelsep}\setlength{\itemsep}{\bibitemsep}\setlength{\parsep}{\bibparsep}}\renewcommand{\makelabel}[1]{##1\hfil}}{\endlist}{\item}
\newcommand{\citesource}[1]{\citefield{#1}{labelprefix}\,\citefield{#1}{labelnumber}}


% Was soll das machen?
\newcommand{\textstyleFootnoteSymbol}{}



% Was ist das???? St. Mü. 30.10.2021
%Kann weg. Damit waren die bücker transkripte aligniert. Habe das jetzt mit tabularx und hphantom gemacht

%\newlength{\calength} %tmp length to store the space 1. until [; 2. until ].
%
%%first argument speaker ID, second argument text. Optional argument left margin indicator (arrow or similar)
%\newcommand{\cabox}[3][]{\parbox{0mm}{\hspace*{-1cm}#1}%
%\parbox{1.5cm}{#2}%
%\parbox{9.6cm}{#3}\\%
%}
%
%%translation. First parbox is empty, second parbox takes the translation text
%\newcommand{\trsbox}[1]{\parbox{1.5cm}{~}%
%\parbox{9.6cm}{\itshape #1}\\%
%}
%
%%store the width of a string.
%\newcommand{\settablength}[1]{\settowidth{\calength}{#1}\global\calength=\calength}
%
%%print string and store its width. Useful if the first item of the aligned set is also the longest
%\newcommand{\inittab}[1]{#1\settablength{#1}}
%
%%insert horizontal white space equivalent to the stored width
%\newcommand{\skiptab}{\parbox{\calength}{~}}
%
%%print the argument and fill up with horizontal white space until the stored width is reached.
%\newcommand{\filledtab}[1]{\parbox{\calength}{#1}}



% for standalone compilations Felix: This is in the class already
%\let\thetitle\@title
%\let\theauthor\@author 
\makeatletter
\newcommand{\togglepaper}[1][0]{ 
\bibliography{../bib-abbr,../stmue,../localbibliography,
collection.bib}
  %% hyphenation points for line breaks
%% Normally, automatic hyphenation in LaTeX is very good
%% If a word is mis-hyphenated, add it to this file
%%
%% add information to TeX file before \begin{document} with:
%% %% hyphenation points for line breaks
%% Normally, automatic hyphenation in LaTeX is very good
%% If a word is mis-hyphenated, add it to this file
%%
%% add information to TeX file before \begin{document} with:
%% \include{localhyphenation}
\hyphenation{
Arsch
anaph-o-ra
Bü-cking
con-stit-u-ents
Dor-drecht
For-schungs-ge-mein-schaft
Ge-schich-te
ha-ben
pho-nol-o-gy
pro-so-dic
pro-so-di-cally
Sal-pe-ter
sei-nen
Wil-liams
}
\hyphenation{
Arsch
anaph-o-ra
Bü-cking
con-stit-u-ents
Dor-drecht
For-schungs-ge-mein-schaft
Ge-schich-te
ha-ben
pho-nol-o-gy
pro-so-dic
pro-so-di-cally
Sal-pe-ter
sei-nen
Wil-liams
}
  % \memoizeset{
  %   memo filename prefix={hpsg-handbook.memo.dir/},
  %   % readonly
  % }
  \papernote{\scriptsize\normalfont
    \@author.
    \titleTemp. 
    To appear in: 
    Ulrike Freywald \& Horst Simon (eds.) Headedness and/or grammatical anarchy?
    Berlin: Language Science Press. [preliminary page numbering]
  }
  \pagenumbering{roman}
  \setcounter{chapter}{#1}
  \addtocounter{chapter}{-1}
}
\makeatother



% This does a linebreak for \gll for long sentences leaving space for the language at the right
% margin. The factor .0989 is needed since otherwise starred examples cause a linebreak.
% St.Mü. 17.06.2021 08.02.2021
\newcommand{\longexampleandlanguage}[2]{%
%\begin{tabularx}{.99\linewidth}[t]{@{}X@{}p{\widthof{(#2)}}@{}}%
%\begin{minipage}[t]{.99\linewidth}%
\begin{tabularx}{\linewidth}[t]{@{}X@{}p{\widthof{(#2)}}@{}}%
\begin{minipage}[t]{\linewidth}%
#1%
\end{minipage} & (\ili{#2})%
\end{tabularx}}

% ORCIDs in langsci-affiliations 
\usepackage{orcidlink}
\definecolor{orcidlogocol}{cmyk}{0,0,0,1}
\ProvideDocumentCommand{\LinkToORCIDinAffiliations}{ +m }
  {%
    \orcidlink{#1}
  }

   %% hyphenation points for line breaks
%% Normally, automatic hyphenation in LaTeX is very good
%% If a word is mis-hyphenated, add it to this file
%%
%% add information to TeX file before \begin{document} with:
%% %% hyphenation points for line breaks
%% Normally, automatic hyphenation in LaTeX is very good
%% If a word is mis-hyphenated, add it to this file
%%
%% add information to TeX file before \begin{document} with:
%% %% hyphenation points for line breaks
%% Normally, automatic hyphenation in LaTeX is very good
%% If a word is mis-hyphenated, add it to this file
%%
%% add information to TeX file before \begin{document} with:
%% \include{localhyphenation}
\hyphenation{
Arsch
anaph-o-ra
Bü-cking
con-stit-u-ents
Dor-drecht
For-schungs-ge-mein-schaft
Ge-schich-te
ha-ben
pho-nol-o-gy
pro-so-dic
pro-so-di-cally
Sal-pe-ter
sei-nen
Wil-liams
}
\hyphenation{
Arsch
anaph-o-ra
Bü-cking
con-stit-u-ents
Dor-drecht
For-schungs-ge-mein-schaft
Ge-schich-te
ha-ben
pho-nol-o-gy
pro-so-dic
pro-so-di-cally
Sal-pe-ter
sei-nen
Wil-liams
}
\hyphenation{
Arsch
anaph-o-ra
Bü-cking
con-stit-u-ents
Dor-drecht
For-schungs-ge-mein-schaft
Ge-schich-te
ha-ben
pho-nol-o-gy
pro-so-dic
pro-so-di-cally
Sal-pe-ter
sei-nen
Wil-liams
}

   \togglepaper[7]
}{}


\title{The Left-Left Constraint: A structural constraint on adjuncts}

\author{Hubert Haider\orcid{0000-0001-7052-7607}\affiliation{University of Salzburg}}
% \chapterDOI{} %will be filled in at production

% \epigram{}

\abstract{Left-hand adjuncts of left-headed (= \textit{head-initial}) phrases are constrained in a particular way. The constraint, which is absent for adjuncts of head-final phrases, is this. The head of the adjunct must be in the absolute phrase-final position. Anything that follows the head disqualifies the phrase as an adjunct. The effect of this \textit{head-final-constraint} is adjacency between the head of the adjunct and the phrase the adjunct is adjoined to. This holds for adverbials, viz. adjuncts to VPs and APs, as well as for adnominal attributes.
	
This constraint does not follow from any established conditions on phrase structuring. It will be shown to arise from a licensing requirement that holds for phrases merged with a given phrase. Left adjuncts of head-\textit{initial} phrases are outside of the structural licensing domain of the head of the phrase and therefore they are in need of an alternative way of getting structurally licensed. This alternative way – proper attachment – results in the hitherto unaccounted adjacency effect.}


\begin{document}

\maketitle

\section{The issue}\label{sec-issue}

\emph{Left}-adjunction to \emph{left}-headed major lexical phrases\footnote{%
  ``Major lexical phrases'' are phrases headed by word-level categories, such as A$^0$, N$^0$, V$^0$, and to a
  limited extent P$^0$. These heads, unlike functional-category heads, license the phrases they combine with directionally. The LLC is a constraint on left-adjoining to left-head phrases of argument-taking heads.}  
is subject to a constraint that is absent for (left-)adjunctions to
right-headed phrases (\citealp[782--785]{Haider2004}; \citealp[194]{Haider2010}; \citealp[13--16,
34--37]{Haider2013}). For ease of reference in this paper, this constraint shall be referred to as
the \emph{Left-Left-Constraint} (LLC). In strictly head-initial languages such as \ili{English}, the LLC
applies to left adjuncts of any major lexical phrase, and in particular to adjuncts of NPs as well
as VPs. For all these adjuncts, their heads must be strictly adjacent to the phrase they are
adjoined to. The adjunct phrase may be extended on its own left side, for instance by degree
modifiers, but its head must be phrase-final in order to be adjacent to the host phrase. Note that
the adjacency requirement holds for the \emph{head} of the adjunct relative to the \emph{phrase} to
which it is adjoined. In other words, it is not a \emph{head-to-head}-adjacency but a
\emph{head-to-phrase} adjacency. This is particularly clear when several adjuncts are involved. Each
adjunct must be adjacent to the phrase it adjoins, even if this phrase already contains a
left-adjoined adjunct.

In \ili{German} and \ili{Dutch}%
\footnote{\citet[292]{Broekhuis2013} formulates a ``Head-final Filter on attributive adjectives'':
``The structure [\sub{NP} \ldots\ [\sub{AP} ADJ XP] N\sharp] is unacceptable, when XP is phonetically non-null and N\sharp{} is a bare head noun or a noun preceded by an adjective phrase: [(AP) N].''
}
and all other \ili{Germanic} OV languages, the LLC applies only to adjuncts of NPs but not to adjuncts of\kern1pt{}
VPs or APs. This is a predictable fact since NPs are head-initial while VPs and APs are head-final phrases in these languages.\footnote{For \ili{Dutch}, see \citet*[291--293]{Broekhuis2013}.}  Since the LLC is a constraint on left-adjunction to \emph{head-initial} phrases, VPs and APs are not in the scope of this constraint. Consequently, in uniformly head-final languages – \ili{Japanese}, for example – there is no context at all for the LLC to apply.

The following \ili{English} examples illustrate the LLC first for adjuncts of VPs (cf.\ (\ref{ex-adjunct-vp}))  and then for adjuncts of NP (cf.\ (\ref{ex-adjunct-np})). Preverbal adverbials may contain modifiers, but the head of the adverbial must be in the final position (\ref{ex-owl}), (\ref{ex-analysed}), in order to meet the LLC. Example (\ref{ex-trump}) illustrates the head-to-phrase adjacency. Each of the two adjuncts must be head-adjacent to the phrase they are adjoined to.
\eal\label{ex-adjunct-vp}
\ex[]{A finch is [[much more \emph{often} (*than an owl)] [heard in Blackwood Forest]].}\label{ex-owl}
\ex[]{He has [[more \emph{carefully} (*than anyone else)] [analysed this problem]].}\label{ex-analysed}
\ex[]{She has [\sub{VP} very often [\sub{VP} publicly [\sub{VP} criticised Trump]]].}\label{ex-trump}
\ex[]{[Much more \emph{often} than an owl], a finch is heard in Blackwood Forest.}\label{ex-finch}
\ex[]{One should \emph{more carefully} analyse such data.}\label{ex-carefully}
\ex[*]{One should \emph{with (more/great) care} analyse such data.}\label{ex-morecare}
\zl


%\largerpage[-3]
\noindent
Example (\ref{ex-carefully}) is instructive in two respects. First it shows that the adjective plus the comparative phrase form a single phrase, and second, it shows that a clause-initial position is not subject to the LLC. The structure of (\ref{ex-carefully}) can be analysed in alternative ways. If the adverbial is in a functional spec position, the absence of LLC is predicted, since it constrains adjoined positions but not spec positions. Alternatively, if the position of the clause-initial adverbial is regarded as an ad-joined position, it is adjoined to a functional projection. In this case, the LLC is not operative since it constrains lexical projections as projections of heads with a grammatically defined directionality property, but it does not apply to functional projections.%
%
\footnote{Examples such as \emph{Rarely in this league do you get two long touchdowns} are clear cases of a Spec-head-configuration, with \emph{do} in the functional head-position. Corpus searches confirm that an adverbial like \emph{at this point} or \emph{in this respect} is not attested in between a finite auxiliary and a verb, but it occurs between the subject and a finite auxiliary, that is, within a functional projection, as in (\ref{ex-nobody}) and (\ref{ex-and}):
    \ea
    \label{ex-nobody} Nobody \emph{at this point} has stepped out.
    \z
    \ea\label{ex-and} [\ldots] and \emph{in this respect} has replaced the Muslim Brotherhood.
    \z
Neither the BNC nor CocA contains a single written token of \emph{``has  at this point''} or \emph{``has in this respect''}, but \emph{``at this point has''} or \emph{``in this respect has''} is attested. In these cases, the adverbial is adjoined to the \emph{functional} projection of the finite auxiliary.}


%\largerpage[-3]
\noindent
Eventually, the contrast between (\ref{ex-carefully}) and (\ref{ex-morecare}) reconfirms that a \emph{structural} condition is at work. The adverbials are in the very same pre-VP position, semantically they are exchangeable, and phonologically, the unacceptable (\ref{ex-morecare}) consists of even less syllables than the acceptable version (\ref{ex-carefully}). Nevertheless, their acceptability is completely opposite, as a search in three big corpora%
%
\footnote{BNC = British National Corpus (100 million: British\il{English!British}, 1980s--1993); CocA = Corpus of contemporary American English\il{English!American} (520 million: US, 1990--2015); NOW = News on the web (5.2 billion: Web news, since 2010).}
confirms. The sequence ``should more *ly'', with ``*'' as a joker for a single word, is attested in each corpus (BNC: 17, COCA 53, NOW 254). However, as expected for a PP in the pre-VP position, the sequences ``should with care'', ``should with great care'', or ``should with more care'' are completely absent in these three corpora, that is, in an aggregated corpus of 5.8 billion words.

The situation illustrated above is parallel in \ili{Romance}. A Google search (Dec. 15, 2017) for ``doit soigneusement'' `must carefully' on news and book sites produced 59 and 14.100 hits, respectively. A search for ``doit avec soin'', `must with care', however, produced zero hits on news sites and only eight in the unfiltered search, some of which are enclosed by commas, although ``avec soin'' is attested 31,000 times for ‘news’ and more than 7 million times in general. In languages with head-final VPs, as for instance \ili{Dutch} or \ili{German}, such a difference does not exist (cf.\ (\ref{ex-dutchgerman})).

As for attributes, \citet*[551]{HuddlestonPullum2002} emphasise the ``virtual exclusion of post-head dependents. Attributive AdjPs, like other attributive modifiers, hardly permit post-head complements or modifiers.'' The hedging by \emph{hardly} is motivated by apparent exceptions of the kind that will be dealt with in Subsection~\ref{subsec-llc} of this paper.

\eal\label{ex-adjunct-np}
\ex an [[obviously much less fascinating (*than the LLC)] [constraint]]
\ex an [[extremely fascinating (*to his audience)] [actor]]
\ex a [very good (*at math)] linguist
\ex a [generous (*to a fault)] examiner
\zl

\noindent
In \ili{German}, the LLC constrains left-adjunction to an NP as a head-initial phrase (cf.\ (\ref{ex-headinitialphr})) in the same manner as in \ili{English}. APs and VPs, however, are head-final and therefore ``immune'' against the LLC (\ref{ex-dutchgerman}a--c). \ili{German} is representative of the \ili{Germanic} OV-languages in this respect, and \ili{Dutch} is, too (\ref{ex-dutchgerman}g,h).

\eal\label{ex-headinitialphr}
\ex
\gll eine [[hervorragend \emph{geeignete} (*dafür)] [Kandidatin]] \\
an	\hphantom{[[}outstandingly eligible \hphantom{(*}for.it \spacebr{}candidate \\
\glt `an outstandingly eligible candidate for this'

\ex
\gll eine [[um Vieles \emph{bessere} (*als gedacht)] [Lösung]] \\
a \hphantom{[[}by much better \hphantom{(*}than thought \spacebr{}solution   \\
\glt `a by much better solution than thought'


% \ex
% \gll eine [[ebenso gut \emph{geeignete} (*wie er)] [Kandidatin]] \\
% an \hphantom{[[}as well eligible \hphantom{(*}as him \spacebr{}candidate  \\
% \glt `a candidate who is as well eligible as him'
\zl

\eal\label{ex-dutchgerman}
\ex\label{ex-kandidatin}
\gll Diese Beschränkung könnte [\sub{VP} [viel \emph{faszinierender} als das EPP] sein]. \\
     this constraint    could  {}        \spacebr{}much more.fascinating than the EPP be \\
\glt `This constraint could be much more fascinating than the EPP.'

\ex
\gll Er hat [\sub{VP} dieses Problem [so \emph{präzise} wie alle anderen] analysiert].  \\
     he has {}        this problem \spacebr{}as precisely as all others\textsc{.agr} analysed  \\
\glt `He has analysed this problem as precisely as all others.'

\ex\label{ex-lösung}
\gll ein [\sub{AP} [viel \emph{häufiger} als jedes andere] verfügbares] Gut  \\
     a   {}        \spacebr{}much more.frequent than any other available\textsc{.agr} asset  \\
\glt `a much more frequent available asset than any other'

\ex
\gll eine [\sub{NP} [\sub{AP} viel faszinierendere (*als das EPP)] Beschränkung] \\
     a    {}        {}        much more.fascinating \hphantom{(*}than the EPP constraint \\
\glt `a much more fascinating constraint than the EPP'

\ex
\gll eine [\sub{NP} [so \emph{präzise} (*wie alle anderen)] Analyse des Problems] \\
     an   {}        \spacebr{}as precise \hphantom{(*}as all others analysis of.the problem \\
\glt `an analysis of the problem that is as precise as all others'

\ex
\gll ein [\sub{NP} [\sub{AP} viel \emph{häufigeres} (*als jedes andere)] Gut] \\
     a   {}        {}        much more.frequent \hphantom{(*}than any other asset \\
\glt `an asset much more frequent than any other'
%\zl

%\eal\label{ex-meer-sneller}
\ex\label{ex-beslissingen}
\longexampleandlanguage{
\gll  dat beslissingen [veel \emph{meer} dan werd gedacht] gedreven worden door emoties\footnotemark \\
      that decisions \spacebr{}much more than was thought driven were by emotions \\}{Dutch}
\glt `that decisions were driven by emotions much more than thought'

\ex\label{ex-dier}
\gll  een veel \emph{sneller} (*dan een paard) dier  \\
a much faster \hphantom{(*}than a horse animal\\
\glt `a much faster animal than a horse'

\footnotetext{\url{http://nha.courant.nu/issue/HD/1935-07-11/edition/0/page/2}, 2022-03-17.}
\zl

\noindent
As a consequence of the LLC, prenominal attributes in uniformly head-\emph{initial} languages such as \ili{Romance}, North-\ili{Germanic} and \ili{English} are complementless since any complement of the head of the attribute would intervene between the head and the target of adjunction and thereby violate the LLC. Complex attributes are obligatorily post-nominal (cf.\ (\ref{ex-complexattributes})). In \ili{Romance} this is a regular option, in \ili{Germanic} this is an instance of an apposition (cf.\ (\ref{ex-schmetterling})).%
%
\footnote{Prosodically, appositions are marked with a separate intonation contour (``comma intonation''; see \citealp[Section~2.3.3]{Dehe2014}) for \ili{English}. The analogous situation holds for \ili{German}. In (\ref{ex-mager}), the AP is an adjunct while in (\ref{ex-arme}) it is appositive and parenthetical. Here, adjectives do not agree and the AP is a separate intonation phrase.
	
\ea\label{ex-mager} 
\gll Die [mager\emph{en} \emph{und} \emph{blau} \emph{geädert}en] Arme ragten aus einem schwarzen T-Shirt. \\ 
     the \spacebr{}meagre\textsc{.agr} and blue veined\textsc{.agr} arms protruded from a black T-shirt \\
\ex\label{ex-arme} 
\gll Die Arme, [mager \emph{und} \emph{blau} \emph{geädert}], ragten aus einem schwarzen T-Shirt. \\ 
     the arms \spacebr{}meagre and blue veined protruded from a black T-shirt \\
\z
}
Unlike in \ili{Romance} languages, adnominal attributes are prenominal. The difference is reflected in the lack of agreement (cf.\ (\ref{ex-schmetterling-a}) vs. (\ref{ex-apollofalter})). Another option is extraposing the intervening phrase, if the grammar admits this (cf.\ (\ref{ex-segelfalter})). \ili{French} is representative for all other \ili{Romance} languages in this respect.

\eal\label{ex-complexattributes}
\ex\label{ex-motherinlaw}  a curious (*about his past) mother-in-law

\ex a mother-in-law, curious about his past

\ex
\gll un [\sub{AP} plus grand (*que le précédent)] nombre de personnes \\
     a  {}        much bigger \hphantom{(*}than the preceding number of persons \\

\ex
\gll un nombre de personnes [\sub{AP} plus grand que le précédent] \\
     a  number of persons   {}        much bigger than the preceding \\
%\glt `a number of persons much bigger than the preceding'

\ex
\gll une femme [\sub{AP} fière de soi] \\
     a   woman {}        proud of herself \\
%	    \glt `a woman proud of herself'

\ex  
\gll une [\sub{AP} fière (*de soi)] femme  \\
     a   {}        proud \hphantom{(*}of herself woman \\
\glt `a woman proud of herself'

\zl


\eal\label{ex-schmetterling}
\ex\label{ex-schmetterling-a}
\gll Ein Schmetterling, \emph{so} \emph{selten} \emph{wie} der Apollofalter, ist der Segelfalter. \\
a butterfly as rare\textsc{.(no agreement)} as the Apollo.butterfly is the Iphiclides.podalirius \\
\glt `A butterfly as rare as the Apollo butterfly is the Iphiclides podalirius.'
%\glt `a butterfly, as rare as the Apollo-butterfly is the Iphiclides-podalirius'

\ex\label{ex-apollofalter}
\gll Ein so selten\emph{er} (*wie der Apollofalter) Schmetterling ist der Segelfalter. \\ 
     a as rare\textsc{.agr} \hphantom{(*}as the Apollo.butterfly butterfly is the Iphiclides.podalirius  \\

\ex \label{ex-segelfalter}
\gll Ein [[so selten\emph{er}] Schmetterling \emph{wie} \emph{der} \emph{Apollofalter}] ist der Segelfalter. \\
     a \hphantom{[[}as rare\textsc{.agr} butterfly as the Apollo.butterfly is the Iphiclides.podalirius \\
%\glt `a such rare butterfly as the Apollo-butterfly is the Iphiclides-podalirius'
\zl

\largerpage
\noindent
The LLC predictably holds for any language with unequivocally head-initial phrases. However, there are alleged SVO languages that apparently violate the LLC, as for instance the \ili{Slavic} languages. Upon closer scrutiny, these languages do not qualify as ``unequivocally head-initial''. In \ili{Slavic} languages, the head position in the phrase is in fact not fixed. It is \emph{flexible}. For details, the reader is referred to \citet{HaiderSzucsich2022a} and \citet{SzucsichHaider2015}.
\ili{Slavic} languages are representative of a ``third type'' of head-positioning, namely \emph{unspecified} head-positioning, in addition to the other two widely acknowledged types, namely
head-\emph{initial} and head-\emph{final}. In such a ``Type-3'' setting, adjuncts of an apparently head-initial phrase are not constrained by the LLC. A grammar with unspecified head positioning, that is, a Type-3 language, allows for alternative serialisations within a phrase. A head may alternatively be in an initial (cf.\ (\ref{ex-swojdom})), final (cf.\ (\ref{ex-dompokazuje})), or intermediate position (cf.\ (\ref{ex-pokazujedom})), that is, sandwiched between its arguments. \ili{Polish} is representative of the majority of \ili{Slavic} languages in this respect.


\eal\label{ex-slavic}
\ex\label{ex-swojdom}
\gll de Basia \emph{pokazuje} Jarkowi swój dom. \\
that Basia\textsc{.nom} shows Jarek\textsc{.dat} her house\textsc{.acc} \\\hfill(\ili{Polish})
\glt `that Basia shows Jarek her house'

\ex\label{ex-dompokazuje}
\gll że Basia Jarkowi swój dom \emph{pokazuje}. \\
that Basia\textsc{.nom} Jarek\textsc{.dat} her house\textsc{.acc} shows \\
\glt `that Basia shows Jarek her house'


\ex\label{ex-pokazujedom}
\gll że Basia Jarkowi \emph{pokazuje} swój dom. \\
that Basia\textsc{.nom} Jarek\textsc{.dat} shows her house\textsc{.acc} \\
\glt `that Basia shows Jarek her house'

\zl

\largerpage
\noindent
In \ili{Slavic} languages (see \citealt[116]{SiewierskaUhlirova1998a}, \citealt[16]{HaiderSzucsich2022a},
\citealt[Section~6]{HaiderSzucsich2022b}), the LLC effect is absent for preverbal adjuncts
(\mex{1}a,b) as well as for prenominal adjuncts (\mex{1}c,d).\footnote{%
\ili{Bosnian}/\ili{Croatian}/\ili{Serbian} and \ili{Czech} do not admit this pattern \citep[116]{SiewierskaUhlirova1998a}.
} The contrast between \ili{English} and \ili{Slavic}
languages in this respect confirms the claim that LCC is a property of adjuncts of genuinely
head-initial phrases, as in \ili{English}, and absent for adjuncts within the directionality domain of the
head of the adjunction site. If in \ili{Slavic}, a head licenses in either direction, it will license
adjuncts in either position.

% \eal\label{ex-\ili{Polish}}
% \judgewidth{??}
% \ex[??]{
% \gll [\sub{NP} [\sub{AP} wierny (swojej żonie)] mąż ] \\
%      {}        {}        faithful \spacebr{}his wife\textsc{.dat} man \\
% \glt `a man who is faithful to his wife'
% }\label{ex-wife}

% \ex[?]{
% \gll W      zeszłym roku [\sub{VP} [\sub{AdvP} \emph{dużo} \emph{więcej} (niż Jarek)] pracowała tylko Katarzyna]. \\
%      during last    year {}        {}          much        more          \spacebr{}than Jarek worked only Katarzyna   \\
% \glt `Only Katarzyna worked much more than Jarek during last year.'
% }\label{ex-Katarzyna-jarek}
% \zl
% %\itdopt{fragezeichen hochstellen}

\eal
\ex 
\longexampleandlanguage{
\gll V  prošlom  godu [gorazdo bol’še čem Igor] vyigrala tol’ko Maša\\
     in previous year \spacebr{}much more than Igor won only Mary\\}{Russian}
\glt `Last year, only Mary has much more won than Igor.'
\ex 
\longexampleandlanguage{
\gll Prošle godine je  [mnogo više od Želimira] radila samo Branka\\      
     last   year   has \spacebr{}much more than Želimir worked only Branka\\}{B/C/S}
\glt `Last year, only Branka has much more worked than Želimir.'
\ex 
\gll [verni-jat             (na žena si)] măž\\
     \spacebr{}faithful-\textsc{def}  \spacebr{}to wife his.\refl{} husband\\\hfill(\ili{Bulgarian})
\glt `a husband faithful to his wife'
\ex 
\gll [wierny            (swojej żonie)] mąż\\
     \spacebr{}faithful \spacebr{}his wife.\dat{} husband\\\hfill(\ili{Polish})
\glt `a husband faithful to his wife'
\zl

\noindent
The absence of the LLC in \ili{Slavic} languages is merely one feature out of a systematic set of contrasts between uncontroversial SVO languages and the \ili{Slavic} languages. They are Type-3 languages that have been misclassified as SVO languages \citep{HaiderSzucsich2022a,SzucsichHaider2015}.

\section{Previous attempts of accounting for LLC-constrained data}\label{sec-preattempts}

%\largerpage
The adjacency property of adnominal attributes and of preverbal adverbials in \ili{English} has each been seen as a theoretical challenge in the literature, but not as a common property of head-initial phrases. As for attributive APs in \ili{English}, \citet{Emonds76a-u} has raised the issue and \citet{Williams1982} has deferred it to a filter-condition (i.e.\ \emph{Generalised Head Final Filter}).

The following accounts\footnote{One anonymous reviewer tells me that – according to a forthcoming
  publication – a ``final-over-final constraint'' (= a constraint that disallows structures where a
  head-initial phrase is contained in a head-\emph{final} phrase in the same extended
  projection/domain) could account for the facts. Evidently, this cannot be the case: in languages
  like \ili{English}, heads are uniformly \emph{initial}, in any phrase, so the constraint cannot be
  operative at all since there are no head-final phrases involved. The LLC applies to
  head-\emph{initial} phrases, and the adjunct phrases are head-initial in \ili{English} as well, and so
  are the NPs and VPs they are adjoined to.} have been tried out, namely a \emph{head-to-head
  adjunction} proposal for adverbs (Section~\ref{subsec-adverbs}), a \emph{head-to-functional-head
  raising} account for adjectives (Section~\ref{subsec-attributes}), a \emph{head-complement}
relation for adjectives (Section~\ref{subsec-adjectives}), and a \emph{processing} account (Section~\ref{subsec-llc}) for
adnominal attributes. In each particular case, adjacency is captured, but each account turns out to
be empirically inadequate. None of these accounts is able to satisfactorily cover both instances –
a \emph{adnominal} attributes and \emph{adverbial} phrases – and the absence of adjacency of adjuncts when the host phrase is head-final. Eventually, even the theoretical null-hypothesis – the apparent correlations are accidental – has found its advocate (Section~\ref{subsec-error}).

\subsection{Adverbs as head-to-head adjoined items?}\label{subsec-adverbs}

The fact that preverbal adverbials very frequently are simple adverbs has duped \citet[409]{Bouchard1995}, who claims that preverbal adverbials in \ili{English} or \ili{French} are head-adjoined to the verbal head and therefore ``simple''. Even if this were a correct option, which it is not, it would not rule out adjoining phrasal adverbials. The hypothesis merely postulates that word-level adverbials may be adjoined to a verbal head. By the same token, however, one would have to assume that phrase-level adverbials would have to be adjoined to the phrase-level category, that is, the VP. Eventually, it would be entirely unclear what to do with adverbs that precede other ad-verbs as in \emph{She'd have surely more deeply regretted it.}

It should be obvious that a head-head adjunction idea misses an essential generalisation. Pre-verbal
adverbials may be phrasal but only to the extent that the head remains phrase-final. For strictly
head-initial languages like \ili{English} or \ili{French} this entails that a preverbal adverbial phrase can be
extended only on its left side and not on the side where the complements would appear. Hence
adverbial phrases in \ili{English} may contain modifiers but no complements (cf.\ (\ref{ex-publish})), as
attested in \ili{English}, and other VO-languages, such as \ili{Romance} languages (cf.\
(\mex{1}b,c)).\footnote{The only licit option for \emph{plus souvent que} [\ldots] in the position
  in (\ref{ex-saintetienne}) is parenthetic. And, indeed, a Google search (2020-01-26), has produced no hit for ``news'' sites, but a single hit (\ref{ex-faim}), with comma signs for parenthesis, on a ``books'' site, although in all other contexts taken together, the phrase \emph{plus souvent que} is attested in a range of well above 5 millions:
\ea\label{ex-faim}
\gll et doit, \emph{plus} \emph{souvent} \emph{que} moi, souffrir de la faim \\
and must more often than me suffer from the hunger \\
\zlast}

%
\largerpage
\enlargethispage{2pt}
\eal
\ex\label{ex-publish} She has \textit{even much} earlier (*\textit{than him/he}) published in this field.
\ex\label{ex-saintetienne}
\gll Saint-Etienne a \emph{plus} \emph{souvent} (*que Lille) gagné.\\  
     Saint-Etienne has more often \hphantom{(*}than Lille won \\\hfill(\ili{French})
\glt `Saint-Etienne has won more often than Lille.'

\ex\label{ex-lasinistra}
\longexampleandlanguage{
\gll La sinistra \emph{ha} \emph{più} \emph{volte} (*\emph{di} Fratelli d'Italia) \emph{vinto} le elezioni. \\
the left has more times \hphantom{(*}than Fratelli d'Italia won the elections \\}{Italian}
\glt `The left party has won the elections more often than Fratelli d´Italia.'
\zl

\subsection{Adjectival attributes with heads raised to functional heads selecting an NP?}\label{subsec-attributes}

A more sophisticated approach is the hypothesis that attributive APs are complements of a functional
head, viz. an agreement head, in combination with the obligatory raising of the adjective to this
functional head position. This is exactly what \citet[291]{Corver1997} has proposed, namely ``the
existence of a head-final functional node Agr (heading AgrP) which can function as a landing site
for adjectival heads that are moved rightward''. 

\eal
\ex\label{ex-base}{\footnotesize [\sub{NP} [\sub{AgrP} PRO [\sub{Agr$'$}[\sub{AP} \ldots\ A$^0$\ldots]
  [\sub{Agr$^0$} e] ]] [\sub{NP} \ldots\ N$^0$ \ldots]]}~ %\jambox{
	base structure
%}
\ex\label{ex-raising}{\footnotesize [\sub{NP} [\sub{AgrP} PRO [\sub{Agr$'$}[\sub{AP} \ldots\
  e\sub{i} \ldots] [\sub{Agr$^0$} A$^0$] ]] [\sub{NP} \ldots\ N$^0$\ldots]]}  %\jambox{
	raising of A$^0$ to Agr$^0$
%}
	\begin{tikzpicture}
			\node at (0,0) {};
			\draw [->] (3.5,.5) to [out=south east,in=south west] (5.0,.5);
		\end{tikzpicture}
\zl
\vspace{-1em}

\noindent
This hypothesis correctly predicts that the adjectival head of an AP attribute in the NP will always be adjacent to NP because the functional head is NP-adjacent. However, the hypothesis demonstrably fails in other respects. There are equally immediate predictions of this hypothesis which are evidently wrong. Head-movement to the right over-generates heavily. It predicts out-comes that do not exist. Here are two areas of counterevidence, one from \ili{English} and one from \ili{German} which are representative of the respective language type, that is, VO and OV, respectively.

%\largerpage[2]
If an adjective were raised out of the AP into a pre-nominal functional head position, then APs with complements (cf.\ (\ref{ex-toimprove}), (\ref{ex-faithful})) would be turned into attributes in which the argument of the AP apparently precedes the adjective, but only in attributes. The predicted results (cf.\ (\ref{ex-eagerscientist}), (\ref{ex-faithfulhusband})) are unquestionably discouraging. \ili{English} is a case for the LLC but \ili{English} obviously does not raise the adjectival head out of an AP attribute (cf.\ (\ref{ex-eagerscientist}), (\ref{ex-faithfulhusband})), and \ili{English} is representative of all other head-initial \ili{Germanic} and \ili{Romance} languages, all of which are constrained by the LLC. By the same token, participial constructions with particle verbs are predicted to strand the particle in \ili{German} or \ili{Dutch}, but in fact they do not (cf.\ (\ref{ex-abraten})).\footnote{
The structure in (\ref{ex-abraten}) is exactly the structure \citet[350]{Corver1997} argues for, with examples such as (\ref{ex-daaraan}):

\ea\label{ex-daaraan}
\gll [\sub{DP} een {[\sub{NP} [\sub{AgrP} PRO [\sub{Agr$'$}} [nauw t\textsubscript{i} daaraan] [\sub{Agr} verwante\textsubscript{i}]]] [\sub{NP} idee]]]\\
{} a {} \spacebr{}closely {} there.to {} related {} idea\\
\glt `an idea closely related to this'
\z

\noindent
The problematic side shows when the moved item is a verbal element, that is, a participle, with an
obligatorily stranded particle. The predicted outcome is clearly deviant. (\ref{ex-abraten}), the participial construction corresponding to (\ref{ex-perte}) – stranded particle \& extraposed PP – is ruled out by the LLC. The well-formed version is (\ref{ex-dissuading}).

\begin{exe}
\ex\label{ex-perte}
\gll Der Experte riet\sub{i} [allen ab-e\sub{i} davon]. \\
     the expert suaded \spacebr{}everyone dis of.it\\ (ab-raten = dis-suade)

\ex\label{ex-dissuading}
\gll der [\sub{AP} allen davon abratende] Experte \\
     the {}        everyone from.it dissuading expert\\
\end{exe}\vspace{-\baselineskip}
}

\eal
\ex[]{He has always been [\sub{AP} eager to improve].}\label{ex-toimprove}

\ex[]{He has always been [\sub{AP} faithful to her].}\label{ex-faithful}

\ex[*]{He has always been a [\sub{AgrP} [\sub{AP} e\textsubscript{i} to improve] eager\textsubscript{i}] scientist.}\label{ex-eagerscientist}

\ex[*]{He has always been a [\sub{AgrP} [\sub{AP} e\textsubscript{i} to her] faithful\textsubscript{i}] husband.}\label{ex-faithfulhusband}

\ex[*]{
\gll der [[allen \emph{ab}-e\textsubscript{i} davon] \emph{ratende}\textsubscript{i}] Experte \\
the \hphantom{[[}everyone.\textsc{acc} dis    of.it advising expert \\ \glt (ab-raten = `dis-advise')
}\label{ex-abraten}
\zl

%\largerpage[2]
\noindent
Raising an adjective to an agreement position would strongly resemble raising the finite verb to the verb second position in V2-languages. In Scandinavian languages, for instance, the finite verb crosses the subject and precedes it in its derived position, and it strands the particle. In the \ili{Germanic} OV-languages, the fronting of the finite verb crosses all of its complements. In the case of adjective movement, the adjective in its derived position would be predicted to cross particles and objects, with such items ending up in a position in which they would precede the raised item. The facts do not support this hypothesis, however.

\ili{German} provides another area of evidence, along the same line. Elements that obligatorily follow the adjective in the AP are banned from the attributive construction in \ili{German} (and in \ili{Dutch} as well) because of the LLC. A comparative PP obligatorily follows the adjectival head (cf.\ (\ref{ex-preis-hoeher}a,b)). But, if the adjective raises, it would cross the comparative PP, resulting in (\ref{ex-wert}). This prediction turns out wrong (cf.\ (\ref{ex-wert})). The adjective in (\ref{ex-wert}) is treated just like the adjective in (\ref{ex-preis}), namely as an adjective with a wrong serialisation.

\eal
\label{ex-preis-hoeher}
\ex[]{
\gll Der Preis ist [\sub{AP} höher \emph{als} \emph{der} \emph{Wert}]. \\
     the price is {} higher than the value \\
%\glt `the price is higher than the value'
}\label{ex-preishöher}

\ex[*]{
\gll Der Preis ist [\sub{AP} \emph{als} \emph{der} \emph{Wert} höher]. \\
     the price is  {} than the value higher  \\
\glt `The price is higher than the value.'
}\label{ex-preis}

\ex[*]{
\gll der [\sub{AgrP} [\sub{AP} e\sub{i} \emph{als} \emph{der} \emph{Wert}] höhere\sub{i}] Preis \\
     the {}          {}        {}       than the value higher price \\
\glt `the price which is higher than the value'
}\label{ex-wert}
\zl

\noindent
Eventually, a movement account for adjectives would merely cover NP"=adjuncts. So, the account would have to be generalised in order to cover VP-adjuncts as well since the LLC applies in both contexts. 

For VP adjuncts, \citet{Cinque99a-u} has worked out a proposal that is based on functional
projections. According to this proposal, which has become a standard assumption in Generative
approaches, adverbials are expressions in spec positions of empty adverbial functional heads
indicated by [\sub{F$^0$-Adv} e ] in (\mex{1}). The idea that preverbal adjuncts are contained in functional projections has been widely adopted since.

\eal
\ex \ldots\ [\sub{AdvP} XP [\sub{Adv$'$} [\sub{F$^0$-Adv} e ] [\sub{VP} V$^0$ \ldots]]]
\ex Hillary has [\sub{AdvP} [very cleverly] [\sub{Adv$'$} [\sub{F$^0$-Adv} e ] [\sub{VP} figured out that]]]
\zl

\noindent
In spite of its wide reception, it is unlikely to be empirically adequate since it is challenged by unequivocal, robust and manifold counterevidence for its core part; see \citet[Section~6.4, 6.5]{Haider2013} and \citet{Haider2004}.%
%
\footnote{Here is a central and straightforward prediction that is wrong in any of the applicable OV languages: any argumental phrase preceding an adverbial phrase in an OV language is predicted to be opaque for extraction. Such a phrase would be in a pre-VP-functional projection. That such positions are opaque for extraction is an established fact in the literature on extraction domains. This prediction is inevitable but empirically wrong. For example, if \emph{vergeblich} `futile' in (\ref{ex-überzeugen}) had to be assigned to a functional spec position, the infinitival clause preceding it is adjoined to or contained in a functional projection. In each case, extraction is predicted to be ungrammatical. However, there is no OV language that would unequivocally confirm Cinque's prediction \citep[see][]{Haider2004}.

\ea\label{ex-überzeugen}
\gll Mich\textsubscript{i} / Wen\textsubscript{i} hat er [e\textsubscript{i} damit zu überzeugen] \emph{vergeblich} versucht? \\
     me                   {} who(m) has he {} with.it to convince vainly attempted \\
\zlast
}

Within Cinque's framework, an LLC-effect is completely unexpected and unpredicted. If an adverbial phrase is a phrase in a spec position, the LLC has no chance at all to apply. Typical and uncontroversial functional spec positions such as the clause-initial position in V2-languages or the subject position in SVO languages are open to phrases of any structural make-up. In particular, there is no evidence for a restriction such as the LLC to apply to phrases in such positions. Such evidence, if it existed, would be surprising since the LLC is a constraint on adjunctions and not on functional specifier position.

Note that in Cinque's account, a raising approach would not be admissible since the VP is regarded as the functional complement of the functional head. A VP with a preverbal adverbial is an adverbial phrase with a VP complement.%
%
\footnote{As an inevitable but unwelcome consequence, each auxiliary in (\ref{ex-theory})
  subcategorizes and selects an adverbial phrase while the very same auxiliaries in (\ref{ex-badly})
  subcategorize and select a VP (see also \citealt[Section~4.6.1.3]{MuellerGT-Eng1}):

\ea\label{ex-theory} 
The new theory certainly may [\sub{AdvP} possibly have [\sub{AdvP} indeed been [\sub{AdvP} badly formulated]]].
\z
\ea\label{ex-badly} 
The new theory may [\sub{VP} have [\sub{VP} been [\sub{VP} formulated badly]]].
\z

\noindent
Moreover, intervening heads, such as a negation particle, trigger \emph{do}-support (\ref{ex-doesnot}). The alleged adverbial heads are predicted to have the same effect but they do not (\ref{ex-never}). The difference follows if adverbials are adjuncts of the VP.

\ea
\label{ex-doesnot} 
It does \emph{not} work that way.
\z
\ea
\label{ex-never} 
It \emph{never} works that way.
\zlast
}
The adverbial phrase is a phrase in the spec position of the functional projection. An adverbial head could not leave the spec position and target the functional head position. In Corver's version, the AP is the complement of the functional head and the functional projection containing the adjective is adjoined to the NP.

In sum, a functional projection accommodating an attributive AP or an adverbial phrase of a VP is not the key for the solution but a road to predictions that fail. Its consequences are counterfactual.

\subsection{Attributive adjectives as heads that select an NP complement?}\label{subsec-adjectives}

A third avenue of attacking the problem has been contemplated by \citet*[339]{Abney87a}. He suggested that the NP following the attributive AP is a complement of the adjective (cf.\ (\ref{ex-outspoken})).

\eal
\ex[]{[\sub{DP} the\sub{D$^0$} [\sub{AP} very [\sub{A$'$} outspoken [\sub{NP} critic of this proposal]]]]}\label{ex-outspoken}

% \ex[]{
% \gll [\sub{DP} ein\sub{D$^0$} [\sub{NP} [\sub{AP} sich seiner Sache sehr \emph{sicherer}]~~~~~~~~~~~~ [\sub{NP} Kritiker des Vorschlages]]]  \\
%      {}        a              {} {} \textsc{refl.dat} his cause.\textsc{gen} very sure {} critic the proposal\textsc{.gen} \\
% \glt `a critic of the proposal who is very sure of his cause'
% }\label{ex-kritiker}

\ex[]{
\gll  [\sub{DP} ein\sub{D$^0$} [\sub{AP} sich [\sub{A$'$} seiner Sache [\sub{A$'$} sehr [\sub{A$'$} \emph{sicherer}\sub{A$^0$} [\sub{NP} Kritiker des Vorschlages]]]]]] \\
      {}        a              {} \textsc{refl.dat} {} his cause.\textsc{gen}  {} very {} sure {} critic the proposal\textsc{.gen}\\
\glt `a critic of the proposal who is very sure of his cause'
}\label{ex-sicher}

\ex[*]{
\gll [\sub{DP} ein\sub{D$^0$} [\sub{AP} sich [\sub{S$'$} seiner Sache [\sub{A$'$} sehr
[\sub{A$'$} [\sub{NP} Kritiker des Vorschlages] \emph{sicherer}\sub{A$^0$} ]]]]] \\
     {}         a   {}        \textsc{refl.dat} {} his cause.\textsc{gen} {} very {} {} critic the proposal\textsc{.gen} sure\textsc{.agr} \\
\glt `a critic of the proposal who is very sure of his cause'
}\label{ex-proposal}
\zl

\noindent
That (\ref{ex-outspoken}) cannot be a correct analysis becomes clear in a language in which the AP is head-final. Here, the complements of the adjective precede the head, but the phrase is nevertheless subject to the LLC if it is an adjunct of a head-initial phrase. \ili{German} is a language with this kind of setting.  Abney's focus is merely on \ili{English}. \ili{German} clearly tells that an analysis that – due to the restriction imposed by LLC – might be contemplated for \ili{English} would not work for \ili{German}, as (\mex{0}b,c) illustrate.

In \ili{German}, just like in \ili{English}, adjectives do not permit accusative objects,\footnote{There is an
  exceptional historical relic (s.\ also \citealt[272]{Mueller99a}): \emph{Gewohnt} `used to' used to govern a genitive, which is identical in form with the accusative (i) for the pronoun \emph{es} `it'. This has facilitated a reanalysis. However, as (ii) shows, \emph{ungewohnt} `not used to' is not acceptable with an accusative. The negative prefix \prefix{un} `un-' selects only adjectives, but no verb or participle.
\ea
\gll Ich bin es             gewohnt – Ich bin das gewohnt. –  Ich bin diesen Lärm         nicht gewohnt. – der ungewohnte Lärm \\
     I   am  it.\GEN/\ACC{} used.to  {} I am this used.to  {} I   am  this   noise.\ACC{} not   used.to  {} the unused.to noise\\
\z
\ea[*]{
\gll Ich bin den Lärm ungewohnt. – Der Lärm ist ungewohnt.\\
     I am the noise.\ACC{} unused.to {} the noise.\NOM{} is unused.to\\ (= unusual)
}
\zlast
} but there are dative, genitive and prepositional objects. The adjective \emph{sicher} `sure' takes a reflexive dative and a genitive NP as objects in a head-\emph{final} AP. If we disregard the implausible semantic compositionality of (\ref{ex-sicher}) for the moment, the NP as a complement in (\ref{ex-sicher}) would be a structurally illicit complement nevertheless since an AP is head-final in \ili{German}. So, NP complements have to precede. Consequently, if the NP were a structural complement of the adjective, it would have to precede (cf.\ (\ref{ex-proposal})). The resulting expression is gibberish, however. No language is known with structures like (\ref{ex-proposal}).

Moreover, the complementation idea would run into difficulties whenever a NP is modified by more than one attribute (cf.\ (\ref{ex-vorschlag})). In this case, in Abney's analysis, the lower adjective \emph{sympathisch} `likeable' would select an NP as complement while the adjectival head \emph{befremdlich} `strange' of the higher attribute would have to select an AP. Clearly, the result would be grammatical only if the lower AP is an attribute of an NP. Since the higher attribute cannot select the lower NP directly, the account will inevitably lead into over-generation.

\ea\label{ex-vorschlag}
\gll  der [\sub{NP} [\sub{AP} für mich befremdliche] [\sub{NP} [\sub{AP} ihm              sympathische] Vorschlag]]\\
      the {}        {}        to  me   strange       {}        {}        him\textsc{.dat} likeable      proposal\\
\glt `the proposal that is strange to me but likeable to him'
\z

\noindent
Abney's idea is in fact similar to Cinque's proposal for adverbials, except that Cinque postulates an empty functional head while Abney takes the adjective to be the selecting head. If Abney updated his analysis, he could join Cinque and postulate a functional head for attribution (or agreement, like Corver), that selects the NP. This analysis would fail, too. If he assumed, following Cinque, that adjuncts are phrases in spec positions, the LLC could not apply and rule out the ungrammatical cases. If, on the other hand, Cinque adopted Abney's analysis and applied it to adverbials, assuming that the head of an adjunct in reality selects the phrase it appears to be adjoined to, then the analysis fails for languages with head-final phrases, since they are not constrained by the LLC.

\subsection{The LLC as a processing effect?}\label{subsec-llc}

In a theoretically uncommitted approach, \citet{Fischer2016} presents a tentative proposal in terms of processing effects. In particular, adjectival agreement is suspected to work as a boundary signal. As a boundary signal for the boundary of the AP it is phrase-final. That's why it is bound to occur at the right \emph{edge} of the AP.

Attractive though it might seem, such a parsing-based account does not satisfactorily work for
various reasons. First, there are languages such as \ili{English} without any adjectival agreement, but
attributes are constrained by the LLC nevertheless. Second, \ili{Norwegian} shows that the inflected
adjective is not strictly adjacent to the following NP since \emph{nok} `enough' may intervene; see
examples in (\ref{ex-28-hai}) below. Third, the LLC effect for attributes is but a subset of the LLC
phenomena. The LLC applies to adverbials as well, but in this case, it could not be treated as a
violation of a morphologically signalled boundary condition since adverbials do not agree. Fourth,
the boundary-signal hypothesis would lead to exactly opposite expectations with respect to the
head-position of the host-phrase of the adjoined phrase. The LLC effect should be absent for
head-initial phrases because here, the phrase-\emph{initial} head of the NP or VP clearly signals
the boundary of the NP relative to the preceding attribute. 

On the other hand, head-final phrases,
such as \ili{German} VPs, are notoriously ambiguous with respect to the boundary of an adjunct. 
If a boundary-signal-triggered condition were favoured by parsing, it ought to disambiguate (\mex{1}a), which is structurally ambiguous between (\mex{1}b) and (\mex{1}c). An adjacency condition would be sufficient for ruling out (\mex{1}b) and thereby disambiguating (\mex{1}a). But, patently, adjunct boundaries are not signalled where signalling would be needed for parsing.

% \eal
% \ex\label{ex-zufrieden}
% \gll Sie ist [zufrieden damit] abgereist. \\
% she has \spacebr{}satisfied therewith left \\
% \glt ‘Satisfied with it, she has left.'

% \ex\label{ex-damitabgereist}
% \gll Sie ist zufrieden [damit abgereist]. \\
% she has satisfied \spacebr{}therewith left \\
% \glt ‘Satisfied, she has left with it.'

% \ex \label{ex-damitzufrieden}
% \gll Sie ist [damit zufrieden] abgereist. \\
% she has \spacebr{}therewith satisfied left \\
% \glt ‘Satisfied with it, she has left.'

% \ex \label{ex-zufrieden-abgereist}
% \gll Sie ist damit [zufrieden abgereist].  \\
% she has therewith \spacebr{}satisfied left \\
% \glt ‘Satisfied, she has left with it.'
% \zl

\eal
\ex 
\gll Sie ist zufrieden damit abgereist.\\
     she is  content   there.with left\\
\ex 
\gll Sie ist [zufrieden damit] abgereist.\\
     she is  \spacebr{}satisfied there.with left \\
\glt `Satisfied with it, she has left'
\ex
\gll Sie ist zufrieden [damit abgereist].\\
     she is  satisfied \spacebr{}there.with left\\     
\glt `Satisfied, she has left with it.'
\zl

\noindent
In sum, the theoretical tool-kit of grammar theory does not offer the promising tool for deriving the LLC in such a way that it simultaneously covers the modifiers of NPs (viz. `attributes') and the modifiers of VPs and APs (viz. `adverbials'). The potential way out by postulating functional projections above an attribute or an adverbial turns out to be empirically as well as theoretically unattractive.

\subsection{Sampling error?}\label{subsec-error}

It rarely happens in research literature that a cross-linguistically uniform and robust pattern is suspected to be a mere coincidence of unrelated grammatical circumstances. This is what \citet{Hinterhoelzl2016} proposes, however. In his view, the adjacency effect in \ili{German} has nothing in common with the corresponding effect in \ili{English}. In other words, it is a sampling error, that is, two unequal things are falsely treated as equal by anyone who seeks a uniform account.

Accordingly, ``the H\textsubscript{ead}F\textsubscript{inal}-effects in the verbal and nominal domain in \ili{English} can be reduced to a metrical condition'' \citep[180]{Hinterhoelzl2016}. For \ili{German}, however, the pertinent constraint for the NP is claimed to be morphological: ``If we assume that inflected words are formed in the syntax and that the adjectival inflection constitutes a phrasal affix, [\ldots] we may assume that affix and head may be joined at M\textsubscript{orphological} F\textsubscript{orm} under the condition of strict adjacency.'' \citep[188]{Hinterhoelzl2016}. Surely, an ``if we assume'' is easily available. The costly part is the demonstration that it is correct.
Unfortunately, this part is missing in the paper. Neither the ``phrasal-affix'' claim nor the allegedly causal metrical conditions are independently justified or at least demonstrated to work for one of the crucial examples.

\largerpage
Had the author dutifully shown how the proposed metrical constraints are supposed to work, it could not have escaped him that they do not. Replacing metrically equivalent subtrees does not change the metrical property of the whole tree. In (\ref{ex-newbuilding}) , the adjectival phrase is branching, but obviously well-formed with the weak subtree \emph{much smaller}. Adding a metrically weak extension such as \emph{than it}, with a weak pronoun, would not change weights. On the other hand, \emph{than it appears} may be strong and this could change the s/w-distribution. Consequently, if (\ref{ex-newbuilding}) is metrically ok, (\ref{ex-muchsmaller}) is metrically ok as well, and the only variant that possibly might be filtered out is (\ref{ex-building}). But this is not what the facts tell. The COCA corpus – 520 million words of present day American English\il{English!American} – contains exactly 741 items of \emph{a much smaller [\ldots]}, but only a single item of the form \emph{a much smaller than}, namely \emph{a much smaller than expected loss}, which is irrelevant (see the discussion in the following section). In terms of corpus frequency, the difference between (\ref{ex-newbuilding})  and (\mex{1}b,c)  is as clear-cut as anyone could ask for.\footnote{%
If meter were at issue, (\ref{ex-turnout}) ought to be as unacceptable as (\ref{ex-building}), which is not the case. An explanation is presented in the following subsection of this paper.

\begin{exe}
\ex\label{ex-turnout}
a much smaller \emph{than expected} turnout \hfill (The Telegraph online, 2020-01-24)
\end{exe}
}
\eal
\ex[]{
a [much smaller] new building}\label{ex-newbuilding}

\ex[*]{
a [much smaller than it] new building}\label{ex-muchsmaller}

\ex[*]{
a [much smaller than it appears] new building}\label{ex-building}
\zl

\largerpage[2]
\noindent
Analogously, \ili{German} and \ili{English} would have to be separated by rigid metrical constraints that make a structure like (\ref{ex-scored}) virtually unstressable in \ili{English}, \ili{Italian} (cf.\ (\ref{ex-italian})) or \ili{Swedish} (cf.\ (\ref{ex-swedish})), but not in \ili{German} (cf.\ (\ref{ex-getroffen})) or \ili{Dutch} (cf.\ (\ref{ex-gescoord})). Independent evidence for the empirical and operational details of the required metrical phonology is wanting, especially since it is implausible that a highly flexible property such as prosody, that adapts to all kinds of structures, could exert a rigid bonding on structuring in exactly this case.\footnote{A line of a poem in a particular meter may be metrically deviant, but there is no meter for prose.}

\eal
\ex\label{ex-scored}
[He [has [[more often (*\emph{than anyone else})] scored.

\ex\label{ex-getroffen}
%\longexampleandlanguage{
\gll [Er [hat [[viel öfter (als jeder andere)] gepunktet.] \\
\spacebr{}he \spacebr{}has \hphantom{[[}much more.often \spacebr{}than anyone else scored \\%}{German}
\glt `He has scored much more often than anyone else.'

\ex\label{ex-gescoord}
\gll Hij heeft vaker      (dan iemand anders) gescoord. \\
     he  has   more.often \spacebr{}than anyone else scored \\\hfill(\ili{Dutch})
\glt `He has scored more often than anyone else.'

\ex\label{ex-italian}
\gll Ha  più  spesso (*di              chiunque altro)  segnato. \\
     has much more.often  \hphantom{(*}than anyone   else scored \\\hfill(\ili{Italian})
\glt `He has scored more often than anyone else.'

\ex\label{ex-swedish}
\longexampleandlanguage{
\gll Hon var lika djupt (*som oss) [\sub{VP} sårad över hans tystnad]. \\
     she was equally deeply \hphantom{(*}as us {} hurt by his silence \\}{Swedish}
\glt `She was hurt by his silence equally deeply as us.'
\zl

\noindent
A metrical constraint fails also with respect to an adequate differentiation between adjuncts of major lexical projections and adjuncts of lexicalized functional projections (see fn. 4).

As for the suspected morphological constraint\footnote{As predicted by the LLC, but in violation of the alleged morphological constraint, the left-hand AP of two \emph{conjoined} attributes may violate the LLC, provided the second and NP-adjacent AP is head-adjacent indeed:

\ea[]{
\gll Jetzt steht dort ein [[genauso breites (\emph{wie} \emph{zuvor})] aber [doppelt so hohes]] Gebäude. \\
     now stands there a \hphantom{[[}just.as wide \hphantom{(}as before but \spacebr{}twice as high building \\
}
\ex[*]{
\gll Jetzt steht dort ein [[doppelt so hohes] aber [genauso breites (\emph{wie} \emph{zuvor})]] Gebäude. \\
     now stands there a   \hphantom{[[}double as high but \spacebr{}just.as wide \hphantom{(}as before building \\
}
\zlast
}
that allegedly separates \ili{German} from \ili{English} and \ili{Italian}, independent evidence is missing. First, adjective inflection in \ili{German} is definitely not a ``phrasal affix''. It is inflection, that is, a paradigm with strong and weak forms and agreement for case and number.  Second, if it were an affix, it ought to parallel the relation between a T$^0$ head and the finite lexical verb of the \ili{English} VP. But, in an \ili{English} finite clause, a pre-VP adverbial (unlike a negation particle) does not prevent joining the T$^0$ present tense affix, viz. the present tense \emph{-s}, and the verb. Moreover, \ili{Russian} and other \ili{Slavic} languages would be wrongly subsumed under the adjacency requirement. In sum, the attempted dismissal of a single source for the cross-linguistically operative LLC effect lacks force. The theoretically stronger and empirically adequate solution is one that does not have to invoke several independent grammatical restrictions, especially when it can be shown \emph{why} a single condition holds across categories as well as across languages (see Section~\ref{sec-con-hai}).

\section{Apparent counterevidence for the LLC-cases of ``acceptable ungrammaticality''}

\ili{English} is a representative instance for a discussion of apparent exceptions. The LLC constrains two independent patterns. First, adverbials in the slot between the subject position and the left boundary of the VP in \ili{English} have to be head-adjacent to a head-initial VP. Second, prenominal attributes of head-initial NPs have to be head-adjacent to the NP. This section presents\pagebreak data that at first glance appear to contradict these predictions and forwards reasons and evidence as to why this is apparent counterevidence only. In (\ref{ex-datacontra}) are examples of the data areas to be discussed:

\eal\label{ex-datacontra}
\judgewidth{??}
\ex[]{
Research has [\emph{at} the same time] come under increased scrutiny.
}\label{ex-research}

\ex[]{
a [\emph{higher} than average/expected] proportion\footnotemark
}\label{ex-proportion}

\ex[??]{
an [\emph{easy} to enter] competition
}\label{ex-competition}

\footnotetext{I am especially grateful to Kerstin Hoge and Amir Zeldes for making me aware of this particular type of data in the discussion period at the workshop from which this volume resulted.}
\zl

\noindent
In (\ref{ex-research}), the head of the adverbial PP is the preposition \emph{at}. The head of the attributive AP in (\ref{ex-proportion}) is the adjective \emph{higher}, and in (\ref{ex-competition}), the head arguably is the adjective \emph{easy}. These heads are not adjacent to the target phrase of adjunction. \citet*[780]{HuddlestonPullum2002} are deliberate when characterising what they call the ``central position'' of adjuncts: ``Central position disfavours long and heavy adjuncts. Thus [\ldots] PPs, NPs are fore the most part less likely in this position than AdvPs.''

As for (\ref{ex-research}), the following table presents instructive search results from the three big corpora (see fn. 5) of \ili{English} for \emph{has at the same time} in comparison to similar expressions with virtually the same structure. The scores show that (\ref{ex-research}) is not representative of PP-adverbials in this position. The number before the slash is the number of occurrences of the given expression. The number following the slash is the number of occurrences of the PP in the respective corpus, that is, \emph{at the same time} in (\ref{ex-hasatthetime}a), and so on, in \emph{any} position.

\eal\label{ex-hasatthetime}%
\begin{tabular}[t]{lllll}
& & BNC & CocA & NOW \\
a. & has at the same time & 6 / 6835 & 11 / 34097 & 105 / 279.400\\
b. & has at the right time & 0 / 244 & 0 / 1208 & 0 / 21421\\
c. & has at a different time & 0 / 18 & 0 / 86 & 0 / 541\\
d. & has at that time & 0 / 2493 & 0 / 9772 & 2 / 83.806\\
e. & has at no time & 1 / 126 & 0 / 367 & 19 / 20.410
\end{tabular}
\zl
%\itdopt{ex 20 einfügen+ref in line 407 and 419 EE:keinen Weg gefunden, ex in tabular zu packen}

\noindent
A side glance on \ili{German}, with its head-final VP shows that it imposes no restraints on adverbial positions preceding the base position of the verb in the VP. This is directly reflected in corpora. A Google search for \emph{hat zu dieser Zeit} (‘has at that time’) – filtered for ``news'' and ``book'' sites – produced 1,380 hits on ``news'' pages and 19,200 hits on ``book'' sites.

The search results for \ili{English} confirm that \emph{at the same time} and, to a very small extent, \emph{at no time} are the odd balls, but in a frequency range well below one tenth of a percent. Both expressions are used like parenthetic%
%
\footnote{Some writers typographically mark the parenthesis, as in the following example:
\ea And it's ridiculous to have someone who has – at various points in his life – paid little or no taxes, \ldots
\z
\url{http://www.post-gazette.com/opinion/Op-Ed/2018/11/26/Ruben-Navarrette-Jr-Donald-Trump-flip-flops-on-immigration-make-for-one-wild-ride/stories/201811260022}, 2022-03-17.
}
idiomatic expressions. Whenever the very same NP headed by \emph{time} has to be interpreted compositionally and therefore structured compositionally, the corpora confirm the LLC-geared prediction at a 100\% level (\ref{ex-hasatthetime}b–d). This indicates that such expressions, viz. (\ref{ex-hasatthetime}a) and (\ref{ex-hasatthetime}e), are treated like an \emph{adverbial idiom}, in place of \emph{simultaneously} or \emph{never}.

The pattern (\ref{ex-than}b) stands for an intriguing class of apparent counterexamples. Again, the exceptions are limited to a small set of candidates. The outstanding items are \emph{expected} and \emph{average} and the profile is uneven again. In each case, the comparative expression intervenes between the head of the attribute and the target phrase of adjunction. This is a structure clearly ruled out by the LLC.

\ea\label{ex-than}
\begin{tabular}[t]{lllll}
& & CocA & BNC & NOW\\
a. & a better than \emph{expected} \ldots & 8 & 3 & 351\\
b. & a better than \emph{average} \ldots & 14 & 8 & 141\\
c. & a better than \emph{necessary} \ldots & 0 & 0 & 0\\
d. & a better than \emph{usual} \ldots & 2 & 0 & 6\\
e. & a higher than \emph{expected} \ldots & 5 & 3 & 220\\
f. & a higher than \emph{average} \ldots & 29 & 13 & 361\\
g. & a higher than \emph{necessary} \ldots & 1 & 0 & 3\\
h. & a higher than \emph{usual} & 5 & 2 & 131\\
i. & a faster than \emph{expected} \ldots & 1 & 0 & 65\\
j. & a faster than \emph{average} \ldots & 0 & 0 & 1\\
k. & a faster than \emph{necessary} \ldots & 0 & 0 & 0\\
l. & a faster than \emph{usual} \ldots & 0 & 0 & 6
\end{tabular}
\z
%\itdopt{ex 21 einfügen + ref in line 419, 487}

\noindent
The key for understanding these findings comes from languages in which the head of the attribute is
inflected. \ili{German} is such a language. Here are the \ili{German} counterparts:

\eal\label{ex-besseres}
\ex[*]{
\gll ein besser\emph{es} als erwartet Ergebnis \\
a better\textsc{.nom/acc.sg.n.} than expected result \\
\glt `a result better than expected'
}

\ex[*]{
\gll ein teurer\emph{es} als nötig Eingreifen \\
a more.expensive\textsc{.nom.sg.f.} than necessary intervention  \\
\glt `an intervention more expensive than necessary'
}

\ex[*]{
\gll den besser\emph{en} als üblich Ausblick  \\
  the better\textsc{.acc.sg.m.} than usual outlook \\
  \glt `the outlook better than usual'
}
\zl

\noindent
They confirm the LLC \emph{and} they show how language users try to outfox it cf.\ (\ref{ex-Ergebnis}), in \ili{German} and in \ili{English}. As expected and predicted, the LLC correctly blocks structures with interveners. In (\ref{ex-besseres}), the head is identified by agreement inflection, it is not adjacent to the NP, and the result is ungrammatical and robustly unacceptable.

Corpus search, however, produces a non-negligible number of specimen of the kind illustrated by (\ref{ex-Ergebnis}). Here, an \emph{adjacent} and \emph{inflectable} item is inflected although it is not the head of the attribute. It is embedded in the comparative phrase introduced by \emph{als} `than'. In fact, the pattern in (\ref{ex-Ergebnis}) is a `fake' fulfilment of the LLC. The adjacent item is treated as if it were the head although it is definitely not the head of the attribute. Why this? The reason is a rule conflict.

\eal\label{ex-Ergebnis}
\ex\label{ex-besser}
\gll [ein besser als erwartet\emph{es}] Ergebnis\footnotemark\\
\spacebr{}a better than expected\textsc{.n.nom.sg.} result \\
\glt `a result better than expected'
\footnotetext{\url{https://invezz.com/de/news/2021/04/29/nokia-meldete-besser-als-erwartetes-ergebnis-fur-q1-hier-nachsten-ziele-fur-kaufer/},
2022-03-19.}

\ex
% \gll die [besser als durchschnittlich\emph{en}] Erfolgsstatistiken\footnotemark\\
% the \spacebr{}better than average\textsc{.fem.nom.sg} statistics.of.success  \\
% \glt `the statistics of success above average'
% \footnotetext{\url{http://www.krebs-kompass.org/showthread.php?t=3605&page=7}, 2017-03-17.}

\gll mit einer [besser als durchschnittlich\emph{en}] Note\footnotemark\\
     with a    \spacebr{}better than average.\fem{}.\NOM{}.\SG{} grade\\
\footnotetext{
\url{https://www.handelsblatt.com/finanzen/steuern-recht/recht/arbeitszeugnis-vor-gericht-note-3-ist-eine-durchschnittliche-leistung/10976846.html},
2022-03-19.
}
\glt `with a grade better than average'

\ex
\gll ein [teurer als nötig\emph{es}] Eingreifen\footnotemark\\
a \spacebr{}more.expensive than necessary\textsc{.nom.n.sg} intervention \\
\glt `an intervention that is more expensive than necessary'
\footnotetext{\url{https://www.gruene-bundestag.de/parlament/bundestagsreden/2011/oktober/gerhard-schick-errichtung-des-europaeischen-finanzaufsichtssystems.html},
2017-03-19.}
 
\ex
\gll den [besser als üblich\emph{en}] Ausblick\footnotemark\\
the \spacebr{}better than usual\textsc{.m.acc.sg} outlook \\
\footnotetext{\url{http://www.finanzen.net/nachricht/aktien/gute-aussichten-gute-branchennews-treiben-chipwerte-infineon-auf-langzeithoch-
5293416}, 2022-03-19.}
\glt `the outlook better than usual'
\zl

\noindent
The rule conflict is this. The LLC enforces an adjacent head position but the comparative construction requires the comparative \emph{than}-phrase to follow the comparative adjective and thereby to intervene. This is a ``catch-22 dilemma'', that is, if one rule is obeyed, the other is violated, and vice versa. In such a situation, speakers tend to waive what they deem to be the minor rule. The results are phenomena of ``acceptable ungrammaticality'', also known as ``grammatical illusions''; see \citet*[159]{Bever1976},\footnote{``Sequences that are ungrammatical but acceptable, that is, cases the grammar marks as ill-formed, but which are acceptable by virtue of their behavioural simplicity.''}  \citet{Haider2011}, \citet{PhillipsLau2011}, \citet{Frazier2015}. Examples such as (\ref{ex-Ergebnis}) sound acceptable and are only recognised as ungrammatical upon closer scrutiny.\footnote{A less frequent but also attested alternative attempt of dodging the conflicting rule situation is inflecting both, the adjectival head plus the NP-adjacent inflectable item. Google (2020-01-16) produces 158 hits for \emph{besseres als erwartes}, as in:  ``ein besser\emph{es} als erwartet\emph{es} Ergebnis'' `a better\textsc{Agr} than expected\textsc{Agr} result'.}  This phenomenon – ``acceptable ungrammaticality'' – is the key for understanding (\ref{ex-than}).

In \ili{German}, but not in \ili{English}, the `fake head'-strategy is betrayed by inflection. The \ili{German} data nevertheless show what happens in \ili{English}. Speakers treat an adjacent item as a fake head for the purposes of the LLC. Let us check this explanation. An immediate prediction is this. Uninflectable items or items of a different category than that of the real head are fully unacceptable. This turns out to be correct. In (\ref{ex-expectedresult}), \emph{expected} is a finite verb, while \emph{bullet} in (\ref{ex-bullet}) and \emph{median} in (\ref{ex-median}) are nouns.

\eal
\ex[*]{\label{ex-expectedresult}
a better than I expected result}

\ex[*]{a faster than a bullet interceptor plane\footnotemark}\label{ex-bullet}

\ex[*]{a higher than the median temperature}\label{ex-median}

\footnotetext{Amir Zeldes (p.c.) made me aware of structures of the type a \emph{ faster-than-light travel}, which could be mistaken for attribute + N structures but are in fact compounds whose initial part is a graft.}
\zl

\noindent
The category mismatch makes the very same strategy unviable in the case of adverbials. A corpus search for the counterparts of attributes in adverbial usage, such as (\ref{ex-problem}), produced zero results. If \emph{better than expected} were a licit adnominal attribute it ought to be a licit adverbial, too. But it is not.

\eal\label{ex-problem}
\ex[*]{She has better than expected solved the problem.}
\ex[*]{She has higher than average scored on this task. }
\zl

\noindent
A special case of the pattern illustrated by (\ref{ex-than}) is triggered by the distribution of \emph{enough} and its cognates in all \ili{Germanic} languages (cf.\ \cite{Haider2011}). This is the only degree modifier that does not precede its target. Example (\ref{ex-genug}) illustrates \emph{big enough} in contrast with \emph{sufficiently big} in three other \ili{Germanic} languages. In all \ili{Germanic} languages, the cognates of \emph{enough} have survived and preserved their exceptional status over a period of more than a millennium, apparently due to its high frequency. Being a degree modifier, it is an uninflected word.

\ea\label{ex-genug}
\begin{tabular}[t]{@{}l@{~}lll@{\hspace{107pt}}r@{}}
a. & sufficiently big & -- & big \emph{enough}\\
b. & genügend groß 	  & --  &	groß \emph{genug} & (\ili{German})\\
c. & voldoende groot   & -- &	groot \emph{genoeg} & (\ili{Dutch})\\
d. & tilstrækkeligt stor	& -- &	stor \emph{nok} & (\ili{Danish})\\
\end{tabular}
\z

\noindent
The fact that this modifier \emph{follows} the head of the AP should disqualify such an AP for attributive usage. It would violate the LLC, and indeed, such constructions are robustly deviant, as the examples (\mex{1}a,b) exemplify. Even the spell-checker of my text software marks them as incorrect. However, the corpora reveal attempts of outwitting the LLC such as the following sample (\ref{ex-genug}c), (\ref{ex-genug}d), which is also confirmed by \citet{Fischer2016}.

\eal
\ex[*]{
\gll keine groß\emph{en} genug Triebwerke \\
no big\textsc{.nom.fem.pl} enough engines \\\hfill(\ili{German})
\glt `no enigines that are big enough'
}\label{ex-triebwerke}

\ex[*]{
\gll auf fest\emph{en} genug Beinen \\
on strong\textsc{.dat.n.pl} enough legs \\
\glt `on legs that are strong enough'
}\label{ex-beine}

\ex[?]{
\gll keine groß genug\emph{en} Triebwerke\footnotemark \\
no big enough\textsc{.nom.fem.pl} engines \\
\glt `no enigines that are big enough'
}\label{ex-groß}
\footnotetext{\url{http://www.kleinezeitung.at/international/5295011/A380-notgelandet_RiesenAirbus-zerriss-es-ein-Triebwerk},
2022-03-17.}

\ex[?]{
\gll auf fest genug\emph{en} Beinen\footnotemark \\
on strong enough\textsc{.dat.n.pl} legs \\
\glt `on legs that are strong enough'
}\label{ex-fest}
\footnotetext{\url{https://www.pickupforum.de/topic/152024-toter-bester-freund-der-freundin-würde-euch-das-stören/?page=2\&tab=comments\#comment-2193863}}
\zl

\noindent
In (\mex{0}c,d), the intervener is inflected, although it is an uninflectable item. In \ili{German}, even in combination with a noun, \emph{genug} `enough' remains uninflected, in either position, prenominal or post-nominal.\footnote{Cf.: \emph{Geld genug} `money enough' – \emph{genug Geld} `enough money' – \emph{genug Münzen}  `enough coins' – \emph{genug Abstand} `enough interspace' – *\emph{genuges Geld} – *\emph{genuge Münzen} – *\emph{genuger Abstand.}} The inflection in (\mex{0}c,d) is a way of compromising the LLC by violating the minor rule (i.e.\ inflecting the uninflectable) for saving the major rule, namely LLC, by pretending that \emph{genug} (enough) is the head, by virtue of being inflected.

\ili{Dutch}, \ili{German}, \ili{English}, and \ili{Norwegian} provide a nice minimal-pairwise setting for relevant contrasts in this respect. \ili{English} does not inflect attributes, but \ili{Dutch} does \citep[454]{Broekhuis2013}, \ili{German} and \ili{Norwegian} \citep[180]{Fabricius-Hansen2010} do, too. Among the inflected group of these languages, \ili{Norwegian} tolerates an inflected adjective followed by \emph{enough}, but \ili{Dutch} and \ili{German} do not. So, these data show that the difference between \ili{English} and \ili{Norwegian} on the one hand, and \ili{German} and \ili{Dutch} on the other hand should not be sought in conditions of attribute inflection. What accounts for the acceptability of (\mex{1}a,d) in contrast to (\mex{1}b,c) cannot be a principle of inflection.

\eal\label{ex-28-hai}
\ex\label{ex-bigenough} a big enough room
\ex[*]{
\gll een groote            genoeg inzet      --  \hfill ?? een groot genoege inzet \\
     a    big.\textsc{agr} enough dedication {}         {} a   big   enough.\textsc{agr} dedication\\%\hfill(\ili{Dutch})
%\glt `a dedication big enough'
}\label{ex-dedication}
\ex[*]{
\gll ein groß\emph{er}         genug Raum 	--	\hfill ?? ein groß genug\emph{er} Raum \\
     a   big.\textsc{nom.sg.m} enough room      {}             {} a   big  enough.\textsc{nom.sg.m} room\\%\hfill(\ili{German})
%\glt `a room big enough'
}\label{ex-bigroom}

\ex[]{
\gll  et stort           nok    rom  --  det store           nok    leveranser\footnotemark \\
      a  big.\textsc{sg} enough room {}  the big.\textsc{pl} enough supplies \\%\hfill(\ili{Norwegian})
%\glt `a room big enough' – `supplies big enough'
}\label{ex-supplies}
\footnotetext{\url{https://www.an.no/bodoby/vi-har-fatt-landets-storste-fiskebutikk/s/1-33-7147807}, 2022-01-01}%\scalebox{.98}{2022-03-17}}
\zl

\noindent
Why are \ili{English} (\mex{0}a) and \ili{Norwegian} (\mex{0}d) tolerant against \emph{enough}
as intervener, but \ili{Dutch} (\mex{0}b) and \ili{German} (\mex{0}c) are not? \ili{English} and \ili{Norwegian} are VO languages and in VO languages, particles of particle verbs follow the verb. Consequently, participial attributive constructions with particle verbs always have particles intervening between the participial attribute and the noun phrase:

\eal
\ex a washed \emph{out} road --  a switched \emph{off} phone -- a rolled \emph{up} ribbon
\ex eine \emph{aus}gewaschene Straße	-- ein \emph{ab}geschaltetes Telefon -- ein \emph{auf}gerolltes Band
\zl

\noindent
Particles of particle verbs do not count as interveners for LLC since particles are part of a complex verb and this complex verb is the head. This opens an escape hatch for (\ref{ex-28-hai}b,c). The degree particle is interpreted as part of a complex adjectival head.%
%
\footnote{Consequently, \emph{enough} should be a tolerated intervener also for preverbal adverbials, which is the case indeed:
\ea ``Security'' has often enough become a stand-in for whatever intelligence operatives decide to do. (NOW)
\zlast
}
This escape is not available in \ili{Dutch} or \ili{German} since in these languages, the particle of complex verbs obligatorily precedes. Hence there is no licit pattern for post-head degree particle to be associated with.

Let us turn now to the third pattern, namely (\ref{ex-competition}). The \ili{German} counterparts (cf.\ (\ref{ex-germancounter})) are unproblematic since inflection shows that the head of the attribute is in final position. The construction is a participial construction in which the adjective serves as an optional adverbial.

\eal\label{ex-germancounter}
\ex 
\gll ein nicht (leicht) zu lösend\emph{es} Problem \\
     a not \spacebr{}easy to solve.\textsc{nom.sg.n} problem \\
\glt `a problem that is not (easy) to solve'
\ex 
\gll eine nicht (einfach) zu beantwortend\emph{e} Frage \\
     a not \spacebr{}easy to answer question.\textsc{nom.sg.f}	\\
\glt `a question that is not (easy) to answer'
\zl

\largerpage[-1]
%\enlargethispage{4pt}
\noindent
For the pattern (\ref{ex-competition}), illustrated once more by (\ref{ex-easytoanswer}), there is a variant in \ili{English}, namely (\ref{ex-easy}), in which the adjective clearly is the adjacent head. Corpus search\footnote{Here are the results for the following searches:
\ea ``is an easy to''	: 	BNC   0;  COCA  1;  NOW:  147.
\ex ``is an easy * to'' (``*'' = joker for a word slot): BNC: 30; CocA: 177; NOW: 1,601.
\z}
shows that the type (\ref{ex-easy}) outnumbers the type (\ref{ex-easytoanswer}) by far. This result is stable across other predicates such as \emph{difficult}, \emph{hard} or \emph{simple}.\footnote{NOW corpus: ``is a \emph{difficult} to'': 26, ``is a \emph{difficult} * to'': 1690; ``is a \emph{hard} to'' 22, ``is a \emph{difficult} * to'' 1416; ``is a \emph{simple} to'' 13, ``is a \emph{simple} * to'' 770.} \citet*[551]{HuddlestonPullum2002} rule out and star \emph{an easy to find place} but admit a \emph{ready to eat TV meal} as having \emph{something of the character of a fixed phrase.}

In fact, (\ref{ex-easy}) is an independent construction, and not merely the extraposition variant of
(\ref{ex-easytoanswer}), as the examples in (\ref{ex-fake-headed}c,d) illustrate. For predicates
such as \emph{convenient} or \emph{comfortable}, a construction type such as (\ref{ex-easytoanswer})
is unquestionably deviant but an infinitival clause as a complement of N is grammatical and
acceptable.

\pagebreak
\eal\label{ex-fake-headed}
\judgewidth{??}
\ex[??]{an \emph{easy to answer} question}\label{ex-easytoanswer}
\ex[]{an \emph{easy} question \emph{to answer}}\label{ex-easy}
\ex[]{a \emph{necessary} price to pay 	-- *a necessary to pay price}\label{ex-price}
\ex[]{a \emph{comfortable} car to drive -- *a comfortable to drive car}\label{ex-car}
\zl

\noindent
Once more, and in analogy to (\ref{ex-besseres}) and (\ref{ex-easytoanswer}) is an instance of a `fake-headed' attribute. Grammars of \ili{English} qualify such a structure as deviant \citep[551]{HuddlestonPullum2002}. This judgement matches the corpus search results. The construction \emph{an easy to N} is totally absent in the BNC. COCA produced a single hit. Even the biggest corpus consulted, namely the NOW corpus, contains merely a single token of the string \emph{an easy to answer}, in the context of \emph{an easy to answer question}. There are other instantiations of this construction type that are somewhat more frequent. For example, there are 146 tokens of \emph{an easy to understand~\ldots} in the NOW corpus, but not a single token is attested in the BNC corpus. The COCA corpus contains four tokens of this expression. Therefore, it seems to be safe to conclude that these are instances of acceptable ungrammaticality.

%\largerpage[-1]
The attributes in (\ref{ex-fake-headed}) are treated as if the infinitival verb were the head. After all, it is the NP that provides the referent for the object slot of \emph{answer} in (\ref{ex-easyquestion}). But even in this situation of acceptable ungrammaticality, LLC is clearly respected since anything to the right of the verb makes the construction strictly deviant in the pattern (\ref{ex-easyquestion}). Here, the LLC is violated, as above, but the fake head strategy would not work, either, since in VO languages, adverbials must not intervene between the verb and a direct object.

\eal
\ex[*]{an [easy to answer \emph{correctly}] question}\label{ex-easyquestion}
\ex[]{an easy question [to answer \emph{correctly}]}
\zl

\noindent
In sum, it is warranted to conclude that the allegedly ``apparent'' counterevidence is apparent indeed. The LLC is not challenged by these data. Taken together with the existing positive evidence, they confirm the existence of such a constraint on adjunction to head-initial phrases.

\section{The grammatical source of the LLC constraint}\label{sec-llcconstraint}

The LLC is real, but a satisfactory account of this constraint is still missing. Let us recapitulate what the desired account has to cover. \emph{First}, it has to capture a directionality property. The LLC constrains left adjuncts of \emph{left}-headed phrases, that is, head-\emph{initial} phrases. It is absent for left adjuncts of right-headed, that is, head-\emph{final} phrases, and it is absent for adjuncts of phrases with unspecified directionality of the head, as for instance in the VPs or NPs\footnote{The South \ili{Slavic} BCS languages, that is \ili{Bosnian}, \ili{Croatian}, \ili{Serbian} and \ili{Slovenian} (see \citet{SzucsichHaider2015}, differ from other \ili{Slavic} languages with respect to the LLC. In the BCS languages and \ili{Slovenian}, the directionality of N$^0$ is specified as ``progressive'', producing \emph{head-initial} NP-structures, which are subject to the LLC.}  of \ili{Slavic} languages.

\emph{Second}, the desired condition has to be category-neutral since the LLC applies to NP adjuncts (i.e.\ attributes) as well as to adjuncts of VPs and APs (i.e.\ adverbials). This disqualifies accounts in terms of an agreement relation between the head of the adjunct and the head of the hosting phrase. In other words, the fact that there are languages in which adnominal attributes agree with the NP they are adjoined to is irrelevant since in the very same languages the adverbials do not agree but both contexts are equally constrained by the LLC.

%\largerpage[-1]
\emph{Third}, the LLC only constrains \emph{adjuncts} of lexical projections but it crucially does not apply to phrases in spec positions. This disqualifies accounts that place attributes or adverbials in spec position of functional heads. Taken these facts together, they call for a fresh approach. The approach suggested here is one in terms of a directionality-based licensing theory \citep{Haider2013,Haider2015}. Directionality of licensing is a property of lexical categories. Functional categories do not have arguments. Their structure is invariant across categories\footnote{Contrary to widely assumed but empirically unfounded assumptions, functional positions as targets of lexical head movement are universally preceding their complement. Hence, their serialisation is not  directionality dependent.}  and languages. The specifier precedes and the complement follows; see \citet{Haider2013,Haider2015}.

Why should the directionality of a lexical head matter at all for adjuncts?\footnote{I am grateful to one of the reviewers for legitimately raising this question.}  After all, adjuncts unlike arguments do not depend on the head. But – and this matters – an adjunct position\footnote{This approach dodges the traditional structural analysis of \emph{adjuncts} (i.e.\ \emph{Chomsky-adjoined phrases}), as phrases that are \emph{adjoined} to their host phrase. Attributes are adjuncts adjoined to NPs; adverbials are adjuncts of verbal, adjectival and in certain cases of nominal constituents}  is a structural position within the phrasal projection of a head and needs to be licensed just as any position within a phrase. \citet*[57]{HornsteinNunes2008} characterise the situation as follows. ``It is fair to say that what adjuncts are and how they function grammatically is not well understood. The current wisdom comes in two parts: (i) a description of some of the salient properties of adjuncts (they are optional, not generally selected, often display island effects, etc.) and (ii) a technology to code their presence (Chomsky-adjunction, different labels, etc.).''

The LLC is part of the ``technology to code their presence''. The LLC is the reflex of a strict structural management of admissible \emph{positions} in a phrasal projection of a lexical head. This paper focuses on the very property imposed on \emph{structure} (and not on the syntactic or semantic content). Adjuncts that \emph{precede} their host phrase are \emph{structurally} constrained if and only if their host phrase is a head-initial phrase. This constraint is absent for adjuncts of head-final phrases or adjunct phrases with flexible head positioning, such as in the \ili{Slavic} languages.

Adjuncts preceding a head-initial NP, VP, or AP are phrases whose position is obviously not in the directionality domain of the progressively licensing head of the phrase. That is the crucial distinction between head-initial and head-final phrases. Adjuncts preceding the head of a head-final phrase, such as ZP in (\ref{ex-headfinal}), are within the directionality domain of the regressively licensing head of the phrase. In (\ref{ex-headinitial}), ZP is not within the directionality domain of \xzero.
%, but in (\ref{ex-headfinal}) it is.

\eal
\ex\label{ex-headfinal}  head-final: ~ [\sub{XP} ZP $_\leftarrow$ [\sub{XP} \ldots\ $_\leftarrow$ \xzero] ]
\ex\label{ex-headinitial} head-initial:	[\sub{XP} ZP [\sub{XP} \xzero $_\rightarrow$ \ldots ] ]
\zl
%\itdopt{ex pfeile einfügen}

\noindent
In (\ref{ex-headfinal}), an adjunct ZP is within the licensing domain of \xzero because \xzero is a \emph{regressively} licensing head and therefore it is directionally licensed by \xzero. In (\ref{ex-headinitial}), ZP is outside the directionality domain of \xzero and therefore not licensed by \xzero. For the details of the licensing system and the derivation of the systematic contrasts between head-final and head-initial phrases, the reader is referred to \citet{Haider2015} and \citet{Haider2020a}.

\largerpage
For the present purpose it is sufficient to realise the directionality difference between (\ref{ex-headfinal}) and (\ref{ex-headinitial}) and to accept the condition that a \emph{structural position} of a phrase to be integrated in another phrase needs to be \emph{directionally licensed} in the containing phrase. This leaves exactly one context of a phrase that is not licensed by the head of the phrase it is a part of. This context is the context of left adjuncts to left-headed phrases. This is the case singled out by the LLC. Here, the `glue' for integrating a phrase is not the directional license by a head. The phrase must produce its own glue for attaching to another phrase. Let us call this relation ``proper attachment'' and define a principle to that effect:

\ealnoraggedright\label{ex-PPA}
\ex \emph{Principle of Proper Attachment }(PPA): \\ A phrase XP adjoined to a constituent YP that is \emph{not within the directional licensing domain} of the head of YP must be properly attached to YP.
\ex A phrase XP is \emph{properly attached} to a constituent YP if it is \emph{minimally distant} from YP.
\ex The head \xzero of XP is \emph{minimally distant} from YP if there is no ZP (∉ projection nodes of \xzero) dominated by XP that is closer to YP in linearization than \xzero.
\zl

\noindent
This principle is sufficient for covering all the phenomena discussed above. The LLC is the joint result of properties of the adjoining phrase and the host phrase. Head-initial host phrases are unable to license their \emph{left} adjuncts directionally. So the adjoined phrase must license its position by itself. It must properly attach to the host phrase. This is the source of the LLC effects. For adjuncts that are properly attached and precede a head-initial phrase, \emph{each node} on the projection line of the adjunct is ``minimally distant'' to the host phrase.

In head-final phrases, adjuncts are directionally licensable in any adjunction position preceding the head. So there is no need for a last resort option for obtaining a positional license via the PPA, whence the complete absence of LLC effects. In Type-3 phrases, the head is free to license in either direction since the directionality is not fixed to a particular value, that is, either progressive or regressive.

As a closing remark, I do not hesitate to admit that the definition in (\ref{ex-PPA}) involves a potentially unwelcome ingredient for an entirely structural condition, namely ``linearization'', that is, a string-based notion. Two items $\alpha$ and $\beta$ are adjacent if there is no \emph{intervening} (= string-based) item $\gamma$. For the time being, I do not see how to dispense with the string-based part of PPA in order to arrive at a purely structure-based definition.

% \itdopt{alpha, beta, gamma symbol einfügen}
\section{Conclusion}\label{sec-con-hai}

\largerpage
The LLC is the effect of a principle that governs the attachment of phrases to other phrases out-side of the directionality domain of the head of the host phrase. It is a principle necessitated by the conditions of licensing phrases in a projection, based on the directionality of a head and its projections (see \citealp{Haider2015} for the details of the licensing system and the systematic syntactic consequences that correlate with the head-initial and head-final property). Phrases \emph{adjoined} to a phrase outside of the directionality domain of the head are nevertheless licensed but under a different condition. They are ``glued'' to the respective phrase, which requires ``tight'' attachment. This is defined as proper attachment by the PPA (cf.\ (\ref{ex-PPA})). \emph{Each} node on the projection line of the head of a PPA-adjoined phrase is \emph{adjacent} to the host phrase since there are no interveners between the head of the adjunct and the boundary of the phrase the adjunct is pre-adjoined to.

For strictly head-initial languages with prenominal attributes, the PPA strips these attributes of all their complements. Apparent counterexamples are cases of acceptable ungrammaticality and reflect the users' attempts to circumvent the PPA. In head-final phrases, all these effects are absent. In sum, the PPA completes the licensing system for joining phrases by defining the licensing condition for adjunct position outside of the directionality domain of the extended projection of the head of the host phrase.

\section{Afterthought in connection with the topic of this volume – headedness or anarchy}

\largerpage
``Adjunct'' is a well-studied concept in terms of its semantical properties. Its appropriate syntactical coverage is still a matter of dispute. As \citet*[33]{Dowty2003a} points out, ``The distinction between `complements' and `adjuncts' has a long tradition in grammatical theory, and it is also included in some way or other in most current formal linguistic theories.'' But he emphasises that ``it is a highly vexed distinction for several reasons, one of which is that no diagnostic criteria have emerged that will reliably distinguish adjuncts from complements in all cases.''

The only reliable distinctive property of adjuncts and complements, which \citet{Dowty2003a} notes, seems to be the following. Adjuncts can be instantiated in an arbitrary number within the same phrase, complements cannot. Grammar does not set a principal upper bound for attributes in combination with a single noun phrase or adverbials in combination with a single verb phrase. ``Multiple adjuncts (an unlimited number), can accompany the same head while only a fixed number of complement(s) can accompany a head (viz. just the one (or two, etc.) subcategorized by the particular head.'' \citep[39]{Dowty2003a}

From a theoretical point of view, this tells us that what we usually call an adjunct is not under strict control of the head of the phrase it is adjoined to. It is neither semantically nor categorially selected. As a consequence, even the number of admissible occurrences is not fixed. Adverbs may ``come or go'' without permission by the head of the phrase they are associated with. Does this certify adjuncts as structural anarchists in the tightly ruled realm of phrasal heads, defying the grammatical authority of their heads? \citet*[58]{HornsteinNunes2008} make out ``a deeply held, though seldom formulated, intuition: the tacit view that adjuncts are the abnormal case, while arguments describe the grammatical norm. We suspect that this has it exactly backwards. In actuality, adjuncts are so well behaved that they require virtually no grammatical support to function properly.''

They are well-behaved indeed, respecting all the constraints which the phrasal and the clausal architecture imposes. Phrase structure determines possible slots for adjuncts. What is a possible slot differs across phrase structure types. Example (\ref{ex-show}) illustrates the differences between \ili{English}, as a head-initial [S[VO]] language, and \ili{German}, with a V2-clause structure based on a head-final verb phrase.

\eal\label{ex-show}
\ex\label{ex-tomorrow}
The second show will be held \emph{tomorrow evening at the same time at the same venue.}

\ex\label{ex-morgen}
\gll [\emph{Morgen} \emph{Abend,} \emph{zur} \emph{selben} \emph{Zeit,} \emph{am} \emph{selben} \emph{Ort}] wird die zweite Show stattfinden. \\
\spacebr{}tomorrow evening at.the same time at.the same venue will the second show happen\\
\glt `Tomorrow evening, the second show will happen at the same time, at the same venue.'

\ex\label{ex-morgenabend}
\gll [\emph{Morgen} \emph{Abend}] wird \emph{am} \emph{selben} \emph{Ort} die zweite Show \emph{zur} \emph{selben} \emph{Zeit} stattfinden. \\
\spacebr{}tomorrow evening will at.the same venue the second show at.the same time happen\\
\glt `Tomorrow evening, the second show will happen at the same time, at the same venue.'

\ex\label{ex-zweiteshow}
\gll [\emph{Morgen} \emph{Abend}] wird die zweite Show stattfinden, \emph{zur} \emph{selben} \emph{Zeit}, \emph{am} \emph{selben} \emph{Ort}. \\
\spacebr{}tomorrow evening will the second show happen at.the same time at.the same venue \\
\glt `Tomorrow evening, the second show will happen at the same time, at the same venue.'
\zl

\largerpage
\noindent
Typically, in clauses based on head-final VPs, unlike head-initial ones (see \citealp[12, 43]{Haider2015,Haider2010}), adverbials may intervene between the arguments of the verb. In addition, there is – like in \ili{English} – room in the clause-initial area. Here, multiple adjuncts, unlike arguments, may be stacked in \ili{German} (cf.\ (\ref{ex-tomorrow})).  Eventually, there is the clause-final areas, the extraposition range at the end of VPs, which provides structural space for extraposed adverbial PPs and clauses. Altogether, this amounts to more than fourteen additional word order variants for (\ref{ex-morgen}), two of which are listed as (\ref{ex-morgenabend}) and (\ref{ex-zweiteshow}). In \ili{English}, the head-initial phrase-structure and the SVO clause structure restrict the kind of adverbial phrases in (\ref{ex-tomorrow}) to the peripheral positions of the clause.

That adjuncts ``require virtually no grammatical support'' is a correct observation, but ``virtually'' is an essential part of this characterisation. The support they need is the availability of a syntactic position. Adjuncts lack this minimal grammatical support whenever the adjunction site is outside the directionality domain of the head of the phrase they are adjoined to. Exactly in this case, LLC comes into play and guarantees that the adjunct is tightly ``glued'' to its host phrase.



\section*{\acknowledgmentsUS}

Gratefully, I acknowledge the helpful feedback by Stefan Müller and by two anonymous reviewers. 

{\sloppy
\printbibliography[heading=subbibliography,notkeyword=this]
}
\end{document}



% en
%      <!-- Local IspellDict: en_GB-ise-w_accents -->

