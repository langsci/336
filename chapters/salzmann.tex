\documentclass[output=paper,colorlinks,citecolor=black,
% hidelinks,
% showindex
]{langscibook}

\IfFileExists{../localcommands.tex}{%hack to check whether this is being compiled as part of a collection or standalone
   \usepackage{orcidlink}

% add all extra packages you need to load to this file


% Haitao Liu
%\usepackage{xeCJK}
%\setCJKmainfont{SimSun}
%\setCJKmainfont[Scale=MatchUppercase,
%                Path=fonts/
%]{SourceHanSerifSC-Regular}

% instead use option:  ,chinesefont % for references in raffelsiefen.tex
% loading the package changes some spacings


\usepackage{multicol}
\usepackage{tikz}\usetikzlibrary{decorations.pathreplacing}
\usepackage{url}
\urlstyle{same}

%\usepackage{listings}
%\lstset{basicstyle=\ttfamily,tabsize=2,breaklines=true}

\usepackage{langsci-basic}
\usepackage{langsci-optional}
\usepackage[danger]{langsci-lgr}

% toggle danger in texlive 2021
%\newcommand{\M}{\textsc{m}\xspace}

% toggle danger in texlive 2021 or uncomment this
% \newcommand{\N}{\textsc{n}\xspace}
% \newcommand{\F}{\textsc{f}\xspace}


\usepackage{./styles/biblatex-series-number-checks}




\usepackage{langsci-gb4e}




% Demske

\usepackage{tipa}
\usepackage{styles/avm+}
%\usepackage{styles/merkmalstruktur}
\avmfont{\sc}
\usepackage{langsci-forest-setup}
\usepackage{xspace}
%\usepackage{styles/abbrev} 

\usepackage{soul}
\usepackage{color}
\newcommand{\rem}[1]{\textcolor{red}{\st{#1}}}
\newcommand{\add}[1]{\textcolor{blue}{\ul{#1}}}


% Salzmann

\usepackage[nocenter]{qtree}


% Müller

% add this to the default preamble 
\forestset{default preamble={
    for tree={anchor=north},
}}


\usepackage{german}

%\usepackage{german}
\selectlanguage{USenglish}

% Mit Babel geht irgendwie die hyphenation nicht richtig
%\usepackage[ngerman,english]{babel}
%\useshorthands{"} 
%\addto\extrasenglish{\languageshorthands{ngerman}}

\usepackage{styles/makros.2020,
styles/abbrev,
styles/merkmalstruktur,
styles/article-ex,styles/eng-date}


\usepackage{todonotes}
\newcommand{\todostefan}[1]{\todo[color=green!40]{\footnotesize #1}\xspace}
\newcommand{\inlinetodo}[1]{\todo[color=green!40,inline]{\footnotesize #1}\xspace}

\newcommand{\inlinetodoopt}[1]{\todo[color=green!40,inline]{\footnotesize #1}\xspace}
\newcommand{\inlinetodoobl}[1]{\todo[color=red!40,inline]{\footnotesize #1}\xspace}

\newcommand{\itdobl}[1]{\inlinetodoobl{#1}}
\newcommand{\itdopt}[1]{\inlinetodoopt{#1}}

\newcommand{\addpages}{\todostefan{add pages}}

%\newcommand{\iaddpages}{\yel[add pages]{pages}\xspace}



% subfigure
\usepackage{subcaption}



% Nolda
%\usepackage[main=british,nil,german,french]{babel}
\newcommand{\foreignlanguagedummy}[2]{#2}
\usepackage{tagpair}
\usepackage{hang}
\usepackage[noconfig]{ntheorem}
\usepackage{pstricks,pst-node,pst-tree}
\usepackage{newunicodechar}



   \newcommand*{\orcid}[1]{}

% do not show the chapter number. It is redundant, since most references to figures are within the
% same chapter.
\renewcommand{\thefigure}{\arabic{figure}}

\newcommand{\rlapsub}[1]{\rlap{\sub{#1}}}

% \SetupAffiliations{output in groups = false, 
%                    separator between two = {\bigskip\\},
%                    separator between multiple = {\bigskip\\},
%                    separator between final two = {\bigskip\\}
%                    }


%%%%%%%%Alte Umlaute
\newcommand{\oldae}{$\stackrel{\textrm{\tiny e}}{\textrm{a}}$}
\newcommand{\oldoe}{$\stackrel{\textrm{\tiny e}}{\textrm{o}}$}
\newcommand{\oldue}{$\stackrel{\textrm{\tiny e}}{\textrm{u}}$}

\newcommand{\refl}{\REFL}
\newcommand{\pst}{\PST}


% Müller
\let\vref\ref


\let\citew\citet

\newcommand{\page}{}

% biblatex stuff
% get rid of initials for Carl J. Pollard and Carl Pollard in the main text:
\ExecuteBibliographyOptions{uniquename=false}




\newcommand{\nom}{\textsc{nom}}
\newcommand{\gen}{\textsc{gen}}
\newcommand{\dat}{\textsc{dat}}
\newcommand{\acc}{\textsc{acc}}


%\newcommand{\spacebr}{\hspaceThis{[}}



\newcommand{\acknowledgmentsEN}{Acknowledgements}
\newcommand{\acknowledgmentsUS}{Acknowledgments}


% no bf!!!111!
\let\textbfemph\emph

\newcommand{\textbfremoved}[1]{#1}
%\newcommand{\emphremoved}[1]{#1}


\newcommand{\noemph}[1]{#1}
\newcommand{\underlineemph}[1]{\emph{#1}}




% for editing, remove later
\usepackage{xcolor}
\newcommand{\added}[1]{{\red #1}}
\newcommand{\addedthis}{\todostefan{added this}}

\newcommand{\changed}[1]{\textcolor{orange}{#1}}




% Nolda

\theorembodyfont{\normalfont}
\let\restriction\relax
\renewtheoremstyle{break}{\item{\itshape ##1\ ##2}\newline\nopagebreak}{\item{\itshape ##1\ ##2\ (##3)}\newline\nopagebreak}
\theoremstyle{break}
\newtheorem{definition}{Definition}
\newtheorem{pattern}{Pattern}
\newtheorem{restriction}{Restriction}
\newunicodechar{‑}{\hbox{-}}
\newunicodechar{…}{\dots}
\newunicodechar{⁡}{\relax}
\newunicodechar{⁣}{\relax}
\newunicodechar{⁀}{\raisebox{+1ex}{\ensuremath\frown}}
\newunicodechar{⁐}{\raisebox{+1ex}{\ensuremath\frown}\setbox1=\hbox{\ensuremath\smile}\hspace{-\wd1}\raisebox{-1ex}{\ensuremath\smile}}
\newunicodechar{⪪}{\ensuremath{<\mathrel{\llap{\ensuremath{-}}}}}
\setkomafont{descriptionlabel}{\normalfont}
\ExecuteBibliographyOptions{labeldate=comp,labelnumber=true,defernumbers=true}
\defbibenvironment{sources}{\list{\printfield{labelprefix}\,\printfield{labelnumber}}{\settowidth{\labelwidth}{S\,0}\setlength{\labelsep}{\biblabelsep}\setlength{\leftmargin}{\labelwidth}\addtolength{\leftmargin}{\labelsep}\setlength{\itemsep}{\bibitemsep}\setlength{\parsep}{\bibparsep}}\renewcommand{\makelabel}[1]{##1\hfil}}{\endlist}{\item}
\newcommand{\citesource}[1]{\citefield{#1}{labelprefix}\,\citefield{#1}{labelnumber}}


% Was soll das machen?
\newcommand{\textstyleFootnoteSymbol}{}



% Was ist das???? St. Mü. 30.10.2021
%Kann weg. Damit waren die bücker transkripte aligniert. Habe das jetzt mit tabularx und hphantom gemacht

%\newlength{\calength} %tmp length to store the space 1. until [; 2. until ].
%
%%first argument speaker ID, second argument text. Optional argument left margin indicator (arrow or similar)
%\newcommand{\cabox}[3][]{\parbox{0mm}{\hspace*{-1cm}#1}%
%\parbox{1.5cm}{#2}%
%\parbox{9.6cm}{#3}\\%
%}
%
%%translation. First parbox is empty, second parbox takes the translation text
%\newcommand{\trsbox}[1]{\parbox{1.5cm}{~}%
%\parbox{9.6cm}{\itshape #1}\\%
%}
%
%%store the width of a string.
%\newcommand{\settablength}[1]{\settowidth{\calength}{#1}\global\calength=\calength}
%
%%print string and store its width. Useful if the first item of the aligned set is also the longest
%\newcommand{\inittab}[1]{#1\settablength{#1}}
%
%%insert horizontal white space equivalent to the stored width
%\newcommand{\skiptab}{\parbox{\calength}{~}}
%
%%print the argument and fill up with horizontal white space until the stored width is reached.
%\newcommand{\filledtab}[1]{\parbox{\calength}{#1}}



% for standalone compilations Felix: This is in the class already
%\let\thetitle\@title
%\let\theauthor\@author 
\makeatletter
\newcommand{\togglepaper}[1][0]{ 
\bibliography{../bib-abbr,../stmue,../localbibliography,
collection.bib}
  %% hyphenation points for line breaks
%% Normally, automatic hyphenation in LaTeX is very good
%% If a word is mis-hyphenated, add it to this file
%%
%% add information to TeX file before \begin{document} with:
%% %% hyphenation points for line breaks
%% Normally, automatic hyphenation in LaTeX is very good
%% If a word is mis-hyphenated, add it to this file
%%
%% add information to TeX file before \begin{document} with:
%% \include{localhyphenation}
\hyphenation{
Arsch
anaph-o-ra
Bü-cking
con-stit-u-ents
Dor-drecht
For-schungs-ge-mein-schaft
Ge-schich-te
ha-ben
pho-nol-o-gy
pro-so-dic
pro-so-di-cally
Sal-pe-ter
sei-nen
Wil-liams
}
\hyphenation{
Arsch
anaph-o-ra
Bü-cking
con-stit-u-ents
Dor-drecht
For-schungs-ge-mein-schaft
Ge-schich-te
ha-ben
pho-nol-o-gy
pro-so-dic
pro-so-di-cally
Sal-pe-ter
sei-nen
Wil-liams
}
  % \memoizeset{
  %   memo filename prefix={hpsg-handbook.memo.dir/},
  %   % readonly
  % }
  \papernote{\scriptsize\normalfont
    \@author.
    \titleTemp. 
    To appear in: 
    Ulrike Freywald \& Horst Simon (eds.) Headedness and/or grammatical anarchy?
    Berlin: Language Science Press. [preliminary page numbering]
  }
  \pagenumbering{roman}
  \setcounter{chapter}{#1}
  \addtocounter{chapter}{-1}
}
\makeatother



% This does a linebreak for \gll for long sentences leaving space for the language at the right
% margin. The factor .0989 is needed since otherwise starred examples cause a linebreak.
% St.Mü. 17.06.2021 08.02.2021
\newcommand{\longexampleandlanguage}[2]{%
%\begin{tabularx}{.99\linewidth}[t]{@{}X@{}p{\widthof{(#2)}}@{}}%
%\begin{minipage}[t]{.99\linewidth}%
\begin{tabularx}{\linewidth}[t]{@{}X@{}p{\widthof{(#2)}}@{}}%
\begin{minipage}[t]{\linewidth}%
#1%
\end{minipage} & (\ili{#2})%
\end{tabularx}}

% ORCIDs in langsci-affiliations 
\usepackage{orcidlink}
\definecolor{orcidlogocol}{cmyk}{0,0,0,1}
\ProvideDocumentCommand{\LinkToORCIDinAffiliations}{ +m }
  {%
    \orcidlink{#1}
  }

   %% hyphenation points for line breaks
%% Normally, automatic hyphenation in LaTeX is very good
%% If a word is mis-hyphenated, add it to this file
%%
%% add information to TeX file before \begin{document} with:
%% %% hyphenation points for line breaks
%% Normally, automatic hyphenation in LaTeX is very good
%% If a word is mis-hyphenated, add it to this file
%%
%% add information to TeX file before \begin{document} with:
%% %% hyphenation points for line breaks
%% Normally, automatic hyphenation in LaTeX is very good
%% If a word is mis-hyphenated, add it to this file
%%
%% add information to TeX file before \begin{document} with:
%% \include{localhyphenation}
\hyphenation{
Arsch
anaph-o-ra
Bü-cking
con-stit-u-ents
Dor-drecht
For-schungs-ge-mein-schaft
Ge-schich-te
ha-ben
pho-nol-o-gy
pro-so-dic
pro-so-di-cally
Sal-pe-ter
sei-nen
Wil-liams
}
\hyphenation{
Arsch
anaph-o-ra
Bü-cking
con-stit-u-ents
Dor-drecht
For-schungs-ge-mein-schaft
Ge-schich-te
ha-ben
pho-nol-o-gy
pro-so-dic
pro-so-di-cally
Sal-pe-ter
sei-nen
Wil-liams
}
\hyphenation{
Arsch
anaph-o-ra
Bü-cking
con-stit-u-ents
Dor-drecht
For-schungs-ge-mein-schaft
Ge-schich-te
ha-ben
pho-nol-o-gy
pro-so-dic
pro-so-di-cally
Sal-pe-ter
sei-nen
Wil-liams
}
    \bibliography{localbibliography}
    \togglepaper[23]
}{}

\author{Martin Salzmann\orcid{0000-0002-6153-3025}\affiliation{University of Pennsylvania}}

\title{The NP vs. DP-debate and notions of headedness}
\abstract{Much of the NP- vs. DP-debate has relied on largely conceptual and theory-internal arguments. In this paper, I instead discuss  well-established concepts of headedness and explore their relevance for the NP vs. DP-debate. I will rely on two simple and arguably theory-neutral concepts: (i) the fact that there is an asymmetric relationship between head and non-head regarding selection and form determination and (ii) the fact that the features of the head are present on the maximal projection and its consequences for distribution, selection and agreement. While not all arguments lead to a conclusive result, the facts overall favor the DP-hypothesis: W.r.t.\ the asymmetry between D and N, we will see some evidence that D selects N(P). Facts from categorial selection,  selection of particular forms of the D-position and from agreement with hybrid nouns suggest that the features of D rather than those of N are present on the maximal projection. This clearly supports the DP-hypothesis.}
  


% %custom footer for preprints
% \papernote{\scriptsize\normalfont
%     Martin Salzmann.
%     The NP vs. DP-debate and notions of headedness.
%     Ulrike Freywald \& Horst J. Simon.  
%     Berlin: Language Science Press. [preliminary page numbering]
% }

\begin{document}
\maketitle

\vspace{-\baselineskip}
\section{Introduction}

\largerpage
It is fair to say that the DP-hypothesis, first proposed in \citet{Abney:1987:Diss}, and illustrated
in the tree diagram in Figure~\ref{fig-dp-the-book}, has been very successful. In most of the current formal syntactic literature, especially that carried out within the Minimalist Program since \citet{Chomsky:1995:minprog}, the DP-hypothesis is usually taken for granted.\footnote{The idea that the determiner is the head of the noun phrase can also be found in literature predating Abney's dissertation, see \citet[77]{Abney:1987:Diss} for references. In the literature on \ili{German}, the NP/DP-debate was most prominent in the late 80s/early 90s, see
\citet{Vater:1984:DeterminantienQuantoren, Vater:1986:NP-StrukturDeutsch} for proponents of the NP-hypothesis and
\citet{Haider:1988:DeutscheNominalphrase,Haider:1992:DP},
\citet{Bhatt:1990:Nominalphrase},
 \citet{Gallmann:1990:DP},
 \citet{Olsen:1991:DP} and
  \citet{Vater:1991:DeterminantienDP} for proponents of the DP-hypothesis; interestingly, the earliest mention of the idea that D is the head can already be found in  \citet[280]{Erben:1980:DeutscheGrammatikAbriss12}.}

\begin{figure}
\Tree [.DP [.XP ] [.D$'$ D\\the NP\\book ]]
\caption{DP-hypothesis}\label{fig-dp-the-book}	 
\end{figure}
Against this background it is somewhat surprising that most of the evidence in its favor is based on theory-internal considerations, either having to do with specific assumptions within GP-theory at the time or presumed parallels between the nominal and clausal structure. Arguments that refer to established concepts of headedness are actually rather rare and have not played a prominent role in the discussion. This paper therefore aims to address this issue by discussing a number of widely-accepted concepts of headedness and applying them to the NP vs. DP-debate. It will be shown that while some of the concepts do not lead to a conclusive result, some actually make clear predictions and help tease apart the two different theories. As we will see, facts from selection and agreement favor the DP-hypothesis, while none of the diagnostics favors the NP-hypothesis. Overall, then, the DP-hypothesis is at an advantage.

This paper is structured as follows: Section~\ref{sec-theory-internal-arguments-for-dp} briefly discusses examples of theory-internal arguments in favor of the DP-hypothesis. Section~\ref{sec-headedness} introduces different concepts of headedness and applies them to the NP vs. DP-debate. Section~\ref{sec-conclusion} concludes.

\section{Theory-internal arguments for the DP-hypothesis}
\label{sec-theory-internal-arguments-for-dp}

%\largerpage[2]
As discussed in \citet{Salzmann:2020:NP-DP}, previous arguments in favor of the DP\hskip0pt-\hskip0pthypothesis can be categorized as follows: 
\renewcommand{\theenumi}{\roman{enumi}}%
\begin{enumerate}
  \item Conceptual arguments that are largely due to specific assumptions of the GB-framework at the time.
  \item  Parallelism arguments based on the presupposition that the clausal and nominal architecture must be very similar.
  \item Constituency arguments showing that N forms a constituent to the exclusion of D.
 \item Head-movement arguments suggesting that there is an X°-position above N. 
\end{enumerate}

\noindent
As shown in \citet{Salzmann:2020:NP-DP}, most if not all of these arguments are inconclusive. The constituency arguments are irrelevant since they do not diagnose headedness. The head-movement arguments are relevant, but strictly speaking, they do not show that the higher head has to be identified with D. For the conceptual and parallelism arguments, reasonable alternatives can be provided within the NP-hypothesis. Since I have discussed this extensively in my previous work (and see also \citealt{Bruening:2009:DP,Bruening:2020:NP-DP}, \citealt{Bruening-et-al:2018:Selection-Idioms-NP}), I will only discuss one concrete case and refer the reader to the references just mentioned for details. 

%\largerpage
This conceptual argument for the DP-hypothesis comes from examples like the following, where in addition to the possessor there is also a prenominal determiner present (such structures are limited in \ili{English} but frequent in other languages, e.g., Hungarian, cf., e.g., \citealt[270-276]{Abney:1987:Diss} and \citealt{Salzmann:2020:NP-DP}):

\ea  John's every secret wish 
\z

\noindent
Such examples posed a problem under the X$'$-theoretic assumptions of the
Government\hskip0pt-\hskip0ptand\hskip0pt-\hskip0ptBinding era because it was assumed that heads
only project one specifier (\citealt[288-297]{Abney:1987:Diss}). Since both the quantifier and the
possessor have to occupy specifier positions of N, analyzing the previous example as in Figure~\ref{fig-np-johns-every-secret-wish} was not an option:

\begin{figure}
	\Tree [.NP [.XP\\John's ] [.N$'$ Det\\every [.N$'$ ZP\\secret N$'$\\wish ]	]]
\caption{Co-occurrence of possessor and determiner under the NP-hypothesis}\label{fig-np-johns-every-secret-wish}
	\end{figure}
	
No such problems arise under the DP-hypothesis, where the possessor occupies the specifier,
while the determiner is in the D-position. The respective structure is given in
  Figure~\ref{fig-dp-johns-every-secret-wish}.
	
\begin{figure}
\Tree [.DP [.XP\\John's ] [.D$'$ D\\every [.NP [.ZP\\secret ] N$'$\\wish ]]]
\caption{Co-occurrence of possessor and determiner under the DP-hypothesis}
\label{fig-dp-johns-every-secret-wish}
\end{figure}

While consistent with the assumptions at the time, the restriction to just one specifier has been
given up in the meantime. One obvious reason for this are languages with multiple
\textit{wh}-fronting where at least under some analyses, all fronted \textit{wh}-phrases occupy
specifiers of the same head. Multiple specifiers are also frequently postulated in scrambling
languages. See, e.g., 
%\citet{Heck-Himmelreich:2017:OpaqueIntervention}
\citeauthor{Heck-Himmelreich:2017:OpaqueIntervention} \citeyearpar{Heck-Himmelreich:2017:OpaqueIntervention} 
for cases where multiple scrambling targets different specifiers of v.

Even in languages like \ili{English} where there there is no overt evidence for several specifiers of the same head being occupied at the same time, multiple specifier configurations can arise during the derivation, e.g., when a \textit{wh}-object undergoes successive-cyclic movement via Spec,vP.

\section{Concepts of headedness and their implications for the NP/DP debate}
\label{sec-headedness}

I take the following concepts of headedness to be well-established and uncontroversial since they rely on simple phrase-structural properties:\footnote{I ignore semantic concepts of headedness as discussed in \citet{Zwicky:1985:heads}, since they generally do not lead to useful results w.r.t.\ the relevant properties: In a PP or an Aux-VP-combination, one will classify the noun or the verb as the semantic head. But with respect to their syntactic behavior, it is clear that the preposition and the auxiliary are the heads instead. Consequently, nothing much can be gained by classifying the noun as the semantic head of the noun phrase.}

\begin{itemize}
	\item the head and the complement are in an asymmetric relationship %the head selects the complement and can determine its features (case, morphological selection)
	
	\item the features of the head are present on the maximal projection 
\end{itemize}%\vspace*{0.2cm}

\noindent
As far as I can tell, these criteria are shared by most contemporary syntacticians, which makes them largely theory-neutral (as long as phrase-structure is adopted).

I will now explore the implications of these concepts for the NP/DP debate. The first property of heads has consequences for selection and form determination between D and N. The second affects the distribution of the constituent of which X is the head and its interactions with material outside its projection w.r.t.\ selection and agreement. Most of these arguments are discussed in more detail in \citet{Salzmann:2020:NP-DP}.

\subsection{Head/non-head asymmetry}

This asymmetry is very clear when we look at verbs and their arguments: It is the verb that selects the argument, e.g., an NP, and it is the verb that determines its form, e.g., by assigning it a case.

\subsubsection{Selection}

When we look at the noun phrase, the result of applying this criterion is not fully clear. On the one hand, D-elements like the definite determiner do not occur without a noun. On the other hand, there are noun phrases without an overt determiner, e.g., with bare plurals:

\ea books
\z

\noindent
This might suggest that the determiner is the head. However, this is in fact far from clear since there is a large body of literature suggesting that in cases like the one just mentioned, there is in fact a silent determiner (I will come back to this below). Note also that the determiner must be present with singular nouns. 

Another aspect to be considered when applying the selection criterion is to ascertain whether one is dealing with syntactic or semantic selection. The fact that the determiner requires a noun could simply be due to the fact that it is specified to combine with a predicate to return an individual (cf. \citealt{Longobardi:1994:Reference-properNames}). However, such interactions also exist between adverbs and verbal projections of different sizes, but nobody would treat the adverb as the syntactic selector. Furthermore, nothing in the semantic composition requires D to be the syntactic head (cf. \citealt[31]{Bruening:2009:DP} for more discussion).

In the case at hand, however, it can be shown that semantic selection is not sufficient. As pointed
out in \citet{Larson:to-appear:DP-hypothesisDP-CP}, determiners cannot combine with just any
predicate. Rather, they require a nominal predicate: 

\ea Every man/happiness/*happy
\z 

\noindent
This suggests indeed that D selects N. Under the assumption that only heads can select (cf. also \citealt{Zwicky:1985:heads} for discussion), this would argue in favor of the DP-hypothesis. While arguably the standard assumption, there have been proposals suggesting that non-heads can select, too, see \citet{Bruening-et-al:2018:Selection-Idioms-NP}, who argue for selection by D under the NP-hypothesis. Thus, one probably shouldn't draw any strong conclusions from this.\footnote{A popular concept of headedness is obligatoriness. It overlaps with the asymmetry argument in that the selector is also obligatory (unless it is elided, which is irrelevant, as this is phonetic deletion). The limits of the argument become clear once cases are considered where the dependent argument cannot be omitted. This not only holds for objects of verbs like \textit{devour}, but also for complements of prepositions. W.r.t.\ the noun phrase, we can observe that both determiner and noun are present and neither can be omitted (assuming there to be a silent determiner with bare plurals). Consequently, the obligatoriness criterion leads to an inconclusive result when applied to the noun.}

\subsubsection{Form determination}

Form determination refers to phenomena like case-government and morphological selection in verbal complexes (where the auxiliary/functional verb determines the form of the lexical verb). In the following example, each verbal element determines the form of the immediately subordinate verb (from \citealt[30]{Bruening:2009:DP}):


\ea I might have been being handed some cocaine (when the police caught me).
\z

\noindent
It is difficult to apply the criterion to the noun phrase, though. \citet{Bruening:2009:DP} argues that the noun should be considered the head because its features determine those of NP-modifiers like adjectives and determiners, which show concord with the head noun in gender and number.\footnote{Double definiteness in \ili{Germanic} could potentially be considered a case where D determines features of the noun, but that largely depends on one's analysis of the phenomenon, and it is far from obvious that this phenomenon should be subsumed under concord, see, e.g., \citet{Schoorlemmer:2012:DefinitenessGermanic} for recent discussion.} However, concord is to be distinguished from government. In government by verbs, a verb governing accusative case on an object is not accusative itself; it does not share a feature with the object; rather, it assigns a feature to the object for which it is not specified itself (case is just a probe feature). Concord, on the other hand, involves the sharing of features. 

 There are, to my knowledge, no cases of form determination within the noun phrase that would clearly identify either D or N as the head. Strong and weak inflection on adjectives in \ili{Germanic} (where the adjective covaries with the definiteness/shape of the D-element) shows that D can affect the form of other constituents within the noun phrase. This may suggest that D is indeed a head, from which one may conclude that it must be the head of the noun phrase. While plausible, it relies on the assumption that probes must be heads and cannot be phrasal. While the predominant view, alternative conceptions have been proposed. Probing by phrasal elements is arguably inevitable if concord within the noun phrase involves a phi-probe on A targeting N (and if A is adjoined to NP).\footnote{Another possibility may be the genitive of quantification in \ili{Slavic}, where the NP in the scope of the quantifier/numeral appears in the genitive. If quantifier/numeral occupy the head of the DP, this would indeed represent a case of form determination. In some of the analyses, however, cf., e.g., \citet{Boskovic:2006:CaseAgreementGenQuant}, the quantifier actually occupies a specifier position, which renders the argument inconclusive.} 
 Thus, no argument can be made for either the NP- or DP-hypothesis on the basis of form determination.

\subsection{The features of the head are present on the maximal projection}

 This property of heads has the consequence that it is the features of the head that are visible to the syntactic context outside the noun phrase. Features of other noun phrase internal constituents, however, are less visible. 
 
 This has the following syntactic implications: First, the head determines the distribution of the phrase since its category label is visible on the projection. Second, the head is the preferred element for higher heads to interact with. This can be seen in that the head/the features of the head are the preferred target for selection and agreement. 

 \subsubsection{Distribution}
 
 The question is whether the distribution of noun phrases is due to the categorial properties of D
 or those of N. Since the distribution of pronouns and noun phrases is very similar and since verbs
 can be combined with both nouns/noun phrases and pronouns, this would seem to favor the
 DP-hypothesis; one could simply state that verbs generally combine with D(Ps). However, given that
 pronouns are frequently reanalyzed as D-heads taking a silent NP-complement
 (\citealt{Elbourne:2005:Situations-Individuals}), it is not inconceivable that they could also be
 reanalyzed as NPs with only the determiner overt as in the right-hand tree in Figure~\ref{fig-determiner-overt}.
 
\begin{figure}
  \begin{subfigure}{.48\textwidth}
    \centering
 {\Tree [.DP D [.\sout{NP} ]]}
\end{subfigure}
  \begin{subfigure}{.48\textwidth}
    \centering
 {\Tree [.NP Det [.\sout{N$'$}  ]]}
\end{subfigure}
\caption{Pronouns under the DP- and NP-hypothesis}\label{fig-determiner-overt}
\end{figure}
  
 Thus, in either case, the similar distribution of nouns/noun phrases and pronouns would be due to whatever categorial feature is on the maximal projection in both cases, viz., either D or N. Thus, arguments from distribution do not help to decide the NP/DP-debate.\footnote{Pronouns sometimes display different distribution than nouns, but that usually concerns weak pronouns. For instance, in \ili{German}, weak pronouns are fronted to the Wackernagel position and weak object pronouns cannot occur in Spec,CP. However, this restriction is not primarily about their categorial status but about their information-structural possibilities and, associated with that, their prosodic weight (which in some works, however, is reanalyzed as a categorial difference): none of these restrictions apply to strong pronouns. 
 
  \ili{German} personal pronouns are different from demonstrative pronouns in important respects that argue against a silent NP-complement, see \citet[6, fn. 4]{Salzmann:2020:NP-DP} for discussion.}
 
 \subsubsection{Selection from outside}
 
 There are three different aspects of selection that are relevant here, viz., selection of particular phi- or definiteness features of noun phrases, categorial selection, and third, selection of particular lexical items within noun phrases in the context of idioms, addressing an argument from \citet{Bruening-et-al:2018:Selection-Idioms-NP}. 
 
 \paragraph{Selecting features of D vs. N}
 
  Morphological selection is assumed to target features of the head in a selector's complement. In the case of verbal complements, this can be features such as +/-- wh, +/-- V2, +/-- subjunctive or specific non-finite forms (participle, bare infinitive, \textit{to}-infinitive) 
  
 As pointed out in \citet{Bruening:2009:DP}, there do not seem to be any cases where the verb selects D-related properties such as a particular definiteness value or particular determiners (but see the next subsection). This does not support the DP-hypothesis. However, one also does not find any cases where a verb selects properties of N. One does find semantic selection, e.g., selection of an animate/inanimate noun, but I am not aware of any cases where the verb selects any particular morpho-syntactic features of N like [gender] or [number] (to the extent that the latter really is a feature of N and not of Num). Consequently, the facts from morphological selection are inconclusive w.r.t.\ the NP/DP-debate.
 
\paragraph{Categorial selection} 
\label{salzmann:sec-categorial-selection}

 \citet[6]{Bruening-et-al:2018:Selection-Idioms-NP} claim that there is an important asymmetry between the selection of verbal and nominal complements: While verbs can select verbal complements of different sizes like CP, TP, vP and VP, there do not seem to be any cases where verbs select nouns of different types, e.g., DP vs. NP. 
 
 While this is generally correct, there are cases that suggest that categorial selection of different types of noun phrases may be necessary after all. This holds quite generally for pseudo-incorporation, which often involves NP-objects, which are predicates and thus compose differently with the verb. Crucially, contrary to the claims in \citet{Bruening-et-al:2018:Selection-Idioms-NP}; \citet{Bruening:2020:NP-DP}, it is not the case that each verb of a language can occur with both `regular' and pseudo-incorporated nouns. The class of verbs that allows pseudo-incorporation is always restricted. For instance, as discussed in \citet{Kallulli:1999:Dissertation-Albanian}, in \ili{Albanian}, a language that generally allows bare count singulars, pseudo-incorporation is blocked with individual-level predicates like `love', `hate', `admire', `respect'. Furthermore,  it is often observed that verb and noun together express an ``institutionalized activity'', see \citet[164-165]{Dayal:2011:HindiPsuedoIncor}. There is thus clearly a selectional component.  Under the DP-hypothesis, one can state that some verbs allow the selection of NPs in addition to DPs. Under the NP-hypothesis, the difference can arguably only be captured by means of semantic selection, viz., some verbs can select predicates in addition to individuals.

However, not all cases can be handled by means of differences in semantic selection. \citet{Erschler:2019:DP-Ossetic} discusses a comitative preposition in \ili{Ossetic} that selects NumP but is crucially not compatible with DP (he shows that the selection must be morphosyntactic rather than semantic). The need for categorial selection is even more obvious w.r.t.\ the distribution of bare noun constructions and weak definites. Both are semantically very similar in that they covary under ellipsis and quantification and do not support anaphora (\citealt{Carlson-et-al:2006:WeakDefinites}, \citealt{Aguilar:2014:WeakDefinites}). The following two examples illustrate covariation under quantification. In both cases, a distributive reading is possible (in fact by far the most salient if not even only reading).

\ea 
\ea {
    \gll Jeder Schüler spielt Klavier.\\
        every student plays piano\\\hfill(\ili{German})
        \glt `Every student plays the piano.' 
        }
\ex {
    \gll Jeder Schüler bleibt im Bett.\\
        every student stays in.the bed\\
        \glt `Every student stays in bed.' 
        }
    \z
    \z    
        
\noindent
What is relevant in the case at hand is that the two construction types are in complementary distribution within a language; given that they have the same semantics, the distribution of the presence/absence of the definite article cannot be captured in semantic terms. The fact that the distribution has to be captured in morphosyntactic terms becomes particularly obvious once \ili{English} is compared with \ili{German}. While there are cases where the languages pattern the same (e.g., uses of the weak definite construction as in `take the bus' or the bare noun construction as in `take to court'), there are several cases where the distribution is the opposite. \ili{English} uses the weak definite in the expressions \textit{play *(the) piano, read *(the) newspaper, listen to *(the) radio}, while \ili{German} uses the bare noun construction in these cases. However, it is not always the case that a weak definite in \ili{English} corresponds to a bare noun in \ili{German}. The reverse can also be found: We find bare nouns in the following \ili{English} expressions \textit{stay in (*the) bed, go to (*the) church, be in (*the) jail}, while \ili{German} requires the weak definite.

How can this distribution be captured and what does it imply for the NP/DP debate? Under the DP-hypothesis, one can handle the distribution by means of categorial selection. Certain verbs or prepositions (in certain collocations) select either a  full DP as in \textit{take the bus}, other verbs or prepositions select a bare NP in certain cases as in \textit{stay in bed}. Under the NP-hypothesis, the challenge arises to ensure that in some cases only a bare noun is possible. However, determiners are modifiers under the NP-hypothesis and therefore it should always be possible to add them (with count nouns). One cannot use selection here: One cannot assume that V/P selects an N which in turn selects nothing. 

 \paragraph{Selection of D-elements in idioms}
 
 \citet{Bruening-et-al:2018:Selection-Idioms-NP} and \citet{Bruening:2020:NP-DP} discuss selection relationships in conventionalized expressions/idioms. They show that these expressions always consist of (potentially a sequence of) local relations, which are mainly government relations between heads. There can be open slots, but they never affect heads in the government sequence but only left branches or the lowest complement. This is illustrated by the following \ili{German} example:

\largerpage
 \ea {
 \gll Gefahr laufen, zu ...\\
    risk run to\\\hfill(\ili{German})
    \glt `run the risk to \ldots'
    }
    \z
 
\noindent
Here, the verb and the object are fixed, as is the specification of the non-finite complement clause attached to the noun, which has to be a \textit{to}-infinitive. However, everything else in that non-finite clause is open. The open slot is the structurally lowest position. Open slots in the middle of the government sequence, however, do not seem to be found. For instance, while there are idioms involving V+P+N, there are no idioms involving just V+N with P being completely open. While this may seem like a complicated and counter-intuitive approach to idioms (and it is not fully clear how such an idiom would be represented in the lexicon), it should be pointed out that this view avoids the pitfalls of constituency-based approaches since idioms crucially need not form a syntactic constituent. Thus, Bruening's approach seems like an interesting proposal to capture what a possible idiom can look like.
 
 What is relevant in the current context is that, according to the authors, there is an important asymmetry between the verbal and the nominal domain. While conventionalized expressions can involve verbs selecting functional heads with particular properties, e.g. +wh-clauses (as in \textit{know which way the wind blows}), the \textit{to}-infinitive discussed above or particular prepositions, there do not seem to be conventionalized expressions involving nominals where the D-position is fixed. They argue that even if there is a default specification for the D (e.g., \textit{the} or \textit{a} or no determiner at all), they argue that the choice of D can always vary. The type of examples Bruening has in mind are as follows: the idiom \textit{foot the bill} normally takes the definite determiner, but one can find variations as in the following example:
 
\largerpage
 \ea Taxpayers must foot another bill
 \z
 
\noindent
 The same goes for idioms with indefinite determiners like \emph{beat a dead horse}. The canonical form of the idiom includes an indefinite determiner, but one can find variants like the following:
 
 \ea politicians who continue to beat the dead horse that all the economy needs to be robust is for rich people \ldots\footnote{
\url{https://www.english-corpora.org/iweb/}}
 \z

\noindent
Crucially, what cannot vary, however, is the content of N. This strongly suggests that N+V form a closer unit than N and D and thus favors the NP-hypothesis. Under the DP-hypothesis, one would expect the D-position to be fixed.
 
 In what follows I will take issue with this argument (see also \citealt{Larson:to-appear:DP-hypothesisDP-CP} for similar criticism). First, both in \ili{English} and in \ili{German}, there is usually a canonical form for the D-position, even if the D-position is to some extent variable. For instance, in the following two \ili{German} idioms, it is clear that the canonical form is either definite or indefinite:

 \ea
 \ea {
 \gll ins Gras beissen\\
    in.the grass bite\\\hfill(\ili{German})
    \glt `die'
    }
\ex {
    \gll jemandem einen Korb geben\\
        to.someone a basket give\\
    \glt    `reject someone'
    }
    \z
    \z
    
\noindent
Thus, if there is a default specification, this means that   we have rather specific knowledge about the form of the D-position. Thus suggests that selection is needed after all. Second, there are many idioms where no flexibility of the D-position can be found, neither via google nor via the DeReKo-corpus, the largest corpus of written \ili{German} (https://www1.ids-mannheim.de/kl/projekte/korpora.html). Here are a few \ili{German} examples where the D-position is either empty (bare singular, bare plural), definite or indefinite:
    
    \ea
    \ea {
    \gll Leine ziehen\\
        leash pull\\\hfill(\ili{German})
       \glt  `to get lost'
    }
    \ex {
    \gll Bände sprechen\\
        volumes speak\\
        \glt `to speak volumes'
    }
    \ex {
    \gll die Flinte ins Korn werfen\\
        the shotgun into.the grain throw\\
        \glt `to give up'
    }
    \ex {
    \gll ein Licht aufgehen\\
        a light appear\\
      \glt  `to dawn upon'
    }
    \z
    \z
   
\noindent
Third, concerning variability, one has to distinguish different cases. There are indeed cases of free variation. Some \ili{German} examples can be found in \citet[209]{Fleischer:1982:Phraseologie}, of which I represent one where the determiner can be `all', `both' or `the':
   
   \ea {
   \gll alle / beide / die Hände voll zu tun haben\\
        all  {} both {} the hands full to do have\\\hfill(\ili{German})
    \glt `have one's hands full' 
   }
   \z
   
\noindent
In many other cases, the variation largely involves creative language use that is generally possible
with idioms but not indicative of an open slot. Often, the effect of choosing different Ds affects
the quantification of the event (cf. \textit{foot another bill} above). As competent speakers and
cooperative hearers, we can also play with this. For instance, while the  following opaque idiom
normally does not allow any variation regarding the D-position, once we force it, one can probably
obtain a plausible interpretation nevertheless (recall that \textit{ins Gras beissen} 
means `to die'): 
    
    \ea {
    \gll in ein anderes Gras beissen als sein Vater\\
    in a different grass bite like his father\\\hfill(\ili{German})
    \glt `die in a different way than his father'
    }
    \z
    
\noindent
Thus, by deviating from the canonical form of the idiom, one can achieve a certain effect that is part of creative language use, but this does not imply that the D-position is generally free. Crucially, since the nature of the D-position cannot be predicted on the basis of semantics, it will have to be regulated by syntax, viz., by selection. 
    
    Consequently, the argument from idioms actually favors the DP-hypothesis because it allows direct selection of the D-position. The presence of absence of a D-element can, in addition, be handled by means of categorial selection (NP vs. DP). Under the NP-hypothesis, serious problems arise. To account for idioms without a D, one would have to prevent the presence of a D, but since determiners are modifiers, there is no obvious way to do that. Furthermore, to ensure that a specific D occurs, one would have to select an N which in turn selects a particular type of D. While doable, this solution would be more complicated than direct selection as under the DP-hypothesis.

\subsubsection{Agreement}
\label{salzmann:sec-agreement}
 
Given that the features of the head of the noun phrase are present on the maximal projection, we
 expect the (features of the) head to be the preferred goal for probes, e.g., those initiating
 agreement, outside the noun phrase. Access to the non-head will be blocked by Relativized
 Minimality/the A-over-A-principle (\citealt{Chomsky:1973:condTrans}). The two different theories
 thus make crucially different predictions here. Under the DP-hypothesis, D will be the preferred
 target, while N and other constituents may be inaccessible. Under the NP-hypothesis, the reverse
 prediction is made: It is N that should be the preferred target, while D should be less
 accessible. The crucial differences are indicated in the tree diagrams given in Figure~\ref{fig-agreement-np-dp} (potentially inaccessible material is set in gray).
 
\begin{figure}
\begin{subfigure}{.48\textwidth}
\centering
 { v + \Tree [.DP \textcolor{gray}{XP} [.D$'$ D [.\textcolor{gray}{NP} ]]] 
	} 
\end{subfigure}
\begin{subfigure}{.48\textwidth}
\centering
{v + \Tree [.NP \textcolor{gray}{DetP} [.N$'$ N ]]
	}
\end{subfigure}
\caption{Minimality under the DP- and NP-hypothesis}\label{fig-agreement-np-dp}
\end{figure}

The predictions are thus rather clear here. However, since D and N normally agree in phi-features through concord, it is difficult to find cases that would help tease apart the two theories. In \citet{Salzmann:2020:NP-DP} I argue that agreement switches within the noun phrase with hybrid nouns in \ili{Bosnian}-\ili{Croatian}-\ili{Serbian} favors the DP-hypothesis: In this language, certain nouns can trigger biological or grammatical agreement on various nominal and clausal agreement targets. Importantly, there can be switches from grammatical to biological agreement but not the other way around. I argue that this follows most naturally under the DP-hypothesis. Suppose that N bears both gender features, that all heads within DP enter Agree with each other and that each head will only target the next lower head. This can lead to a situation where we find biological agreement only on D. When the verb targets the noun phrase in this case, it can only copy the biological gender feature, the grammatical gender feature on N is inaccessible. While this follows under the DP-hypothesis, the reverse would be predicted under the NP-hypothesis: Since N is the head, both gender features would be present on the maximal projection. Thus, a switch back from biological to grammatical gender between D and v should be possible, contrary to fact.\footnote{Another possible argument for the DP-hypothesis could come from agreement with quantified nouns where in some cases agreement can only target the features of the quantifier and not those of the noun, see \citet{Danon:2013:AgreementAlternationQuantifiedNom}, \citet{Driemel-Stojkovic2019:AgreeQP}. This suggests that the quantifier is the head rather than the noun.}

\section{Conclusion}
\label{sec-conclusion}

While much of the literature on the NP/DP-debate discusses conceptual and theory-internal arguments, this paper has focused on arguments that make direct reference to concepts of headedness. The two criteria I have relied on are (i) the asymmetric relationship between head and non-head regarding selection and form determination and (ii) the fact that the features of the head are present on the maximal projection. 

As we have seen, while several arguments turn out to be inconclusive, the facts overall favor the DP-hypothesis. W.r.t.\ the asymmetry between D and N, there is some evidence that D selects NPs. The facts from categorial selection and the selection of the form of D-elements in idioms suggest that it must be possible to select (i) both DP or NP and (ii) different types of D. This can straightforwardly handled by means of the DP-hypothesis, while under the NP-hypothesis, blocking the presence of a D-head in certain collocations/idioms turns out to create insurmountable problems. Furthermore, agreement facts from hybrid agreement suggest that D is closer to noun phrase external agreement probes than N, which supports the DP-hypothesis. I thus conclude that based on simple and arguably theory-neutral diagnostics for headedness, the DP-hypothesis is at an advantage.

\section*{\acknowledgmentsUS}


Some of the ideas in this paper were presented in my seminar on the syntax of the noun phrase in the winter term 2016/17 at the University of Leipzig. I thank the participants of the seminar for helpful comments. Previous versions of this work were presented at the workshop on headedness at the Freie Universität Berlin (May 2017), the NP-DP workshop at the DGfS in Bremen (February 2019), and the syntax reading group at the University of Pennsylvania (September 2019). I thank the audiences at these occasions for helpful feedback; particular thanks go to Klaus Abels, Farruk Akkus, David Embick, Ulrike Freywald, Hubert Haider, Julie Legate, Stefan Müller, Martin Neef, Andreas Nolda, Andreas Pankau, and Horst Simon. The paper has also greatly benefited from discussions with Benjamin Bruening, Imke Driemel, Florian Schwarz and Malte Zimmermann.

\printbibliography[heading=subbibliography,notkeyword=this]

\end{document}


% en
%      <!-- Local IspellDict: en_US-w_accents -->
