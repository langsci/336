%% -*- coding:utf-8 -*-
\documentclass[output=paper
  ,nobabel
  ,draftmode
  ,colorlinks, citecolor=brown
]{langscibook}

\ChapterDOI{10.5281/zenodo.7142710}

\IfFileExists{../localcommands.tex}{%hack to check whether this is being compiled as part of a collection or standalone
   \usepackage{orcidlink}

% add all extra packages you need to load to this file


% Haitao Liu
%\usepackage{xeCJK}
%\setCJKmainfont{SimSun}
%\setCJKmainfont[Scale=MatchUppercase,
%                Path=fonts/
%]{SourceHanSerifSC-Regular}

% instead use option:  ,chinesefont % for references in raffelsiefen.tex
% loading the package changes some spacings


\usepackage{multicol}
\usepackage{tikz}\usetikzlibrary{decorations.pathreplacing}
\usepackage{url}
\urlstyle{same}

%\usepackage{listings}
%\lstset{basicstyle=\ttfamily,tabsize=2,breaklines=true}

\usepackage{langsci-basic}
\usepackage{langsci-optional}
\usepackage[danger]{langsci-lgr}

% toggle danger in texlive 2021
%\newcommand{\M}{\textsc{m}\xspace}

% toggle danger in texlive 2021 or uncomment this
% \newcommand{\N}{\textsc{n}\xspace}
% \newcommand{\F}{\textsc{f}\xspace}


\usepackage{./styles/biblatex-series-number-checks}




\usepackage{langsci-gb4e}




% Demske

\usepackage{tipa}
\usepackage{styles/avm+}
%\usepackage{styles/merkmalstruktur}
\avmfont{\sc}
\usepackage{langsci-forest-setup}
\usepackage{xspace}
%\usepackage{styles/abbrev} 

\usepackage{soul}
\usepackage{color}
\newcommand{\rem}[1]{\textcolor{red}{\st{#1}}}
\newcommand{\add}[1]{\textcolor{blue}{\ul{#1}}}


% Salzmann

\usepackage[nocenter]{qtree}


% Müller

% add this to the default preamble 
\forestset{default preamble={
    for tree={anchor=north},
}}


\usepackage{german}

%\usepackage{german}
\selectlanguage{USenglish}

% Mit Babel geht irgendwie die hyphenation nicht richtig
%\usepackage[ngerman,english]{babel}
%\useshorthands{"} 
%\addto\extrasenglish{\languageshorthands{ngerman}}

\usepackage{styles/makros.2020,
styles/abbrev,
styles/merkmalstruktur,
styles/article-ex,styles/eng-date}


\usepackage{todonotes}
\newcommand{\todostefan}[1]{\todo[color=green!40]{\footnotesize #1}\xspace}
\newcommand{\inlinetodo}[1]{\todo[color=green!40,inline]{\footnotesize #1}\xspace}

\newcommand{\inlinetodoopt}[1]{\todo[color=green!40,inline]{\footnotesize #1}\xspace}
\newcommand{\inlinetodoobl}[1]{\todo[color=red!40,inline]{\footnotesize #1}\xspace}

\newcommand{\itdobl}[1]{\inlinetodoobl{#1}}
\newcommand{\itdopt}[1]{\inlinetodoopt{#1}}

\newcommand{\addpages}{\todostefan{add pages}}

%\newcommand{\iaddpages}{\yel[add pages]{pages}\xspace}



% subfigure
\usepackage{subcaption}



% Nolda
%\usepackage[main=british,nil,german,french]{babel}
\newcommand{\foreignlanguagedummy}[2]{#2}
\usepackage{tagpair}
\usepackage{hang}
\usepackage[noconfig]{ntheorem}
\usepackage{pstricks,pst-node,pst-tree}
\usepackage{newunicodechar}



   \newcommand*{\orcid}[1]{}

% do not show the chapter number. It is redundant, since most references to figures are within the
% same chapter.
\renewcommand{\thefigure}{\arabic{figure}}

\newcommand{\rlapsub}[1]{\rlap{\sub{#1}}}

% \SetupAffiliations{output in groups = false, 
%                    separator between two = {\bigskip\\},
%                    separator between multiple = {\bigskip\\},
%                    separator between final two = {\bigskip\\}
%                    }


%%%%%%%%Alte Umlaute
\newcommand{\oldae}{$\stackrel{\textrm{\tiny e}}{\textrm{a}}$}
\newcommand{\oldoe}{$\stackrel{\textrm{\tiny e}}{\textrm{o}}$}
\newcommand{\oldue}{$\stackrel{\textrm{\tiny e}}{\textrm{u}}$}

\newcommand{\refl}{\REFL}
\newcommand{\pst}{\PST}


% Müller
\let\vref\ref


\let\citew\citet

\newcommand{\page}{}

% biblatex stuff
% get rid of initials for Carl J. Pollard and Carl Pollard in the main text:
\ExecuteBibliographyOptions{uniquename=false}




\newcommand{\nom}{\textsc{nom}}
\newcommand{\gen}{\textsc{gen}}
\newcommand{\dat}{\textsc{dat}}
\newcommand{\acc}{\textsc{acc}}


%\newcommand{\spacebr}{\hspaceThis{[}}



\newcommand{\acknowledgmentsEN}{Acknowledgements}
\newcommand{\acknowledgmentsUS}{Acknowledgments}


% no bf!!!111!
\let\textbfemph\emph

\newcommand{\textbfremoved}[1]{#1}
%\newcommand{\emphremoved}[1]{#1}


\newcommand{\noemph}[1]{#1}
\newcommand{\underlineemph}[1]{\emph{#1}}




% for editing, remove later
\usepackage{xcolor}
\newcommand{\added}[1]{{\red #1}}
\newcommand{\addedthis}{\todostefan{added this}}

\newcommand{\changed}[1]{\textcolor{orange}{#1}}




% Nolda

\theorembodyfont{\normalfont}
\let\restriction\relax
\renewtheoremstyle{break}{\item{\itshape ##1\ ##2}\newline\nopagebreak}{\item{\itshape ##1\ ##2\ (##3)}\newline\nopagebreak}
\theoremstyle{break}
\newtheorem{definition}{Definition}
\newtheorem{pattern}{Pattern}
\newtheorem{restriction}{Restriction}
\newunicodechar{‑}{\hbox{-}}
\newunicodechar{…}{\dots}
\newunicodechar{⁡}{\relax}
\newunicodechar{⁣}{\relax}
\newunicodechar{⁀}{\raisebox{+1ex}{\ensuremath\frown}}
\newunicodechar{⁐}{\raisebox{+1ex}{\ensuremath\frown}\setbox1=\hbox{\ensuremath\smile}\hspace{-\wd1}\raisebox{-1ex}{\ensuremath\smile}}
\newunicodechar{⪪}{\ensuremath{<\mathrel{\llap{\ensuremath{-}}}}}
\setkomafont{descriptionlabel}{\normalfont}
\ExecuteBibliographyOptions{labeldate=comp,labelnumber=true,defernumbers=true}
\defbibenvironment{sources}{\list{\printfield{labelprefix}\,\printfield{labelnumber}}{\settowidth{\labelwidth}{S\,0}\setlength{\labelsep}{\biblabelsep}\setlength{\leftmargin}{\labelwidth}\addtolength{\leftmargin}{\labelsep}\setlength{\itemsep}{\bibitemsep}\setlength{\parsep}{\bibparsep}}\renewcommand{\makelabel}[1]{##1\hfil}}{\endlist}{\item}
\newcommand{\citesource}[1]{\citefield{#1}{labelprefix}\,\citefield{#1}{labelnumber}}


% Was soll das machen?
\newcommand{\textstyleFootnoteSymbol}{}



% Was ist das???? St. Mü. 30.10.2021
%Kann weg. Damit waren die bücker transkripte aligniert. Habe das jetzt mit tabularx und hphantom gemacht

%\newlength{\calength} %tmp length to store the space 1. until [; 2. until ].
%
%%first argument speaker ID, second argument text. Optional argument left margin indicator (arrow or similar)
%\newcommand{\cabox}[3][]{\parbox{0mm}{\hspace*{-1cm}#1}%
%\parbox{1.5cm}{#2}%
%\parbox{9.6cm}{#3}\\%
%}
%
%%translation. First parbox is empty, second parbox takes the translation text
%\newcommand{\trsbox}[1]{\parbox{1.5cm}{~}%
%\parbox{9.6cm}{\itshape #1}\\%
%}
%
%%store the width of a string.
%\newcommand{\settablength}[1]{\settowidth{\calength}{#1}\global\calength=\calength}
%
%%print string and store its width. Useful if the first item of the aligned set is also the longest
%\newcommand{\inittab}[1]{#1\settablength{#1}}
%
%%insert horizontal white space equivalent to the stored width
%\newcommand{\skiptab}{\parbox{\calength}{~}}
%
%%print the argument and fill up with horizontal white space until the stored width is reached.
%\newcommand{\filledtab}[1]{\parbox{\calength}{#1}}



% for standalone compilations Felix: This is in the class already
%\let\thetitle\@title
%\let\theauthor\@author 
\makeatletter
\newcommand{\togglepaper}[1][0]{ 
\bibliography{../bib-abbr,../stmue,../localbibliography,
collection.bib}
  %% hyphenation points for line breaks
%% Normally, automatic hyphenation in LaTeX is very good
%% If a word is mis-hyphenated, add it to this file
%%
%% add information to TeX file before \begin{document} with:
%% %% hyphenation points for line breaks
%% Normally, automatic hyphenation in LaTeX is very good
%% If a word is mis-hyphenated, add it to this file
%%
%% add information to TeX file before \begin{document} with:
%% \include{localhyphenation}
\hyphenation{
Arsch
anaph-o-ra
Bü-cking
con-stit-u-ents
Dor-drecht
For-schungs-ge-mein-schaft
Ge-schich-te
ha-ben
pho-nol-o-gy
pro-so-dic
pro-so-di-cally
Sal-pe-ter
sei-nen
Wil-liams
}
\hyphenation{
Arsch
anaph-o-ra
Bü-cking
con-stit-u-ents
Dor-drecht
For-schungs-ge-mein-schaft
Ge-schich-te
ha-ben
pho-nol-o-gy
pro-so-dic
pro-so-di-cally
Sal-pe-ter
sei-nen
Wil-liams
}
  % \memoizeset{
  %   memo filename prefix={hpsg-handbook.memo.dir/},
  %   % readonly
  % }
  \papernote{\scriptsize\normalfont
    \@author.
    \titleTemp. 
    To appear in: 
    Ulrike Freywald \& Horst Simon (eds.) Headedness and/or grammatical anarchy?
    Berlin: Language Science Press. [preliminary page numbering]
  }
  \pagenumbering{roman}
  \setcounter{chapter}{#1}
  \addtocounter{chapter}{-1}
}
\makeatother



% This does a linebreak for \gll for long sentences leaving space for the language at the right
% margin. The factor .0989 is needed since otherwise starred examples cause a linebreak.
% St.Mü. 17.06.2021 08.02.2021
\newcommand{\longexampleandlanguage}[2]{%
%\begin{tabularx}{.99\linewidth}[t]{@{}X@{}p{\widthof{(#2)}}@{}}%
%\begin{minipage}[t]{.99\linewidth}%
\begin{tabularx}{\linewidth}[t]{@{}X@{}p{\widthof{(#2)}}@{}}%
\begin{minipage}[t]{\linewidth}%
#1%
\end{minipage} & (\ili{#2})%
\end{tabularx}}

% ORCIDs in langsci-affiliations 
\usepackage{orcidlink}
\definecolor{orcidlogocol}{cmyk}{0,0,0,1}
\ProvideDocumentCommand{\LinkToORCIDinAffiliations}{ +m }
  {%
    \orcidlink{#1}
  }

   %% hyphenation points for line breaks
%% Normally, automatic hyphenation in LaTeX is very good
%% If a word is mis-hyphenated, add it to this file
%%
%% add information to TeX file before \begin{document} with:
%% %% hyphenation points for line breaks
%% Normally, automatic hyphenation in LaTeX is very good
%% If a word is mis-hyphenated, add it to this file
%%
%% add information to TeX file before \begin{document} with:
%% %% hyphenation points for line breaks
%% Normally, automatic hyphenation in LaTeX is very good
%% If a word is mis-hyphenated, add it to this file
%%
%% add information to TeX file before \begin{document} with:
%% \include{localhyphenation}
\hyphenation{
Arsch
anaph-o-ra
Bü-cking
con-stit-u-ents
Dor-drecht
For-schungs-ge-mein-schaft
Ge-schich-te
ha-ben
pho-nol-o-gy
pro-so-dic
pro-so-di-cally
Sal-pe-ter
sei-nen
Wil-liams
}
\hyphenation{
Arsch
anaph-o-ra
Bü-cking
con-stit-u-ents
Dor-drecht
For-schungs-ge-mein-schaft
Ge-schich-te
ha-ben
pho-nol-o-gy
pro-so-dic
pro-so-di-cally
Sal-pe-ter
sei-nen
Wil-liams
}
\hyphenation{
Arsch
anaph-o-ra
Bü-cking
con-stit-u-ents
Dor-drecht
For-schungs-ge-mein-schaft
Ge-schich-te
ha-ben
pho-nol-o-gy
pro-so-dic
pro-so-di-cally
Sal-pe-ter
sei-nen
Wil-liams
}
%   \bibliography{../localbibliography}
    \togglepaper[6]
}{}


\title{Categoryless heads in morphology?} 

\author{Manuela Korth\orcid{0000-0002-3750-6131}\affiliation{Universität Potsdam}}
% \chapterDOI{} %will be filled in at production

% \epigram{}

\abstract{This article presents a symmetrical approach to headedness in German morphology. All affixes are assumed to be heads irrespective of their position and independent of their function. Categorical projection is no longer seen as the central criterion for morphological headedness. The head in an affixed word is rather defined by morphological minimality and selectional restrictions. A consequence is the existence of categoryless heads. It is shown how structure building processes operate and how projection and feature-percolation mechanisms work in such an approach. Several challenging examples for theories of morphological headedness are discussed -- especially inflected forms, prefixed words, and diminutives. The findings are evaluated by inspecting the result of stress assignment processes in affixed words.}

\begin{document}

\maketitle

\section{Introduction to headedness}\label{sec-intro-kor}
Heads are a common concept in the analysis of linguistic structures. The discussion about headedness in modern linguistics started several decades ago in syntax and was based on work by \citet{Bloomfield1933}, who distinguishes endocentric from exocentric syntactic constructions. An endocentric construction contains a head, by which the headed construction can be replaced. An exocentric construction instead is unheaded. Neither of its subparts can stand for the superordinate construction.\footnote{An overview of endocentricity and exocentricity is given by \citet{Hincha1961} and \citet{Barri1975}.}  Exocentric constructions are the anarchists in grammar but have become rare over the years. Nowadays, most syntacticians manage without exocentricity. Syntactic structures can be analyzed as completely endocentric due to the introduction of functional projections and the abandonment of substitution as a central criterion for headedness.

The debate about heads in morphology started later. \citet{Harris1946} already mentioned the role of morpheme classes for endocentric constructions in syntax and thereby gave an idea of headedness in morphology, but the main discussion did not begin until the 1980s. Influential approaches to morphological heads are the ones by \citet{Williams1981}, \citet{Lieber1981, Lieber1992}, \citet{Selkirk1982}, and \citet{DiSciulloWilliams1987}. The notion of head in \ili{German} morphology is subject to the approaches by \citet{Hoehle1982}, \citet{Reis1983}, \citet{Olsen1986, Olsen1990} and many others. Headedness in compounding is still under discussion,\footnote{Exocentric compounds are examined by e.g.\ \citet{Bauer2008} and \citet{ScaliseEtAl2009}. Further challenges to headedness in compounding – next to exocentricity – are studied by \citet{ScaliseFabregas2010}.} but the debate about heads in derived words and inflected forms faded out in the 1990s. It is time to renew the discussion and to bring back the topic into the morphological discourse. There has been progression in all areas of linguistic research. New developments bring along further insights and different solutions to challenging data. 

We will pick up the symmetrical approach of \citet{Lieber1981} and take a closer look at derivational and inflectional affixes in \ili{German}. Our assumption is that all affixes are heads – irrespective of their position and function. Prefixes as well as suffixes and inflectional as well as derivational affixes serve as heads. Categorical determination is often seen as a central criterion for the identification of morphological heads because heads typically bear a category, which they share with the immediately dominating node. We will change the perspective on headedness by setting up a headedness condition which is based on requirement and selection and refers to the projection level of constituents. The consequence is that some morphological heads are categoryless. Percolation problems like in Lieber's approach are avoided by taking selectional restrictions into account. Such a proposal parallels morphological and syntactic analyses, in that it uses a uniform headedness condition for phrasal and word-internal structures. It thereby allows for graduality between syntax and morphology on the one hand and between derivation and inflection on the other hand, so that e.g. diachronic processes can easier be handled. 

The next two sections constitute the base for our analyses. Section~\ref{sec-headmorph} gives an overview over the notion of head in morphology and looks at different conceptions to word-internal headedness. Section~\ref{sec-parallelsyn} afterwards shows parallels to heads in syntax. The following four sections discuss central phenomena, which are challenging for theories of morphological headedness. The head status of inflectional affixes is subject to Section~\ref{sec-inflection}. Section~\ref{sec-dimsuf} examines diminutive suffixes and considers examples from colloquial \ili{German}, in which diminutive forms of categories other than nouns exists. Derivational prefixes are the topic of Section~\ref{sec-verbpre}. Most of them are transparent with respect to categorical specification but the verbal ones seem to determine the category of the complex verb. Section~\ref{sec-inflpre} picks them up and inspects the difficulties in combination with inflection. Section~\ref{sec-prosodic} checks the findings of the previous sections at the interface to phonology and shows that morphological heads behave not differently to syntactic heads with respect to prosodic interface conditions. A short conclusion finishes this study.

\section{Heads in morphology}\label{sec-headmorph}

Concatenative morphological processes combine morphological constituents to more complex morphological constituents, which can be new words or inflected forms. One immediate subpart of every morphologically complex constituent should head the respective structural level to minimize grammatical anarchy. But there is no consensus about the point which part of a complex word the important leadership status should be given to. Some criteria can help to determine heads in morphology. The head of an endocentric compound is often identified by the semantic relation of hyponymy. Most compounds are a hyponym of its head and can be replaced by it in syntax. The right-hand subpart \emph{juice} of the \ili{English} compound \emph{apple juice} in (\ref{ex-hugojuice}) can thereby be diagnosed as head. The drinking of apple juice implicates the drinking of juice, so that the sentence with the compound is subaltern to the sentence which contains the compound's head instead.\footnote{See e.g. \citet[Section~4.3]{Loebner2003} or \citet[Section~5.2]{SchwarzChur2001} for implications and other semantic relations between sentences.}

\eal
\ex\label{ex-hugojuice} Hugo drinks apple juice.
%%1st subexample: change \ea\label{...} to \ea\label{...}\ea; remove \z  
%%further subexamples: change \ea to \ex; remove \z  
%%last subexample: change \z to \z\z 
\ex Hugo drinks juice.
\zl

\noindent The morphological head determines the properties of the complex word. Its features percolate one level up to be shared with the immediately dominating node. The compound in (\ref{ex-haushoch}) gets the category from its head \emph{hoch} `high', moreover, the compound in (\ref{ex-hochhaus}) and its head \emph{Haus} `house' do not only correspond in category but also in gender.

\eal
\ex\label{ex-haushoch}
\noemph{haus-hoch} \hspace{20pt} [N, neut] $+$ A $\rightarrow$ A \\
house-high\\
`extremely high'
\ex\label{ex-hochhaus}
\noemph{Hoch-haus} \hspace{20pt} A $+$ [N, neut] $\rightarrow$ [N, neut]\\
high-house \\
`skyscraper'
\zl

\noindent Affixal heads typically have selectional restrictions, \citep[cf.][77]{Hoehle1982}. The \ili{German} suffix -\emph{bar} productively combines with verbal bases to form adjectives.\footnote{See \citet{Riehemann1998} for a detailed analysis of \emph{bar}-adjectives in \ili{German}.} Nominal and adjectival bases are excluded from derivation with -\emph{bar}. There are only some marginal and marked exceptions. Affixal heads in word formation can also take influence on the valency of their base. The \ili{German} suffix -\emph{bar} and its \ili{English} counterpart -\emph{able} in (\ref{ex-beer}) induce a passive-like transformation.

\eal\label{ex-beer}
\ex Linguists drink beer.
\ex Beer is drinkable. 
\zl

\noindent Different theories have been developed to predict which constituent qualifies as the morphological head of a complex word. \citet{Williams1981} proposes the Right-hand Head Rule (= RHR), which gives the head status to the rightmost subpart of a complex morphological constituent. The morphological head can thereby be the right-hand member of a compound, a derivational or inflectional suffix, or the base which a prefix combines with. But the RHR does not come without exceptions. Some \ili{Germanic} prefixes seem to determine the category of the derived word and should be considered to serve as heads, whereas inflectional suffixes do not take influence to the category of the inflected form, so that their classification as head is rather doubtful. Furthermore, several compounds in \ili{Romance} languages are left-headed \citep[cf. e.g.][Section~5]{Scalise1988}.

\largerpage[2]
The RHR is nevertheless the most common conception regarding headedness in morphology. Some modified versions have been proposed to capture more phenomena. One of them has been suggested by \citet{Selkirk1982}. Her version gives the head status to the right-hand constituent which shares the category with the dominating node. A consequence of such a modification is that inflectional affixes can never be heads in morphology, whereas derivational prefixes can occasionally get head status. Selkirk's approach describes the situation in complex words but does not prognosticate the head of a new combination without further assumptions. A combination of a verbal and a following nominal morphological constituent can potentially lead to a verbal or nominal word-formation product.\footnote{\citet{Olsen1990} modifies Selkirk's version of the RHR, so that it can predict the head of a new word, but it neither allows for prefixal heads nor for left-headed compounds.}   Furthermore, Selkirk's approach inconsistently analyses the verbal prefix \emph{en}- as head in (\ref{ex-ennoble}) but as non-head in (\ref{ex-enclose}).\footnote{This oddity has already been mentioned by \citet[167]{TrommelenZonneveld1986}.}

\eal
\ex\label{ex-ennoble}
ennoble \hspace{48pt} V\textsuperscript{af} $+$ N $\rightarrow$ V
\ex\label{ex-enclose} enclose  \hspace{50pt} V\textsuperscript{af} $+$ V $\rightarrow$ V / X\textsuperscript{af} $+$ V $\rightarrow$ V
\zl

\noindent \citet{Lieber1981} comes up with a different conception regarding the determination of morphological heads. She offers a symmetrical approach, in which affixes qualify as heads – prefixes as well as suffixes, inflectional as well as derivational affixes. Her approach contains a feature percolation mechanism with four conventions, which controls the feature transfer to the dominating node. It allows the non-head to percolate features if the head is not specified for them. Such a mechanism is necessary for the process of inflection because features of several affixes and the stem together constitute the feature complex of the inflected word (cf.\ (\ref{ex-ging})). The feature percolation mechanism is indifferent to the question whether the categorical specification comes from the head or a non-head. A side effect is that heads can be categoryless.

\eanoraggedright
\label{ex-ging}
\noemph{ging} [V, past] $+$ -\noemph{e} [conj]\textsuperscript{af} $+$ -\noemph{st}  [sg, 2ps]\textsuperscript{af} $\rightarrow$ \noemph{gingest}  [V, past, conj, sg, 2ps] \\
`(you) would go'
\z

\largerpage[2]
\noindent 
One decade later, \citet{Lieber1992} revised her symmetrical theory to use a modified version of the RHR instead and assigned the head status in derivational and inflectional morphology to the rightmost constituent which matches in category with the dominating node. We will base our examinations in the following sections on Lieber's symmetrical approach and assume that all affixes are heads.


\section{Parallelism to syntax}\label{sec-parallelsyn}

\citet[413]{Givon1971} wrote the frequently cited sentence ``today's morphology is yesterday's syntax'',  which highlights the connection between syntax and morphology; however, it did not remain uncriticized. Syntactic structures can change to morphological structures during grammaticalization processes, (cf. \cite{HopperTraugott1993} or \cite{Lehmann2015} among many others). Syntax and morphology should therefore have a lot in common. Furthermore, it is still under discussion whether morphology constitutes a separate component of grammar as it is assumed by \citet{Sadock1991, Sadock2012} as well as by \citet{Borer1988} and \citet{Spencer1991}\footnote{\citet{Borer1988} and \citet{Spencer1991} set up a separate morphological component but assume an influence of morphological processes at different stages of the structure building process in syntax and phonology.} or whether it represents at least partially a subcomponent of syntax as in the conceptions of \citet{Jackendoff1997} and \citet{AckemaNeeleman2004}, in which morphology is split up into three portions – one morphological fragment is part of the syntactic component of grammar, the other two fragments belong to the phonological and to the semantic (resp. conceptual) component.\footnote{An early approach with a split-up morphology is the one by \citet{ShibataniKageyama1988}, in which some morphological processes belong to the lexicon, others to syntax and phonology.}

Syntactic structures in \ili{German} can be left-headed as in (\ref{ex-fürreise}) or right-headed as in (\ref{ex-reisewegen}). So, it is to be expected that morphological structures allow for both options, too.

\eal 
\ex\label{ex-fürreise}
\gll für die Reise  \hspace{45pt} P $+$ DP $\rightarrow$ PP\\ for the.\textsc{acc} journey  \\
\glt `for the journey'
\ex\label{ex-reisewegen}
\gll der Reise  wegen\hspace{30pt} DP $+$ P $\rightarrow$ PP\\ the.\textsc{gen}   journey   because.of  \\
\glt `because of the journey'
\zl

%\largerpage
\noindent 
Syntax combines constituents with either the same level of projection as in Figure~\ref{ex-adj-str} or different levels of projection as in Figure~\ref{ex-hc-str}. We use a rather simple syntactic structure here without intermediate projections and specifiers so that we need to deal with only two kinds of structural relationships – adjunction structures and head-complement structures.

\begin{figure}
\begin{subfigure}{.48\textwidth}
\centering	
\begin{forest}
	[XP
		[YP]
		[XP]
	]
\end{forest}
\caption{Adjunction structure}\label{ex-adj-str}
\end{subfigure}
\begin{subfigure}{.48\textwidth}
\centering
\begin{forest}
	[XP
		[YP]
		[X]
	]
\end{forest}
\caption{Head-complement structure}\label{ex-hc-str}
\end{subfigure}
\caption{Syntactic structures}\label{ex-adj-hc-str}
\end{figure}

%\largerpage
%\enlargethispage{1pt}
%\noindent 
This is paralleled in morphology. Members of compounds show the same level of projection, cf.\ Figure~\ref{ex-samepro}, whereas the subconstituents of derivations, cf.\ Figure~\ref{ex-diffpro} – and those of inflected words – have different projection levels.

\begin{figure}
\begin{subfigure}{.48\textwidth}
\centering
\begin{forest}
	[X
		[X]
		[Y]
	]
\end{forest}
\caption{Adjunction structure}\label{ex-samepro}
\end{subfigure}
\begin{subfigure}{.48\textwidth}
\centering
\begin{forest}
	[X
		[Y]
		[X$^{\textnormal{af}}$]
	]
\end{forest}
\caption{Head-complement structure}\label{ex-diffpro}
\end{subfigure}
\caption{Morphological structures}
\end{figure}


We will concentrate our examinations mainly on structures with different projection levels in order to investigate the head status of affixes. Syntax designates the subconstituent with the lower projection level as head. Transmitting this method to morphology, affixes should be heads because their projection level is lower than that of the base. We can set up the headedness condition in (\ref{ex-headcond}) for structures whose subconstituents differ in the level of projection.

\ea\label{ex-headcond}
Headedness condition: \\ 
The head of a complex constituent is the subconstituent with 	the lower projection level.
\z

\noindent The projection level is related to the requirements of a constituent. Heads typically demand for a complement; non-heads are satisfied without a partner. The determiner in (\ref{mess}) needs a partner in syntax; the suffix in (\ref{messy}) looks for a complement in morphology. Both are heads under this perspective. So, we can say for short: The head is the subconstituent which requires a complement.

\eal
\ex\label{mess} This is a *(mess).
\ex\label{messy}
This is *(mess)-y.
\zl

%\largerpage
\noindent 
The headedness condition does not distinguish between derivational and inflectional affixes. It is symmetrical and provides equal rights for prefixes and suffixes. A consequence is the presence of heads without a categorial feature. At a first glance, such categorylessness occurs with inflectional affixes as in (\ref{ex-tage}), possibly with diminutive suffixes as in (\ref{ex-tischchen}) and with derivational prefixes as in (\ref{ex-un-wahr}). The abbreviation MSEL in (\ref{ex-MSEL}) stands for morphological selection. N\textsuperscript{4} in (\ref{ex-tage}) indicates a noun which belongs to the plural inflection class 4 \citep[cf.][29]{IDS97}.

\eal\label{ex-MSEL}
\ex\label{ex-tage}
\noemph{Tag-e}  \hspace{41pt}\noemph{-e} [pl | MSEL: N\textsuperscript{4}]\textsuperscript{af}\\
`days'
\ex\label{ex-tischchen}
\noemph{Tisch-chen} \hspace{15.5pt}\noemph{-chen} [neut | MSEL: N]\textsuperscript{af}\\
`small table'
\ex\label{ex-un-wahr}
\noemph{un-wahr} \hspace{27pt}\noemph{un-} [MSEL: A]\textsuperscript{af}\\
`untrue' 
\zl

\noindent Affixal heads without a categorical specification like the ones in (\ref{ex-MSEL}) allow for feature percolation from the non-head. The selectional restrictions of the respective affix indirectly guarantee that the correct category is transferred from the non-head to the dominating node. Most affixes are specialized for bases of one specific category. Some derivational affixes are less strict and combine with bases of different categories. The adjectival suffix -\emph{lich} is compatible with adjectival, nominal, and verbal bases (cf.\ \cite[260--263]{FleischerBarz1995} and \cite[141--142]{AltmannKemmerling2000}). A certain variability with respect to the category of the base also holds for the prefix \emph{un}- \citep[cf.][]{Schnerrer1982}. Several derivations with nominal bases can be observed in addition to the pattern in (\ref{ex-un-wahr}), although the adjectival \emph{un}- is more productive than the nominal one. The suffix -\emph{lich}, which carries a categorical specification, leads to an adjective independently of the category of the base, whereas the categorically underspecified prefix \emph{un}{}- is transparent for the category of its partner. A combination with an adjective produces an adjective. A combination with a noun results in a noun. An adjusted notation for the selectional restrictions of the prefix \emph{un}{}- is given in (\ref{ex-unwahr}). The representation A/N can be unified to $+$N if a binary system is used to distinguish categories.

\eal
\label{ex-unwahr}
\ex un-wahr \hspace{10pt} un- [MSEL: A/N]\textsuperscript{af}\\`untrue' 
\ex Un-sinn \hspace{11.5pt} un- [MSEL: A/N]\textsuperscript{af}\\`nonsense'
\zl

%\largerpage[2]
\noindent 
We started this section with the diachronic connection of syntactic and morphological structures, and we will close it now with a remark on historical processes. A model which analyzes affixes as heads allows for an easier description of grammaticalization processes. There is no radical change by which function words lose their category, their unboundedness, and their status as head at once. They rather undergo a gradual shift from a word to an affix. Their autonomy decreases more and more, but the headedness remains constant. Beyond this, the same grammatical features can either be expressed by analytic or by synthetic constructions in different languages as well as during different periods of time.\footnote{An overview of historical changes in analytic and synthetic realizations in \ili{German} is given by \citet[Section~11]{NueblingEtAl2010}.} Analytic and synthetic realizations even exist in parallel as can be seen in (\ref{conj}).

\largerpage
\eal\label{conj}
\ex
\gll Er    käm-e. \\ 
     he   come.\textsc{past-conj}  \\
\glt  `He would come.'
\ex
\gll Er    würd-e   komm-en.\\ 
     he   \textsc{aux.past-conj} come-\textsc{inf}  \\
\glt `He would come.'
\zl

\noindent Such a graduality also holds for compound parts which become derivational affixes.\footnote{An intermediate state on the way to a derivational affix is often called affixoid (cf.\ e.g. \cite{Elsen2009} and \cite{Szatmari2011} for affixoids in \ili{German} as well as \cite{Schmidt1987} for a critical view.).} The connection of compounding and derivation is further highlighted in the model of \citet{Hoehle1982}, who treats derivations in parallel to compounds.\footnote{But see \citet{Reis1983} for a critical discussion of Höhle's theory.} Graduality has further been observed on the passage between syntax and compounding\footnote{See e.g. \citet[Section~2]{NueblingSzczepaniak2009} for the origin of linking elements in \ili{German} compounds, which derive from inflectional affixes in former genitive constructions.} as well as between derivation and inflection. \citet{Bybee1985} sets up a relevance hierarchy, which demonstrates regularities in the succession of inflectional feature classes. Feature classes with more semantic relevance are closer to the stem than feature classes with less semantic relevance.\footnote{This has a strong connection to the set of linguistic universals, \citep[cf.][]{Greenberg1963}. Less relevant feature classes occur in fewer languages and presuppose the existence of more relevant feature classes.} \citet{EisenbergSayatz2002} take the way from the other side and look at the order of derivational affixes. Some derivational affixes are close to the root, others – like e.g.\ diminutive suffixes \citep[cf.][]{Dressler1994} – are next to the fuzzy border to inflectional suffixes. Still others – like suffixes for comparative adjectives – have a rather doubtful status with respect to the classification as derivational or inflectional.

%\largerpage[2.5]
The different continua are symbolized in Figure~\ref{ex-continua}. Only a model which does not strictly divide syntax, compounding, derivation, and inflection can handle the graduality among them. Consequently, we treat morphological heads as similar to syntactic heads and inflectional affixes similarly to derivational affixes.

\begin{figure}
\centering
\begin{tikzpicture}
	\draw[<->] (1.05,1) -- (3.95,1);
	\draw[<->] (4,.95) -- (4,-1.95);
	\draw[<->] (3.95,-2) -- (1.05,-2);
	\draw[<->] (1,-1.95) -- (1,.95);
	
	\node at (.3,1.1) {syntax};
	\node at (4.8,1.2) {inflection};
	\node at (4.9,-2.2) {derivation};
	\node at (-0.2,-2.2) {compounding};
\end{tikzpicture}\hspace{1cm}
\caption{(Morpho-)syntactic subcomponents with gradual relationship}\label{ex-continua}
\end{figure}



We now take a closer look at the three groups of potentially categoryless affixes in (\ref{ex-MSEL}) during the next sections. We start with inflectional suffixes in Section~\ref{sec-inflection}.

\section{Inflection}\label{sec-inflection}

\largerpage
Morphological theories are confronted with the peculiarities of feature percolation in inflected words. Some morphologists avoid the challenge by using inflectional paradigms \citep[cf.][]{Stump2001}. Our discussion focuses on a morpheme-based analysis, so that we are faced with questions of feature percolation and headedness in inflected words.

The feature complex of finite verbs in \ili{German} includes the category feature as well as features for tense, mood, number, and person, which percolate from different morphemes. The structure in Figure~\ref{ex-sagtest} represents the analysis for an inflected form of the weak verb \emph{sag} `say'. The category feature originates from the stem. Tense and mood percolate from the suffix closest to the stem. The outer suffix is responsible for number and person.

\begin{figure}
\begin{forest}
	sm edges, empty nodes
	[{[V, past, ind, sg, 2ps]}
		[{[V, past, ind]}
			[V [sag]]
			[{[past, ind]\textsuperscript{af}} [-te]]
		]
		[[{[sg, 2ps]\textsuperscript{af}} [-st, name=st]]]
	]
\end{forest}
\caption{Structure for \emph{sagtest} `(you\textsubscript{\textsc{sg}}) said'}\label{ex-sagtest}
\end{figure}

\largerpage
All three morphemes provide important components for the entire word. That is traditionally the mission of the head, but only one morphological constituent can qualify as head at any binary branching point. Following the reflections in the last two sections, the outer suffix -\emph{st} functions as head of the whole inflected form, whereas the inner suffix -\emph{te} can be seen as the head of the left immediate subconstituent. All features percolate to the highest node in Figure~\ref{ex-sagtest}. None of them is blocked, because the heads are not prespecified for features which are of the same kind as the features of the non-heads. The inner suffix -\emph{te} does not have a category feature and allows for the percolation of the category from the stem. The outer suffix -\emph{st} is not only underspecified for a category but also lacks features for tense and mood. It takes over the respective information from its partner.

\citet{Lieber1992} mentions a problem for her original model from 1981, in which she works with categoryless affixal heads. The feature percolation mechanism therein seems not to be restricted enough to exclude some non-existing feature combinations. It would potentially be possible to percolate the gender feature of the nominal base \emph{Bild} `image' in Figure~\ref{ex-bildlich} to the higher adjectival node.

\begin{figure}
\begin{forest}
	[$^{\ast}${[A, neut]}
		[{[N, neut]} [bild]]
		[A\textsuperscript{af} [-lich, name=lich]]
	]
\end{forest}
\caption{Structure for \emph{bildlich} `figurative' (inadmissible percolation)}\label{ex-bildlich}
\end{figure}


Lieber therefore changes her original theory radically. She now analyzes only category-determining affixes as heads and introduces a categorical signature to capture inflection. Nouns, verbs, and adjectives have underspecified features which can be valued through the process of inflection. This is shown for the noun \emph{Tag} `day' in Figure~\ref{ex-tagen}. Percolation from the non-head only serves to fill the unvalued features in the categorical signature of the head. No independent feature values of the non-head are permitted to percolate.\footnote{The notation in Figure~\ref{ex-tagen} differs from the notation of \citet{Lieber1992} in that she uses a binary system to represent features.}

\begin{figure}
\begin{forest}
	sm edges
	[N\\ \hline \textsc{gen}: masc\\\textsc{num}: pl\\\textsc{case}: dat, align=left, draw
		[N\\ \hline \textsc{gen}: masc\\\textsc{num}: pl\\\textsc{case}: $\beta$, align=left, draw
			[N\\ \hline \textsc{gen}: masc\\\textsc{num}: $\alpha$\\\textsc{case}: $\beta$, align=left, draw [Tag]]
			[{[\textsc{num}: pl]\textsuperscript{af}} [-e]]
		]
		[{[\textsc{case}: dat]\textsuperscript{af}} [-n, name=n]]
	]
\end{forest}
\caption{Structure with categorical signature for \emph{Tagen} `days'}\label{ex-tagen}
\end{figure}

\largerpage
But the categorical signature comes with new problems. Inflectional features are irrelevant in word formation processes like derivation and composition, so that the underspecified features in the categorical signature remain unvalued. Such unvalued features should crash the structure-building process. An alternative could be to value them by default, but such a mechanism seems to be quite odd inside compounds and derived words. It would furthermore be necessary to have two different categorical signatures for verbs in \ili{German}. \ili{German} verbs can either occur as finite forms with the feature complex in Figure~\ref{ex-said} or as non-finite forms with a status feature in Figure~\ref{ex-say} \citep[cf.][]{Bech1983}.

\begin{figure}
\begin{subfigure}[b]{.48\textwidth}
\centering
\begin{forest}
		[V\\\hline\textsc{tense}: $\alpha$\\\textsc{mood}: $\beta$\\\textsc{num}: $\gamma$\\\textsc{pers}: $\delta$, align=left, draw]
\end{forest}
\caption{Feature complex for finite forms}\label{ex-said}
\end{subfigure}
\begin{subfigure}[b]{.48\textwidth}
\centering
\begin{forest}
		[V\\\hline\textsc{status}: $\alpha$, align=left, draw]
\end{forest}
\caption{Feature for non-finite forms}\label{ex-say}
\end{subfigure}
\caption{Two categorical signatures for verbs}
\end{figure}


Categorical signatures have been invented to solve problems of percolation, but they provide no better results for examples like the one in Figure~\ref{ex-bildlich} than models which do without a categorical signature. The value for the gender feature cannot be prevented from percolating in Figure~\ref{ex-bildlich} resp. Figure~\ref{ex-19kor}, because adjectives have an underspecified gender feature in their categorical signature.

\begin{figure}
\begin{forest}
	[A\\\hline\textsc{gen}: neut\\\textsc{num}: $\beta$\\\textsc{case}: $\gamma$, align=left, draw,name=AStar
		[N\\\hline\textsc{gen}: neut\\\textsc{num}: $\alpha$\\\textsc{case}: $\beta$, align=left, draw [bild]]
		[A\textsuperscript{af}\\\hline\textsc{gen}: $\alpha$\\\textsc{num}: $\beta$\\\textsc{case}: $\gamma$, align=left, draw [-lich, name=lich2]]
	]{\draw[black](AStar.west)+(83:2.35em)node[anchor=east,align=center]{$^\ast $};}
\end{forest}
\caption{Structure with categorical signatures for \emph{bildlich} `figurative' (inadmissible percolation)}\label{ex-19kor}
\end{figure}

The underspecified features of adjectives are normally valued by agreement with\-in DP. Only adjectives in attributive function as in (\ref{ex-darstellung}) show inflection in \ili{German}. Adjectives in predicative use as in (\ref{ex-darstellung-b}) occur uninflected, although agreement was historically possible \citep[cf.][107]{Szczepaniak2009}.

\eal\label{20}
\ex\label{ex-darstellung}
\gll die bildlich-e   Darstellung   überzeugte   ihn \\ the.\textsc{fem.sg.nom}   figurative-\textsc{fem.sg.nom}  representation.\textsc{fem.sg.nom} convinced him \\
\glt `The figurative representation convinced him.'
\ex\label{ex-darstellung-b}
\gll Die Darstellung war   bildlich.\\ 
     the.\textsc{fem.sg.nom}    representation.\textsc{fem.sg.nom}  was   figurative  \\
\glt `The representation was figurative.'
\zl

\noindent The lack of inflection comes unexpected, if adjectives and other lexical categories are stored with a categorical signature in the mental lexicon. Underspecified features should be valued in context. It seems to be necessary therefore to add the categorical signature later and assign it only in contexts which force inflectional marking. Such a method would require phonologically and semantically empty morphological elements which carry the respective categorical signature and open a lexical category for a specific context. But morphological elements without a phonological or semantic counterpart are not desirable. So, we will do without the doubtful categorical signature here and try to explain the percolation independently. 

Spurious connections with an arbitrary percolation like the one in Figure~\ref{ex-altst} are excluded by the selectional restrictions of inflectional affixes. The suffix -\emph{st}, which is responsible for number and person, looks for a verb. It can therefore not be bound to an adjectival or nominal stem.

\begin{figure}
\centering
\begin{forest}
	[\textsuperscript{$\ast$}{[A, sg, 2ps]}
		[A [alt]]
		[{[sg, 2ps]}\textsuperscript{af} [-st]]
	]
\end{forest}
\caption{Inadmissible connection of -\emph{st} to \emph{alt} `old'}\label{ex-altst}
\end{figure}


But the suffix -\emph{st} is not satisfied with any verbal base. It is specialized for bases with features for tense and mood, whereas the suffix -\emph{te}, which introduces these features, looks for a pure verb without inflectional marking. The selectional restrictions of both suffixes are given in (\ref{ex-22kor}). They ensure that the suffixes occur in the correct order.\footnote{The order of affixes is displayed in the relevance hierarchy of \citet{Bybee1985}. See also \citet[205]{Eisenberg2006} for a hierarchy of verbal feature classes in \ili{German}.}

\eal\label{ex-22kor}
\ex \noemph{-st} [sg, 2ps | MSEL: V – $\alpha$TENSE, $\beta$MOOD]\textsuperscript{af}
\ex \noemph{-te} [past, ind | MSEL: V]\textsuperscript{af}
\zl

\noindent The percolation is restricted to categories and free features. Features which are already connected to a category only percolate with the category together. The features for tense and mood in Figure~\ref{ex-sagtest23} are free, i.e.\ they are not bound to a category. They percolate from the suffixal head to the immediately dominating node. Their categorical underspecification allows for a percolation of the verbal category from the non-head. The formerly free features for tense and mood connect to the category. This connection is symbolized by the dash.

\begin{figure}
\centering
\begin{forest}
	sm edges, empty nodes
	[{[V -- past, ind, sg, 2ps]}
		[{[V -- past, ind]}
			[V [sag]]
			[{[past, ind]}\textsuperscript{af} [-te]]
		]
		[[{[sg, 2ps]}\textsuperscript{af} [-st, name=st2]]]
	]
\end{forest}
\caption{Modified structure for \emph{sagtest} `(you\textsubscript{\textsc{sg}}) said'}\label{ex-sagtest23}
\end{figure}


The suffix -\emph{st} is added at the next higher level in Figure~\ref{ex-sagtest23}. The features for number and person are free and percolate up. The category is taken from the non-head again. But the category does not come alone this time. Features for tense and mood are already bound to it and must percolate with the category together. The formerly free features for person and number connect to the category. The verb is fully specified now.

The problematic percolation in Figure~\ref{ex-bildlich} can be excluded with such a mechanism. The gender feature is connected to the noun and cannot percolate independently. The dominating adjectival node in Figure~\ref{ex-figurative} remains correctly without a gender feature.

\begin{figure}
\centering
\begin{forest}
	[A
		[{[N -- neut]} [bild]]
		[A\textsuperscript{af} [-lich, name=lich3]]
	]
\end{forest}
\caption{Structure for \emph{bildlich} `figurative'}\label{ex-figurative}
\end{figure}


But do inflectional affixes really occur without a categorical specification? There is another option to handle the data. We know from modern generative models that inflectional features are assigned (or at least checked) by special functional heads like T or Agr. Transferring this to morphology would mean that inflectional suffixes have a category. Suffixes with a tense feature are of category T, whereas suffixes which mark a verb for person and number can be given the category Agr. This simplifies the selectional restrictions (cf.\ (\ref{MSEL2})). The suffix -\emph{st} looks now for a partner with the category T and not for a verb with a prespecification for tense and mood. The result at this single step of the structure-building process is nearly the same but we do not need to deal with categorylessness.

\eal\label{MSEL2}
\ex \noemph{-st} [Agr – sg, 2ps | MSEL: T]\textsuperscript{af}
\ex \noemph{-te} [T – past, ind | MSEL: V]\textsuperscript{af}
\zl

\noindent A problem seems to be that the features which are bound to T must disconnect from T to percolate to higher nodes. This is excluded by the percolation mechanism. But any further percolation of tense and mood is not necessary. The syntax only needs number and person for agreement in \ili{German}. Tense does not matter, as can be seen with many examples from novels, in which past tense verbs are often combined with non-past adverbials. The modified structure for Figure~\ref{ex-sagtest23} is given in Figure~\ref{ex-sagtest26}.

\begin{figure}
\centering
\begin{forest}
	sm edges, empty nodes
	[{[Agr -- sg, 2ps]}
		[{[T -- past, ind]}
			[V [sag]]
			[{[T -- past, ind]}\textsuperscript{af} [-te]]
		]
		[[{[Agr -- sg, 2ps]}\textsuperscript{af} [-st, name=st3]]]
	]
\end{forest}
\caption{Alternative structure for \emph{sagtest} `(you\textsubscript{\textsc{sg}}) said'}\label{ex-sagtest26}
\end{figure}


Thus, whether inflectional suffixes have or lack a category depends on the complexity of syntactic structures in the respective conception of grammar. A model which allows for the functional projections T and Agr in syntax can handle the \ili{German} data by using the same categories in morphology. A simpler syntax which manages without them can deal with categorylessness. We will stick to the categoryless variant in the analyses below and will briefly come back to a challenge with the alternative variant in Section~\ref{sec-inflpre}.

\section{Diminutive suffixes}\label{sec-dimsuf}

Diminutive suffixes show a special behavior with respect to headedness in several languages. \ili{German} has the diminutive suffixes -\emph{chen} and -\emph{lein} next to some regional variants, which are related to the two standard forms.\footnote{Diminution can furthermore be expressed by i-formation, which often occurs with truncation of the base. The ending <i> has not yet reached the full status of a suffix \citep[cf.][]{Koepcke2002}.} We will concentrate our discussion on the suffix -\emph{chen} here, but -\emph{lein} and regional variants behave similarly. \ili{German} diminutive suffixes standardly build nouns out of nouns. They do not affect the categorical specification but determine the gender of the resulting noun. All diminutive nouns have neuter gender irrespective of the gender of their base (cf. (\ref{ex-tischchen2})).

\eal\label{ex-tischchen2}
\ex \noemph{Tisch} [N – masc]\\`table'
\ex \noemph{Tischchen} [N – neut]\\`small table'
\zl

\noindent Diminutive suffixes seem to be categoryless heads, but examples like (\ref{ex-dumm}) and (\ref{ex-früh}) point in another direction. It looks as if the suffix is category-changing here and transfers an adjective to a noun with neuter gender \citep[cf.\ also][85]{Hoehle1982}.

\eal\label{ex-dumm}
\ex \noemph{dumm} [A]\\`stupid'
\ex \noemph{Dummchen} [N – neut]\\`stupid person'
\zl
\eal\label{ex-früh}
\ex \noemph{früh} [A]\\`early'
\ex \noemph{Frühchen} [N – neut]\\`preemie'
\zl

\noindent We should be cautious with such diminutives because we cannot exclude that there is an intermediate step in the derivation. It is equally possible, that the adjective first becomes a noun (cf.\ (\ref{ex-dumme})), before it combines with the suffix. However, \citet{Wiese2006} points out that diminution with -\emph{chen} gets along without such an intermediate step synchronically.

\ea\label{ex-dumme}
\noemph{der/die Dumme} [N – masc/fem]\\`(the) stupid person'
\z

\noindent Other examples support the hypothesis of categorylessness. Diminutive suffixes can be bound to adjectives as in (\ref{adj-chen}) and interjections as in (\ref{attr-chen}) without affecting the category or other features. Examples like those in (\ref{attr-chen}) are used quite often in colloquial \ili{German}, whereas a combination with adjectives is only rarely found. Adjectives with diminutive suffixes almost ever occur in predicative use, where no inflection is required. But a few attributive cases, in which the inflection follows a truncated diminutive suffix, are also attested, cf. (\ref{trunc-dim}).

\ea\label{adj-chen}
\begin{tabular}[t]{@{}l@{~}lll@{}}
	a.& müde [A]          && müdchen [A]\\
	  & `tired'           && `(a little) tired'\\
	b.& \noemph{gut} [A]  && \noemph{gutchen} [A]\\
	  & `good'            && `(a little) good'\\
	c.& \noemph{spät} [A] && \noemph{spätchen} [A]\\
	  & `late'            && `(a little) late'\\
\end{tabular}
\z

\ea\label{attr-chen}
\begin{tabular}[t]{@{}l@{~}lll@{}}
		a.&\noemph{tschüss}&&\noemph{tschüsschen}\\
		&`bye'&&`bye'\\
		b.&\noemph{hallo}&&\noemph{hallöchen}\\
		&`hello'&&`hello'\\
		c.&\noemph{okay}&&\noemph{okaychen}\\
		&`okay'&&`okay'\\
\end{tabular}
\z

\eal\label{trunc-dim}
\ex
\gll ein   müd-ch-es Lächeln\footnotemark \\
      a     tired-\textsc{dim-neut.sg.nom}  smile\\
\glt `a tired smile'\footnotetext{
\url{https://www.gamestar.de/xenforo/threads/offizieller-mechwarrior-mechcommander-thread.58151/page-20} (2020-08-23).}

\ex
\gll ein  klein-ch-es fein-ch-es Jung-chen\footnotemark \\
     a     small-\textsc{dim-neut.sg.acc}   fine-\textsc{dim-neut.sg.acc} boy-\textsc{dim.neut.sg.acc}\\  
\glt  `a nicens little boy' (James Joyce)
\footnotetext{
\url{http://www.meine-lieblingsbuecher.de/AnfangeEinleitung/AnfangeJ/body_anfangej.html} (2020-08-23).}
\zl

\noindent The structural analysis of diminutives faces some problems. A completely transparent suffix can handle the data in (\ref{adj-chen}) and (\ref{attr-chen}) but is not fully compatible with prototypical examples. The neuter gender of nominal diminutives would come out of nowhere. The suffix must therefore be responsible for the gender. A first approximation to a structural analysis is given in Figure~\ref{ex-tischdim}. The gender feature of the suffix is free and percolates to the dominating node. The category is taken from the base. Such a percolation conflicts with our assumptions in the previous section because the gender feature of the base is bound to the category and cannot be left behind. It has to percolate with the category together, but this would lead to double gender marking, which is not possible in \ili{German}.

\begin{figure}
\centering
\begin{forest}
	[\textsuperscript{$\ast$}{[N -- neut]}
		[{[N -- masc]} [Tisch]]
		[{[neut]\textsuperscript{af}} [-chen, name=chen]]
	]
\end{forest}
\caption{Structure for \emph{Tischchen} `small table' (inadmissible percolation)}\label{ex-tischdim}
\end{figure}

Furthermore, a categoryless diminutive suffix with a gender feature is incompatible with adjectival and interjectional examples. The free gender feature should percolate to the top, but \emph{müdchen} in (\ref{adj-chen}) and \emph{tschüsschen} in (\ref{attr-chen}) as well as similar examples do not bear a gender feature. So, there are in fact two diminutive suffixes, (cf.\ (\ref{ex-MSELchen})).\footnote{\citet{Wiese2006} assumes a family resemblance in the sense of \citet{Wittgenstein1953} for the two variants.}

\eal\label{ex-MSELchen}
\ex\label{ex-MSELchen1}
\noemph{-chen}\textsubscript{1} {[N – neut {\textbar} MSEL: N/X]}\textsuperscript{af}
\ex\label{ex-MSELchen2}
\noemph{-chen}\textsubscript{2} [MSEL: X]\textsuperscript{af}
\zl

\noindent The suffix in (\ref{ex-MSELchen1}) is the original diminutive suffix which creates neuter nouns. It has a category to which the gender feature is bound, so that no illegitimate percolation must be assumed. It is not obvious that the suffix has a category, because it normally demands for nouns, so that no categorical change can be observed. But examples like (\ref{ex-dumm}), (\ref{ex-früh}), and (\ref{ex-neinchen}) point in the direction of a potentially category changing nature of the nominal diminutive suffix.

\eal\label{ex-neinchen}
\ex \noemph{nein}\\ 
    `no' 
\ex \noemph{Neinchen} [N – neut]\\
    `child who habitually says \emph{no}'
\zl

\noindent This is further motivated by the inflectional behavior of diminutive nouns. All diminutive nouns standardly belong to one and the same inflection class C1 independently of the inflection class of the base noun.\footnote{See again \citet[29]{IDS97} for inflection classes.} Diminutives are marked by -\emph{s} in their genitive singular form and occur unmarked in their plural form. The contrast is shown for the noun \emph{Bär} `bear' in (\ref{bär}) and its diminutive \emph{Bärchen} `little bear' in (\ref{bär-b}).\footnote{Colloquial \ili{German} also allows for \emph{s}-plurals. There are furthermore some marginal exceptions with internal inflection. Bases with \emph{er}-plural like \emph{Kind}/\emph{Kinder} `child/children' can keep their plural form in the combination with a diminutive suffix. The plural of \emph{Kindchen} is either realized as \emph{Kindchen} or as \emph{Kinderchen} in Standard \ili{German}.} The inflection class should be bound to a category. A change of the inflection class is thus not expectable with a categoryless diminutive suffix.

\eal\label{bär}
\ex \noemph{(des) Bär-en} [N – masc, sg, gen{}]\\`(of the) bear' 
\ex \noemph{(die) Bär-en} [N – neut, pl{}]\\ `(the) bears'
\zl

\eal\label{bär-b} 
\ex \noemph{(des) Bär-chen-s} [N – neut, sg, gen]\\ `(of the) little bear'
\ex \noemph{(die) Bär-chen-ø} [N – neut, pl]\\ `(the) little bears'
\zl

\noindent The suffix in (\ref{ex-MSELchen2}) is derived from the original variant. It has lost its morphological features and has become non-selective with respect to the category of the base.\footnote{Such a neutral diminutive suffix with respect to category also exists in \ili{Romance} languages like \ili{Italian} \citep[cf.\ e.g.][]{Scalise1988}.}  Furthermore, it is semantically bleached.\footnote{Semantic bleaching is typical for diachronic processes. It accompanies grammaticalization processes \citep[cf.][Section~3.2]{Szczepaniak2009} as well as the transition of free lexical morphemes to derivational affixes \citep[cf.][Section~2]{NueblingEtAl2010}.} It does not add any meaning related to smallness to the base meaning. It has a more social function and makes an utterance friendlier. The salutation \emph{hallöchen} `hello' is not a small \emph{hallo}, which could be interpreted as impolite, but a friendly \emph{hallo} in the communication with family and friends. 

It would potentially be possible to combine the categoryless variant of the diminutive suffix to nouns. That would lead to a preservation of the gender of the base, so that words like \emph{Bärchen} `little bear' would have masculine instead of neuter gender. Such an expansion of the categoryless variant has not yet been attested in \ili{German}, but see \citet{Edelhoff2017} for \ili{Luxembourgish}. The original variant of the diminutive suffix is more specific. It outranks the more general categoryless variant in contexts where both variants could apply.\footnote{This parallels the assumption that specific rules have priority over general ones, which is known from the Elsewhere Condition by \citet{Kiparsky1973}.}

\section{Verbal prefixes}\label{sec-verbpre}

Prefixes normally do not influence the morphosyntactic properties of the derived word. Some of them do not even care about the category of the base. The prefix \emph{miss}{}- is compatible with adjectives, nouns, and verbs. The category of the base corresponds to the category of the prefixed word. Prefixes like \emph{miss}{}- in (\ref{ex-miss}) and \emph{un}- in (\ref{ex-unwahr}) before can be analyzed as categoryless.

\eal\label{ex-miss}
\ex
\gll miss-launig  [A] \\ dis-joyful \\
\glt `bad-tempered'
%\noemph{miss-launig} [A]\\`bad-tempered'
\ex
\gll Miss-ernte [N] \\ dis-harvest \\
\glt `crop failure'
%\noemph{Miss-ernte} [N]\\ `crop failure'
\ex
\gll miss-acht [V] \\ dis-respect \\
\glt `to disregard'
%\noemph{miss-acht} [V]\\ `disregard'
\zl

\noindent 
Verbal prefixes are special. They combine with bases of different categories, but the prefixation always results in a verb. This is shown with the prefix \emph{ent}- in (\ref{ex-ent-}), (\ref{ex-ent-stone}), and (\ref{ex-ent-say}).

\eal\label{ex-ent-}
\ex \noemph{fern} [A]\\ `distant'
\ex \noemph{entfern} [V]\\ `to remove'
\zl

\eal\label{ex-ent-stone}
\ex \noemph{Stein} [N]\\ `stone'
\ex \noemph{entstein} [V]\\ `to remove stones (out of fruits)'
\zl

\eal\label{ex-ent-say}
\ex \noemph{sag} [V]\\ `to say'
\ex \noemph{entsag} [V]\\ `to renounce'
\zl

\noindent Verbal prefixes also determine the thematic structure of the derived word \citep[cf.\ e.g.][]{Wunderlich1987}. They are able to add new arguments. The direct object in (\ref{studentsleepsin}) is introduced by the prefix and cannot occur with the base verb in (\ref{studentsleeps}), which only demands for a subject.

\eal
\ex\label{studentsleeps}
\gll Der               Student   schläft.    \\ 
     the.\textsc{nom}  student   sleeps   \\
\glt `The student sleeps.'
\ex\label{studentsleepsin}
\gll Der             Student   ver-schläft   die           Vorlesung. \\ 
     the.\textsc{nom}   student   \textsc{pref}-sleeps   the.\textsc{acc}   lecture \\
\glt `The student misses the lecture by oversleeping.'
\zl

\noindent Prefixes also have the power to suppress arguments of the base. The verb in (\ref{ex-house}) occurs with a subject and a direct object. The prefix verb in (\ref{ex-view}) replaces the direct object with a new one. The original object is reintroduced as PP. This property of prefixes is parallel to the behavior of verbal particles (cf.\ (\ref{ex-noview})).\footnote{See \citet{Hoekstra1988} for similar examples with \ili{Dutch} particle verbs.}

\eal
\ex\label{ex-house}
\gll Er baut die Häuser. \\
	he.\textsc{nom} builds the.\textsc{acc} houses\\
\glt	`He builds the houses.'

\ex\label{ex-view}
\gll Er ver-baut die schöne Aussicht (mit den Häusern). \\
	he.\textsc{nom} \textsc{pref}-builds the.\textsc{acc} beautiful view with   the.\textsc{dat}   houses \\
\glt `He obstructs the beautiful view (by building houses).'

\ex\label{ex-noview}
\gll Er              baut   die            schöne      Aussicht    (mit    den           Häusern)   zu.\\
     he.\textsc{nom}   builds   the.\textsc{acc}   beautiful   view           with the.\textsc{dat} houses \textsc{ptcl} \\
\glt	`He obstructs the beautiful view (by building houses).'
\zl

%\largerpage
\noindent 
Some prefix verbs show a phonological peculiarity. Their adjectival and nominal bases occur with
umlauts as in (\ref{ex-umlaut}) and (\ref{ex-kraft}). The umlaut cannot be caused by the
prefix. Umlauting is more typically an effect of suffixes (cf.\ \citealt[101]{Eschenlohr1999} and references therein).

\begin{exe}
\ex
  \begin{xlist}
\label{ex-umlaut}
\ex\label{jüng} \noemph{jung} [A]\\`young'
\ex \noemph{verjüng} [V]\\`to rejuvenate'
\zl
\eal 
% \end{xlist}
%\ex
%  \begin{xlist}
\label{ex-kraft}
\ex\noemph{Kraft} [N -- fem]\\`power'
\ex \noemph{entkräft} [V]\\`to weaken'
%\zl
\end{xlist}
\end{exe}

\noindent We can also observe doublets, where the same prefix and the same base result either in a derived word with umlaut as in (\ref{ex-kält}) or in a derived word without umlaut as in (\ref{ex-cold}). Umlauts furthermore appear in verbalizations without prefixes, cf. (\ref{umlaut3}) and (\ref{lüft}).

%\eal
\begin{exe}
\ex
  \begin{xlist}
\label{ex-kält}
\ex \noemph{kalt} {}[A{}]\\`cold'
\ex \noemph{erkält} {}[V{}]\\`to catch a cold'
\zl
\eal
% \end{xlist}
%\ex
%  \begin{xlist}
\label{ex-cold}
\ex \noemph{kalt} {}[A{}]\\`cold'
\ex \noemph{erkalt} {}[V{}]\\`to cool down' 
% \end{xlist}
%\ex
%  \begin{xlist}
\zl
\eal
\label{umlaut3}
\ex\label{schwärz}\noemph{schwarz} {}[A{}]\\`black'
\ex \noemph{schwärz} {}[V{}]\\`to blacken'
% \end{xlist}
%\ex
%  \begin{xlist}
\zl
\eal
\label{lüft}
\ex\noemph{Luft} {}[N -- fem{}]\\`air'
\ex\noemph{lüft} {}[V{}]\\`to ventilate'
%\zl
\end{xlist}
\end{exe}


%\largerpage
\noindent 
The umlaut in these examples is introduced by an independent verbalizing suffix which corresponds to the feature [+front] in phonology. The suffix cares for the category, determines the umlaut, and influences the valency of the base. The umlauting suffix goes back to the \ili{Germanic} stem-building suffix -\emph{ian} and is typically accompanied by transitivization, in which a causer is added to the argument structure of the base, cf. (\ref{ex-papierschwarz}) vs. (\ref{ex-schwarzespapier}). Transitivization through umlauting also appears in verbal doublets like (\ref{ex-baum}).\footnote{See \textcites[90--93]{Sonderegger1997}[Section~5.3.7]{Sonderegger2003} and \citew[217--220]{Schmidt2004} for -\emph{ian} and further \ili{Germanic} stem-building suffixes.} Not all examples are as clear as the ones in (\ref{ex-papier}) and (\ref{ex-baum}) because of semantic changes of lexicalized verbs on the one hand and analogical processes on the other hand.

\eal\label{ex-papier}
\ex\label{ex-papierschwarz}
\gll Das            Papier   ist   schwarz. \\
	the.\textsc{nom}   paper   is     black \\
\glt `The paper is black.'

\ex\label{ex-schwarzespapier}
\gll Der             Junge   schwärzt   das          Papier. \\
	the.\textsc{nom}   boy       blackens     the.\textsc{acc}   paper \\
\glt	`The boy blackens the paper.'
\zl

\eal\label{ex-baum}
\ex
\gll Die             Bäume   fallen. \\
	the.\textsc{nom}   trees      fall \\
\glt	`The trees fall.'

\ex
\gll Die             Männer   fällen   die           Bäume. \\
	the.\textsc{nom}   men        fell        the.\textsc{acc}   trees\\
\glt	`The men fell the trees.'
\zl

\noindent 
The structure of prefixed verbs with umlaut can now be represented as in Figure~\ref{ex-verjüng}.

\begin{figure}
\centering
\begin{forest}
	sm edges, empty nodes
	[V
		[[X\textsuperscript{af} [ver-]]]
		[V
			[A [jung]]
			[V\textsuperscript{af} [{[$+$front]}, name=front]]
		]
	]
\end{forest}
\caption{Structure for \emph{verjüng} `to rejuvenate'}\label{ex-verjüng}
\end{figure}

%\largerpage[-1]
There are still a lot of prefix verbs without umlaut. We have two options for their formation. We can represent their structure either as in Figure~\ref{ex-erdolch-a} or as in Figure~\ref{ex-erdolch-b}. The verbal prefix in Figure~\ref{ex-erdolch-a} directly combines with the noun and determines the category of the derived word. The alternative structure in Figure~\ref{ex-erdolch-b} parallels the representation in Figure~\ref{ex-verjüng} in that it contains an additional suffix which is responsible for the category. In contrast to Figure~\ref{ex-verjüng}, the verbal suffix in Figure~\ref{ex-erdolch-b} has no phonological effect. The prefix does not need a category feature anymore. It is a categoryless head. Its head status can be based on the property that it causes semantic changes and influences the argument structure of the base verb.\footnote{Semantic changes are shown by \citet{Stiebels1996}. Changes in argument structure are represented by \citet{Wunderlich1987}, who, for example, assumes a passive-like transformation for verbs with the prefix \emph{be}-.}

\begin{figure}
\begin{subfigure}[b]{.48\textwidth}
\centering
\scalebox{.92}{
\begin{forest}
	[V
		[V\textsuperscript{af} [er-]]
		[{[N -- masc]} [dolch, name=dolch]]
	]
\end{forest} }
\caption{Prefix projects category}\label{ex-erdolch-a}
\end{subfigure}
\begin{subfigure}[b]{.48\textwidth}
\centering
\scalebox{.92}{
\begin{forest}
	sm edges, empty nodes
	[V
		[[X\textsuperscript{af} [er-]]]
		[V
			[{[N -- masc]} [dolch]]
			[V\textsuperscript{af} [-$\emptyset$, name=empty]]
		]
	]
\end{forest} }
\caption{Suffix projects category}\label{ex-erdolch-b}
\end{subfigure}
\caption{Two alternative structures for \emph{erdolch} `to stab'}
\end{figure}

The structure in Figure~\ref{ex-erdolch-a} has the advantage that we do not need a phonologically empty suffix, but the structure in Figure~\ref{ex-erdolch-b} gives more uniformity to the analysis of prefix verbs. A phonologically empty verbalizing suffix – as it is assumed in Figure~\ref{ex-erdolch-b} – is independently needed for examples like (\ref{ex-beach}) and (\ref{ex-healthy}), in which nouns and adjectives become verbs by conversion or zero-derivation.\footnote{Different views on conversion and zero-derivation in general are collected in \citet{BauerValera2005}. A further alternative would be the categorial underspecification of the base, cf. \citet{Motsch1965} and \citet{BergenholtzMugdan1979} for early approaches as well as \citet{HarleyNoyer1999}, among others, for assumptions in the framework of Distributed Morphology.}

\eal\label{ex-beach}
\ex\noemph{Strand} [N – masc]\\`beach'
\ex\noemph{strand} [V]\\`to strand'
\zl

\eal\label{ex-healthy}
\ex\noemph{gesund} [A]\\`healthy'
\ex\noemph{gesund} [V]\\`to recover'
\zl

%\largerpage[-1]
\noindent 
The phonologically empty suffix for Figure~\ref{ex-erdolch-b} and the examples (\ref{ex-beach}) and (\ref{ex-healthy}) also derives from \ili{Germanic} stem-building suffixes. The suffixes -\emph{ēn} and -\emph{ōn}, which existed besides -\emph{ian}, did not trigger umlauting.\footnote{See again \textcites[90--93]{Sonderegger1997}[Section~5.3.7]{Sonderegger2003} and \citet*[217--220]{Schmidt2004} for \ili{Germanic} stem-building suffixes.} They are conflated synchronically to a suffix which stands for those semantic representations that are not covered by the umlauting verbal suffix. So, we decide for the structure in Figure~\ref{ex-erdolch-b} here. An argument against this analysis could be that several bases of prefixed verbs do seemingly not occur independently as verbs, so that the intermediate step in the derivation is missing.

\vbox{
\ea
\begin{tabular}[t]{lllll}
	a.&\noemph{Glas} [N – neut]~~&&$\rightarrow$&$\surd$\\
	&`glass'&&&\\
	b.&\noemph{glas} [V]&&$\rightarrow$&?\\
	c.&\noemph{verglas} [V]&&$\rightarrow$&$\surd$\\
	&`to glaze'&&&\\
\end{tabular}
\z

\ea
\begin{tabular}[t]{lllll}
	a.&\noemph{Dolch} [N -- masc]&&$\rightarrow$&$\surd$\\
	&`dagger' &&&\\
	b.&\noemph{dolch} [V]&&$\rightarrow$&?\\
	c.&\noemph{erdolch} [V]&&$\rightarrow$&$\surd$\\
	&`to stab' &&&\\
\end{tabular}
\z
}

%\largerpage
\noindent 
The verbs which represent this intermediate step are generally possible, but speakers merely do not need all of them (at least nowadays). A careful look at the data brings us to examples like (\ref{ex-glast}), in which the missing verbs can be found. The verbs at the intermediate step occur, but some of them not as frequent as the prefixed verb.

\eal\label{ex-glast}
\ex
\gll Also, wie man glast habe ich ja schon lesen können, {aber [\ldots]}\footnotemark\\
well how one.\textsc{nom} glasses have I.\textsc{nom} \textsc{ptcl}  already read can    but \\
\glt `Well, I could already read how to glass, but…'
\footnotetext{\url{http://forum.longboardz.de/showthread.php?5570-Glasing} (2017-10-30).}

\ex\label{49b}
\gll {Obsessiv [\ldots]}  dolcht er mit einem Kuli für sein Recht, dabei Blut statt Tinte spritzend.\footnotemark\\ 
obsessive daggers he.\textsc{nom} with a.\textsc{dat} biro for his.\textsc{acc} right thereby blood.\textsc{acc} instead ink.\textsc{acc} sprinkling\\
\glt `He obsessively daggers for his right with a biro, thereby sprinkling blood
\footnotetext{\url{https://kid37.blogger.de/topics/Super+8/?start=20} (2017-10-30).}
instead of ink.'
\zl

%\largerpage
\noindent 
The verbal prefixes in Figure~\ref{ex-verjüng} and Figure~\ref{ex-erdolch-b} seem to be functionless, because the verbalizing suffixes determine the category and  influence the argument structure. But examples like (\ref{ex-staub}) and (\ref{ex-staub2}) show that prefixes provide for additional changes in valency.

\eal\label{ex-staub}
\ex\noemph{Staub} [N – masc]\\`dust'
\ex\noemph{staub} [V]\\`to raise dust'
\ex\noemph{ent-staub} [V]\\`to free from dust'
\zl

\eal\label{ex-staub2}
\ex
\gll Der Teppich staubt. \\
     the.\textsc{nom} carpet dusts \\
\glt `The carpet raises dust.'
\ex
\gll Der             Mann   ent-staubt   den          Teppich. \\
     the.\textsc{nom}   man      \textsc{pref}-dust    the.\textsc{acc}   carpet \\
\glt `The man frees the carpet from dust.'
\zl

\noindent Thus, verbal prefixes are still heads, but rather unobtrusive ones, which leave the attention to their partner. They are categoryless and demand for a verbal base (cf. (\ref{ex-MSEL3})).

\ea\label{ex-MSEL3}
\noemph{ver-} [MSEL: V]
\z

\section{Inflected prefix verbs}\label{sec-inflpre}

We take a closer look at the possibilities of inflection for prefixed verbs now. Verbal prefixes are not responsible for the inflectional behavior of the derived verb. Whether a prefix verb belongs to the strong or to the weak inflection class depends on the base. Strong verbs like \emph{schreib} `to write' mark their past tense form by ablaut, whereas weak verbs like \emph{sag} `to say/tell' use the suffix   -\emph{te}. The inflectional behavior is transferred to prefix verbs with the respective bases. This is shown for the strong verb \emph{beschreib} `to describe' in (\ref{ex-schreib}) and for the weak verb \emph{besag} `say/mean' in (\ref{ex-sag}). We will concentrate on the tense feature here (as well as in the following examples) and set aside the mood feature and other features for ease of presentation.

\eal\label{ex-schreib}
\ex\noemph{schreib}  {}[V – pres{}] / \noemph{schrieb} {}[V – past{}]\\`to write'
\ex\noemph{beschreib} {}[V – pres{}] / \noemph{beschrieb} {}[V – past{}]\\`to describe'
\zl

\eal\label{ex-sag}
\ex\noemph{sag} {}[V – pres{}] / \noemph{sagte} {}[V – past{}]\\`to say/tell'
\ex\noemph{besag} {}[V – pres{}] / \noemph{besagte} {}[V – past{}]\\`to say/mean'
\zl

\noindent Deadjectival and denominal verbs (with and without a prefix) as the ones in (\ref{ex-red}) and (\ref{ex-staub3}) belong to the weak inflection class. The weak inflection is regular and represents the default variant.

\eal\label{ex-red}
\ex\noemph{rot} {}[A{}]\\`red'
\ex\noemph{röt} {}[V – pres{}] / \noemph{rötete} {}[V – past{}]\\ `to redden'
\ex\noemph{erröt} {}[V – pres{}] / \noemph{errötete} {}[V – past{}]\\`to blush'
\zl

\eal\label{ex-staub3}
\ex\noemph{Staub} {}[N – masc{}]\\`dust'
\ex\noemph{staub} {}[V – pres{}] / \noemph{staubte} {}[V – past{}]\\`to raise dust'
\ex\noemph{entstaub} {}[V – pres{}] / \noemph{entstaubte} {}[V – past{}]\\`to free from dust'
\zl

\largerpage
\noindent 
We have to explain now how the strong inflection in (\ref{ex-schreib}) comes to the base verb. One possibility would be to percolate an ablaut feature from the base to the next higher node. \citet{Lieber1992} denies such a percolation, because ablaut is marked by a diacritic feature and must therefore be excluded from percolation. Another possibility would be to allow certain variability in structure, so that inflectional processes can operate before derivational processes set in. The effects of such a structural variability are shown for the weak denominal prefix verb \emph{verjähr} `to become time-barred' in Figure~\ref{ex-verjahr}. The tense suffix -\emph{te} looks for a verb. It can either bind to the lower verb as in Figure~\ref{ex-verjährte} or to the higher verb as in Figure~\ref{ex-verjährte-b}.

\begin{figure}
\begin{subfigure}{.48\textwidth}
\centering
\scalebox{.8}{
\begin{forest}
	sm edges, empty nodes
	[{[V -- past]}
		[[[X\textsuperscript{af} [ver-]]]]
		[{[V -- past]}
			[V
				[N [jahr]]
				[V\textsuperscript{af} [{[$+$front]}]]
			]
			[[{[past | MSEL: V]}\textsuperscript{af} [-te, name=te]]]
		]
	]
\end{forest} }
\caption{Derivation follows inflection}\label{ex-verjährte}
\end{subfigure}
\begin{subfigure}{.48\textwidth}
\centering
\scalebox{.8}{
\begin{forest}
	sm edges, empty nodes
	[{[V -- past]}
		[V
			[[X\textsuperscript{af} [ver-]]]
			[V
				[N [jahr]]
				[V\textsuperscript{af} [{[$+$front]}]]
			]
		]
		[[[{[past | MSEL: V]}\textsuperscript{af} [-te, name=te2]]]]
	]
\end{forest} }
\caption{Inflection follows derivation}\label{ex-verjährte-b}
\end{subfigure}
\caption{Two alternative structures for \emph{verjährte} `became time-barred'}\label{ex-verjahr}
\end{figure}

\largerpage
The tense suffix of strong verbs corresponds to an ablaut feature in phonology, which changes the vowel of the stem. The structures in Figure~\ref{ex-verschlang} represent two possible analyses for the past tense form of the strong prefix verb \emph{verschling} `to devour' under the assumption of structural variability. The structure in Figure~\ref{ex-verschling} seems to be more adequate for prefix verbs which belong to the strong inflection class, because the base and the tense feature form a constituent and can easily amalgamate to \emph{schlang}, which is part of the paradigm of \emph{schling} `to gulp'.

\begin{figure}
\begin{subfigure}[b]{.48\textwidth}
\centering
\scalebox{.9}{
\begin{forest}
	sm edges, empty nodes
	[{[V -- past]}
		[[X\textsuperscript{af} [ver-]]]
		[{[V -- past]}
			[V [schling, name=schling]]
			[{[past | MSEL: V]}\textsuperscript{af} [\textsc{ablaut}, name=abl]]
		]			
	]
	\draw[decorate,decoration={brace, amplitude=6pt}] (abl.south east)-- (schling.south west) node[midway, below=5pt]{schlang};
\end{forest} }
\caption{Derivation follows inflection}\label{ex-verschling}
\end{subfigure}
\begin{subfigure}[b]{.48\textwidth}
\centering
\scalebox{.9}{
\begin{forest}
	sm edges, empty nodes
	[{[V -- past]}
		[V
			[X\textsuperscript{af} [ver-]]
			[V [schling, name=schling]]
		]
		[[{[past | MSEL: V]}\textsuperscript{af} [\textsc{ablaut}, name=abl2]]]
	]
\end{forest} }
\caption{Inflection follows derivation}\label{ex-verschling-b}
\end{subfigure}
\caption{Two alternative structures for \emph{verschlang} `devoured'}\label{ex-verschlang}
\end{figure}

The structures in Figure~\ref{ex-verjährte} and \ref{ex-verschling} do not come without problems. Combinations of a verbal prefix and its base are used independently in derivational processes like (\ref{ex-56kor}) and should therefore correspond to a constituent in the structural analysis.
%\itdopt{Soll es so wie in 52a. oder 52b. gemacht werden? Also die Übersetzung des Verbs mit rein oder nicht (wurde sich gewünscht, MK würde es aber lieber ohne machen, weil die Verben schon früher mit Übersetzung erwähnt wurden.)}
\eal\label{ex-56kor}
%\ex\noemph{verjähr} [V] `to become time-barred' $+$ -\noemph{ung} [N -- fem]\textsuperscript{af}
%$\rightarrow$  \noemph{Verjährung} [N -- fem]\\`limitation of time'
%\ex\noemph{verjähr} [V] $+$ -\noemph{ung} [N -- fem]\textsuperscript{af} $\rightarrow$  \noemph{Verjährung} [N -- fem]\\`limitation of time'
%\ex\noemph{beschreib} [V] $+$ -\noemph{bar} [A]\textsuperscript{af} $\rightarrow$ \noemph{beschreibbar} [A]\\`describable'
\ex
\gll {\noemph{verjähr} [V] $+$} {-\noemph{ung} [N -- fem]\textsuperscript{af} $\rightarrow$} {\noemph{Verjährung} [N -- fem]}\\
     {become time-barred}       {\hphantom{-}\textsc{nmlz}                                 } {limitation of time}\\
\ex
\gll {\noemph{beschreib} [V] $+$} {-\noemph{bar} [A]\textsuperscript{af} $\rightarrow$} \noemph{beschreibbar} [A]\\
     {describe                  } {-able         }                                      describable\\   
\zl

\noindent Information from inflectional marking is furthermore needed in syntax. That does not hold for inherent inflection like tense and mood, but for contextual features like number and person, which are relevant for agreement in syntax \citep[cf.][]{Booij1996}. Inflectional markings should therefore be highest in morphological structure, so that the syntactic component can easily get access to it. Even if the tense feature is not needed in syntax, it should be higher in structure than the prefix, because it classifies the whole event as past and not only a subpart of the verbal action. Not \emph{schling} `to gulp' alone is interpreted as past in Figure~\ref{ex-verschlang} but \emph{(etwas) verschling} `to devour (something)'. 

Additional problems appear if we use T and Agr as inflectional categories instead of assuming free features. In this case, the prefix would need selectional restrictions for a pure V in Figure~\ref{ex-verjährte-b} as well as Figure~\ref{ex-verschling-b}, for T in Figure~\ref{ex-verjährte} as well as Figure~\ref{ex-verschling}, and possibly for Agr in other examples. It is therefore less complicated to use analyses without structural variability and to connect derivational affixes to the structure before inflectional affixes are added. So, we need another solution for the task how to ablaut the root.

%\largerpage
Structures which are generated by different modules of grammar are not necessarily isomorphic. This is well known from bracketing paradoxes \citep[cf.\ e.g.][]{Spencer1988}, and from the interface of syntax and phonology (cf.\ \cite{Shattuck-HufnagelTurk1996}, among many others). Structural mismatches also occur at the morphology-phonology interface. \citet{AckemaNeeleman2007} assume the rule in (\ref{ex-ifthen}) to translate (morpho-)syntactic into phonological representations.

\ea\emph{Input correspondence} by \citet[344]{AckemaNeeleman2007}\label{ex-ifthen}
\begin{tabular}{@{}ll}
	If &an AFFIX selects (a category headed by) X,\\
	& the AFFIX is phonologically realized as /affix/, and\\
	& X is phonologically realized as /x/,\\
	then & /affix/ takes /x/ as its host.	
\end{tabular}
\z

\begin{sloppypar}
\noindent 
The rule in (\ref{ex-ifthen}) is able to create mismatches. We can see this with one of Spencer's bracketing paradoxes in example (\ref{ex-transgrammar}). The suffix -\emph{ian} semantically combines with the complex constituent \emph{transformational grammar} because the whole phrase describes a person who makes their studies in this specific framework. \citet{AckemaNeeleman2007} assume an isomorphism between the semantic and the morphosyntactic organization in examples of this kind. The bracketing in (\ref{ex-transgrammarian}) represents (a simplification of) the morphosyntactic structuring, whereas the bracketing in (\ref{ex-transfogrammarian}) shows the division into phonological units.
\end{sloppypar}

\eal\label{ex-transgrammar}
\ex\label{ex-transgrammarian}
{[[transformational grammar] -ian]}
\ex\label{ex-transfogrammarian}
(transformational) (grammarian)
\zl

\noindent The morphosyntactic constituent \emph{transformational grammar} is too complex to constitute a partner for the suffix in phonology. The suffix looks for a simpler phonological base and decides to combine with the category-determining head of its morphosyntactic sister. It syllabifies together with \emph{grammar}, so that we get the phonological constituents \emph{transformational} and \emph{grammarian} in (\ref{ex-transfogrammarian}).\footnote{\ili{German} bracketing paradoxes of this kind cannot result from a mismatch between morphosyntax and phonology because the prenominal adjective agrees with the suffix in gender. The mismatch must rather be shifted to the interface between morphosyntax and semantics. Such a difference does not influence our analysis for Figure~\ref{ex-verjahr} and \ref{ex-verschlang}, in which we are faced with a real mismatch to phonology.}

%\largerpage
The rule in (\ref{ex-ifthen}) is based on the common concept of head, in which the head corresponds to the category-determining constituent. Our present analysis differentiates between heads and category-determining constituents. A head typically determines the category, but it can also be categoryless. We can apply the rule in (\ref{ex-ifthen}) to our examples in Figure~\ref{ex-verjahr} and \ref{ex-verschlang}, if we interpret X as the category-determining element, which does not necessarily coincide with the head. The tense feature of our examples is morphologically bound to the whole prefix verb but combines with the category-determining root in phonology. So, we get the morphological bracketing in (\ref{59a}) and the phonological structuring in (\ref{59b}) and (\ref{59c}).\footnote{This analysis gives us a further argument for sorting out the structure under Figure~\ref{ex-erdolch-a} in the last section. The prefix in Figure~\ref{ex-erdolch-a} is represented with a category feature and should therefore attract the inflectional suffix phonologically. Such structures are ungrammatical.}

\eal\label{59}
\ex\label{59a}
[[X\textsuperscript{af} V] T\textsuperscript{af}]
\ex\label{59b}
 (ver) (jährte)
 \ex\label{59c}
 (ver) (schlang)
\zl

\noindent The prefix in (\ref{59}) is phonologically less integrated than the inflectional suffix. That comes unexpected with regard to (\ref{ex-ifthen}). But prefixes behave in a peculiar way. Several phonologists assume that prefixes are mapped onto a separate phonological word in \ili{German} \citep[cf.][Section~3.4]{Wiese1996}.\footnote{\citet{Booij1985} assumes that verbal prefixes with a reduced syllable like \emph{be}- and \emph{ge}- in \ili{German} and \ili{Dutch} do not correspond to a phonological word. They rather constitute an appendix to the phonological word of the base.} The inflectional affix -\emph{te} instead must be integrated into an adjacent phonological word because it consists of a reduced syllable, which cannot receive stress.

But how does morphology know that the past form of verbs like \emph{verschling} `to devour' or \emph{beschreib} `to describe' is realized by ablauting the root? Morphology does not know anything about the phonological realization. Morphology is only interested in the category of the individual constituents and the inflectional features, especially in the past value for the tense feature in our examples in Figure~\ref{ex-verjahr} and \ref{ex-verschlang}. Lexicon and phonology do the rest. Following the rule in (\ref{ex-ifthen}), suffixes phonologically integrate into the category-determining element. So, phonology but not morphology has access to a constituent of root and tense.
Phonology now asks the lexicon whether it has stored a suitable entry for such a constituent. The lexicon offers the form \emph{schlang} `gulped', which is accepted by phonology. If the lexicon has no entry on the whole word route for a specific request, the past form is realized by combining the root with the suffix -\emph{te}.\footnote{See \citet{CaramazzaEtAl1988} and \citet{Plag2006} for an interaction of whole word route and decomposition route.}

We have seen now that different kinds of affixes can be analyzed as heads in \ili{German}. Typical affixal heads bear a category, which is projected to the dominating node; the less typical ones lack a category and allow for categorical projection from their base. Morphological heads can no longer be identified by categorical projection alone. The crucial criteria for headedness are requirement and selection instead. Affixes are heads because they require a partner and select it by its properties.

\section{Prosodic behavior of heads}\label{sec-prosodic}

\largerpage
An argument for the head status of affixes comes from the interface to prosody. Heads in syntax are prosodically subordinated to their complement due to the stress condition in (\ref{ex-stresscon1}). This is shown in Figure~\ref{ex-book}, where the noun \emph{Buch} `book' receives the strongest stress in the bottom line. The strongest stress is marked for each constituent by the value 1.\footnote{The stress notation (but not the stress assignment process used here) goes back to \citet{ChomskyHalle1968}.} The stress value of the determiner \emph{ein} `a', which constitutes the head of the DP, is lowered by 1, whereas the nominal complement keeps its stress level. The same holds for the verb \emph{lesen} `to read', which heads the VP. Its stress level is reduced, while the stress pattern of the DP does not change.

\ea\label{ex-stresscon1}
Stress condition I (neutral stress)\footnote{This is a simplified part of the stress assignment condition by \citet[253]{Korth2014}. Similar conceptions of stress assignment in syntax are given by \citet{Jacobs1993} and \citet{Truckenbrodt2007}.}\\
Heads have a lower stress level than their complement.
\z

\begin{figure}
\centering
\begin{forest}
	sm edges, empty nodes
	[VP
		[DP
			[D [ein]]
			[NP [Buch, roof]]
		]
		[[V [lesen, name=lesen]]]
	]
\end{forest}\\
\begin{tabular}{p{13pt} p{10pt} p{5pt}}
	1&1&1\\
	\cline{1-2}
	2&1&\\
	\hline
	2&1&2\\
\end{tabular}
\caption{Structure and stress pattern for \emph{ein Buch lesen} `to read a book'}\label{ex-book}
\end{figure}

\largerpage
Affixal heads do not differ from syntactic heads in this point. The prefix in Figure~\ref{ex-vertreibung} is subordinate to the verbal root, and the nominal suffix is subordinate to the prefix verb.

\begin{figure}
	\centering
	\begin{forest}
		sm edges, empty nodes
		[N
		[V
		[X\textsuperscript{af} [ver-]]
		[V [treib]]
		]
		[[N\textsuperscript{af} [-ung, name=ung]]]
		]
	\end{forest}\\
	\begin{tabular}{p{15pt} p{13pt} p{7pt}}
		1&1&1\\
		\cline{1-2}
		2&1&\\
		\hline
		2&1&2\\
	\end{tabular}
	\caption{Structure and stress pattern for \emph{Vertreibung} `expulsion'}\label{ex-vertreibung}
\end{figure}

Some affixal heads seem to resist the stress condition. Among them are negation prefixes and several non-native affixes. The peculiarities in the phonological behavior of non-native affixes go back to the source languages and must be stored in the mental lexicon. In contrast to that, the stress pattern in words with negation prefixes is not accidental. Some examples are given in (\ref{ex-un-}) and (\ref{ex-in-a}). The prefixes in (\ref{ex-in-a}) are nonnative ones, but – regarding stress – they behave similarly to the native prefixes in (\ref{ex-un-}). The stressed syllable in (\ref{ex-un-}) and (\ref{ex-in-a}) and in most of the following examples is marked by italicization.

\eal\label{ex-un-}
\ex\label{ex-unweit}
\gll \underlineemph{un}-weit [A] \\ \textsc{neg}-far \\ 
\glt `not far'
\ex
\gll \underlineemph{Un}-kraut [N] \\ \textsc{neg}-herb \\ 
\glt `weed'
\zl
\eal\label{ex-in-a}
\ex\label{ex-unstabil}
\gll \underlineemph{in}-stabil [A] \\ \textsc{neg}-stable \\
\glt `unstable'
\ex
\gll \underlineemph{a}-tonal [A] \\ \textsc{neg}-tonal \\
\glt `atonal'
\zl

%\itdopt{HS+UF: Morphem-für-Morphem-Glossierung einfügen}
\noindent Words with negation prefixes usually realize a contrast to their unprefixed positive counterpart, which has already been mentioned by \citet[108]{AltmannKemmerling2000}. Contrast is accompanied by a focus feature on the contrasting element in the structural analysis. The focused constituent attracts stress due to the condition in (\ref{ex-stresscon2}), which outranks the condition in (\ref{ex-stresscon1}). This is shown for \emph{unehrlich} `dishonest' in Figure~\ref{ex-dishonest}.

\ea\label{ex-stresscon2}
Stress condition II (focus)\footnote{This is again a part of the stress assignment condition by \citet[253]{Korth2014}. \citet{Gussenhoven1992} makes similar assumptions with respect to focused constituents in syntax.}

Focused constituents have a higher stress level than non-focused constituents in the same   domain independently of the   structural relationship.
\z

\begin{figure}
\centering
\begin{forest}
	sm edges, empty nodes
	[A
		[[X\textsuperscript{af}\textsubscript{F} [un-]]]
		[A
			[N [ehr]]
			[A\textsuperscript{af} [-lich, name=lich]]
		]
	]
\end{forest}\\
\begin{tabular}{p{18pt} p{11pt} p{6pt}}
	1&1&1\\
	\cline{2-3}
	&1&2\\
	\hline
	1&2&3\\
\end{tabular}
\caption{Structure and stress pattern for \emph{unehrlich} `dishonest'}\label{ex-dishonest}
\end{figure}


Focus features are restricted to focus domains \citep[cf.][]{Rooth1992}.\footnote{Such domains are sometimes called foreground domains \citep[by e.g.][]{Heusinger1999}.} The focus features on the adjectives in (\ref{ex-schach}) highlight the adjectives inside the DPs but do not project any higher. The adjectives would otherwise receive stronger stresses than the noun \emph{Schach} `chess'.

\ea\label{ex-schach}
\gll Er sah {}[einen \underlineemph{alt}en\textsubscript{\textsc{f}} Mann{}]\textsubscript{\textsc{fd}} und {}[einen \underlineemph{jung}en\textsubscript{\textsc{f}} Mann{}]\textsubscript{\textsc{fd}} \underlineemph{Schach} spielen. \\
     he.\textsc{nom} saw \spacebr{}a.\textsc{acc} old man and \spacebr{}a.\textsc{acc} young man chess.\textsc{acc} play \\
\glt `He saw an old man and a young man playing chess.'
\z

\noindent Similar effects can be observed for word-internal foci \citep[cf.][Section~4.5]{Korth2014}. Focus projection does not cross the word level. The stress assignment outside the focus domain in (\ref{ex-untrue}) is not influenced by the word-internal focus on the negation prefix. The prefix and the designated syllables of the two nouns get equally high metrical prominences.

\ea\label{ex-untrue}
\gll Der              Mi\emph{nis}ter machte eine           {}[\emph{un}-\textsubscript{\textsc{f}} wahre]\textsubscript{\textsc{fd}} \emph{Aus}sage.\\ 
     the.\textsc{nom} minister made   a.\textsc{acc} \spacebr{}un-           true                       statement\\
\glt `The minister made an untrue statement.'
\z

\noindent There are words in which the negation prefix is not stressed. But instead of contradicting our previous explanation, words like the ones in (\ref{ex-unverbesserlich}) and (\ref{ex-unverantwortlich}) rather support it.

\eal\label{ex-unverbesserlich}
\ex\noemph{unver\underlineemph{bes}serlich}\\`incorrigible'
\ex\noemph{unaus\underlineemph{weich}lich}\\`inevitable'
\ex\noemph{un\underlineemph{glaub}lich}\\`unbelievable'
\ex\noemph{unver\underlineemph{wüst}lich}\\`indestructible'
\zl

\eal\label{ex-unverantwortlich}
\ex\noemph{unver\underlineemph{ant}wortlich}\\`irresponsible'
\ex\noemph{unver\underlineemph{gess}lich}\\`unforgettable'
\zl

\noindent Most words with this prosodic behavior do not express a genuine contrast. Their adjectival bases do not occur independently, so that there is no need for a word-internal focus marking on the prefix. The positive counterparts of the respective adjectives can be expressed by alternative words with similar meaning.

\ea
\begin{tabular}[t]{lllllll}
	a.&Er&ist&verbesserlich.&&$\rightarrow$&?\\
	&he.\textsc{nom}&is&corrigible&&&\\
	&`He is corrigible.'&&&&&\\
	b.&Er&ist&korrigierbar.&&$\rightarrow$&$\surd$\\
	  &he.\textsc{nom}&is&corrigible&&&\\
          &`He is corrigible.'
\end{tabular}
\z
\ea
\begin{tabular}[t]{lllllll}
	a.&Das&ist&ausweichlich.&&$\rightarrow$&?\\
	&that.\textsc{nom}&is&avoidable&&&\\
	& `That is avoidable.'&&&&&\\
	b.&Das&ist&vermeidbar.&&$\rightarrow$&$\surd$\\
	  &that.\textsc{nom}&is&avoidable&&&\\
          & `That is avoidable.'
\end{tabular}
\z
%\itdopt{An MK: Glossen fehlen jeweils bei a in Beispiel 64 und 65}

\noindent The positive counterparts of the adjectives in (\ref{ex-unverantwortlich}) are used independently, but in other contexts than the negated versions. A genuine contrast is absent once more.

\eal
\ex
\gll Der              Minister ist dafür     verantwortlich. \\
     the.\textsc{nom} minister is  there.for responsible \\
\glt `The minister is responsible for this.'

\ex
\gll Diese Entscheidung ist unverantwortlich. \\
	this.\textsc{nom} decision is irresponsible\\
\glt `This decision is irresponsible.'
\zl


\eal
\ex
\gll Der Minister ist vergesslich. \\
	the.\textsc{nom} minister is forgetful\\
\glt `The minister is forgetful.'

\ex
\gll Seine letzte Rede ist unvergesslich. \\
	his.\textsc{nom} last speech is unforgettable\\
\glt `His last speech is unforgettable.'
\zl

\noindent Stress on the negation prefix is not completely blocked. It is only less preferred in isolation as well as in predicative use, because of the missing genuine contrast. But speakers can interpret the words in (\ref{ex-unverbesserlich}) and (\ref{ex-unverantwortlich}) as contrasting with potential positive counterparts. Furthermore, the prefix tends to be stressed in attributive use. Such a stress results from a prominence shift in contexts where another strong stress follows. The phenomenon of prominence shift has been mentioned in earlier studies, e.g. by \citet{ChomskyHalle1968} and \citet{Selkirk1995} for \ili{English}, and \citet{Wiese1996} for \ili{German}.

\eal
\ex
\gll Dieses Werkzeug ist unent\underlineemph{behr}lich. \\
this.\textsc{nom} tool is essential \\
\glt `This tool is essential.'

\ex
\gll ein \underlineemph{un}entbehrliches Werkzeug \\
	a essential tool\\
\glt `an essential tool'
\zl

\largerpage
\noindent 
Some of the adjectives with a positive counterpart optionally show a stress pattern like the words in (\ref{ex-unverbesserlich}) and (\ref{ex-unverantwortlich}). Speakers vary in marking the contrast explicitly where it is useful and doing without it where it is not necessary. 

\eal
\ex
\gll Er ist be\underlineemph{lehr}bar. \\
	he.\textsc{nom} is teachable\\
\glt	`He is teachable.'
\ex 
%er \hspace{19pt} ist \underlineemph{un}belehrbar
\gll Er              ist \underlineemph{un}belehrbar.\\ 
     he.\textsc{nom} is  unteachable\\
\glt `He is unteachable.'
\ex
\gll Er              ist unbe\underlineemph{lehr}bar. \\
     he.\textsc{nom} is  unteachable\\
\glt `He is unteachable.'
\zl

\noindent 
Native affixes satisfy the phonological requirements for heads. They are prosodically subordinated to their base unless they carry focus features.

\section{Conclusion}\label{conclu-kor}

The previous sections discussed the head status of affixes in Standard \ili{German} and examined the hypothesis that all affixes are heads. We argued for a general head status of affixes based on different criteria which set affixes in parallel with syntactic heads. Affixes require a complement, have a lower projection level than their partner, and show effects of prosodic subordination unless they express a contrast. These characteristics equally hold for derivational and inflectional affixes. Prefixes are treated similarly to suffixes.

Several affixes are atypical heads. They lack a central property by which heads are normally identified. They do not bear a category feature, so that they do not directly influence the category of the superordinate morphological constituent. It is therefore necessary to disconnect the category determination from the identification criteria for morphological heads. Typical affixal heads have a categorical specification, which they share with the immediately dominating node, but direct category determination is not necessarily a criterion for the head status of morphemes.

Adjectival and nominal prefixes as well as the semantically bleached variants of diminutive suffixes appeared to be categoryless. Even verbal prefixes turned out to be categoryless heads. Whether inflectional suffixes come up without a category too, depends on the conception of grammar. We can label them with categories like T and Agr or analyze them as categoryless. Categorically underspecified affixal heads allow the non-head to project its category. Models in which inflectional suffixes are classified as categoryless must assume that features can percolate from head and non-head at once. The percolation is thereby restricted to categories and free features, whereby head features have priority over non-head features. Bound features cannot be untied from the category to which they are connected. Affixes are heads, but some of them are quite unusual ones. Thus, affixation in \ili{German} does not show anarchistic tendencies. Affixes have a rather temperate and diplomatic nature, leaving nearly all glory and attention to their partner.


{\sloppy
\printbibliography[heading=subbibliography,notkeyword=this]
}
\end{document}


% en
%      <!-- Local IspellDict: en_US-w_accents -->
