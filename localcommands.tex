\newcommand*{\orcid}[1]{}

% do not show the chapter number. It is redundant, since most references to figures are within the
% same chapter.
\renewcommand{\thefigure}{\arabic{figure}}

\newcommand{\rlapsub}[1]{\rlap{\sub{#1}}}

% \SetupAffiliations{output in groups = false, 
%                    separator between two = {\bigskip\\},
%                    separator between multiple = {\bigskip\\},
%                    separator between final two = {\bigskip\\}
%                    }


%%%%%%%%Alte Umlaute
\newcommand{\oldae}{$\stackrel{\textrm{\tiny e}}{\textrm{a}}$}
\newcommand{\oldoe}{$\stackrel{\textrm{\tiny e}}{\textrm{o}}$}
\newcommand{\oldue}{$\stackrel{\textrm{\tiny e}}{\textrm{u}}$}

\newcommand{\refl}{\REFL}
\newcommand{\pst}{\PST}


% Müller
\let\vref\ref


\let\citew\citet

\newcommand{\page}{}

% biblatex stuff
% get rid of initials for Carl J. Pollard and Carl Pollard in the main text:
\ExecuteBibliographyOptions{uniquename=false}




\newcommand{\nom}{\textsc{nom}}
\newcommand{\gen}{\textsc{gen}}
\newcommand{\dat}{\textsc{dat}}
\newcommand{\acc}{\textsc{acc}}


%\newcommand{\spacebr}{\hspaceThis{[}}



\newcommand{\acknowledgmentsEN}{Acknowledgements}
\newcommand{\acknowledgmentsUS}{Acknowledgments}


% no bf!!!111!
\let\textbfemph\emph

\newcommand{\textbfremoved}[1]{#1}
%\newcommand{\emphremoved}[1]{#1}


\newcommand{\noemph}[1]{#1}
\newcommand{\underlineemph}[1]{\emph{#1}}




% for editing, remove later
\usepackage{xcolor}
\newcommand{\added}[1]{{\red #1}}
\newcommand{\addedthis}{\todostefan{added this}}

\newcommand{\changed}[1]{\textcolor{orange}{#1}}




% Nolda

\theorembodyfont{\normalfont}
\let\restriction\relax
\renewtheoremstyle{break}{\item{\itshape ##1\ ##2}\newline\nopagebreak}{\item{\itshape ##1\ ##2\ (##3)}\newline\nopagebreak}
\theoremstyle{break}
\newtheorem{definition}{Definition}
\newtheorem{pattern}{Pattern}
\newtheorem{restriction}{Restriction}
\newunicodechar{‑}{\hbox{-}}
\newunicodechar{…}{\dots}
\newunicodechar{⁡}{\relax}
\newunicodechar{⁣}{\relax}
\newunicodechar{⁀}{\raisebox{+1ex}{\ensuremath\frown}}
\newunicodechar{⁐}{\raisebox{+1ex}{\ensuremath\frown}\setbox1=\hbox{\ensuremath\smile}\hspace{-\wd1}\raisebox{-1ex}{\ensuremath\smile}}
\newunicodechar{⪪}{\ensuremath{<\mathrel{\llap{\ensuremath{-}}}}}
\setkomafont{descriptionlabel}{\normalfont}
\ExecuteBibliographyOptions{labeldate=comp,labelnumber=true,defernumbers=true}
\defbibenvironment{sources}{\list{\printfield{labelprefix}\,\printfield{labelnumber}}{\settowidth{\labelwidth}{S\,0}\setlength{\labelsep}{\biblabelsep}\setlength{\leftmargin}{\labelwidth}\addtolength{\leftmargin}{\labelsep}\setlength{\itemsep}{\bibitemsep}\setlength{\parsep}{\bibparsep}}\renewcommand{\makelabel}[1]{##1\hfil}}{\endlist}{\item}
\newcommand{\citesource}[1]{\citefield{#1}{labelprefix}\,\citefield{#1}{labelnumber}}


% Was soll das machen?
\newcommand{\textstyleFootnoteSymbol}{}



% Was ist das???? St. Mü. 30.10.2021
%Kann weg. Damit waren die bücker transkripte aligniert. Habe das jetzt mit tabularx und hphantom gemacht

%\newlength{\calength} %tmp length to store the space 1. until [; 2. until ].
%
%%first argument speaker ID, second argument text. Optional argument left margin indicator (arrow or similar)
%\newcommand{\cabox}[3][]{\parbox{0mm}{\hspace*{-1cm}#1}%
%\parbox{1.5cm}{#2}%
%\parbox{9.6cm}{#3}\\%
%}
%
%%translation. First parbox is empty, second parbox takes the translation text
%\newcommand{\trsbox}[1]{\parbox{1.5cm}{~}%
%\parbox{9.6cm}{\itshape #1}\\%
%}
%
%%store the width of a string.
%\newcommand{\settablength}[1]{\settowidth{\calength}{#1}\global\calength=\calength}
%
%%print string and store its width. Useful if the first item of the aligned set is also the longest
%\newcommand{\inittab}[1]{#1\settablength{#1}}
%
%%insert horizontal white space equivalent to the stored width
%\newcommand{\skiptab}{\parbox{\calength}{~}}
%
%%print the argument and fill up with horizontal white space until the stored width is reached.
%\newcommand{\filledtab}[1]{\parbox{\calength}{#1}}



% for standalone compilations Felix: This is in the class already
%\let\thetitle\@title
%\let\theauthor\@author 
\makeatletter
\newcommand{\togglepaper}[1][0]{ 
\bibliography{../bib-abbr,../stmue,../localbibliography,
collection.bib}
  %% hyphenation points for line breaks
%% Normally, automatic hyphenation in LaTeX is very good
%% If a word is mis-hyphenated, add it to this file
%%
%% add information to TeX file before \begin{document} with:
%% %% hyphenation points for line breaks
%% Normally, automatic hyphenation in LaTeX is very good
%% If a word is mis-hyphenated, add it to this file
%%
%% add information to TeX file before \begin{document} with:
%% %% hyphenation points for line breaks
%% Normally, automatic hyphenation in LaTeX is very good
%% If a word is mis-hyphenated, add it to this file
%%
%% add information to TeX file before \begin{document} with:
%% \include{localhyphenation}
\hyphenation{
Arsch
anaph-o-ra
Bü-cking
con-stit-u-ents
Dor-drecht
For-schungs-ge-mein-schaft
Ge-schich-te
ha-ben
pho-nol-o-gy
pro-so-dic
pro-so-di-cally
Sal-pe-ter
sei-nen
Wil-liams
}
\hyphenation{
Arsch
anaph-o-ra
Bü-cking
con-stit-u-ents
Dor-drecht
For-schungs-ge-mein-schaft
Ge-schich-te
ha-ben
pho-nol-o-gy
pro-so-dic
pro-so-di-cally
Sal-pe-ter
sei-nen
Wil-liams
}
\hyphenation{
Arsch
anaph-o-ra
Bü-cking
con-stit-u-ents
Dor-drecht
For-schungs-ge-mein-schaft
Ge-schich-te
ha-ben
pho-nol-o-gy
pro-so-dic
pro-so-di-cally
Sal-pe-ter
sei-nen
Wil-liams
}
  % \memoizeset{
  %   memo filename prefix={hpsg-handbook.memo.dir/},
  %   % readonly
  % }
  \papernote{\scriptsize\normalfont
    \@author.
    \titleTemp. 
    To appear in: 
    Ulrike Freywald \& Horst Simon (eds.) Headedness and/or grammatical anarchy?
    Berlin: Language Science Press. [preliminary page numbering]
  }
  \pagenumbering{roman}
  \setcounter{chapter}{#1}
  \addtocounter{chapter}{-1}
}
\makeatother



% This does a linebreak for \gll for long sentences leaving space for the language at the right
% margin. The factor .0989 is needed since otherwise starred examples cause a linebreak.
% St.Mü. 17.06.2021 08.02.2021
\newcommand{\longexampleandlanguage}[2]{%
%\begin{tabularx}{.99\linewidth}[t]{@{}X@{}p{\widthof{(#2)}}@{}}%
%\begin{minipage}[t]{.99\linewidth}%
\begin{tabularx}{\linewidth}[t]{@{}X@{}p{\widthof{(#2)}}@{}}%
\begin{minipage}[t]{\linewidth}%
#1%
\end{minipage} & (\ili{#2})%
\end{tabularx}}

% ORCIDs in langsci-affiliations 
\usepackage{orcidlink}
\definecolor{orcidlogocol}{cmyk}{0,0,0,1}
\ProvideDocumentCommand{\LinkToORCIDinAffiliations}{ +m }
  {%
    \orcidlink{#1}
  }
